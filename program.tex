\section{Program Model}~\label{pmodel}
%
Let $Prog$ be a program. Let $Var$ be the set of variables in the program $Prog$
and let $Val$ be the set of values that each variable can take and may be a 
scalar number in set of integers $\mathbb{Z}$ or set of rationals
$\mathbb{Q}$ or reals $\mathbb{R}$.
%
We consider \emph{programs} with bounded loops and finite recursion depth 
along with safety properties given as a set of assertions, $\assertions$, 
in the program.  All the loop bounds in the program are known apriori. 
All functions calls in the program are inlined before analysis.  
%We also support pointers.  \rmcmt{other program model supported}.
%
\section{Semantic Representations of Program}
%
The \emph{concrete semantics} of a program is the most precise mathematical
description of the program behavior. Any other semantic representation of a
program obtained through static analysis such as {\emph{data-flow
analysis}~\cite{flow-book} or \emph{set-constraint}~\cite{Cousot:1995} 
based analysis, is an abstract semantics which is derived from the concrete 
semantics via Galois connections. 


Several semantic representation~\cite{Cousot04,semantic02} of a program have been proposed
in the literature to analyze program properties.  Cousot~\cite{Cousot04} defines
a hierarchy of abstraction semantics of a state transition system $M$ corresponding to a 
control-flow graph $G$.  Cousot starts from partial trace semantics and derive successive
approximations via Galois connection which follows the sequence -- a) \emph{partial 
trace semantic}, b) \emph{reflexive transitive closure semantic}, 
c) \emph{reachability semantic}, d) \emph{interval semantic}. Each semantic 
representation differ from the other in the precision of the information.  An 
abstract semantics is less precise than their concrete counterparts and hence 
can prove less program properties, but are cheaper to compute or approximate.  
Cousot in~\cite{semantic02} gives a complete range of abstract semantics of a program.
%


\para{Partial Trace Semantics}
%
The semantics of a program computes the \emph{states} of a program which gives a
concise mathematical meaning of a program.  The collecting semantics of a
program computes all possible memory states that can occur during the execution
of a program.
%
Recall that a trace $\pi$ of a state transition system $M=(\Sigma, \mathcal{R},
\mathcal{I})$ is a finite sequence of states that follows the transition
relation $\mathcal{R}$.  Let $\Sigma_{M}^n$ denote the set of finite execution 
traces of length $n$. Then, $\Sigma_M^0=\emptyset$, while  
$\Sigma_M^1=\{s \mid s\in \Sigma\}$. 
Further, a trace of length $(n+1)$ can be expressed recursively as 
$\Sigma_M^{(n+1)}=\{\sigma ss' \mid \sigma s\in \Sigma_{M}^n \wedge (s,s') \in
\mathcal{R}\}$.   
%


%Thus, a \emph{concrete semantics}~\cite{Cousot04} of a program is the set of all finite 
%execution traces, 
We will denote the \emph{partial trace semantics} as $(\llbracket M \rrbracket_{t})$ 
which is powerset of traces, $(\powerset(\Pi), \subseteq)$.  
%
A fixed point characterization of the trace semantics is defined as follows. 
%
\[
  \mathit{lfp}\; F_{M}^{t}\; \text{where}\; F_{M}^{t}(X) \mathrel{\hat=} \{s \mid s \in
  \Sigma\} \cup \{\sigma ss' \mid \sigma s \in X \wedge (s,s') \in \mathcal{R} \}
\]
%
\para{Collecting Semantics over Traces}
%
The collecting semantics~\cite{Cousot04} of $M$, $\Sigma_{M}^{*}$, 
is the set of all such finite execution traces of transition system $M$, and is given by 
$\Sigma_M^*\mathrel{\hat=} \bigcup_{n\geq0}\Sigma_M^{n}$. 
%
%
\para{Reachability Semantics}
%
%Whereas, Min\'{e}~\cite{minepaper} starts from 
A \emph{reachability} semantics~\cite{minethesis} of a state transition 
system $M=(\Sigma, \mathcal{R},\mathcal{I})$ is the set of states that are 
reachable from the initial states $\mathcal{I}$. A reachability
semantics is a complete lattice of powerset of
states defined as, $(\powerset(V \times (Var \rightarrow Value)), \subseteq,
\cup, \cap)$ and is denoted by $(\llbracket M \rrbracket_{r})$. 
%
A fixed point characterization of the reachability semantics is
defined as follows.
\[
  \mathit{lfp}\; F_M^{r}\; \text{where}\; F_{M}^{r}(S) \mathrel{\hat=} \mathcal{I} \cup \{s' \mid
   \exists s \in S \wedge (s,s') \in \mathcal{R} \} 
\]
%
Cousot in~\cite{Cousot04} shows that a reachability semantics can be derived
from a partial trace semantics through a sequence of abstractions.  Thus, 
the trace semantics $\llbracket M \rrbracket_{t}$ is more precise than the 
reachability semantics $\llbracket M \rrbracket_{r}$.  However, it is
practically infeasible to compute the trace semantics of a program 
that finds all memory states which occur during a program execution. 
%
\rmcmt{continuation text}
\section{Trace Transformers and CFG Safety}
%
The semantics of a state transition system $M$ corresponding to a control-flow graph
$G$ can be defined in terms of \emph{state semantics}~\cite{Cousot04} or \emph{trace
semantics}~\cite{Cousot04}.  Informally, a state semantics $(\llbracket M \rrbracket_{s})$ 
or reachable semantics defines the reachable states of $M$ starting from an initial state.  
Whereas, a trace semantics $(\llbracket M \rrbracket_{t})$ of $M$ is the set of well-formed 
traces of $M$. 
%
In section~\ref{state-transformer}, we define various classical 
\emph{state transformers} of $M=(\Sigma, \mathcal{R}, \mathcal{I})$ 
that operates over the powerset of states, $\mathcal{P}(\Sigma)$.
%
We now define various trace transformers that operates over $\powerset(\Pi)$ 
and formulate \emph{CFG safety} in terms of these transformers. The term
$\sigma_i \in \Sigma$ refers to a state and $\pi \in \Pi$ ranges over a set of 
traces in $\Sigma^*$. 
%
\para{Strongest Postcondition} A \emph{strongest postcondition}
  transformer, $tpost(T)$, of a set of traces $T \subseteq \Pi$ is defined as  
  $tpost(T) \mathrel{\hat=} \{\pi.\sigma_l \mid \exists \pi \in \Pi.\; \pi \in T
  \wedge \{\sigma_a \ldots \sigma_k\} \in \pi \wedge \sigma_k \rightarrow \sigma_l\}$. 
  The term $\pi.\sigma_l$ denotes extending trace $\pi$ with its postcondition
  state $\sigma_l$.\\
%
\para{Weakest Precondition} A \emph{weakest precondition}
  transformer, $\widehat{tpre(T)}$, of a set of traces $T \subseteq \Pi$ is defined as  
  $\widehat{tpre(T)} \mathrel{\hat=} 
  \{\sigma_b\pi \mid \forall \sigma_a \in \Sigma.
  \sigma_a\sigma_b\pi \in T \vee \sigma_a \not\rightarrow
  \sigma_b \}$. \\
%
\para{Existential Precondition} An \emph{existential precondition}
  transformer, $tpre(T)$, of a set of traces $T \subseteq \Pi$ is defined as  
  $tpre(T) \mathrel{\hat=} \{\sigma_a.\pi \mid \exists \pi \in \Pi.\; \pi \in T
  \wedge \pi = \{\sigma_b \ldots \sigma_k\} \wedge \sigma_a \rightarrow
  \sigma_b\}$. 
  The term $\sigma_a.\pi$ denotes extending trace $\pi$ with its precondition
  state $\sigma_a$.\\
%
\para{Universal Postcondition} An \emph{universal postcondition}
  transformer, $\widehat{tpost(T)}$, of a set of traces $T \subseteq \Pi$ is defined as  
  $\widehat{tpost(T)} \mathrel{\hat=}
  \{\pi\sigma_k  \mid \forall \sigma_l \in \Sigma.
  \pi\sigma_k\sigma_l \in T \vee \sigma_k \not\rightarrow
  \sigma_l \}$. 
%
%-------------------------------------------------------------------------------
\para{CFG Safety}  
%
Let $G = (V, E, \mathcal{S}, \mathcal{T}, I, E)$ 
  be a CFG with a special error node $\err \in V$.  A trace $\pi
  \in G$ is safe if it does not terminate in the error location. A CFG $G$ is
  \emph{safe} w.r.t. $\err$ if all traces $\pi$ of $G$ are safe
  w.r.t $\err$.
%
We define two trace transformers over $\powerset(\Pi)$ for safety checking of a
CFG, an unsafe trace transformer, $f_{unsafe}$, and a safe trace transformer,
$f_{safe}$. A CFG $G$ is \emph{safe} exactly if $f_{unsafe}^{G}(\Pi)=\emptyset$.
%
\begin{equation}\label{eq:unsafe}
f_{unsafe}^{G}(T) \mathrel{\hat=} \{\pi \in \Pi \mid \pi \in T \wedge
\pi\;\text{ is well formed and not safe} \} 
\end{equation}
%
\begin{equation}\label{eq:safe}
f_{safe}^{G}(T) \mathrel{\hat=} \{\pi \in \Pi \mid \pi \in T \vee \pi\;
\text{is not well formed or safe} \} 
\end{equation}
%
\para{Fixed-point Characterization of Unsafe Trace Transformer} 
%
We present a fixed point characterization of the unsafe safe transformer for
counterexample search and conflict generalization. A counterexample search 
corresponds to finding an abstract representation of a counterexample trace 
over $\powerset(\Pi)$. 
%
Let $\chi$ denote the set of states $\{(v,\omega)\mid v = \err \}$ that 
are at the error location.  Then, the unsafe trace transformer of CFG $G$, can 
be characterised as follows.
%
\begin{proposition}
%
  $f_{unsafe}^{G}(T) = T \cap (\mathit{lfp}\; Z.\;\mathcal{I} \cup tpost(Z)) \cap 
    (\mathit{lfp}\; Z.\;\chi \cup tpre(Z))$
%
\end{proposition}
%
\begin{proof}
  The fixed point characterization of unsafe trace transformer is described as
  follows. 
  Let $C1  = (lfp Z.\;\mathcal{I} \cup tpost(Z))$ and $C2 = (lfp Z.\; \chi \cup
  tpre(Z))$. The set $C1$ gives set of traces that starts from an initial
  state $\mathcal{I}$ and follows the transition relation $\mathcal{R}$.  
  The set $C2$ gives the set of traces that follows the transition relation 
  and terminates in an error state.  The set $C1 \cap C2$ is the set of 
  well-formed traces that terminates in an error state. It follows 
  that $f_{unsafe}^{G}(T) = T \cap C1 \cap C2$.  
\end{proof}
%
Note that the two fixed points above, for computing $C1$ and $C2$, represents a
forward and backward analysis respectively. Combination of forward and backward
analysis provide strictly greater precision than applying either in
isolation~\cite{Cousot99} in the abstract.  
%
%%%%%%%%%%%%%%%%% TRACE SEMANTICS APPROXIMATION %%%%%%%%%%%%%%%%%%%
\section{Approximations of Trace Semantics}
%
Figure~\ref{fig:semantic} shows the sequence of syntactic translation steps 
and semantic approximations of concrete set of traces, $\powerset(\Pi)$.  
A program is represented by a Control-flow Graph $G=(V,E,S,\mathcal{T},I,E)$.  
Classical abstract interpretation interprets a CFG as equation 
system~\cite{minethesis,Schmidt98,tacas12}, which is described next.  The box in 
\emph{red} shows the corresponding lattices that an abstract interpreter
operates on.  The ACDLP procedure presented in this paper operates on a
different lattice which is obtained by syntactic translation step
$\mathcal{T}_2$ that translates a CFG to SSA form. The bounded box in 
\emph{green} shows the corresponding lattices for ACDLP. In this section, 
we first describe an equation system that is derived from a CFG which is commonly
used for abstract interpretation.  Later, we describe a syntactic translation 
of CFG to Static Single Assignment form which is used for ACDLP procedure.   
In section~\ref{semantic-trace}, we desribe the corresponding lattices over 
these structures and various transformers to operate on these lattices. 
%
\begin{figure}[htbp]
\centering
\scalebox{.70}{\import{figures/}{semantic.pspdftex}}
\caption{Semantic Representation of Program \label{fig:semantic}}
\end{figure}
%
\rmcmt{\para{Program Transformation $\mathcal{T}_1$}}
%
\subsection{CFG to Constraint System}~\label{ssa-cite}
%
Recall that an operational semantics of a CFG can be defined using a 
state transition system. An equation 
system can be derived from a state transition system, which introduces a set
values varaible $(S_v)_{v \in V}$ to each program location that takes values in
powerset of environments, $\powerset(Var \mapsto Value)$.  These equations perform 
point-wise lifting of $\powerset(Var \mapsto Value)$ to $V$.   The variables $S_v$
are related through the transformers associated with program statements that express the 
data-flow equations between individual control-flow nodes in the CFG.  Haller
et. al. call this equation system~\cite{minethesis} 
\emph{static analysis equation}~\cite{tacas12}.  Solution to these equations 
yields the data-flow information of $G$.  The resulting lattice that describes 
these static analysis equations is given by $(\mathcal{F}_G, \sqsubseteq)$.   
%


Figure~\ref{fig:se} gives the CFG representation of a program (in left) 
and its corresponding static analysis equations (in right).  The 
function $post(S)$ computes the successor state of a set of states 
$S$ that can be reached in one step.  A set-valued variable $S_v$ is 
introduced at every control location $v$ in the CFG.  Note that the variables
$S_v$ are related through the post-condition transformer $post$ associated with
the program statements, for example, the variable $S_{n6}$ at the loop head
merges the control-flow from $S_{n5}$ and $S_{n7}$.  The CFG is \emph{safe} if
$S_{Error}$ is empty $(\emptyset)$. 
%
\para{Collecting Semantics over Equation Systems}
%
A collecting semantics over static analysis equations of a program gathers for 
each program variables and program location its value during program execution.
%
\begin{figure}[t]
%\scriptsize
\centering
\begin{tabular}{c|c|}
\hline
  Control-Flow Graph & Static Analysis Equation \\
\hline
\begin{minipage}{4.2cm}
\scalebox{.65}{\import{figures/}{semantic-example.pspdftex}}
\end{minipage}
&
\begin{minipage}{10cm}
$\begin{array}{l}
     S_{n1} = \top, \\
     S_{n2} = post_{y>0}(S_{n1}) \\
     S_{n3} = post_{y=0}(S_{n1}) \\
     S_{n4} = post_{y<0}(S_{n1}) \\
     S_{n5} = post_{x:=2}(S_{n2}) \cup post_{x:=0}(S_{n3}) \cup post_{x:=-2}(S_{n4}), \\
     S_{n6} = post_{y:=x*y}(S_{n5}) \cup post_{y:=y+2}(S_{n7}), \\ 
     S_{n7} = post_{y \leq 20}(S_{n6}), \\
     S_{Error} = post_{y<0}(S_{n6})
\end{array}$
\end{minipage}
\\
\hline
\end{tabular}
\caption{\label{fig:se} A CFG and its static analysis equation}
\end{figure}
%
\subsection{Logical Encoding of Program Semantics}~\label{ssa-cite}
%
\para{Program Transformation $\mathcal{T}_2$}
%
The program transformation $\mathcal{T}_2$ shown in Figure~\ref{fig:semantic}
involves two separate steps. 
\begin{enumerate}
  \item Generation of a \emph{bounded program}
  \item Translation of acyclic control-flow of bounded program to SSA
\end{enumerate}
%
\para{Generation of Bounded Program}
%
A \emph{bounded program} is obtained from an input program by a transformation 
that unfolds loops and recursions a finite number of times, which generates an
acyclic control-flow structure.  Figure~\ref{fig:unwind} and figure~\ref{fig:unwind-cfg} 
gives an input program and the corresponding bounded program obtained through
loop unrolling, respectively.
%
\begin{figure}[htbp]
\centering
\vspace*{-0.2cm}
  \scalebox{.90}{\import{figures/}{unwind.pspdftex}}
\caption{Input Program with bounded loops
  \label{fig:unwind}}
\end{figure}
%
\begin{figure}[htbp]
\centering
\vspace*{-0.2cm}
  \scalebox{.90}{\import{figures/}{unwind-cfg.pspdftex}}
\caption{Generating Bounded Program throug loop unrolling
  \label{fig:unwind-cfg}}
\end{figure}
%
The CFG representation of the bounded program is 
translated to a Static Single Assignment (SSA) form 
%through the syntactic translation step, $\mathcal{T}_2$, as illustrated in Figure~\ref{fig:semantic}.  
The translation from CFG to SSA is a well known technique 
in optimizing compiler and verification~\cite{ssa1,ssa2,ssa1988,ssa1991}.  We 
briefly explain the details of the CFG to SSA translation since this is standard. 
%
\para{CFG to Static Single Assignment Form}
%
The translation from CFG to SSA follows two separate steps.  
The first step gives a unique \emph{index} to each
definition of a program variable, and each use of that variable is given the
index of the definition that reaches it; the second step insert special 
$\phi$-functions at control-flow join points where a given variable may have
more than one reaching definition.  The argument to a $\phi$-function is the 
set of all indexed instance of the variable that could reach the join point.
Based on the current execution trace, the $\phi$-function select an appropriate
instance of the variable and assign it to a new instance of the variable. The
translation involves the following steps. 
%
\begin{enumerate}
\item A unique name for each definition point in the procedure, shown in
  figure~\ref{fig:ssa-simple}. 
\item Identifies points in the procedure that merges different values from
  distinct control-flow paths, shown in figure~\ref{fig:ssa-simple}  
\item Identification of induction variables in loops becomes easy by inserting a
  $\phi$-function for any variable that is modified inside the
    loop~\cite{Gerlek:1995}.
\end{enumerate}
%
\begin{figure}[htbp]
\centering
\vspace*{-0.2cm}
  \scalebox{.90}{\import{figures/}{ssa-simple.pspdftex}}
\caption{CFG to SSA Translation Steps
  \label{fig:ssa-simple}}
\end{figure}
%
Figure~\ref{fig:unwind-ssa} gives an example of SSA translation for the bounded
program in Figure~\ref{fig:unwind-cfg}.
%for a simple CFG, following the above mentioned steps.
%
\begin{figure}[htbp]
\centering
\vspace*{-0.2cm}
  \scalebox{.90}{\import{figures/}{unwind-ssa.pspdftex}}
  \caption{Static Single Assignment form for Bounded Program of
  Figure~\ref{fig:unwind-cfg}
  \label{fig:unwind-ssa}}
\end{figure}
%
\para{Collecting semantics over SSA}
%
A collecting semantics over SSA of a bounded (loop free) program gathers for each 
SSA variables its value during program execution.
%
\para{Advantages of SSA}
%
Previous researches have shown that Static Single Assignment form has
advantages over other forms of programs representation which allows compilers to
facilitate program analysis through simplification of data-flow analysis such as
def-use and use-def chains and design of optimization
algorithms~\cite{ssa1991,ssa1,ssa2}. 
%
\para{Exact SSA semantics} An SSA semantics is \emph{exact} if it is
constructed from a bounded program.  Recall that a bounded program is 
obtained through \emph{complete} unwindings of all loops and recursions, which
gives an acyclic code.  
Therefore, for acyclic code, an SSA exactly represents the strongest 
post-condition computation of the code.
The lattice that describe these SSA constraints over SSA 
variables is given by $(\mathcal{SA}_G, \subseteq_{SA})$.  
%
%Chapter~\ref{six} gives the \emph{exact}, \emph{overapproximate} and \emph{underapproximate} SSA semantics for an input program. 
%
In this paper,  we restrict our formalizations to bounded programs which 
gives an exact SSA semantics of a program.  
%
\para{SSA Safety}
%
The translation from CFG to SSA is represented by a set 
$\constraints=\program\cup\{\neg \bigwedge_{\assertion\in\assertions} \assertion\}$,
where $\program$ contains an encoding of the statements in the program as
constraints, obtained after translating the program into single static 
assignment (SSA) form~\cite{ssa88,ssa1988,ssa1991}. 
%
Based on the above program representation, we define a \textit{safety formula}
($\formula$) as the conjunction of all elements in $\constraints$, that is,  
$\formula:= \bigwedge_{\constraint\in\constraints} \constraint$.  The formula 
$\formula$ is unsatisfiable if and only if the program is safe.
\rmcmt{Intuitively, $\formula$ is the usual SAT encodings of BMC for
reachability.} 
%
\begin{example}
The safety formula $\formula$ for the bounded program of 
figure~\ref{fig:unwind-ssa} is shown below.  Note that $\formula$ is obtained
by taking conjunction of all SSA and the negation of set of assertions. 
%
\begin{equation}\label{eq:ssa}
\begin{array}{l}
  \formula = (x_0 = 0) \wedge
     (x_0 \leq 2) \wedge
     (x_1 = x_0 + 1) \wedge
     (x_1 \leq 2) \wedge
     (x_2 = x_1 + 1) \wedge
     ((x_1 > 2) \vee (x_2 > 2)) 
\end{array}
\end{equation}
\end{example}
%
\para{Translation from Static Single Assignment to Program Trace}
%
\begin{figure}[htbp]
\centering
\vspace*{-0.2cm}
  \scalebox{.90}{\import{figures/}{syn-map.pspdftex}}
\caption{Concretization-Based Approximation of Trace Semantics in ACDLP
  \label{fig:syn-map}}
\end{figure}
%
Recall that $(\powerset(\Pi),\subseteq)$ and $(\mathcal{SA}_G, \subseteq_{SA})$ 
forms a \emph{concretization-based galois connection} through a concretization
operator $\gamma_T$, as shown in figure~\ref{fig:lattice}. Cousot and Cousot in
~\cite{CC92} propose a relaxation of galois connection framework that allows to work
only with a concretization operator $\gamma$ or dually an abstraction operator
$\alpha$.  We will use this concept to establish the relationship between 
$(\powerset(\Pi),\subseteq)$ and $(\mathcal{SA}_G, \subseteq_{SA})$ only through a 
concretization operator, $\gamma_T$.  Figure~\ref{fig:syn-map} illustrates that 
$\gamma_{T}$ can be achieved through the syntactic back-translation steps, 
$\mathcal{T}_1'$ and $\mathcal{T}_2'$.  
%
\para{Program Transformation $\mathcal{T}_2'$}
%
The program transformation $\mathcal{T}_2'$ shown in Figure~\ref{fig:syn-map}
involves two separate steps. 
%
\begin{enumerate}
  \item Copy-insertion~\cite{Srikant:2007} to remove $\phi$-nodes. 
  \item \rmcmt{Variable renaming in each basic blocks}
\end{enumerate}
%
The translation of SSA to an original program following these syntactic translation
steps is a well known technique in compiler-based code
optimization~\cite{Briggs:1998,ssa1991,Srikant:2007}.
Briggs et. al.~\cite{Briggs:1998} give an algorithm for translation of SSA 
to the original program by replacing $\phi$-functions with appropriately-placed
\emph{copy} instructions.  The program semantics is preserved by inserting a 
copy statement for each argument of $\phi$-function, at the end of each 
predecessor block of the $\phi$-node. Figure~\ref{sp} shows an example of copy
insertion to replace $\phi$-function in SSA.  We skip the details of the translation 
from SSA back to original program since this is standard and beyond the scope of 
this paper.
%
\rmcmt{what happens to ssa variables}.
%
\begin{figure}[htbp]
\centering
\vspace*{-0.2cm}
  \scalebox{.70}{\import{figures/}{sp.pspdftex}}
  \caption{Back-translation from SSA to CFG using copy insertion
  \label{sp}}
\end{figure}
%
\begin{definition} (Concretization) 
  \[
    \gamma_T \colon (\mathcal{SA}_G,\subseteq_{SA})
    \overset{\gamma_{T}}{\rightharpoonup}
    (\powerset(\Pi), \subseteq) 
  \]
$\gamma_T$ is automatically monotonic since $\gamma_T(A) = \pi$, where $A \in
  \mathcal{SA}_G$ is an exact abstraction of $\pi \in \Pi$.  That is, 
  $(\mathcal{SA}_G, \subseteq_{SA})$ is an exact abstraction of 
  $(\powerset(\Pi),\subseteq)$ for the programs considered in this paper (see
  section~\ref{pmodel}).
\end{definition}
%  
\rmcmt{Given a SSA representation of a program, the semantics of a program 
can be understood as a constraint system which is a conjunction of all 
SSA constraints.  A satisfying model of safety formula ($\formula$) 
containing SSA constraints corresponds to an unsafe trace (or counterexample) 
of the original program.  Later, we present a procedure for finding a satisfiable
assignment of $\formula$ in section~\ref{acdlp}.  Note that the model of
$\formula$ contains concrete assignments to SSA variables, which has to be
mapped back to the original program.}
%
%%%%%%%%%%%%%%%%%%%%%%%%%%%%%%%%%%%%%%%%%%%%%%%%%%%%%%%%%%%%%%%%%%%
%%%%%%%%%%%%%%%%%%%%%%% Trace-based Abstraction  %%%%%%%%%%%%%%%%%%%%%
%
\section{Lattice for Approximation of Trace Semantics}~\label{semantic-trace}
%
In this section, we describe the various lattices over the static analysis
equations and static single assignment form for a given control-flow graph.  
Figure~\ref{fig:lattice} shows the various lattices that approximate the
concrete trace semantics.
%


An important property of a clause learning SAT solver is that each meet
irreducibles of the partial assignments domain admits precise
complements~\cite{sas12}.  This property enables learning of domain 
elements that navigates the search away from the conflicting region.   
In order to lift CDCL to program analysis, it is imperative to find a suitable
trace based abstraction that admits precise complementation property. 
To this end, we first show that conventional means to represent a program through 
abstraction of a static analysis equation does not admit precise 
complements with respect to unsafe traces.   In this paper, we present an 
abstraction of program traces using logical encoding of CFG that bounded 
model checkers use.  We will show in Section~\ref{complement} that this 
representation allows us to lift CDCL to program analysis.  
%
%Hence, our formalizations can be understood as a full abstract interpretation 
%account of bounded model checking.
%\rmcmt{with the hope that this leads to ideas of how
%loops can be approximated in bounded model checking.} 
%
\begin{figure}[htbp]
\centering
\vspace*{-0.2cm}
  \scalebox{.70}{\import{figures/}{lattice.pspdftex}}
  \caption{Approximation of Trace Semantics used by ACDLP (right) and Abstract
  Interpretation (left)
  \label{fig:lattice}}
\end{figure}
%
\subsection{Concrete Flow Lattice}
%
Abstract interpretation technique for program analysis computes 
a fixed-point over static analysis equations~\cite{CC79,octagon}.  
The advantage of using approximations of static analysis equation 
over trace based abstraction is that the former only requires 
abstractions of memory states.  \\
%
We now define the \emph{concrete flow lattice} that describes these static 
analysis equations with respect to the set of control locations. 
%
\begin{definition} (Concrete Flow Lattice). A \emph{concrete flow lattice}
  corresponding to a CFG $G= (V, E, \mathcal{S}, \mathcal{T}, I)$  
  is a complete lattice $(\mathcal{F}_{G}, \sqsubseteq, \sqcap, \sqcup)$, 
  defined as follows. 
\end{definition}
  \[
    \mathcal{F}_{G} \mathrel{\hat=} V \rightarrow \powerset({\Omega}) \qquad \forall a,b \in
     \mathcal{F}_{G}.\; a \sqsubseteq b\; \text{if}\; \forall v \in V.\; a(v)
     \subseteq b(v) 
  \]

  \[
     a \sqcap b \mathrel{\hat=} v \mapsto a(v) \cap b(v) \qquad 
     a \sqcup b \mathrel{\hat=} v \mapsto a(v) \cup b(v)
  \]
%
%The static analysis equations corresponding to a control flow graph is shown in
%Figure~\ref{fig:se}.
%
\Omit{
Recall that a CFG statement $s$ corresponds to a state transition, that is, 
$\mathcal{ST}_s \subseteq \Omega \times \Omega$.  Hence, we define a
\emph{postcondition transformer} and \emph{precondition transformer} 
for $\mathcal{T}(u,v)=s$.
%
\[ 
  post_{u,v}(S) \mathrel{\hat=} \{\omega_k \mid \exists\; \omega_j \in S.\;
  (\omega_j,\omega_k) \in \mathcal{ST}_s \}
\]
%
\[ 
  pre_{u,v}(S) \mathrel{\hat=} \{\omega_j \mid \exists\; \omega_k \in S.\;
  (\omega_j,\omega_k) \in \mathcal{ST}_s \} 
\]
%
\begin{definition} (Concrete Flow Lattice Transformers). We define a strongest
  postcondition transformer and an existential precondition transformer of a 
  concrete flow lattice, $\mathcal{F}_{G}$ as follows.  
\end{definition}
  \[
    fpost \colon \mathcal{F}_{G} \rightarrow \mathcal{F}_{G} 
    \qquad 
    fpre \colon \mathcal{F}_{G} \rightarrow \mathcal{F}_{G} 
  \]
  \[
    fpost(v \mapsto a(v)) \mathrel{\hat=} \underset{(u,v) \in E}{\bigcup}
    post_{u,v} \circ a(u) 
    \qquad 
    fpre(v \mapsto a(v)) \mathrel{\hat=} \underset{(v,w) \in E}{\bigcup}
    pre_{v,w} \circ a(w) 
  \]
%
}
The concrete flow lattice is an over-approximation of the concrete powerset of
traces and there exists a galois connection between the two as shown below. 
\[
  (\powerset({\Pi}),\subseteq,\cup,\cap)
    \galois{\alpha_{F}}{\gamma_{F}}
    (\mathcal{F}_{G},\sqsubseteq,\sqcup,\sqcap)
\]
We define the abstraction and concretization functions between the concrete
trace domain $\powerset({\Pi})$ and concrete flow lattice $\mathcal{F}_{G}$.
\[
  \alpha_{F}(T) \mathrel{\hat=} \lambda v. \{\omega \mid \exists
  \pi \in T.\; (v,\omega) \in \pi \} 
\]
\[
  \gamma_{F}(a) \mathrel{\hat=} \{\pi \in \Pi \mid \forall (v,\omega) \in \pi.\;
  \omega \in a(v) \wedge \pi \in \Pi_{wf} \}
\]
%
It is easy to see that $\alpha_F \circ \gamma_{F}$ is deflationary while 
$\gamma_F \circ \alpha_F$ is inflationary, and the pair $(\alpha_F, \gamma_F)$
forms a galois connection. We prove this in proposition~\ref{ag}. 
%


For the purpose of static analysis, the concrete semantics of program's
control-flow graph is over-approximated using abstract interpretation
methodology~\cite{CC79}. The concrete domain of memory states 
$\powerset({\Omega})$ is abstracted to construct an abstract program
model which is then used for abstract program analysis.  Most static program
analyzers build an abstraction $A$ of the memory states $\Omega$, shown below. 
\[
  (\powerset({\Omega}),\subseteq)
    \galois{\alpha_{AF}}{\gamma_{AF}}
    (\mathcal{A},\sqsubseteq_{AF})
\]
%
\begin{definition} (Abstract Flow Lattice). An \emph{Abstract Flow Lattice}
  corresponding to a CFG $G= (V, E, \mathcal{S}, \mathcal{T}, I)$ and 
  an abstraction $\mathcal{A}$ over concrete memory states $\Omega$
  is a complete lattice $(\widehat{\mathcal{F}_{G}}, \sqsubseteq_{AF}, \sqcap_{AF},
  \sqcup_{AF})$, 
  which is defined as follows. 
\end{definition}
  \[
    \widehat{\mathcal{F}_{G}} \mathrel{\hat=} V \rightarrow \mathcal{A} \qquad \forall a,b \in
     \widehat{\mathcal{F}_{G}}.\; a \sqsubseteq b\; \text{if}\; \forall v \in V.\; a(v)
     \sqsubseteq_{AF} b(v) 
  \]
%
  \[
    a \sqcap_{AF} b \mathrel{\hat=} v \mapsto a(v) \sqcap_{AF} b(v) \qquad 
     a \sqcup_{AF} b \mathrel{\hat=} v \mapsto a(v) \sqcup_{AF} b(v)
  \]
%
\subsection{Complementation in Abstract Flow Lattice}~\label{complement-fg}
%
A meet irreducible of abstract flow lattice $\widehat{\mathcal{F}_{G}}$ is defined as
follows.
\[
  \sqcap_{irrd}(\widehat{\mathcal{F}_{G}}) \mathrel{\hat=} 
  \left\{\begin{array}{l@{\quad}l}
    n \mapsto a & \text{if}\; n \in V \wedge a \in \sqcap_{irrd}(\mathcal{A}) \\
    n \mapsto \top & \text{otherwise} \\
  \end{array}\right.
\]  
%
Here, a node $n \in V$ is mapped to an abstract memory state $a \in \mathcal{A}$, 
where $a$ is a meet irreducible of $A$.  Let $n \mapsto \bar{a}$ be a precise 
complement of $n \mapsto a$, where $\bar{a}$ is a precise complement of $a$.    
However, the two sets are not equal, that is, 
$\gamma_{AF}(n \mapsto a) \neq \neg \gamma_{AF}(n \mapsto \bar{a})$.
The element $\neg \gamma_{AF}(n \mapsto \bar{a})$ gives the set of traces 
where some occurrences of $n$ are contained in the concretization of
$\gamma_{AF}(a)$.  Whereas, the element $\gamma_{AF}(n \mapsto a)$ gives 
the set of traces where all occurrences of $n$ are contained in the 
concretization of $\gamma_{AF}(a)$.  This is explained below.
%
\[
  \gamma_{AF}(n \mapsto a) \mathrel{\hat=} \{\pi \in \Pi \mid \forall (n,\omega) \in
  \pi. \omega \in \gamma_{AF}(a) \}
\]
%
\[
  \neg \gamma_{AF}(n \mapsto \bar{a}) \mathrel{\hat=} \neg \{\pi \in \Pi \mid \forall (n,\omega) \in
  \pi. \omega \not\in \gamma_{AF}(a) \}
\]
%
\[
  \neg \gamma_{AF}(n \mapsto \bar{a}) \mathrel{\hat=} \{\pi \in \Pi \mid \exists (n,\omega) \in \pi. \omega \in \gamma_{AF}(a) \}
\]
%
%An important requirement of ACDLP is that the meet irreducibles in the domain 
%must be precisely complementable with respect to the unsafe trace transformer.  
The abstract control flow lattice $\widehat{\mathcal{F}_{G}}$ does not always have 
complementable meet irreducibles.  So, we define a domain over logical encodings 
of the CFG, the elements of which admits precise complementation property.
%
\subsection{Lattice Structure over SSA}~\label{state-transformer}
%
We now define a lattice over SSA structures that admits precise complementation 
property which is necessary for lifting CDCL to program analysis. Let
$Var_{ssa}$ denote the set of SSA variables in the SSA form and $Value$ denote the
set of values that these SSA variables can take.  Note that the set $Var_{ssa}$ is
obtained from the CFG $G= (V, E, \mathcal{S}, \mathcal{T}, I)$ through program
transformation $\mathcal{T}_2$. 
%
\begin{definition} (Static Assignment Lattice). A \emph{Static Assignment Lattice}
  corresponding to a CFG $G= (V, E, \mathcal{S}, \mathcal{T}, I)$ is a complete 
  lattice $(\mathcal{SA}_{G}, \subseteq_{SA}, \cup_{SA}, \cap_{SA})$, defined 
  as follows.
\end{definition}
%
  \[
    \mathcal{SA}_{G} \mathrel{\hat=} \powerset(Var_{ssa} \rightarrow Value)
    \quad \forall A,B \in
    \mathcal{SA}_{G}.\; A \subseteq_{SA} B\; \text{if}\; \forall a \in A.\; 
    a \in B 
  \]
\rmcmt{  
  \[
    A \cap_{SA} B \mathrel{\hat=} \lambda\; x. ( \{f(x) \cap g(x) \mid 
    f \in A, g \in B \}) 
    \qquad 
    A \cup_{SA} B \mathrel{\hat=} \lambda\; x. ( \{f(x) \cup g(x) \mid 
    f \in A, g \in B \}) 
  \]
}
%
%%%%%%%%%%%%%%%% Static Single Assignment Lattice %%%%%%%%%%%%%%%%%%
%
Figure~\ref{fig:concrete} gives an example of concrete static assignment lattice
over SSA variables $p$ of type boolean and $x$, $y$ of numerical types. Note
that the lattice $\mathcal{SA}_G$ is a set of concrete environments over SSA
variables.  We will call these \emph{concrete SSA environments}.  Following the
result of Briggs et. al.~\cite{Briggs:1998}, the elements of $\mathcal{SA}_G$
can be mapped back to a trace of the original program.  The elements marked in bold 
in Figure~\ref{fig:concrete} corresponds to concrete assignments to SSA variables 
that maps to a concrete program trace. 
%
\begin{figure}[htbp]
\centering
\vspace*{-0.2cm}
  \scalebox{.90}{\import{figures/}{concrete_env.pspdftex}}
\caption{Concrete Static Assignment Lattice over a boolean SSA variable $p$ and two
  numerical SSA variables $x$ and $y$ that takes values in Integer domain \label{fig:concrete}}
\end{figure}
%


An SSA statement (or program transformer) $s \in \constraints$ transforms the 
memory state of a program and is 
therefore associated with a strongest postcondition or an existential 
precondition transformer. Let us assume that each SSA statement is also 
associated with a transition relation, $\mathcal{ST}_{s}^{\Omega}$. 
%
\begin{definition} (Strongest Postcondition and Existential Precondition).
  \[
     post_{s}(A) \mathrel{\hat=} \{\omega' \mid \exists \omega \in
     \Omega. \omega \in A \wedge (\omega,\omega') \in \mathcal{ST}_{s}^{\Omega} \} 
  \] 
  \[ 
     pre_{s}(A) \mathrel{\hat=} \{\omega \mid \exists \omega' \in
     \Omega. \omega' \in A \wedge (\omega,\omega') \in \mathcal{ST}_{s}^{\Omega}\} 
  \]
\end{definition}
%
\begin{definition} (Weakest Precondition and Universal Postcondition). 
  \[
    \widehat{pre_s}(A) \mathrel{\hat=} \{\omega \mid \forall \omega' \in
    \Omega.\; \omega' \in A \vee (\omega,\omega') \not\in \mathcal{ST}_{s}^{\Omega} \}.
  \]
  \[ 
     \widehat{post_s}(A) \mathrel{\hat=} \{\omega' \mid \forall \omega \in
     \Omega.\; \omega \in A \vee (\omega,\omega') \not\in
     \mathcal{ST}_{s}^{\Omega} \}.
  \]
\end{definition}
%
\Omit{
\begin{definition} (Static Assignment Transformers). 
   $
     post_{\mathcal{SA}}, pre_{\mathcal{SA}} : \mathcal{SA}_G
      \rightarrow \mathcal{SA}_{G} 
   $
   \[
     post_{\mathcal{SA}}(a) \mathrel{\hat=} \bigcup_{s \in \rmcmt{\constraints}}
     post_{s}(a) 
     \quad
     pre_{\mathcal{SA}}(a) \mathrel{\hat=} \bigcup_{s \in \rmcmt{\constraints}}
     pre_{s}(a) 
   \]
\end{definition}
}
%
\begin{figure}[t]
\centering
\vspace*{-0.2cm}
\scalebox{.70}{\import{figures/}{state_transformer.pspdftex}}
  \caption{Strongest postcondition $post_s$ (Blue), Existential precondition
  $pre_s$ (Red), Universal postcondition $\widehat{post_s}$ (Pink), 
  Weakest precondition $\widehat{pre_s}$ (Green)}
\label{state-transformer}
\end{figure}
%
Figure~\ref{state-transformer} graphically shows the result of various
state transformers for a given memory state $A$. A strongest postcondition 
maps a set of states $A$ to the set of all successor states that can be reached
in one step, whereas an existential precondition maps a set of states $A$ to
states the program \emph{may} have before executing $s$. A weakest precondition maps a set of states $A$ to the set
of states which can \emph{only} reach elements of $A$. Whereas, an universal
postcondition maps $A$ to set of states the program \emph{must} reach after
executing $s$.
%

Recall that the trace transformers $tpost$ and $tpre$ are defined over
$\powerset(\Pi)$. The transformers, $post_{s}$ and $pre_{s}$, soundly 
approximate $tpost$ and $tpre$ respectively.  \rmcmt{how ? --relate to syntactic
transformation T2}
%
The global static assignment transformers for the lattice $\mathcal{SA}_G$ over 
a set of SSA constraints $\constraints$ can be easily derived from state
transformers, $post_s, pre_s$.  This is defined next.
%
\begin{definition} (Global Static Assignment Transformers). 
  \[ 
  post_{\constraints}, pre_{\constraints}, 
  \widehat{post_{\constraints}}, \widehat{pre_{\constraints}}, 
  \colon \mathcal{SA}_G \rightarrow \mathcal{SA}_G
  \]
    
  \[
  post_{\constraints}(a) \mathrel{\hat=} \underset{\sigma \in
  \constraints}{\bigcap} post_\sigma \circ a
  \qquad  
  pre_{\constraints}(a) \mathrel{\hat=} \underset{\sigma \in
  \constraints}{\bigcap} pre_\sigma \circ a
  \]
  
  \[
  \widehat{post_{\constraints}}(a) \mathrel{\hat=} \underset{\sigma \in
  \constraints}{\bigcap} \widehat{post_\sigma} \circ a
  \qquad   
  \widehat{pre_{\constraints}}(a) \mathrel{\hat=} \underset{\sigma \in
  \constraints}{\bigcap} \widehat{pre_\sigma} \circ a
  \]
\end{definition}
%
\rmcmt{The transformers, $post_{\constraints}$ and $pre_{\constraints}$, soundly 
approximate $tpost$ and $tpre$ respectively. (say how?) }
%
We now define an abstraction of static assignment lattice which we 
call \emph{abstract static assignment} domain. 
The concrete static assignment domain is a complete lattice of concrete 
SSA environments.  The abstractions of concrete static assignment domain fall
into two categories; an abstraction that preserves the relationship between 
SSA variables and the other which does not preserve any relation. 
%

First, we give an abstraction of static assignment lattice that does not
preserve relationship between SSA variables.  Then, we present a generic 
abstraction using a \emph{template-based abstract domain} that can express 
relational as well as non-relational abstractions of SSA environments.  
%An advantage of a template-based domain is that it can be instantiated with 
%arbitrary templates over relational or non-relational domains.  
This provides the flexibility to instantiate CDCL over arbitrary abstract domains for 
program analysis.  
%
\para{Abstract Static Assignment Lattice Over Non-relational Domains}
%
An abstraction of $\mathcal{SA}_G$ over non-relational domain requires each SSA
variables in $SSA_{var}$ to be abstracted independently of each other.  
The concrete domain of $\powerset(Var_{ssa} \rightarrow Value)$ is abstracted 
through a function which maps each SSA variable to an abstract value.  We 
assume that there is a galois connection $(\alpha_v,\gamma_v)$ such that
  $(Value, \subseteq)
   \galois{\alpha_{v}}{\gamma_{v}}
   (\widehat{Value}, \sqsubseteq)$.   
%
Then, an abstraction of $\mathcal{SA}_G$ over non-relational domain, denoted by
$\mathcal{SA}_G^\dagger = \powerset(Var_{ssa} \rightarrow \widehat{Value})$, 
can be constructed using the following galois connection 
$(\alpha^\dagger,\gamma^\dagger)$.
\[
   (\mathcal{SA}_{G},\subseteq_{SA})
   \galois{\alpha^\dagger}{\gamma^\dagger}
   ({\mathcal{SA}_{G}}^\dagger,\sqsubseteq^\dagger)
\]
\[
  \alpha^\dagger(c) \mathrel{\hat=} \lambda x.\alpha_v(\bigcup\{\tau(x) \mid \tau
  \in c\})
\]
\[
  \gamma^\dagger(a) \mathrel{\hat=} \{\lambda x.v' \mid v' \in (\gamma_v \circ
  a)(x)\}
\]
%
\begin{example}
%
An example of $({\mathcal{SA}_{G}}^\dagger,\sqsubseteq^\dagger)$ over an
Interval domain $Itv$ maps each SSA variable on an Interval, where 
$(\alpha_v,\gamma_v) = (\alpha_{Itv}, \gamma_{Itv})$.  Consider an SSA 
$c=\{x_1:=2, x_2 \geq 5\}$, then $\alpha^\dagger(c) = \{x_1=[2,2],x_2=[5,\infty]\}$
\end{example}
%
\para{Abstract Static Assignment Lattice Over Template-based Domains}
%
A folk wisdom in abstract interpretation community is that an analysis using 
non-relational domains are effective but imprecise. However, for practical
purposes, it is imperative to keep track of relations between program variables.  
A relational domain can express relationship between variables, though with varying
expressivity, depending on a weak or strong relational domain.  We now construct
an abstraction of $\mathcal{SA}_G$ using a template-based domain, denoted by 
$\widehat{\mathcal{SA}_{G}}$, which can be instantiated with arbitrary templates 
over relational or non-relational domains. Note that from this point onwards, we 
will operate on $\widehat{\mathcal{SA}_{G}}$ instead of ${\mathcal{SA}_{G}}^\dagger$. 
%
\begin{definition}~\label{assl} (Abstract Static Assignment Lattice). An \emph{Abstract Static 
Assignment Lattice} corresponding to a CFG $G= (V, E, \mathcal{S}, \mathcal{T}, I)$ 
and a set $X$ of values of vector $\vec{x}$ of variables in the 
corresponding SSA, is a complete lattice 
  $(\widehat{\mathcal{SA}_{G}}, \sqsubseteq_{SA}, \sqcap_{SA}, \sqcup_{SA})$, which 
is defined as follows. 
\[
  \widehat{\mathcal{SA}_{G}} \mathrel{\hat=} C\vec{x} \leq \vec{d},\; \text{where}\; 
  \vec{d}\;\text{is a constant vector and C is a coefficient matrix} 
\]
\end{definition}
%
The abstraction and concretization functions $(\alpha_{SA}, \gamma_{SA})$ form 
a galois connection.
%
\[
  (\mathcal{SA}_{G},\subseteq_{SA})
   \galois{\alpha_{SA}}{\gamma_{SA}}
   (\widehat{\mathcal{SA}_{G}},\sqsubseteq_{SA})
\]
%
\[
  \alpha_{SA}(\numconcval) = \min \{\vec{\numabsval}\mid
  \mat{C}\vec{\numvar}\leq\vec{\numabsval}, \vec{\numvar}\in
  \numconcval\},\;\text{where}\; \min\; \text{is applied component-wise}.  
\]  
%  
\[  
   \gamma_{SA}(\vec{\numabsval}) \mathrel{\hat=} \{\vec{\numvar}\mid
   \mat{C}\vec{\numvar}\leq\vec{\numabsval}, \vec{\numvar} \in X\} 
   \qquad \gamma_{SA}(\bot)=\emptyset
\]
%



%
Figure~\ref{fig:abstract} shows the abstract static assignment lattice over an
Interval abstract domain.  The elements marked in bold corresponds to concrete 
assignments to SSA variables that maps to a concrete program trace. 
%
\begin{figure}[htbp]
\centering
\vspace*{-0.2cm}
  \scalebox{.90}{\import{figures/}{abstract_env.pspdftex}}
\caption{Abstract Static Assignment Lattice over a boolean SSA variable $p$ and two
  numerical SSA variables $x$ and $y$ that takes values in Interval domain
  \label{fig:abstract}}
\end{figure}
%
\begin{example}
For example, consider the SSA elements $\{x_1\geq 0, x_1-z\leq 30\}$, then
abstract domain value is $\vec{\numabsval}=\vecv{0}{30}$,
with $\mat{C}=\qmat{-1}{0}{1}{-1}$ and $\vec{\numvar}=\vecv{x_1}{z}$. \\ 
\end{example}
%
\begin{definition} (Meet Irreducibles of $\widehat{\mathcal{SA}_G})$
%

A meet irreducible of \emph{abstract static assignment} domain
$\widehat{\mathcal{SA}_{G}}$ where $\mat{C_i}$ is $i$-th row vector  
of a matrix of size $N \times M$ and $\vec{\numabsval}$ is a vector 
  of size $N$, is defined below.  \rmcmt{Here, $MAX$ is the largest 
  value of $\vec{d}$ determined by the matrix $\mat{C}$}.
%   
\rmcmt{
  \[
     \meet_{irrd}(\widehat{\mathcal{SA}_{G}}) \mathrel{\hat=} 
     \{\vec{d} \mid \exists_{=1}\;d_i \in \vec{\numabsval}.\;(d_i \neq MAX) \wedge
     C_{i}\vec{x} \leq d_i\} \;\text{where}\;(i \leq N)
  \]
}  
Informally, a meet irreducible of $\widehat{\mathcal{SA}_{G}}$ is the abstract
value $\vec{d}$ such that there exists exactly one element of $\vec{d}$ that is
not $MAX$.
%
\end{definition}
%
%\rmcmt{Show how MAX is computed}.
We now discuss how $MAX$ is computed.  For as SSA element 
$\hmat{1}{-1} \vecv{x}{y} \leq d$ where $x$ and $y$ are 
32-bit signed integers (marked as s32 in figure~\ref{fig:max}), 
the total number of bits to represent $d$ is 34.  Thus, the 
value of $MAX$ is the largest value representable in 34 bits.
%
\begin{figure}[htbp]
\centering
\vspace*{-0.2cm}
  \scalebox{.90}{\import{figures/}{max.pspdftex}}
\caption{Computing $MAX$ from bit-width of $d$
  \label{fig:max}}
\end{figure}
%


Meet irreducibles of $\widehat{\mathcal{SA}_G}$ over an \emph{Interval} domain 
with template $[l,u]$ such that $\vecv{-1}{1}(\vec{x})\leq \vecv{l}{u}$ is given
by $\{\vecv{-l}{MAX}, \vecv{MAX}{u}\}$.
%
\begin{example}
  An example of meet irreducible in $\widehat{\mathcal{SA}_{G}}$ over 
  $\vec{\numvar} = \vecv{a}{b}$ that takes values in \emph{Interval} domain 
  is given by, $\meet_{irrd}(\{a:[5,7]\}) = \{\vecv{-5}{MAX}, \vecv{7}{MAX}\}$ 
  with $\mat{C}=\qmat{1}{0}{0}{0}$. Here, the abstract values of variable 
  $b$ is equal to $MAX$. \\
\end{example}  
%
\begin{example} \rmcmt{Octagon}
  An example of meet irreducible in $\widehat{\mathcal{SA}_{G}}$ over 
  $\vec{\numvar} = \vecv{a}{b}$ that takes values in \emph{Octagon} domain 
  is given by, 
  \[\meet_{irrd}(\{a+b \leq 5\}) =
  \vecvvv{MAX}{MAX}{MAX}{MAX}{5}{MAX}{MAX}{MAX}\]
    
  \[\text{with}\;\; \mat{C}=
  \begin{bmatrix}
    0 & 1 \\
    1 & 0 \\
    0 & -1 \\
    -1 & 0 \\
    1 & 1 \\
    1 & -1 \\
    -1 & 1 \\
    -1 & -1
  \end{bmatrix}
  \]
  %\mat{c}=\qqqmat{0}{1}{1}{0}{0}{-1}{-1}{0}{1}{1}{1}{-1}{-1}{1}{-1}{-1}\] 
   Here, the abstract values of $a$, $b$, $-a$, $-b$, $(a-b)$, $(-a+b)$, $(-a-b)$ are
  equal to $MAX$. \\
\end{example}  
%


An advantage of abstract static assignment domain $\widehat{\mathcal{SA}_{G}}$ 
is that it can be instantiated over arbitrary relational or non-relational 
numerical abstract domains.  This gives the flexibility to instantiate CDCL for 
safety verification over arbitrary abstract domains. 
%
We will use the template-based abstraction for the rest of the paper. 
%%%%%%%%%%%%%%%%%%%%%%%%% PROOF %%%%%%%%%%%%%%%%%%%%%%%%
\begin{proposition}~\label{ag}
  $
  (\powerset({\Pi}),\subseteq)
    \galois{\alpha_{T}}{\gamma_{T}}
  (\mathcal{SA}_{G},\subseteq_{SA})
   \galois{\alpha_{SA}}{\gamma_{SA}}
   (\widehat{\mathcal{SA}_{G}},\sqsubseteq_{SA})
  $ 
\end{proposition}
%
\begin{proof}
  The lattice $\mathcal{SA}_G$ is obtained from the concrete lattice of 
  traces $\powerset(\Pi)$ via syntactic translation steps $\mathcal{T}_1$ and
  $\mathcal{T}_2$, as shown in figure~\ref{fig:semantic}.  Both translation
  steps are exact, that is, $\powerset(\Pi)$ and $\mathcal{SA}_G$ form an 
  exact abstraction via the galois 
  connection $(\alpha_T, \gamma_T)$. This implies, $\alpha_T(\pi) = \tau$ where
  $\tau \in \mathcal{SA}_G$.  Also, $\gamma_T(\tau) = \pi \in \powerset(\Pi)$,
  since an SSA can be exactly mapped back to the program trace following the
  syntactic translation steps proposed in~\cite{Briggs:1998}.   
  Thus, there is no loss or gain of information between elements of 
  $\mathcal{SA}_{G}$ and $\powerset(\Pi)$.


  Furthermore, the lattice $\widehat{\mathcal{SA}_{G}}$ is obtained via 
  two-step galois connection $(\alpha_{T} \circ \alpha_{SA})$ from
  the concrete lattice of traces $\powerset(\Pi)$. The abstraction $\alpha_{T}$
  is exact, whereas, $\alpha_{SA}$ is an over-approxation by
  definition~\ref{assl}.
  By \rmcmt{transitivity, (find reference)}, the lattice 
  $\widehat{\mathcal{SA}_{G}}$ over-approximates $\powerset(\Pi)$.
\end{proof}
%
\Omit{
\begin{definition}
A trace $\pi$ in abstract static assignment domain $\widehat{\mathcal{SA}_{G}}$ 
is a sequence of abstract states of the form $\{C_{ij}\vec{x} \leq \vec{d_{i}}$,
starting with an abstract element $\top$. We use $\Pi$ to denote set of traces. 
\end{definition}
}
%
The abstract transformers for the lattice $\widehat{\mathcal{SA}_{G}}$ 
transforms a memory state of a program and therefore associated with state 
transformers.  For every SSA statement $s \in \constraints$, let
$apost_{s}$ and $apre{s}$ be sound over-approximations of $post_{s}$ and
$pre_{s}$ in the lattice $\widehat{\mathcal{SA}_G} \rightarrow 
\widehat{\mathcal{SA}_G}$, respectively. 
%
Th global abstract static assignment transformers for the 
lattice $\widehat{\mathcal{SA}_G}$ over a set of SSA constraints 
$\constraints$ are obtained from abstract state transformers, 
$apost_s, apre_s$. This is defined next.
%
\begin{definition} (Global Overapproximate Abstract Static Assignment Transformers). 
  \[ 
     apost_{\constraints}, apre_{\constraints} : \widehat{\mathcal{SA}_G}
     \rightarrow \widehat{\mathcal{SA}_{G}} 
   \]
   \[
     apost_{\constraints}(a) \mathrel{\hat=} 
     \underset{\sigma \in \constraints}{\bigsqcap} apost_{\sigma} \circ a 
   \]
  \[
     apre_{\constraints}(a) \mathrel{\hat=} 
     \underset{\sigma \in \constraints}{\bigsqcap} apre_{\sigma} \circ a 
   \]
\end{definition}
%
The transformers $apost_{\constraints}$ and $apre_{\constraints}$ soundly overapproximates 
$post_{\constraints}$ and $pre_{\constraints}$, respectively. 
%
\subsection{Complementation in Abstract Static Assignment Lattice}~\label{complement}
%
Recall that every meet irreducibles of a partial assignments domain admits a
precise complement.  This property of the partial assignments domain enables a
CDCL solver to learn a conflict clause that is obtained by complementing a 
conflict reason.  In section~\ref{complement-fg}, we showed that the elements 
of abstract flow lattice $\widehat{\mathcal{F}_G}$ does not always have precise 
complements.  In this section, we show that the meet irreducibles of 
$\widehat{\mathcal{SA}_G}$ have precise complements. 
%
\begin{definition}~\label{meet-decomp}
A \emph{meet decomposition} $\decomp(\absval)$ of an abstract 
element $\absval \in \widehat{\mathcal{SA}_{G}}$ is a set of meet 
irreducibles $M \subseteq \widehat{\mathcal{SA}_{G}}$ such that 
$\absval=\bigsqcap_{m\in M} m$.
\end{definition}
%
\begin{example} The meet decomposition of the interval domain element
$decomp(2\leq x\leq 4 \wedge 3\leq y\leq 5)$ is
the set $\{x\geq 2, x\leq 4, y\geq 3, y\leq 5\}$.
\end{example}
%
Recall that a precise complement of an element $a \in \widehat{\mathcal{SA}_G}$ 
is an element $\bar{a} \in \widehat{\mathcal{SA}_G}$ such that $\neg \gamma_{SA}(\bar{a}) =
\gamma_{SA}(a)$.  The meet irreducibles of Abstract Static Assignment lattice 
$\widehat{\mathcal{SA}_G}$ admits precise complements. This is explained below. 
%
\[
   \gamma_{SA}(\vec{\numabsval}) \mathrel{\hat=} \{\vec{\numvar}\mid
   \mat{C}\vec{\numvar}\leq\vec{\numabsval}\} 
\]
\[
   \gamma_{SA}(\bar{\vec{\numabsval}}) \mathrel{\hat=} \{\vec{\numvar}\mid
   \mat{C}\vec{\numvar} > \vec{\numabsval}\} 
\]
\[
   \neg \gamma_{SA}(\bar{\vec{\numabsval}}) \mathrel{\hat=} \{\vec{\numvar}\mid
   \mat{C}\vec{\numvar} \leq \vec{\numabsval}\} 
\]
%
\begin{example}
Let an abstract SSA environment be $a = \{p:t,x\geq0,y\geq0\}$, 
then $\bar{a} = \{p:f,x<0,y<0\}$ and $\neg \bar{a} = \{p:t,x\geq0,y\geq0\}$.
\end{example}
%
\begin{example}
The precise complement of a meet irreducible $(x \leq 2)$ in
the interval domain over integers is $(x \geq 3)$, or the precise complement
of the meet irreducible $(x+y \leq 1)$ in the octagon domain over integers
is $(x+y \geq 2)$.  
\end{example}
%
\begin{table}
\scriptsize
\begin{center}
{
\begin{tabular}{l|l|l|l|l}
\hline
  Domain & Concrete & Abstract & Meet & Complemented Meet \\ 
  Name  & Domain & Elements & Irreducible & Irreducible \\ \hline
Interval & $\powerset(Var \rightarrow \mathbb{Z}) \cup \bot$  & $x = [l,u]$ & $x\leq N$ & $x > N$ \\ \hline
Octagon &  $\powerset(Var \times Var \rightarrow (\mathbb{Z} \cup
\{\infty\})) \cup \bot$ & $(\pm x_i \pm x_j \leq d)$ & $(x+y \leq N)$ & $(-x-y < N)$ \\ \hline
  Equality & $\powerset(Var \rightarrow \mathbb{Z}) \cup \bot$ & $(x=y)$ &
  $(x=y)$ & $(x \neq y)$ \\ \hline 
 Zones &  $\powerset(Var \times Var \rightarrow (\mathbb{Z} \cup
\{\infty\})) \cup \bot$ & $(x_i - x_j \leq d)$ & $(x+y < N)$ & $(-x-y \leq N)$ \\ \hline
\end{tabular}
}
\end{center}
\label{fig:complement}
\end{table}
%
Table~\ref{fig:complement} shows that most abstract domains admit precise
complements.  We now show that the meet irreducibles of 
$\widehat{\mathcal{SA}_{G}}$ have precise complements when instantiated over
arbitrary abstract domains. 
%
\begin{proposition} 
  $\gamma_{SA}(C_{i}\vec{x} \leq \vec{d_{i}}) = 
  (\neg \gamma_{SA}(C_{i}\vec{x} > \vec{d_{i}}))$
\end{proposition}
%
\begin{proof}
  Assuming that $\vec{d}$ is chosen from a non-relational domain such as 
  Interval domain $Itvs$, the meet 
  irreducibles are of the form $\{\vec{x} \diamond \vec{d}\}$, where
  $\diamond =
  \{<,>,\leq,\geq\}$. It is easy to see that the meet irreducibles of $Itvs$ are easily
  complementable.  Similarly, assuming that $\vec{d}$ is chosen from a
  relational domain such as Octagon 
  domain $Octs$, the meet irreducibles are of the form $\{\mat{C}\vec{x}
  \diamond \vec{d}\}$, which also admits precise complements.  However, for
  element $a \in \widehat{\mathcal{SA}_{G}}$, that does not admit precise 
  complementation, it can be decomposed into half spaces, 
  $decomp(a)=\{m_1, m_2, \ldots m_k\}$, such that each $m_i$ admits precise
  complements. 
\end{proof}
%
\Omit{
\para{Elements in $\mathcal{SA}_G$ gives a Program Trace}
%
We use the result of~\cite{Briggs:1998} in Lemma~\ref{ssa-model} that shows 
a model of $\varphi$ corresponds to a concrete counterexample trace in the 
original program. 
%
\begin{lemma}
  \[ \mathcal{T}_2(\gamma(a)) = \{\pi \mid \pi \in \Pi \wedge a \in
  \widehat{\mathcal{SA}_G} \} \]
\end{lemma}
%
\begin{proof}

\end{proof}
}
%

%%%%%%%%%%%%%%%%%%%%%%%%%%%%%%%%%%%%%%%%%%%%%%%%%%%%%%%%%%%%%%%%%%%

%%%%%%%%%%%%%%%%%%%%%%%%%%%%%%%%%%%%%%%%%%%%%%%%%%%%%%%%%%%%%%%%%%%
\section{An Example of Abstract Interpretation}
%
\begin{figure}[t]
\centering
\begin{tabular}{c|c}
\hline
Control-Flow Graph & Static Analysis Equation \\
\hline
\begin{minipage}{5.0cm}
\scalebox{.65}{\import{figures/}{ai-example.pspdftex}}
\end{minipage}
&
\begin{minipage}{7cm}
$\begin{array}{l}
     S_{n1} = \top, \\
     S_{n2} = post_{x:=2}(S_{n1}), \\
     S_{n3} = post_{x:=x*x}(S_{n2}) \cup post_{x:=x+2}(S_{n4}), \\ 
     S_{n4} = post_{x < 10}(S_{n3}), \\
     S_{n5} = post_{x\geq 10}(S_{n3}) \\
     S_{Error} = post_{x\neq10}(S_{n5}) \\
     S_{Safe} = post_{x=10}(S_{n5}) \\
\end{array}$
\end{minipage}
\\
\hline
\end{tabular}
\caption{\label{fig:se} A CFG and its static analysis equation}
\end{figure}
%
We present an example of a classical abstract interpretation of program.  
Figure~\ref{fig:se} shows a CFG and its corresponsing static analysis equations.  
A static analysis equation encodes the data-flow between individual control-flow 
nodes in the CFG and is given by a set valued variable $S_n$ for each location $n$ 
in the CFG.  Recall that each statement $s$ in the program is associated with a
postcondition transformer, $post_s(S_n)$, that computes the successor state of a 
statement $s$ starting from $S_n$ that can be reached in one step.  Static analysis 
using abstract interpretation usually computes a fixed point over the static analysis 
equation obtained from the Control Flow Graph (CFG) representation of a program~\cite{CC79}.  
%

Let us consider the equations in Figure~\ref{fig:se} that models the loop, 
$S_{n3} = post_{x:=x*x}(S_{n2}) \cup post_{x:=x+2}(S_{n4})$, 
$S_{n4} = post_{z \leq 10}(S_{n3})$.  These equations can be written as 
a function, $F(X)= \{4\} \cup \{x+2 \mid x \in X, x \leq 10\}$, where 
$post_{x:=x*x}(S_{n2}) = {4}$.  Assuming variable $x$ is an integer, the 
lattice of integers with a subset relation is $(\powerset(\mathbb{Z}),\subseteq)$. 
%

Standard means to infer loop invariant is to compute fixed points of function 
over this lattice structure~\cite{CC79,octagon}.  The fixed point of the above 
function gives the set $X$ that satisfies $F(X)=X$.  The loop invariant is given 
by $x\geq 4 \wedge x\leq 10 \wedge x\equiv 0\;(\bmod 2)$.  The set of values of 
$x$ satisfying the loop invariant is the least fixed point. 
%

Abstract interpretation of program computes a fixed point of an abstract
function (called abstract transformer), over an abstract lattice.  Assuming an
interval lattice $Intv$ which maps a set of integer values of a variable to 
its smallest interval that contains it, the function $F$ above is abstracted 
over an interval lattice as shown below.
%
\[ F^{\sharp}([a,b]) = [4,4] \sqcup ([a,b] \sqcap [-\infty,10]) +_{Intv} [2,2] \]
%
The initial values of $x$ is $[4,4]$ which is obtained by interval analyis of
the static analysis equation, $S_{n3} = post_{x:=x*x}(S_{n2})$. The 
function $F^{\sharp}$ computes an interval at each iteration where the interval 
below 10 is incremented by 2 and $(+_{Intv})$ denotes an addition operation in
the interval lattice. 
%
Figure~\ref{fig:fixpoint} shows the fixed point computation of the loop in 
Figure~\ref{fig:se}, over a lattice of intervals.  Each column denotes an
iteration of the fixed point computation which associates an interval with each
location in the program.  The initial value of $x$ is $\top$ at $n1$, while the
locations $n2$, $n3$, $n4$, $n5$, $Error$, $Safe$ are considered unreachable. 
Each iteration of the loop computes a bound on the variable $x$. 
The interval $x\colon[4,10]$ at the loop head $n3$ in the last iteration 
is the loop invariant. 
In practise, the total number of iterations may be too large to reach the fixed 
point. Hence, techniques like widening and narrowing are used to accelerate
convergence. 
%
\begin{figure}[t]
\centering
\begin{tabular}{ccccc}
\hline
  Control Location & Iteration 1 & Iteration 2 & Iteration 3 & Iteration 4\\
\hline
  $n1$ & $x\colon\top$ & $x\colon\top$ & $x\colon\top$  & $x\colon\top$ \\ 
  $n2$ & $x\colon[2,2]$ & $x\colon[2,2]$ & $x\colon[2,2]$ & $x\colon[2,2]$ \\
  $n3$ & $x\colon[4,4]$ & $x\colon[4,6]$ & $x\colon[4,8]$ & $x\colon[4,10]$ \\
  $n4$ & $x\colon[6,6]$ & $x\colon[6,8]$ & $x\colon[6,10]$ & $x\colon[6,10]$ \\
  $n5$ & $x\colon\bot$ & $x\colon\bot$ & $x\colon\bot$ & $x\colon[10,10]$ \\
  $Error$ &$x\colon\bot$ & $x\colon\bot$ & $x\colon\bot$ & $x\colon\bot$ \\
  $Safe$ &$x\colon\bot$ & $x\colon\bot$ & $x\colon\bot$ & $x\colon[10,10]$ \\
\hline
\end{tabular}
  \caption{\label{fig:fixpoint} Fixed point computation of program in
  Figure~\ref{fig:se}}
\end{figure}
%

%%%%%%%%%%%%%%%%%%%%%%%%%%%%%%%%%%%%%%%%%%%%%%%%%%%%%%%%%%%%%%%%%%%


