\documentclass[a4paper]{llncs}

\usepackage{scalerel}
\usepackage{import}
\usepackage{times}
\usepackage{multirow}
\usepackage{microtype}
\usepackage{array}
\usepackage{galois}
\usepackage{graphicx,wrapfig}
\usepackage[noadjust]{cite}
\usepackage{tikz}
\usetikzlibrary{plotmarks}
\usepackage{pgfplotstable}
\usepackage{filecontents}
\usepackage{pgfplots}
\usepackage{amsfonts}
\usepackage{amssymb}
\usepackage{amsmath}
\usepackage{stmaryrd}
\usepackage{color}
\usepackage{listings}
\usepackage{verbatim}
\usepackage{psfrag}
\usepackage{epsfig}
\usepackage{wasysym} 
%\usepackage{subfigure}
\usepackage{paralist}
%\usepackage{dingbat}
\usepackage[algo2e,linesnumbered,ruled,lined]{algorithm2e}
\usepackage{hyperref}
\usepackage[subnum]{cases}
\makeatother

\newcommand{\extendedonly}[1]{}
\newcommand{\paperonly}[1]{#1}
\newtheorem{exmp}{Example}
%\newtheorem{theorem}{Theorem}[section]
%\newtheorem{lemma}[theorem]{Lemma}
%\numberwithin{mytheorem}{subsection} 

\newcommand{\tool}[1]{\textsc{#1}\xspace}
\newcommand{\cbmcv}{\tool{cbmc 5.0}}
\newcommand{\mydef}[1]{\begin{definition}#1\end{definition}}
\newcommand{\rmcmt}[1]{{\color{magenta}{#1}}}%#1
\newcommand{\pscmt}[1]{{\color{blue}{#1}}}%#1
\newcommand{\dkcmt}[1]{{\color{red!70!black}{#1}}}%#1
\newcommand{\tmcmt}[1]{{\color{orange}{#1}}}%#1
\newcommand{\lhcmt}[1]{{\color{purple}{#1}}}%#1

\newif\ifextended
\extendedfalse        % For FM'16 publishers version

%%% Standard text acronyms

% TOO MANY VARIANTS if we INTRODUCE THIS.
\newcommand{\para}[1]{
  \subsubsection*{#1}}
 
%\noindent
%\textbf{#1~}}

\newcommand{\cnf}{\textsc{cnf}\xspace}
\newcommand{\sat}{\textsc{sat}\xspace}
\newcommand{\smt}{\textsc{smt}\xspace}
\newcommand{\dpll}{\textsc{dpll}\xspace}
\newcommand{\fpdpll}{\textsc{fdpll}\xspace}
\newcommand{\cfg}{\textsc{cfg}\xspace}
\newcommand{\cfgs}{\textsc{cfg}s\xspace}
\newcommand{\acfg}{\textsc{acfg}\xspace}
\newcommand{\smpp}{\textsc{smpp}\xspace}
\newcommand{\cegar}{\textsc{cegar}\xspace}
\newcommand{\cdfl}{\textsc{cdfl}\xspace}
\newcommand{\cdflitv}{\textsc{cdfl}$(\itvdom)$\xspace}
\newcommand{\cdcl}{\textsc{cdcl}\xspace}
\newcommand{\cbmc}{\textsc{cbmc}\xspace}
\newcommand{\bmc}{\textsc{bmc}\xspace}
\newcommand{\cdflalgo}{\textsc{\textsf{cdfl}}\xspace}
\newcommand{\ieeefp}{\textsc{\textsf{IEEE 754}}\xspace}


%%%%%%%%%%%%%%%%%%%%%%%%%%%%%%% 
%%% Mathematical Symbols %%%
%%%%%%%%%%%%%%%%%%%%%%%%%%%%%%% 

\renewcommand{\vec}[1]{{\overrightarrow #1}}
\newcommand{\vecv}[2]{{\left(\begin{array}{@{}c@{}} #1 \\ #2\end{array}\right)}}
\newcommand{\qmat}[4]{{\left(\begin{array}{@{}cc@{}} #1 & #2 \\  #3 & #4\end{array}\right)}}
\newcommand{\mat}[1]{{\boldsymbol #1}}

%%% Sets and functions

\newcommand{\powerset}{\ensuremath{\wp}}
\newcommand{\set}[1]{\ensuremath{\left\{#1\right\}}}
\newcommand{\setneg}[1]{\overline{#1}}
\newcommand{\setsep}{\ensuremath{~|~}}
\newcommand{\tuple}[1]{\ensuremath{(#1)}}

\newcommand{\bnf}{\ensuremath{\mathrel{\mathop{::}}=}}
\newcommand{\bnfsep}{~\mid~}

\newcommand{\true}{\mathsf{true}}
\newcommand{\false}{\mathsf{false}}

%%% Lattices
\newcommand{\sle}{\ensuremath{\sqsubseteq}}
\newcommand{\sles}{\ensuremath{\sqsubset}}
\newcommand{\sge}{\ensuremath{\sqsupseteq}}
\newcommand{\sges}{\ensuremath{\sqsupset}}
\newcommand{\join}{\ensuremath{\sqcup}}
\newcommand{\bigjoin}{\ensuremath{\bigsqcup}}
\newcommand{\meet}{\ensuremath{\sqcap}}
\newcommand{\bigmeet}{\ensuremath{\bigsqcap}}


%%% Program model

\newcommand{\program}{\mathit{Prog}}
\newcommand{\assertions}{\mathit{Assn}}
\newcommand{\assertion}{\mathit{a}}
\newcommand{\constraints}{\Sigma}
\newcommand{\constraint}{\sigma}
\newcommand{\formula}{\varphi}

\newcommand{\vars}{\mathit{Vars}}
\newcommand{\subvars}{\mathit{V}}
\newcommand{\boolvars}{\mathit{BVars}}
\newcommand{\numvars}{\mathit{NVars}}

\newcommand{\booldomain}{\mathcal{B}}
\newcommand{\reldomain}{\mathcal{TP}}

\newcommand{\numvar}{x}
\newcommand{\numconcval}{\mathit{X}}
\newcommand{\numabsval}{d}
\newcommand{\decisionvar}{\mathit{q}}
\newcommand{\conflictset}{\mathit{C}}
\newcommand{\exclude}{\mathit{exclude}}

\newcommand{\abstrans}[2]{\llbracket #2 \rrbracket_{#1}}
\newcommand{\abdgen}[3]{abdgen_{#1,#2}^{#3}}
\newcommand{\abd}[2]{abd_{#1}^{#2}}
\newcommand{\uabd}[3]{uabd_{#1,#2}^{#3}}
\newcommand{\ded}[2]{ded_{#1}^{#2}}
\newcommand{\genabs}{\phi_{g}}
\newcommand{\dedresult}{\phi}
\newcommand{\abstransset}{\mathcal{A}}
\newcommand{\abstransel}[1]{\mathit{ded}^{#1}}
\newcommand{\domain}{\mathit{D}}
\newcommand{\subdomain}{\mathit{L}}
\newcommand{\var}{\mathit{s}}
\newcommand{\val}{\mathit{u}}
\newcommand{\concval}{\mathit{concVal}}
\newcommand{\absval}{\mathit{a}}
\newcommand{\allval}{\mathit{A}}
\newcommand{\newdeductions}{\mathit{v}}
\newcommand{\intervals}{\mathit{Itvs}}
\newcommand{\octagons}{\mathit{Octs}}
\newcommand{\onlynew}{\mathit{onlyNew}}
\newcommand{\aunit}{\mathit{AUnit}}

\newcommand{\abs}{\mathit{abs}}
\newcommand{\trail}{\mathcal{T}}
\newcommand{\reasons}{\mathcal{R}}
\newcommand{\worklist}{\mathit{worklist}}
\newcommand{\initworklist}{\mathit{initWorklist}}
\newcommand{\updateworklist}{\mathit{updateWorklist}}
\newcommand{\makesubdomain}{\mathit{MakeL}}

\newcommand{\decide}{\mathit{decide}}
\newcommand{\deduce}{\mathit{deduce}}
\newcommand{\analyzeconflict}{\mathit{analyzeConflict}}

\newcommand{\propheur}{\mathit{H_P}}
\newcommand{\decheur}{\mathit{H_D}}
\newcommand{\confheur}{\mathit{H_C}}


\newcommand{\cvars}{\mathit{CVar}}
\newcommand{\vals}{\mathit{Val}}
\newcommand{\exps}{\mathit{Exp}}
\newcommand{\expr}{\mathit{exp}}
\newcommand{\bexps}{\mathit{B\!Exp}}
\newcommand{\envs}{\mathit{Env}}
\newcommand{\env}{\varepsilon}
\newcommand{\aenv}{\hat{\varepsilon}}
\newcommand{\assg}{\ensuremath{\mathrel{\mathop:}=}}
\newcommand{\cond}[1]{\ensuremath{[#1]}}
\newcommand{\choice}[1]{\ensuremath{\mathit{choose}\{#1\}}}
\newcommand{\loopstmt}[1]{\ensuremath{\mathit{loop}\{#1\}}}
\newcommand{\nondetstmt}[1]{\ensuremath{\mathit{nondet}\{#1\}}}
\newcommand{\lfp}{\mathsf{lfp}}
\newcommand{\gfp}{\mathsf{gfp}}

\newcommand{\locs}{\mathit{Loc}}
\newcommand{\stmt}{\ensuremath{\mathit{stmt}}}
%\newcommand{\err}{\ensuremath{\lightning}}
\newcommand{\err}{\Xi}
\newcommand{\init}{\ensuremath{\mathit{init}}}
\newcommand{\reach}{\mathit{Reach}}
\newcommand{\loopfree}{\mathit{Loop-Free}}

\newcommand{\badprog}{\ensuremath{\mathit{Err}}}
\newcommand{\progexec}{\ensuremath{\mathit{Exec}}}
\newcommand{\absbadprog}{\ensuremath{\hat{\mathit{Err}}}}
% \newcommand{\inv}{\ensuremath{\mathit{Inv}}}
\newcommand{\absinv}{\ensuremath{\hat{\mathit{Inv}}}}
\newcommand{\safe}{\ensuremath{\mathit{Safe}}}
\newcommand{\abssafe}{\ensuremath{\hat{\mathit{Safe}}}}

%%% Standard Abstract interpretation 
%%%
\newcommand{\aenvs}{\mathit{A\!Env}}
\newcommand{\post}{\mathit{post}}
\newcommand{\lpost}[1]{\mathit{post}_{#1}}
\newcommand{\abspost}{\mathit{\hat{post}}}
\newcommand{\abslpost}[1]{\mathit{\hat{post}}_{#1}}

\newcommand{\absuawp}{\mathit{\breve{wp}}}


%%% Fixedpoint-DPLL
%%%
\newcommand{\cvals}{\mathit{CVals}}
\newcommand{\avals}{\mathit{AVals}}

\newcommand{\aval}{\hat{\mathit{v}}}
\newcommand{\sem}[2]{\ensuremath{\llbracket #1 \rrbracket_{#2}}}
\newcommand{\rsem}[3]{\ensuremath{\llbracket #1 \rrbracket_{#2}^{#3}}}
\newcommand{\asem}[2]{\ensuremath{\|#1 \|_{#2}}}

\newcommand{\comps}{\mathit{Comp}}
%%\newcommand{\ccomp}[1]{\tilde{#1}}
\newcommand{\ccomp}[1]{\mathord{\sim}{#1}}
\newcommand{\rest}{R}
\newcommand{\makerest}[1]{\mathit{res}(#1)}
\newcommand{\restrictions}{\mathit{Res}}
\newcommand{\lrestrictions}{\mathit{LRes}}
\newcommand{\restrict}[2]{\ensuremath{#1\mathord{\downharpoonright_{#2}}}}
\newcommand{\implabel}{\mathit{st}}
\newcommand{\gimplabel}{\mathit{stg}}

\newcommand{\decs}{\mathit{Dec}}
\newcommand{\dec}{\mathsf{d}}
\newcommand{\imp}{\mathsf{i}}
\newcommand{\emptystack}{\epsilon}

\newcommand{\lit}{\ell}
\newcommand{\lits}{\mathit{Lit}}
\newcommand{\litseq}{L}
\newcommand{\stackres}[1]{\lfloor #1\rfloor}
\newcommand{\litrest}[1]{\langle#1\rangle}
\newcommand{\litres}{\mathit{lit}}

\newcommand{\solver}[2]{#1 ~\|~ #2}

%%% Pseudocode
\newcommand{\pdeduce}{\mathsf{deduce}}
\newcommand{\pdecide}{\mathsf{decide}}
\newcommand{\pmaximal}{\mathsf{maximal}}
\newcommand{\plearn}{\mathsf{learn}}
\newcommand{\pbacktrack}{\mathsf{backtrack}}
\newcommand{\pfail}{\mathsf{fail}}
\newcommand{\psafe}{\mathsf{safe}}

 \newcommand{\fpfont}[1]{\textbf{\sffamily{#1}}}
 \newcommand{\fpfn}[1]{\sffamily{#1}}
 \SetKwSty{fpfont}
 \SetFuncSty{fpfn}

 \SetKwFunction{Decide}{decide}
 \SetKwFunction{TransProp}{tprop}
 \SetKwFunction{UnitProp}{uprop}
 \SetKwFunction{Refines}{refines}
 \SetKwFunction{Deduce}{deduce}
 \SetKwFunction{IsUnit}{is\_unit}
 \SetKwFunction{Propagate}{propagate}
 \SetKwFunction{Invariant}{invariant}
 \SetKwFunction{Atomic}{atomic}
 \SetKwFunction{Learn}{learn}
 \SetKwFunction{Backtrack}{backtrack}
 \SetKwFunction{Clausegen}{clauseGen}
 \SetKwFunction{Invariantgen}{invariantGen}

 \SetKw{Let}{let}

%%% Generalization
\SetKwFunction{Asserting}{asserting}
\SetKwFunction{Cutheur}{cutheur}
\SetKwFunction{Generalise}{generalise}
\SetKwFunction{Analyse}{analyse}
\SetKwFunction{ClauseGeneralise}{clauseGeneralise}
\SetKwFunction{WpGeneralise}{wpGeneralise}
\SetKw{MyCase}{case}
\SetKwFunction{SetReason}{setReason}
\SetKwFunction{PickAssertingCut}{assertingCut}
\newcommand{\implicationgraph}{\mathit{implicationGraph}}

%%% Operators
\newcommand{\wrt}{w.r.t.\ }
% \newcommand{\inp}{\mathit{input}}
\newcommand{\M}{{\cal M}}
\newcommand{\Tr}{{\mathit{tr}}}
\newcommand{\clip}{{\mathit{clip}}}
\newcommand{\decomp}{{\mathit{decomp}}}
\newcommand{\form}{{\mathit{Form}}}
\newcommand{\clausecompl}{\mathit{clcomp}}
\newcommand{\covers}{{\mathit{covers}}}
\newcommand{\cfgdecomps}{{\cal D}_{\mathit{CFG}}}
\newcommand{\cfgrefine}{{\cal R}_{\mathit{CFG}}}
\newcommand{\literals}{{\cal L}}
\newcommand{\clauses}{{\cal C}}
\newcommand{\proofstate}{\mathit{proof}}
\newcommand{\literaldecomp}[1]{\mathit{lits(}#1\mathit{)}}
\newcommand{\transded}{\mathit{trans}}
\newcommand{\unitded}{\mathit{unit}}

\newcommand{\prog}{P}

\renewcommand{\min}{\mathit{min}}
\renewcommand{\max}{\mathit{max}}
\newcommand{\itvdom}{{\mathit{IEnv}}}
\newcommand{\atoms}{\mathit{Atoms}}

\newcommand{\irred}{\mathit{Irred}_\meet}

\newcommand{\implsu}{I_\mathsf{u}}
\newcommand{\implst}{I_\mathsf{t}}

\newcommand{\predec}{\mathit{predec}}
\newcommand{\conflictnode}{\mathit{conflict}}

\newcommand{\dred}[1]{\textcolor{red!60!black}{#1}}
\newcommand{\dgreen}[1]{\textcolor{green!40!black}{#1}}



\begin{document}

% ---------------------------------------------------------------------
% Title, authors, abstract, keywords
% ---------------------------------------------------------------------
\title{Precise Abstract Interpretation Using CDCL}

\author{Rajdeep Mukherjee\inst{1} \and Peter Schrammel\inst{2} \and 
Leopold Haller\inst{3} \and \\ 
Daniel Kroening\inst{1} \and Tom Melham\inst{1}}

\authorrunning{Mukherjee, Schrammel, Haller, Kroening, Melham}

\institute{University of Oxford, UK \and University of Sussex, UK \and Google Inc., USA}

\maketitle

%===============================================================================
\begin{abstract}
%===============================================================================
%
  This paper shows that Conflict Driven Clause Learning (CDCL) algorithm
  employed in modern day satisfiability solvers can be used to compute 
  fixed point approximations over lattice of program traces to determine 
  safety.  Satisfiability or model finding can be seen as a property of fixed 
  points of unsafe trace transformers over this lattice.  CDCL algorithm 
  alternates between an overapproximate model search phase and an 
  underapproximate conflict analysis phase.  We show that 
  the model search computes a greatest fixed point over unsafe trace transformer
  using deductions and decisions.  Whereas, conflict analysis computes a 
  least fixed point of safe trace transformer over a downset lattice with 
  a heuristic choice of conflict reasons.  Thus, CDCL architecture can be seen 
  as a natural tool to build a precise program analyzer that uses decision and 
  learning to automatically refine the precision of an analysis. This paper
  present a sound mathematical framework for building a precise abstract 
  interpretation based program analyzer using CDCL architecture as well as a practical tool 
  that implements these ideas for automatic bounded safety verification of C programs. 
  We~evaluate the performance of our analyser by comparing with CBMC, which
  uses Boolean CDCL, and Astr{\'e}e, a commercial abstract interpretation tool. 
  We~observe two orders of magnitude reduction in the number of decisions,
  propagations, and conflicts as well as a~1.5x speedup in runtime compared to
  CBMC.  Compared to Astr{\'e}e, ACDLP solves twice as many benchmarks and has
  much higher precision.  
  %Our work is the first step towards lifting CDCL algorithm for design and 
  %implementation of a precise static analyzers.
%
\end{abstract}

%===============================================================================
\section{Introduction}

\para{Conflict Driven Clause Learning Solvers} 
%
A Conflict Driven Clause Learning algorithm~\cite{cdcl} is a boolean satisfiability
procedure that alternates between a model search phase and conflict analysis
phase to solve a propositional formula given in conjunctive normal form (CNF).  
The model search phase uses \emph{decisions} and \emph{Boolean Constraint
Propagation} (BCP) to search for a satisfying assignment of the formula.  A 
decision uses branching to assign the chosen value to a chosen branching variable.  
Decision is followed by BCP step. BCP uses repeated application of unit rule where 
an unit rule identifies variables which must be assigned specific Boolean value. 
If BCP identifies an unsatisfied clause, the CDCL algorithm is said to be in 
\emph{conflict} and the algorithm \emph{backtracks} non-chronologically. A
backjumping procedure undoes all branching steps until the decision level 
where the solver state is consistent (non-conflicting).  This is a key 
difference with a DPLL solver which just backtracks to the previous 
decison level and flips it. The backjumping level in CDCL is determined 
by analyzing the most recent conflict and learning a new clause from the 
conflict.  If the backtracking level is the root of the search tree, then 
the solver terminates with no models, that is, the formula is \emph{unsatisfiable}.  
Else, the model search phase is repeated with the learnt clause and the procedure 
continues until all variables has been assigned in which case, the 
algorithm returns a satisfiable assignment, that is, the formula is \emph{satisfiable}. 
%

\para{Connection between CDCL and Program Analysis}
%
Thus, CDCL solvers can be understood as a procedure to compute approximations of fixed 
points over a lattice of partial assignments~\cite{sas12}.  A key insight 
that connects CDCL to program analysis is that they use an imprecise 
over-approximate domain of partial assignments to gain efficiency and 
techniques like decision and clause learning to improve precision.  
%
Silva et. al. in~\cite{popl2014} propose an understanding of CDCL in the language
of lattices and transformers suggesting a ``\emph{Grand Unification}" of SAT
and static analysis.
%
In this paper, we take one step towards this unification goal by characterizing CDCL
as a procedure for computing fixed point approximations over a lattice
of program traces to determine program safety.  
%learning based program analyzers.  
A practical benefit of this characterization is that the CDCL architecture can
be used to build a precise learning based static analyzers that operates over 
arbitrary non-distributive abstract domains~\cite{atva2017}. \rmcmt{update
reference} \\
%


CDCL can be understood as a general algorithmic framework, parameterized 
by a concrete domain $C$ and an abstract domain $A$, where $C$ is the 
set of propositional truth assignment and $A$ the domain of partial assignments.  
A characterization 
of CDCL as a program analyzer requires the concrete domain to 
be instantiated over a lattice of program traces that may lead to an error.  
Given a \emph{safe trace transformer} which returns a set of safe or 
invalid traces and an \emph{unsafe trace transformer} which returns a set 
of unsafe traces, satisfiability can then be seen as a property of fixed points of
unsafe trace transformer over this lattice.  A model search
overapproximates the unsafe trace transformer and conflict analysis
underapproximates the safe trace transformer.  Decisions refines a downward
iteration sequence and learning overapproximates set 
of unsafe traces.  This paper presents a theoretical recipe and a mathematical
basis for instantiating CDCL architecture for program analysis using the
framework of abstract interpretation.  
%to formalize an abstract interpretation account of Bounded Model Checking using CDCL
%architecture.  
In this paper, we restrict our formalizations to programs 
with bounded loops and finite recursion depth.
We expect that the characterizations presented in this paper would contribute 
to the development of precise and efficient static analyzers that can 
automatically recover from imprecision without employing expensive 
abstract domains such as trace partitioning~\cite{toplas07}.  
One such practical instantiation of CDCL as learning based program
analyzer~\cite{atva2017} \rmcmt{update reference} over a template based abstract domains is 
demonstrated in this paper. 
%
\para{Structure of the paper}
%
In this paper, we first briefly explain the work of Silva et. al.~\cite{sas12,
popl2014} that gives an abstract interpretation account of CDCL algorithm. 
We then present a novel abstract interpretation account of bounded model
checking using CDCL algorithm.  To this end, we first show that conventional
trace based abstractions are not ideal for lifting CDCL to program analysis due
to lack of precise complementation property. So,
we introduce an abstraction of program traces over a logical encoding of program 
using Static Single Assignment (SSA) form.  We then present a novel safety
verification algorithm called \emph{Abstract Conflict Driven Learning for Programs (ACDLP)} 
that uses decision and learning techniques to precisely reason over a lattice of
SSA elements.  We conclude that a practical benefit of abstract interpretation 
account of BMC using CDCL algorithm gives a new, learning based architecture for 
implementing precise program analysis tools.   
%
\para{Contributions}
In this paper, we make the following contributions.
%
\begin{enumerate}
  \item We characterize satisfiability as a property of fixed points of
  unsafe trace transformer over the lattice of program traces.
  \item We present an abstraction of trace semantics that is suitable for
  lifting CDCL architecture to program analysis. 
  \item We characterize model search as a technique to compute a greatest
    fixed point of an unsafe trace transformer using deductions and decisions.
  \item A conflict analysis is characterized as a technique to compute
    a least fixed point of safe trace transformer over a downset lattice 
    with heuristic choice of conflict reasons.  We show that learning is 
    a reduction over unsafe trace transformers, parameterized by an element of downset lattice.  
\end{enumerate}
%

~\label{intro}
%===============================================================================

%===============================================================================
\section{Solver Architectures}
%
We first discuss various architectures of DPLL based satisfiability solvers and then 
concentrate on extending CDCL solver to richer logics.  From architectural standpoint, 
the following classification of DPLL family of solvers gives a clear overview 
of the various similarities and differences between them.  

\para{CDCL}
Figure~\ref{cdcl} gives the architecture of CDCL solver~\cite{cdcl}, which 
has two key components -- \emph{model search} and \emph{conflict analysis}.  
%
\begin{figure}
\centering
\scalebox{.65}{\import{figures/}{cdcl.pspdftex}}
  \caption{Architecture of CDCL \label{cdcl}}
\end{figure} 
%
%
\begin{figure}
\centering
\scalebox{.65}{\import{figures/}{dpll.pspdftex}}
  \caption{Architecture of Classical DPLL(T) \label{dpll}}
\end{figure} 
%
\begin{figure}
\centering
\scalebox{.65}{\import{figures/}{natural_domain_smt.pspdftex}}
  \caption{Architecture of Natural Domain SMT Solver \label{natural-domain-smt}}
\end{figure} 
%
\begin{figure}
\centering
\scalebox{.65}{\import{figures/}{natural_domain_smt.pspdftex}}
  \caption{\rmcmt{Architecture of Model Constructing Satisfiability Calculus} \label{mcsat}}
\end{figure} 
%
%%%%%%%%%%%%%%%%%%%% DPLL(T) %%%%%%%%%%%%%%%%%%%%%%%%%%
\para{DPLL(T)}
Figure~\ref{dpll} gives the architecture of DPLL(T) solver~\cite{cdcl,smt1,smt2}, also 
called \emph{Lazy SMT}.  
DPLL(T) uses CDCL solver as a black box which is used to enumerate the 
assignments of the boolean abstraction of the formula. The candidate boolean 
assignment is then checked with a solver for conjunctive fragment of theory to 
check satisfiability of quantifier-free formula in a theory.  Thus, decisions and 
learning in DPLL(T) is delegated to the propositional CDCL solver which is used to 
enumerate theory facts.  Whereas, the theory solver adds blocking clauses in terms of 
the existing atoms that represent partial assingment infeasible in the theory.   The 
work of~\cite{} introduced \emph{Splitting on demand} 
which encodes new theory facts as propositional variable and adds to a formula.  

There are various first-order 
theories such as arithmetic, bit-vector, arrays, uninterpreted functions, and datatypes.  
One of the benefits of the DPLL(T) architecture is that it is directly impacted by any 
advances and performance improvements of the CDCL SAT solver.    


%
%%%%%%%%%%%%%%%%%%%% Natural Domain SMT %%%%%%%%%%%%%%%%%%%%%%%%%%
\para{Natural Domain SMT}
Figure~\ref{natural-domain-smt} gives the architecture of natural domain 
SMT solver, proposed by Cotton et al., that integrates the theory solvers 
inside the CDCL procedure with techniques such as theory propagation and theory learning.  
The unification of the theory solver and CDCL search into a common architecture 
enables direct search of the model of formula over the natural domain of variables. 
This lead to the development of new solver procedures that performs direct model 
construction complemented with conflict resolution in some restricted first-order 
domains such as floating-point~\cite{DBLP:journals/fmsd/BrainDGHK14}, linear real
arithmetic~\cite{ndsmt,linear-real}, linear integer arithmetic~\cite{linear-int}, 
nonlinear arithmetic~\cite{nonlinear}.

%%%%%%%%%%%%%%%%%%%% Generalized DPLL %%%%%%%%%%%%%%%%%%%%%%%%%%
\para{Generalised DPLL}
McMillan et al.~\cite{DBLP:conf/cav/McMillanKS09} presented a Generalised DPLL 
procedure called \emph{GDPLL} which performs decisions and learning directly 
in theory.  A key difference between lazy SMT and GDPLL is that the theory 
deductions in GDPLL only occurs in response to a conflict in model search in contrast 
to lazy SMT which restricts theory deductions only in response to satisfying 
partial assignments.  Additionally, GDPLL allows theory learning in contrast to lazy SMT 
approach which cannot learn theory facts.  Similar theory-specific approaches are presented for 
equality~\cite{DBLP:journals/iandc/BadbanPTZ07} and integer linear 
arithmetic~\cite{DBLP:conf/cade/JovanovicM11}.


%%%%%%%%%%%%%%%%%%%% MCSAT %%%%%%%%%%%%%%%%%%%%%%%%%%
\para{Model Constructing Satisfiability Calculus}
While natural domain SMT procedures are quite effective in their respective
first-order theories, they have limitations in pure boolean reasoning and
are not compatible with DPLL(T) procedures.  A \emph{model-constructing 
satisfiability calculus} (mcSAT)~\cite{mcsat1,mcsat2} extends DPLL(T) 
procedure through \emph{model assignment} which allows decisions on theory variables.  
Furthermore, it allows propagations and learning in terms of theory variables 
that are not present in the given formula.

\begin{example}~\label{mcsatex}
%
Let us illustrate mcSAT procedure through a simple example. Consider a formula 
$\phi = (x \geq 5) \wedge (\neg (x \geq 2) \vee y \geq 2) \wedge (x^2 + y^2 \leq 2 \vee xy > 2)$.  
Let clause $C1 = (x \geq 5)$, $C2 = (\neg (x \geq 2) \vee y \geq 2)$ and 
$C3 = (x^2 + y^2 \leq 2 \vee xy > 2)$.  The procedure performs obvious deductions initially 
that sets $C1$ to \emph{true}, that is, $(x \geq 5)$ is deduced.  This infers $x \geq 2$ 
in $C2$ which implies $y \geq 2$ in $C2$.  The current partial assignment 
is extended by a decision which sets $x^2 + y^2 \leq 2$ to \emph{true} and performs a 
\emph{model assignment}, $x \rightarrow 5$.  Decisions in mcSAT constitute a truth 
assignment to decided literal as well as concrete assignments to first-order uninterpreted 
symbol.  The later is a model assignment.  The trail after decision is shown in figure~\ref{dec}.
Clearly, the decision leads to a conflict since the trail element $x^2 + y^2  \leq 2$ is 
unsatisfiable under the current model assignment.  The procedure investigates the 
reason for the conflict and learns a generalized clause, 
$C4 = \neg (x^2 + y^2 \leq 2) \vee (x \leq 1)$.  An alternative but not so productive 
learned clause could be $\neg(x^2 + y^2 \leq 2) \vee \neg(x \rightarrow 5)$.  
The contents of the trail at this point are $\{x \geq 5, x \geq 2, y \geq 2, x^2+y^2 \leq 2 \}$.  Clause 
$C4$ infers a new element, $x \leq 1$ which contradicts with the trail element $x \geq 5$.  
Conflict resolution between clauses $C4 =  \neg (x^2 + y^2 \leq 2) \vee (x \leq 1)$ and 
$\neg (x \geq 5) \vee \neg (x \leq 1)$ yields a learned clause $C5 = \neg(x \geq 5) \vee \neg(x^2 + y^2 \leq 2)$.  
The contents of the trail after conflict analysis are $\{x \geq 5, x \geq 2, y \geq 2 \}$. Clause $C5$ infers a 
new element, $\neg (x^2 + y^2 \leq 2)$ which is added to the trail.  The trail obtained 
through propagation after learning is shown in figure~\ref{mcfinal}.  The final content of 
the trail is $\{x \geq 5, x \geq 2, y \geq 2, \neg(x^2 + y^2 \leq 2) \}$ which satisfies 
all clauses.  Hence, the satisfying solution is $\{x=5, y=2\}$.
\end{example}
%
\begin{figure}
\centering
\scalebox{.70}{\import{figures/}{trail_mc.pspdftex}}
  \caption{Content of trail after decision for example~\ref{mcsatex} \label{dec}}
\end{figure} 
%
%
\begin{figure}
\centering
\scalebox{.70}{\import{figures/}{trail_mc_final.pspdftex}}
  \caption{Content of trail after learning for example~\ref{mcsatex} \label{mcfinal}}
\end{figure} 
%

\para{ACDLP versus DPLL/CDCL-based solvers}
%
There are few fundamental differences between ACDLP and solver procedures discussed above. 
ACDLP is different from DPLL(T) procedure in that it does not require a propositional solver and 
the reasoning happens directly in an abstract domain which corresponds to a fragment of theory.  
ACDLP is also different from GDPLL.  This can be explained with an example borrowed from
~\cite{DBLP:conf/cav/McMillanKS09}.  Consider the formula, 
$\phi = (a < b) \wedge (a < c) \wedge (b < d \vee c < d) \wedge (d < a)$, where $a,b,c,d$ are integers.  
GDPLL determines that $\phi$ is unsatisfiable using decisions and shadow rule for conflict analysis.  
Whereas, ACDLP do not need to make any decision and can prove unsatisfiability through deductions 
in the abstract domain of Inequality.  An Inequality abstract domain contains only conjunctions of 
inequalities. The abstract value computed through evaluating each clause in the Inequality abstract 
domain is given by $\pi = \{a < b, a < c, d < a\}$.  The greatest fixed point iteration for $\phi$ under
$\pi$ is computed by first applying best abstract transformer for each singleton clause, which gives 
$\pi'= \pi \cup \{ d < b, d < c \}$.  Now, the clause $(b < d \vee c < d)$ is unsatisfiable under 
$\pi'$.  Hence $\phi$ is unsatisfiable. 


Whereas, one of the fundamental differences between ACDLP and mcSAT is that decisions in mcSAT 
are restricted to concrete model assignments to theory variables and boolean assignments to 
theory atoms in the formula.  Whereas, decisions in ACDLP can be made over set of all meet 
irreducibles representable by the underlying abstract domains.



Another key difference between ACDLP and the solver procedures discussed above is that 
ACDLP directly operates over SSA representation of a program whereas a DPLL(T) based solver 
operates on a formula which is a conjunction of sentences in a given logic where a sentence 
(or clause) is a disjunction of theory atoms. This formula is usually specified in 
SMT-LIB2~\cite{smt-lib2} language.  For example, a sentence in the logic formula of 
quatifier-free linear rational arithmetic (QFLRA) is of the form 
$c_1x_1 + c_2x_2 + \ldots + c_nx_n \diamond c_0$, where $c_i$ is a rational constant 
and $x_i$ is a rational variable, and $\diamond$ is either $\geq$ or $>$.  A model of 
QFLRA is an assignment of rational values to variables $x_i$ such that it satisfies 
every clause of the formula.  \rmcmt{In contrast, an SSA may not be exactly represented 
in a clausal form.  Recall from section~\cite{ssa-cite}, elements of SSA are obtained directly 
from program statements through syntactic translation steps.  Hence, a clause in SSA 
representation constitute an equality or inequality (obtained from assignment statements 
or an assertion) or disjunctions (obtained from control-flow branches).} 

%===============================================================================

%===============================================================================
\section{Program Model}~\label{pmodel}
%
Let $Prog$ be a program. Let $Var$ be the set of variables in the program $Prog$
and let $Val$ be the set of values that each variable can take and may be a 
scalar number in set of integers $\mathbb{Z}$ or set of rationals
$\mathbb{Q}$ or reals $\mathbb{R}$.
%
We consider \emph{programs} with bounded loops and finite recursion depth 
along with safety properties given as a set of assertions, $\assertions$, 
in the program.  All the loop bounds in the program are known apriori. 
All functions calls in the program are inlined before analysis.  
%We also support pointers.  \rmcmt{other program model supported}.
%
\section{Semantic Representations of Program}
%
The \emph{concrete semantics} of a program is the most precise mathematical
description of the program behavior. Any other semantic representation of a
program obtained through static analysis such as {\emph{data-flow
analysis}~\cite{flow-book} or \emph{set-constraint}~\cite{Cousot:1995} 
based analysis, is an abstract semantics which is derived from the concrete 
semantics via Galois connections. 


Several semantic representation~\cite{Cousot04,semantic02} of a program have been proposed
in the literature to analyze program properties.  Cousot~\cite{Cousot04} defines
a hierarchy of abstraction semantics of a state transition system $M$ corresponding to a 
control-flow graph $G$.  Cousot starts from partial trace semantics and derive successive
approximations via Galois connection which follows the sequence -- a) \emph{partial 
trace semantic}, b) \emph{reflexive transitive closure semantic}, 
c) \emph{reachability semantic}, d) \emph{interval semantic}. Each semantic 
representation differ from the other in the precision of the information.  An 
abstract semantics is less precise than their concrete counterparts and hence 
can prove less program properties, but are cheaper to compute or approximate.  
Cousot in~\cite{semantic02} gives a complete range of abstract semantics of a program.
%


\para{Partial Trace Semantics}
%
The semantics of a program computes the \emph{states} of a program which gives a
concise mathematical meaning of a program.  The collecting semantics of a
program computes all possible memory states that can occur during the execution
of a program.
%
Recall that a trace $\pi$ of a state transition system $M=(\Sigma, \mathcal{R},
\mathcal{I})$ is a finite sequence of states that follows the transition
relation $\mathcal{R}$.  Let $\Sigma_{M}^n$ denote the set of finite execution 
traces of length $n$. Then, $\Sigma_M^0=\emptyset$, while  
$\Sigma_M^1=\{s \mid s\in \Sigma\}$. 
Further, a trace of length $(n+1)$ can be expressed recursively as 
$\Sigma_M^{(n+1)}=\{\sigma ss' \mid \sigma s\in \Sigma_{M}^n \wedge (s,s') \in
\mathcal{R}\}$.   
%


%Thus, a \emph{concrete semantics}~\cite{Cousot04} of a program is the set of all finite 
%execution traces, 
We will denote the \emph{partial trace semantics} as $(\llbracket M \rrbracket_{t})$ 
which is powerset of traces, $(\powerset(\Pi), \subseteq)$.  
%
A fixed point characterization of the trace semantics is defined as follows. 
%
\[
  \mathit{lfp}\; F_{M}^{t}\; \text{where}\; F_{M}^{t}(X) \mathrel{\hat=} \{s \mid s \in
  \Sigma\} \cup \{\sigma ss' \mid \sigma s \in X \wedge (s,s') \in \mathcal{R} \}
\]
%
\para{Collecting Semantics over Traces}
%
The collecting semantics~\cite{Cousot04} of $M$, $\Sigma_{M}^{*}$, 
is the set of all such finite execution traces of transition system $M$, and is given by 
$\Sigma_M^*\mathrel{\hat=} \bigcup_{n\geq0}\Sigma_M^{n}$. 
%
%
\para{Reachability Semantics}
%
%Whereas, Min\'{e}~\cite{minepaper} starts from 
A \emph{reachability} semantics~\cite{minethesis} of a state transition 
system $M=(\Sigma, \mathcal{R},\mathcal{I})$ is the set of states that are 
reachable from the initial states $\mathcal{I}$. A reachability
semantics is a complete lattice of powerset of
states defined as, $(\powerset(V \times (Var \rightarrow Value)), \subseteq,
\cup, \cap)$ and is denoted by $(\llbracket M \rrbracket_{r})$. 
%
A fixed point characterization of the reachability semantics is
defined as follows.
\[
  \mathit{lfp}\; F_M^{r}\; \text{where}\; F_{M}^{r}(S) \mathrel{\hat=} \mathcal{I} \cup \{s' \mid
   \exists s \in S \wedge (s,s') \in \mathcal{R} \} 
\]
%
Cousot in~\cite{Cousot04} shows that a reachability semantics can be derived
from a partial trace semantics through a sequence of abstractions.  Thus, 
the trace semantics $\llbracket M \rrbracket_{t}$ is more precise than the 
reachability semantics $\llbracket M \rrbracket_{r}$.  However, it is
practically infeasible to compute the trace semantics of a program 
that finds all memory states which occur during a program execution. 
%
\rmcmt{continuation text}
\section{Trace Transformers and CFG Safety}
%
The semantics of a state transition system $M$ corresponding to a control-flow graph
$G$ can be defined in terms of \emph{state semantics}~\cite{Cousot04} or \emph{trace
semantics}~\cite{Cousot04}.  Informally, a state semantics $(\llbracket M \rrbracket_{s})$ 
or reachable semantics defines the reachable states of $M$ starting from an initial state.  
Whereas, a trace semantics $(\llbracket M \rrbracket_{t})$ of $M$ is the set of well-formed 
traces of $M$. 
%
In section~\ref{state-transformer}, we define various classical 
\emph{state transformers} of $M=(\Sigma, \mathcal{R}, \mathcal{I})$ 
that operates over the powerset of states, $\mathcal{P}(\Sigma)$.
%
We now define various trace transformers that operates over $\powerset(\Pi)$ 
and formulate \emph{CFG safety} in terms of these transformers. The term
$\sigma_i \in \Sigma$ refers to a state and $\pi \in \Pi$ ranges over a set of 
traces in $\Sigma^*$. 
%
\para{Strongest Postcondition} A \emph{strongest postcondition}
  transformer, $tpost(T)$, of a set of traces $T \subseteq \Pi$ is defined as  
  $tpost(T) \mathrel{\hat=} \{\pi.\sigma_l \mid \exists \pi \in \Pi.\; \pi \in T
  \wedge \{\sigma_a \ldots \sigma_k\} \in \pi \wedge \sigma_k \rightarrow \sigma_l\}$. 
  The term $\pi.\sigma_l$ denotes extending trace $\pi$ with its postcondition
  state $\sigma_l$.\\
%
\para{Weakest Precondition} A \emph{weakest precondition}
  transformer, $\widehat{tpre(T)}$, of a set of traces $T \subseteq \Pi$ is defined as  
  $\widehat{tpre(T)} \mathrel{\hat=} 
  \{\sigma_b\pi \mid \forall \sigma_a \in \Sigma.
  \sigma_a\sigma_b\pi \in T \vee \sigma_a \not\rightarrow
  \sigma_b \}$. \\
%
\para{Existential Precondition} An \emph{existential precondition}
  transformer, $tpre(T)$, of a set of traces $T \subseteq \Pi$ is defined as  
  $tpre(T) \mathrel{\hat=} \{\sigma_a.\pi \mid \exists \pi \in \Pi.\; \pi \in T
  \wedge \pi = \{\sigma_b \ldots \sigma_k\} \wedge \sigma_a \rightarrow
  \sigma_b\}$. 
  The term $\sigma_a.\pi$ denotes extending trace $\pi$ with its precondition
  state $\sigma_a$.\\
%
\para{Universal Postcondition} An \emph{universal postcondition}
  transformer, $\widehat{tpost(T)}$, of a set of traces $T \subseteq \Pi$ is defined as  
  $\widehat{tpost(T)} \mathrel{\hat=}
  \{\pi\sigma_k  \mid \forall \sigma_l \in \Sigma.
  \pi\sigma_k\sigma_l \in T \vee \sigma_k \not\rightarrow
  \sigma_l \}$. 
%
%-------------------------------------------------------------------------------
\para{CFG Safety}  
%
Let $G = (V, E, \mathcal{S}, \mathcal{T}, I, E)$ 
  be a CFG with a special error node $\err \in V$.  A trace $\pi
  \in G$ is safe if it does not terminate in the error location. A CFG $G$ is
  \emph{safe} w.r.t. $\err$ if all traces $\pi$ of $G$ are safe
  w.r.t $\err$.
%
We define two trace transformers over $\powerset(\Pi)$ for safety checking of a
CFG, an unsafe trace transformer, $f_{unsafe}$, and a safe trace transformer,
$f_{safe}$. A CFG $G$ is \emph{safe} exactly if $f_{unsafe}^{G}(\Pi)=\emptyset$.
%
\begin{equation}\label{eq:unsafe}
f_{unsafe}^{G}(T) \mathrel{\hat=} \{\pi \in \Pi \mid \pi \in T \wedge
\pi\;\text{ is well formed and not safe} \} 
\end{equation}
%
\begin{equation}\label{eq:safe}
f_{safe}^{G}(T) \mathrel{\hat=} \{\pi \in \Pi \mid \pi \in T \vee \pi\;
\text{is not well formed or safe} \} 
\end{equation}
%
\para{Fixed-point Characterization of Unsafe Trace Transformer} 
%
We present a fixed point characterization of the unsafe safe transformer for
counterexample search and conflict generalization. A counterexample search 
corresponds to finding an abstract representation of a counterexample trace 
over $\powerset(\Pi)$. 
%
Let $\chi$ denote the set of states $\{(v,\omega)\mid v = \err \}$ that 
are at the error location.  Then, the unsafe trace transformer of CFG $G$, can 
be characterised as follows.
%
\begin{proposition}
%
  $f_{unsafe}^{G}(T) = T \cap (\mathit{lfp}\; Z.\;\mathcal{I} \cup tpost(Z)) \cap 
    (\mathit{lfp}\; Z.\;\chi \cup tpre(Z))$
%
\end{proposition}
%
\begin{proof}
  The fixed point characterization of unsafe trace transformer is described as
  follows. 
  Let $C1  = (lfp Z.\;\mathcal{I} \cup tpost(Z))$ and $C2 = (lfp Z.\; \chi \cup
  tpre(Z))$. The set $C1$ gives set of traces that starts from an initial
  state $\mathcal{I}$ and follows the transition relation $\mathcal{R}$.  
  The set $C2$ gives the set of traces that follows the transition relation 
  and terminates in an error state.  The set $C1 \cap C2$ is the set of 
  well-formed traces that terminates in an error state. It follows 
  that $f_{unsafe}^{G}(T) = T \cap C1 \cap C2$.  
\end{proof}
%
Note that the two fixed points above, for computing $C1$ and $C2$, represents a
forward and backward analysis respectively. Combination of forward and backward
analysis provide strictly greater precision than applying either in
isolation~\cite{Cousot99} in the abstract.  
%
%%%%%%%%%%%%%%%%% TRACE SEMANTICS APPROXIMATION %%%%%%%%%%%%%%%%%%%
\section{Approximations of Trace Semantics}
%
Figure~\ref{fig:semantic} shows the sequence of syntactic translation steps 
and semantic approximations of concrete set of traces, $\powerset(\Pi)$.  
A program is represented by a Control-flow Graph $G=(V,E,S,\mathcal{T},I,E)$.  
Classical abstract interpretation interprets a CFG as equation 
system~\cite{minethesis,Schmidt98,tacas12}, which is described next.  The box in 
\emph{red} shows the corresponding lattices that an abstract interpreter
operates on.  The ACDLP procedure presented in this paper operates on a
different lattice which is obtained by syntactic translation step
$\mathcal{T}_2$ that translates a CFG to SSA form. The bounded box in 
\emph{green} shows the corresponding lattices for ACDLP. In this section, 
we first describe an equation system that is derived from a CFG which is commonly
used for abstract interpretation.  Later, we describe a syntactic translation 
of CFG to Static Single Assignment form which is used for ACDLP procedure.   
In section~\ref{semantic-trace}, we desribe the corresponding lattices over 
these structures and various transformers to operate on these lattices. 
%
\begin{figure}[htbp]
\centering
\scalebox{.70}{\import{figures/}{semantic.pspdftex}}
\caption{Semantic Representation of Program \label{fig:semantic}}
\end{figure}
%
\rmcmt{\para{Program Transformation $\mathcal{T}_1$}}
%
\subsection{CFG to Constraint System}~\label{ssa-cite}
%
Recall that an operational semantics of a CFG can be defined using a 
state transition system. An equation 
system can be derived from a state transition system, which introduces a set
values varaible $(S_v)_{v \in V}$ to each program location that takes values in
powerset of environments, $\powerset(Var \mapsto Value)$.  These equations perform 
point-wise lifting of $\powerset(Var \mapsto Value)$ to $V$.   The variables $S_v$
are related through the transformers associated with program statements that express the 
data-flow equations between individual control-flow nodes in the CFG.  Haller
et. al. call this equation system~\cite{minethesis} 
\emph{static analysis equation}~\cite{tacas12}.  Solution to these equations 
yields the data-flow information of $G$.  The resulting lattice that describes 
these static analysis equations is given by $(\mathcal{F}_G, \sqsubseteq)$.   
%


Figure~\ref{fig:se} gives the CFG representation of a program (in left) 
and its corresponding static analysis equations (in right).  The 
function $post(S)$ computes the successor state of a set of states 
$S$ that can be reached in one step.  A set-valued variable $S_v$ is 
introduced at every control location $v$ in the CFG.  Note that the variables
$S_v$ are related through the post-condition transformer $post$ associated with
the program statements, for example, the variable $S_{n6}$ at the loop head
merges the control-flow from $S_{n5}$ and $S_{n7}$.  The CFG is \emph{safe} if
$S_{Error}$ is empty $(\emptyset)$. 
%
\para{Collecting Semantics over Equation Systems}
%
A collecting semantics over static analysis equations of a program gathers for 
each program variables and program location its value during program execution.
%
\begin{figure}[t]
%\scriptsize
\centering
\begin{tabular}{c|c|}
\hline
  Control-Flow Graph & Static Analysis Equation \\
\hline
\begin{minipage}{4.2cm}
\scalebox{.65}{\import{figures/}{semantic-example.pspdftex}}
\end{minipage}
&
\begin{minipage}{10cm}
$\begin{array}{l}
     S_{n1} = \top, \\
     S_{n2} = post_{y>0}(S_{n1}) \\
     S_{n3} = post_{y=0}(S_{n1}) \\
     S_{n4} = post_{y<0}(S_{n1}) \\
     S_{n5} = post_{x:=2}(S_{n2}) \cup post_{x:=0}(S_{n3}) \cup post_{x:=-2}(S_{n4}), \\
     S_{n6} = post_{y:=x*y}(S_{n5}) \cup post_{y:=y+2}(S_{n7}), \\ 
     S_{n7} = post_{y \leq 20}(S_{n6}), \\
     S_{Error} = post_{y<0}(S_{n6})
\end{array}$
\end{minipage}
\\
\hline
\end{tabular}
\caption{\label{fig:se} A CFG and its static analysis equation}
\end{figure}
%
\subsection{Logical Encoding of Program Semantics}~\label{ssa-cite}
%
\para{Program Transformation $\mathcal{T}_2$}
%
The program transformation $\mathcal{T}_2$ shown in Figure~\ref{fig:semantic}
involves two separate steps. 
\begin{enumerate}
  \item Generation of a \emph{bounded program}
  \item Translation of acyclic control-flow of bounded program to SSA
\end{enumerate}
%
\para{Generation of Bounded Program}
%
A \emph{bounded program} is obtained from an input program by a transformation 
that unfolds loops and recursions a finite number of times, which generates an
acyclic control-flow structure.  Figure~\ref{fig:unwind} and figure~\ref{fig:unwind-cfg} 
gives an input program and the corresponding bounded program obtained through
loop unrolling, respectively.
%
\begin{figure}[htbp]
\centering
\vspace*{-0.2cm}
  \scalebox{.90}{\import{figures/}{unwind.pspdftex}}
\caption{Input Program with bounded loops
  \label{fig:unwind}}
\end{figure}
%
\begin{figure}[htbp]
\centering
\vspace*{-0.2cm}
  \scalebox{.90}{\import{figures/}{unwind-cfg.pspdftex}}
\caption{Generating Bounded Program throug loop unrolling
  \label{fig:unwind-cfg}}
\end{figure}
%
The CFG representation of the bounded program is 
translated to a Static Single Assignment (SSA) form 
%through the syntactic translation step, $\mathcal{T}_2$, as illustrated in Figure~\ref{fig:semantic}.  
The translation from CFG to SSA is a well known technique 
in optimizing compiler and verification~\cite{ssa1,ssa2,ssa1988,ssa1991}.  We 
briefly explain the details of the CFG to SSA translation since this is standard. 
%
\para{CFG to Static Single Assignment Form}
%
The translation from CFG to SSA follows two separate steps.  
The first step gives a unique \emph{index} to each
definition of a program variable, and each use of that variable is given the
index of the definition that reaches it; the second step insert special 
$\phi$-functions at control-flow join points where a given variable may have
more than one reaching definition.  The argument to a $\phi$-function is the 
set of all indexed instance of the variable that could reach the join point.
Based on the current execution trace, the $\phi$-function select an appropriate
instance of the variable and assign it to a new instance of the variable. The
translation involves the following steps. 
%
\begin{enumerate}
\item A unique name for each definition point in the procedure, shown in
  figure~\ref{fig:ssa-simple}. 
\item Identifies points in the procedure that merges different values from
  distinct control-flow paths, shown in figure~\ref{fig:ssa-simple}  
\item Identification of induction variables in loops becomes easy by inserting a
  $\phi$-function for any variable that is modified inside the
    loop~\cite{Gerlek:1995}.
\end{enumerate}
%
\begin{figure}[htbp]
\centering
\vspace*{-0.2cm}
  \scalebox{.90}{\import{figures/}{ssa-simple.pspdftex}}
\caption{CFG to SSA Translation Steps
  \label{fig:ssa-simple}}
\end{figure}
%
Figure~\ref{fig:unwind-ssa} gives an example of SSA translation for the bounded
program in Figure~\ref{fig:unwind-cfg}.
%for a simple CFG, following the above mentioned steps.
%
\begin{figure}[htbp]
\centering
\vspace*{-0.2cm}
  \scalebox{.90}{\import{figures/}{unwind-ssa.pspdftex}}
  \caption{Static Single Assignment form for Bounded Program of
  Figure~\ref{fig:unwind-cfg}
  \label{fig:unwind-ssa}}
\end{figure}
%
\para{Collecting semantics over SSA}
%
A collecting semantics over SSA of a bounded (loop free) program gathers for each 
SSA variables its value during program execution.
%
\para{Advantages of SSA}
%
Previous researches have shown that Static Single Assignment form has
advantages over other forms of programs representation which allows compilers to
facilitate program analysis through simplification of data-flow analysis such as
def-use and use-def chains and design of optimization
algorithms~\cite{ssa1991,ssa1,ssa2}. 
%
\para{Exact SSA semantics} An SSA semantics is \emph{exact} if it is
constructed from a bounded program.  Recall that a bounded program is 
obtained through \emph{complete} unwindings of all loops and recursions, which
gives an acyclic code.  
Therefore, for acyclic code, an SSA exactly represents the strongest 
post-condition computation of the code.
The lattice that describe these SSA constraints over SSA 
variables is given by $(\mathcal{SA}_G, \subseteq_{SA})$.  
%
%Chapter~\ref{six} gives the \emph{exact}, \emph{overapproximate} and \emph{underapproximate} SSA semantics for an input program. 
%
In this paper,  we restrict our formalizations to bounded programs which 
gives an exact SSA semantics of a program.  
%
\para{SSA Safety}
%
The translation from CFG to SSA is represented by a set 
$\constraints=\Prog\cup\{\neg \bigwedge_{\assertion\in\assertions} \assertion\}$,
where $\Prog$ contains an encoding of the statements in the program as
constraints, obtained after translating the program into single static 
assignment (SSA) form~\cite{ssa88,ssa1988,ssa1991}. 
%
Based on the above program representation, we define a \textit{safety formula}
($\formula$) as the conjunction of all elements in $\constraints$, that is,  
$\formula:= \bigwedge_{\constraint\in\constraints} \constraint$.  The formula 
$\formula$ is unsatisfiable if and only if the program is safe.
\rmcmt{Intuitively, $\formula$ is the usual SAT encodings of BMC for
reachability.} 
%
\begin{example}
The safety formula $\formula$ for the bounded program of 
figure~\ref{fig:unwind-ssa} is shown below.  Note that $\formula$ is obtained
by taking conjunction of all SSA and the negation of set of assertions. 
%
\begin{equation}\label{eq:ssa}
\begin{array}{l}
  \formula = (x_0 = 0) \wedge
     (x_0 \leq 2) \wedge
     (x_1 = x_0 + 1) \wedge
     (x_1 \leq 2) \wedge
     (x_2 = x_1 + 1) \wedge
     ((x_1 > 2) \vee (x_2 > 2)) 
\end{array}
\end{equation}
\end{example}
%
\para{Translation from Static Single Assignment to Program Trace}
%
\begin{figure}[htbp]
\centering
\vspace*{-0.2cm}
  \scalebox{.90}{\import{figures/}{syn-map.pspdftex}}
\caption{Concretization-Based Approximation of Trace Semantics in ACDLP
  \label{fig:syn-map}}
\end{figure}
%
Recall that $(\powerset(\Pi),\subseteq)$ and $(\mathcal{SA}_G, \subseteq_{SA})$ 
forms a \emph{concretization-based galois connection} through a concretization
operator $\gamma_T$, as shown in figure~\ref{fig:lattice}. Cousot and Cousot in
~\cite{CC92} propose a relaxation of galois connection framework that allows to work
only with a concretization operator $\gamma$ or dually an abstraction operator
$\alpha$.  We will use this concept to establish the relationship between 
$(\powerset(\Pi),\subseteq)$ and $(\mathcal{SA}_G, \subseteq_{SA})$ only through a 
concretization operator, $\gamma_T$.  Figure~\ref{fig:syn-map} illustrates that 
$\gamma_{T}$ can be achieved through the syntactic back-translation steps, 
$\mathcal{T}_1'$ and $\mathcal{T}_2'$.  
%
\para{Program Transformation $\mathcal{T}_2'$}
%
The program transformation $\mathcal{T}_2'$ shown in Figure~\ref{fig:syn-map}
involves two separate steps. 
%
\begin{enumerate}
  \item Copy-insertion~\cite{Srikant:2007} to remove $\phi$-nodes. 
  \item \rmcmt{Variable renaming in each basic blocks}
\end{enumerate}
%
The translation of SSA to an original program following these syntactic translation
steps is a well known technique in compiler-based code
optimization~\cite{Briggs:1998,ssa1991,Srikant:2007}.
Briggs et. al.~\cite{Briggs:1998} give an algorithm for translation of SSA 
to the original program by replacing $\phi$-functions with appropriately-placed
\emph{copy} instructions.  The program semantics is preserved by inserting a 
copy statement for each argument of $\phi$-function, at the end of each 
predecessor block of the $\phi$-node. Figure~\ref{sp} shows an example of copy
insertion to replace $\phi$-function in SSA.  We skip the details of the translation 
from SSA back to original program since this is standard and beyond the scope of 
this paper.
%
\rmcmt{what happens to ssa variables}.
%
\begin{figure}[htbp]
\centering
\vspace*{-0.2cm}
  \scalebox{.70}{\import{figures/}{sp.pspdftex}}
  \caption{Back-translation from SSA to CFG using copy insertion
  \label{sp}}
\end{figure}
%
\begin{definition} (Concretization) 
  \[
    \gamma_T \colon (\mathcal{SA}_G,\subseteq_{SA})
    \overset{\gamma_{T}}{\rightharpoonup}
    (\powerset(\Pi), \subseteq) 
  \]
$\gamma_T$ is automatically monotonic since $\gamma_T(A) = \pi$, where $A \in
  \mathcal{SA}_G$ is an exact abstraction of $\pi \in \Pi$.  That is, 
  $(\mathcal{SA}_G, \subseteq_{SA})$ is an exact abstraction of 
  $(\powerset(\Pi),\subseteq)$ for the programs considered in this paper (see
  section~\ref{pmodel}).
\end{definition}
%  
\rmcmt{Given a SSA representation of a program, the semantics of a program 
can be understood as a constraint system which is a conjunction of all 
SSA constraints.  A satisfying model of safety formula ($\formula$) 
containing SSA constraints corresponds to an unsafe trace (or counterexample) 
of the original program.  Later, we present a procedure for finding a satisfiable
assignment of $\formula$ in section~\ref{acdlp}.  Note that the model of
$\formula$ contains concrete assignments to SSA variables, which has to be
mapped back to the original program.}
%
%%%%%%%%%%%%%%%%%%%%%%%%%%%%%%%%%%%%%%%%%%%%%%%%%%%%%%%%%%%%%%%%%%%
%%%%%%%%%%%%%%%%%%%%%%% Trace-based Abstraction  %%%%%%%%%%%%%%%%%%%%%
%
\section{Lattice for Approximation of Trace Semantics}~\label{semantic-trace}
%
In this section, we describe the various lattices over the static analysis
equations and static single assignment form for a given control-flow graph.  
Figure~\ref{fig:lattice} shows the various lattices that approximate the
concrete trace semantics.
%


An important property of a clause learning SAT solver is that each meet
irreducibles of the partial assignments domain admits precise
complements~\cite{sas12}.  This property enables learning of domain 
elements that navigates the search away from the conflicting region.   
In order to lift CDCL to program analysis, it is imperative to find a suitable
trace based abstraction that admits precise complementation property. 
To this end, we first show that conventional means to represent a program through 
abstraction of a static analysis equation does not admit precise 
complements with respect to unsafe traces.   In this paper, we present an 
abstraction of program traces using logical encoding of CFG that bounded 
model checkers use.  We will show in Section~\ref{complement} that this 
representation allows us to lift CDCL to program analysis.  
%
%Hence, our formalizations can be understood as a full abstract interpretation 
%account of bounded model checking.
%\rmcmt{with the hope that this leads to ideas of how
%loops can be approximated in bounded model checking.} 
%
\begin{figure}[htbp]
\centering
\vspace*{-0.2cm}
  \scalebox{.70}{\import{figures/}{lattice.pspdftex}}
  \caption{Approximation of Trace Semantics used by ACDLP (right) and Abstract
  Interpretation (left)
  \label{fig:lattice}}
\end{figure}
%
\subsection{Concrete Flow Lattice}
%
Abstract interpretation technique for program analysis computes 
a fixed-point over static analysis equations~\cite{CC79,octagon}.  
The advantage of using approximations of static analysis equation 
over trace based abstraction is that the former only requires 
abstractions of memory states.  \\
%
We now define the \emph{concrete flow lattice} that describes these static 
analysis equations with respect to the set of control locations. 
%
\begin{definition} (Concrete Flow Lattice). A \emph{concrete flow lattice}
  corresponding to a CFG $G= (V, E, \mathcal{S}, \mathcal{T}, I)$  
  is a complete lattice $(\mathcal{F}_{G}, \sqsubseteq, \sqcap, \sqcup)$, 
  defined as follows. 
\end{definition}
  \[
    \mathcal{F}_{G} \mathrel{\hat=} V \rightarrow \powerset({\Omega}) \qquad \forall a,b \in
     \mathcal{F}_{G}.\; a \sqsubseteq b\; \text{if}\; \forall v \in V.\; a(v)
     \subseteq b(v) 
  \]

  \[
     a \sqcap b \mathrel{\hat=} v \mapsto a(v) \cap b(v) \qquad 
     a \sqcup b \mathrel{\hat=} v \mapsto a(v) \cup b(v)
  \]
%
%The static analysis equations corresponding to a control flow graph is shown in
%Figure~\ref{fig:se}.
%
\Omit{
Recall that a CFG statement $s$ corresponds to a state transition, that is, 
$\mathcal{ST}_s \subseteq \Omega \times \Omega$.  Hence, we define a
\emph{postcondition transformer} and \emph{precondition transformer} 
for $\mathcal{T}(u,v)=s$.
%
\[ 
  post_{u,v}(S) \mathrel{\hat=} \{\omega_k \mid \exists\; \omega_j \in S.\;
  (\omega_j,\omega_k) \in \mathcal{ST}_s \}
\]
%
\[ 
  pre_{u,v}(S) \mathrel{\hat=} \{\omega_j \mid \exists\; \omega_k \in S.\;
  (\omega_j,\omega_k) \in \mathcal{ST}_s \} 
\]
%
\begin{definition} (Concrete Flow Lattice Transformers). We define a strongest
  postcondition transformer and an existential precondition transformer of a 
  concrete flow lattice, $\mathcal{F}_{G}$ as follows.  
\end{definition}
  \[
    fpost \colon \mathcal{F}_{G} \rightarrow \mathcal{F}_{G} 
    \qquad 
    fpre \colon \mathcal{F}_{G} \rightarrow \mathcal{F}_{G} 
  \]
  \[
    fpost(v \mapsto a(v)) \mathrel{\hat=} \underset{(u,v) \in E}{\bigcup}
    post_{u,v} \circ a(u) 
    \qquad 
    fpre(v \mapsto a(v)) \mathrel{\hat=} \underset{(v,w) \in E}{\bigcup}
    pre_{v,w} \circ a(w) 
  \]
%
}
The concrete flow lattice is an over-approximation of the concrete powerset of
traces and there exists a galois connection between the two as shown below. 
\[
  (\powerset({\Pi}),\subseteq,\cup,\cap)
    \galois{\alpha_{F}}{\gamma_{F}}
    (\mathcal{F}_{G},\sqsubseteq,\sqcup,\sqcap)
\]
We define the abstraction and concretization functions between the concrete
trace domain $\powerset({\Pi})$ and concrete flow lattice $\mathcal{F}_{G}$.
\[
  \alpha_{F}(T) \mathrel{\hat=} \lambda v. \{\omega \mid \exists
  \pi \in T.\; (v,\omega) \in \pi \} 
\]
\[
  \gamma_{F}(a) \mathrel{\hat=} \{\pi \in \Pi \mid \forall (v,\omega) \in \pi.\;
  \omega \in a(v) \wedge \pi \in \Pi_{wf} \}
\]
%
It is easy to see that $\alpha_F \circ \gamma_{F}$ is deflationary while 
$\gamma_F \circ \alpha_F$ is inflationary, and the pair $(\alpha_F, \gamma_F)$
forms a galois connection. We prove this in proposition~\ref{ag}. 
%


For the purpose of static analysis, the concrete semantics of program's
control-flow graph is over-approximated using abstract interpretation
methodology~\cite{CC79}. The concrete domain of memory states 
$\powerset({\Omega})$ is abstracted to construct an abstract program
model which is then used for abstract program analysis.  Most static program
analyzers build an abstraction $A$ of the memory states $\Omega$, shown below. 
\[
  (\powerset({\Omega}),\subseteq)
    \galois{\alpha_{AF}}{\gamma_{AF}}
    (\mathcal{A},\sqsubseteq_{AF})
\]
%
\begin{definition} (Abstract Flow Lattice). An \emph{Abstract Flow Lattice}
  corresponding to a CFG $G= (V, E, \mathcal{S}, \mathcal{T}, I)$ and 
  an abstraction $\mathcal{A}$ over concrete memory states $\Omega$
  is a complete lattice $(\widehat{\mathcal{F}_{G}}, \sqsubseteq_{AF}, \sqcap_{AF},
  \sqcup_{AF})$, 
  which is defined as follows. 
\end{definition}
  \[
    \widehat{\mathcal{F}_{G}} \mathrel{\hat=} V \rightarrow \mathcal{A} \qquad \forall a,b \in
     \widehat{\mathcal{F}_{G}}.\; a \sqsubseteq b\; \text{if}\; \forall v \in V.\; a(v)
     \sqsubseteq_{AF} b(v) 
  \]
%
  \[
    a \sqcap_{AF} b \mathrel{\hat=} v \mapsto a(v) \sqcap_{AF} b(v) \qquad 
     a \sqcup_{AF} b \mathrel{\hat=} v \mapsto a(v) \sqcup_{AF} b(v)
  \]
%
\subsection{Complementation in Abstract Flow Lattice}~\label{complement-fg}
%
A meet irreducible of abstract flow lattice $\widehat{\mathcal{F}_{G}}$ is defined as
follows.
\[
  \sqcap_{irrd}(\widehat{\mathcal{F}_{G}}) \mathrel{\hat=} 
  \left\{\begin{array}{l@{\quad}l}
    n \mapsto a & \text{if}\; n \in V \wedge a \in \sqcap_{irrd}(\mathcal{A}) \\
    n \mapsto \top & \text{otherwise} \\
  \end{array}\right.
\]  
%
Here, a node $n \in V$ is mapped to an abstract memory state $a \in \mathcal{A}$, 
where $a$ is a meet irreducible of $A$.  Let $n \mapsto \bar{a}$ be a precise 
complement of $n \mapsto a$, where $\bar{a}$ is a precise complement of $a$.    
However, the two sets are not equal, that is, 
$\gamma_{AF}(n \mapsto a) \neq \neg \gamma_{AF}(n \mapsto \bar{a})$.
The element $\neg \gamma_{AF}(n \mapsto \bar{a})$ gives the set of traces 
where some occurrences of $n$ are contained in the concretization of
$\gamma_{AF}(a)$.  Whereas, the element $\gamma_{AF}(n \mapsto a)$ gives 
the set of traces where all occurrences of $n$ are contained in the 
concretization of $\gamma_{AF}(a)$.  This is explained below.
%
\[
  \gamma_{AF}(n \mapsto a) \mathrel{\hat=} \{\pi \in \Pi \mid \forall (n,\omega) \in
  \pi. \omega \in \gamma_{AF}(a) \}
\]
%
\[
  \neg \gamma_{AF}(n \mapsto \bar{a}) \mathrel{\hat=} \neg \{\pi \in \Pi \mid \forall (n,\omega) \in
  \pi. \omega \not\in \gamma_{AF}(a) \}
\]
%
\[
  \neg \gamma_{AF}(n \mapsto \bar{a}) \mathrel{\hat=} \{\pi \in \Pi \mid \exists (n,\omega) \in \pi. \omega \in \gamma_{AF}(a) \}
\]
%
%An important requirement of ACDLP is that the meet irreducibles in the domain 
%must be precisely complementable with respect to the unsafe trace transformer.  
The abstract control flow lattice $\widehat{\mathcal{F}_{G}}$ does not always have 
complementable meet irreducibles.  So, we define a domain over logical encodings 
of the CFG, the elements of which admits precise complementation property.
%
\subsection{Lattice Structure over SSA}~\label{state-transformer}
%
We now define a lattice over SSA structures that admits precise complementation 
property which is necessary for lifting CDCL to program analysis. Let
$Var_{ssa}$ denote the set of SSA variables in the SSA form and $Value$ denote the
set of values that these SSA variables can take.  Note that the set $Var_{ssa}$ is
obtained from the CFG $G= (V, E, \mathcal{S}, \mathcal{T}, I)$ through program
transformation $\mathcal{T}_2$. 
%
\begin{definition} (Static Assignment Lattice). A \emph{Static Assignment Lattice}
  corresponding to a CFG $G= (V, E, \mathcal{S}, \mathcal{T}, I)$ is a complete 
  lattice $(\mathcal{SA}_{G}, \subseteq_{SA}, \cup_{SA}, \cap_{SA})$, defined 
  as follows.
\end{definition}
%
  \[
    \mathcal{SA}_{G} \mathrel{\hat=} \powerset(Var_{ssa} \rightarrow Value)
    \quad \forall A,B \in
    \mathcal{SA}_{G}.\; A \subseteq_{SA} B\; \text{if}\; \forall a \in A.\; 
    a \in B 
  \]
\rmcmt{  
  \[
    A \cap_{SA} B \mathrel{\hat=} \lambda\; x. ( \{f(x) \cap g(x) \mid 
    f \in A, g \in B \}) 
    \qquad 
    A \cup_{SA} B \mathrel{\hat=} \lambda\; x. ( \{f(x) \cup g(x) \mid 
    f \in A, g \in B \}) 
  \]
}
%
%%%%%%%%%%%%%%%% Static Single Assignment Lattice %%%%%%%%%%%%%%%%%%
%
Figure~\ref{fig:concrete} gives an example of concrete static assignment lattice
over SSA variables $p$ of type boolean and $x$, $y$ of numerical types. Note
that the lattice $\mathcal{SA}_G$ is a set of concrete environments over SSA
variables.  We will call these \emph{concrete SSA environments}.  Following the
result of Briggs et. al.~\cite{Briggs:1998}, the elements of $\mathcal{SA}_G$
can be mapped back to a trace of the original program.  The elements marked in bold 
in Figure~\ref{fig:concrete} corresponds to concrete assignments to SSA variables 
that maps to a concrete program trace. 
%
\begin{figure}[htbp]
\centering
\vspace*{-0.2cm}
  \scalebox{.90}{\import{figures/}{concrete_env.pspdftex}}
\caption{Concrete Static Assignment Lattice over a boolean SSA variable $p$ and two
  numerical SSA variables $x$ and $y$ that takes values in Integer domain \label{fig:concrete}}
\end{figure}
%


An SSA statement (or program transformer) $s \in \constraints$ transforms the 
memory state of a program and is 
therefore associated with a strongest postcondition or an existential 
precondition transformer. Let us assume that each SSA statement is also 
associated with a transition relation, $\mathcal{ST}_{s}^{\Omega}$. 
%
\begin{definition} (Strongest Postcondition and Existential Precondition).
  \[
     post_{s}(A) \mathrel{\hat=} \{\omega' \mid \exists \omega \in
     \Omega. \omega \in A \wedge (\omega,\omega') \in \mathcal{ST}_{s}^{\Omega} \} 
  \] 
  \[ 
     pre_{s}(A) \mathrel{\hat=} \{\omega \mid \exists \omega' \in
     \Omega. \omega' \in A \wedge (\omega,\omega') \in \mathcal{ST}_{s}^{\Omega}\} 
  \]
\end{definition}
%
\begin{definition} (Weakest Precondition and Universal Postcondition). 
  \[
    \widehat{pre_s}(A) \mathrel{\hat=} \{\omega \mid \forall \omega' \in
    \Omega.\; \omega' \in A \vee (\omega,\omega') \not\in \mathcal{ST}_{s}^{\Omega} \}.
  \]
  \[ 
     \widehat{post_s}(A) \mathrel{\hat=} \{\omega' \mid \forall \omega \in
     \Omega.\; \omega \in A \vee (\omega,\omega') \not\in
     \mathcal{ST}_{s}^{\Omega} \}.
  \]
\end{definition}
%
\Omit{
\begin{definition} (Static Assignment Transformers). 
   $
     post_{\mathcal{SA}}, pre_{\mathcal{SA}} : \mathcal{SA}_G
      \rightarrow \mathcal{SA}_{G} 
   $
   \[
     post_{\mathcal{SA}}(a) \mathrel{\hat=} \bigcup_{s \in \rmcmt{\constraints}}
     post_{s}(a) 
     \quad
     pre_{\mathcal{SA}}(a) \mathrel{\hat=} \bigcup_{s \in \rmcmt{\constraints}}
     pre_{s}(a) 
   \]
\end{definition}
}
%
\begin{figure}[t]
\centering
\vspace*{-0.2cm}
\scalebox{.70}{\import{figures/}{state_transformer.pspdftex}}
  \caption{Strongest postcondition $post_s$ (Blue), Existential precondition
  $pre_s$ (Red), Universal postcondition $\widehat{post_s}$ (Pink), 
  Weakest precondition $\widehat{pre_s}$ (Green)}
\label{state-transformer}
\end{figure}
%
Figure~\ref{state-transformer} graphically shows the result of various
state transformers for a given memory state $A$. A strongest postcondition 
maps a set of states $A$ to the set of all successor states that can be reached
in one step, whereas an existential precondition maps a set of states $A$ to
states the program \emph{may} have before executing $s$. A weakest precondition maps a set of states $A$ to the set
of states which can \emph{only} reach elements of $A$. Whereas, an universal
postcondition maps $A$ to set of states the program \emph{must} reach after
executing $s$.
%

Recall that the trace transformers $tpost$ and $tpre$ are defined over
$\powerset(\Pi)$. The transformers, $post_{s}$ and $pre_{s}$, soundly 
approximate $tpost$ and $tpre$ respectively.  \rmcmt{how ? --relate to syntactic
transformation T2}
%
The global static assignment transformers for the lattice $\mathcal{SA}_G$ over 
a set of SSA constraints $\constraints$ can be easily derived from state
transformers, $post_s, pre_s$.  This is defined next.
%
\begin{definition} (Global Static Assignment Transformers). 
  \[ 
  post_{\constraints}, pre_{\constraints}, 
  \widehat{post_{\constraints}}, \widehat{pre_{\constraints}}, 
  \colon \mathcal{SA}_G \rightarrow \mathcal{SA}_G
  \]
    
  \[
  post_{\constraints}(a) \mathrel{\hat=} \underset{\sigma \in
  \constraints}{\bigcap} post_\sigma \circ a
  \qquad  
  pre_{\constraints}(a) \mathrel{\hat=} \underset{\sigma \in
  \constraints}{\bigcap} pre_\sigma \circ a
  \]
  
  \[
  \widehat{post_{\constraints}}(a) \mathrel{\hat=} \underset{\sigma \in
  \constraints}{\bigcap} \widehat{post_\sigma} \circ a
  \qquad   
  \widehat{pre_{\constraints}}(a) \mathrel{\hat=} \underset{\sigma \in
  \constraints}{\bigcap} \widehat{pre_\sigma} \circ a
  \]
\end{definition}
%
\rmcmt{The transformers, $post_{\constraints}$ and $pre_{\constraints}$, soundly 
approximate $tpost$ and $tpre$ respectively. (say how?) }
%
We now define an abstraction of static assignment lattice which we 
call \emph{abstract static assignment} domain. 
The concrete static assignment domain is a complete lattice of concrete 
SSA environments.  The abstractions of concrete static assignment domain fall
into two categories; an abstraction that preserves the relationship between 
SSA variables and the other which does not preserve any relation. 
%

First, we give an abstraction of static assignment lattice that does not
preserve relationship between SSA variables.  Then, we present a generic 
abstraction using a \emph{template-based abstract domain} that can express 
relational as well as non-relational abstractions of SSA environments.  
%An advantage of a template-based domain is that it can be instantiated with 
%arbitrary templates over relational or non-relational domains.  
This provides the flexibility to instantiate CDCL over arbitrary abstract domains for 
program analysis.  
%
\para{Abstract Static Assignment Lattice Over Non-relational Domains}
%
An abstraction of $\mathcal{SA}_G$ over non-relational domain requires each SSA
variables in $SSA_{var}$ to be abstracted independently of each other.  
The concrete domain of $\powerset(Var_{ssa} \rightarrow Value)$ is abstracted 
through a function which maps each SSA variable to an abstract value.  We 
assume that there is a galois connection $(\alpha_v,\gamma_v)$ such that
  $(Value, \subseteq)
   \galois{\alpha_{v}}{\gamma_{v}}
   (\widehat{Value}, \sqsubseteq)$.   
%
Then, an abstraction of $\mathcal{SA}_G$ over non-relational domain, denoted by
$\mathcal{SA}_G^\dagger = \powerset(Var_{ssa} \rightarrow \widehat{Value})$, 
can be constructed using the following galois connection 
$(\alpha^\dagger,\gamma^\dagger)$.
\[
   (\mathcal{SA}_{G},\subseteq_{SA})
   \galois{\alpha^\dagger}{\gamma^\dagger}
   ({\mathcal{SA}_{G}}^\dagger,\sqsubseteq^\dagger)
\]
\[
  \alpha^\dagger(c) \mathrel{\hat=} \lambda x.\alpha_v(\bigcup\{\tau(x) \mid \tau
  \in c\})
\]
\[
  \gamma^\dagger(a) \mathrel{\hat=} \{\lambda x.v' \mid v' \in (\gamma_v \circ
  a)(x)\}
\]
%
\begin{example}
%
An example of $({\mathcal{SA}_{G}}^\dagger,\sqsubseteq^\dagger)$ over an
Interval domain $Itv$ maps each SSA variable on an Interval, where 
$(\alpha_v,\gamma_v) = (\alpha_{Itv}, \gamma_{Itv})$.  Consider an SSA 
$c=\{x_1:=2, x_2 \geq 5\}$, then $\alpha^\dagger(c) = \{x_1=[2,2],x_2=[5,\infty]\}$
\end{example}
%
\para{Abstract Static Assignment Lattice Over Template-based Domains}
%
A folk wisdom in abstract interpretation community is that an analysis using 
non-relational domains are effective but imprecise. However, for practical
purposes, it is imperative to keep track of relations between program variables.  
A relational domain can express relationship between variables, though with varying
expressivity, depending on a weak or strong relational domain.  We now construct
an abstraction of $\mathcal{SA}_G$ using a template-based domain, denoted by 
$\widehat{\mathcal{SA}_{G}}$, which can be instantiated with arbitrary templates 
over relational or non-relational domains. Note that from this point onwards, we 
will operate on $\widehat{\mathcal{SA}_{G}}$ instead of ${\mathcal{SA}_{G}}^\dagger$. 
%
\begin{definition}~\label{assl} (Abstract Static Assignment Lattice). An \emph{Abstract Static 
Assignment Lattice} corresponding to a CFG $G= (V, E, \mathcal{S}, \mathcal{T}, I)$ 
and a set $X$ of values of vector $\vec{x}$ of variables in the 
corresponding SSA, is a complete lattice 
  $(\widehat{\mathcal{SA}_{G}}, \sqsubseteq_{SA}, \sqcap_{SA}, \sqcup_{SA})$, which 
is defined as follows. 
\[
  \widehat{\mathcal{SA}_{G}} \mathrel{\hat=} C\vec{x} \leq \vec{d},\; \text{where}\; 
  \vec{d}\;\text{is a constant vector and C is a coefficient matrix} 
\]
\end{definition}
%
The abstraction and concretization functions $(\alpha_{SA}, \gamma_{SA})$ form 
a galois connection.
%
\[
  (\mathcal{SA}_{G},\subseteq_{SA})
   \galois{\alpha_{SA}}{\gamma_{SA}}
   (\widehat{\mathcal{SA}_{G}},\sqsubseteq_{SA})
\]
%
\[
  \alpha_{SA}(\numconcval) = \min \{\vec{\numabsval}\mid
  \mat{C}\vec{\numvar}\leq\vec{\numabsval}, \vec{\numvar}\in
  \numconcval\},\;\text{where}\; \min\; \text{is applied component-wise}.  
\]  
%  
\[  
   \gamma_{SA}(\vec{\numabsval}) \mathrel{\hat=} \{\vec{\numvar}\mid
   \mat{C}\vec{\numvar}\leq\vec{\numabsval}, \vec{\numvar} \in X\} 
   \qquad \gamma_{SA}(\bot)=\emptyset
\]
%



%
Figure~\ref{fig:abstract} shows the abstract static assignment lattice over an
Interval abstract domain.  The elements marked in bold corresponds to concrete 
assignments to SSA variables that maps to a concrete program trace. 
%
\begin{figure}[htbp]
\centering
\vspace*{-0.2cm}
  \scalebox{.90}{\import{figures/}{abstract_env.pspdftex}}
\caption{Abstract Static Assignment Lattice over a boolean SSA variable $p$ and two
  numerical SSA variables $x$ and $y$ that takes values in Interval domain
  \label{fig:abstract}}
\end{figure}
%
\begin{example}
For example, consider the SSA elements $\{x_1\geq 0, x_1-z\leq 30\}$, then
abstract domain value is $\vec{\numabsval}=\vecv{0}{30}$,
with $\mat{C}=\qmat{-1}{0}{1}{-1}$ and $\vec{\numvar}=\vecv{x_1}{z}$. \\ 
\end{example}
%
\begin{definition} (Meet Irreducibles of $\widehat{\mathcal{SA}_G})$
%

A meet irreducible of \emph{abstract static assignment} domain
$\widehat{\mathcal{SA}_{G}}$ where $\mat{C_i}$ is $i$-th row vector  
of a matrix of size $N \times M$ and $\vec{\numabsval}$ is a vector 
  of size $N$, is defined below.  \rmcmt{Here, $MAX$ is the largest 
  value of $\vec{d}$ determined by the matrix $\mat{C}$}.
%   
\rmcmt{
  \[
     \meet_{irrd}(\widehat{\mathcal{SA}_{G}}) \mathrel{\hat=} 
     \{\vec{d} \mid \exists_{=1}\;d_i \in \vec{\numabsval}.\;(d_i \neq MAX) \wedge
     C_{i}\vec{x} \leq d_i\} \;\text{where}\;(i \leq N)
  \]
}  
Informally, a meet irreducible of $\widehat{\mathcal{SA}_{G}}$ is the abstract
value $\vec{d}$ such that there exists exactly one element of $\vec{d}$ that is
not $MAX$.
%
\end{definition}
%
%\rmcmt{Show how MAX is computed}.
We now discuss how $MAX$ is computed.  For as SSA element 
$\hmat{1}{-1} \vecv{x}{y} \leq d$ where $x$ and $y$ are 
32-bit signed integers (marked as s32 in figure~\ref{fig:max}), 
the total number of bits to represent $d$ is 34.  Thus, the 
value of $MAX$ is the largest value representable in 34 bits.
%
\begin{figure}[htbp]
\centering
\vspace*{-0.2cm}
  \scalebox{.90}{\import{figures/}{max.pspdftex}}
\caption{Computing $MAX$ from bit-width of $d$
  \label{fig:max}}
\end{figure}
%


Meet irreducibles of $\widehat{\mathcal{SA}_G}$ over an \emph{Interval} domain 
with template $[l,u]$ such that $\vecv{-1}{1}(\vec{x})\leq \vecv{l}{u}$ is given
by $\{\vecv{-l}{MAX}, \vecv{MAX}{u}\}$.
%
\begin{example}
  An example of meet irreducible in $\widehat{\mathcal{SA}_{G}}$ over 
  $\vec{\numvar} = \vecv{a}{b}$ that takes values in \emph{Interval} domain 
  is given by, $\meet_{irrd}(\{a:[5,7]\}) = \{\vecv{-5}{MAX}, \vecv{7}{MAX}\}$ 
  with $\mat{C}=\qmat{1}{0}{0}{0}$. Here, the abstract values of variable 
  $b$ is equal to $MAX$. \\
\end{example}  
%
\begin{example} \rmcmt{Octagon}
  An example of meet irreducible in $\widehat{\mathcal{SA}_{G}}$ over 
  $\vec{\numvar} = \vecv{a}{b}$ that takes values in \emph{Octagon} domain 
  is given by, 
  \[\meet_{irrd}(\{a+b \leq 5\}) =
  \vecvvv{MAX}{MAX}{MAX}{MAX}{5}{MAX}{MAX}{MAX}\]
    
  \[\text{with}\;\; \mat{C}=
  \begin{bmatrix}
    0 & 1 \\
    1 & 0 \\
    0 & -1 \\
    -1 & 0 \\
    1 & 1 \\
    1 & -1 \\
    -1 & 1 \\
    -1 & -1
  \end{bmatrix}
  \]
  %\mat{c}=\qqqmat{0}{1}{1}{0}{0}{-1}{-1}{0}{1}{1}{1}{-1}{-1}{1}{-1}{-1}\] 
   Here, the abstract values of $a$, $b$, $-a$, $-b$, $(a-b)$, $(-a+b)$, $(-a-b)$ are
  equal to $MAX$. \\
\end{example}  
%


An advantage of abstract static assignment domain $\widehat{\mathcal{SA}_{G}}$ 
is that it can be instantiated over arbitrary relational or non-relational 
numerical abstract domains.  This gives the flexibility to instantiate CDCL for 
safety verification over arbitrary abstract domains. 
%
We will use the template-based abstraction for the rest of the paper. 
%%%%%%%%%%%%%%%%%%%%%%%%% PROOF %%%%%%%%%%%%%%%%%%%%%%%%
\begin{proposition}~\label{ag}
  $
  (\powerset({\Pi}),\subseteq)
    \galois{\alpha_{T}}{\gamma_{T}}
  (\mathcal{SA}_{G},\subseteq_{SA})
   \galois{\alpha_{SA}}{\gamma_{SA}}
   (\widehat{\mathcal{SA}_{G}},\sqsubseteq_{SA})
  $ 
\end{proposition}
%
\begin{proof}
  The lattice $\mathcal{SA}_G$ is obtained from the concrete lattice of 
  traces $\powerset(\Pi)$ via syntactic translation steps $\mathcal{T}_1$ and
  $\mathcal{T}_2$, as shown in figure~\ref{fig:semantic}.  Both translation
  steps are exact, that is, $\powerset(\Pi)$ and $\mathcal{SA}_G$ form an 
  exact abstraction via the galois 
  connection $(\alpha_T, \gamma_T)$. This implies, $\alpha_T(\pi) = \tau$ where
  $\tau \in \mathcal{SA}_G$.  Also, $\gamma_T(\tau) = \pi \in \powerset(\Pi)$,
  since an SSA can be exactly mapped back to the program trace following the
  syntactic translation steps proposed in~\cite{Briggs:1998}.   
  Thus, there is no loss or gain of information between elements of 
  $\mathcal{SA}_{G}$ and $\powerset(\Pi)$.


  Furthermore, the lattice $\widehat{\mathcal{SA}_{G}}$ is obtained via 
  two-step galois connection $(\alpha_{T} \circ \alpha_{SA})$ from
  the concrete lattice of traces $\powerset(\Pi)$. The abstraction $\alpha_{T}$
  is exact, whereas, $\alpha_{SA}$ is an over-approxation by
  definition~\ref{assl}.
  By \rmcmt{transitivity, (find reference)}, the lattice 
  $\widehat{\mathcal{SA}_{G}}$ over-approximates $\powerset(\Pi)$.
\end{proof}
%
\Omit{
\begin{definition}
A trace $\pi$ in abstract static assignment domain $\widehat{\mathcal{SA}_{G}}$ 
is a sequence of abstract states of the form $\{C_{ij}\vec{x} \leq \vec{d_{i}}$,
starting with an abstract element $\top$. We use $\Pi$ to denote set of traces. 
\end{definition}
}
%
The abstract transformers for the lattice $\widehat{\mathcal{SA}_{G}}$ 
transforms a memory state of a program and therefore associated with state 
transformers.  For every SSA statement $s \in \constraints$, let
$apost_{s}$ and $apre{s}$ be sound over-approximations of $post_{s}$ and
$pre_{s}$ in the lattice $\widehat{\mathcal{SA}_G} \rightarrow 
\widehat{\mathcal{SA}_G}$, respectively. 
%
Th global abstract static assignment transformers for the 
lattice $\widehat{\mathcal{SA}_G}$ over a set of SSA constraints 
$\constraints$ are obtained from abstract state transformers, 
$apost_s, apre_s$. This is defined next.
%
\begin{definition} (Global Overapproximate Abstract Static Assignment Transformers). 
  \[ 
     apost_{\constraints}, apre_{\constraints} : \widehat{\mathcal{SA}_G}
     \rightarrow \widehat{\mathcal{SA}_{G}} 
   \]
   \[
     apost_{\constraints}(a) \mathrel{\hat=} 
     \underset{\sigma \in \constraints}{\bigsqcap} apost_{\sigma} \circ a 
   \]
  \[
     apre_{\constraints}(a) \mathrel{\hat=} 
     \underset{\sigma \in \constraints}{\bigsqcap} apre_{\sigma} \circ a 
   \]
\end{definition}
%
The transformers $apost_{\constraints}$ and $apre_{\constraints}$ soundly overapproximates 
$post_{\constraints}$ and $pre_{\constraints}$, respectively. 
%
\subsection{Complementation in Abstract Static Assignment Lattice}~\label{complement}
%
Recall that every meet irreducibles of a partial assignments domain admits a
precise complement.  This property of the partial assignments domain enables a
CDCL solver to learn a conflict clause that is obtained by complementing a 
conflict reason.  In section~\ref{complement-fg}, we showed that the elements 
of abstract flow lattice $\widehat{\mathcal{F}_G}$ does not always have precise 
complements.  In this section, we show that the meet irreducibles of 
$\widehat{\mathcal{SA}_G}$ have precise complements. 
%
\begin{definition}~\label{meet-decomp}
A \emph{meet decomposition} $\decomp(\absval)$ of an abstract 
element $\absval \in \widehat{\mathcal{SA}_{G}}$ is a set of meet 
irreducibles $M \subseteq \widehat{\mathcal{SA}_{G}}$ such that 
$\absval=\bigsqcap_{m\in M} m$.
\end{definition}
%
\begin{example} The meet decomposition of the interval domain element
$decomp(2\leq x\leq 4 \wedge 3\leq y\leq 5)$ is
the set $\{x\geq 2, x\leq 4, y\geq 3, y\leq 5\}$.
\end{example}
%
Recall that a precise complement of an element $a \in \widehat{\mathcal{SA}_G}$ 
is an element $\bar{a} \in \widehat{\mathcal{SA}_G}$ such that $\neg \gamma_{SA}(\bar{a}) =
\gamma_{SA}(a)$.  The meet irreducibles of Abstract Static Assignment lattice 
$\widehat{\mathcal{SA}_G}$ admits precise complements. This is explained below. 
%
\[
   \gamma_{SA}(\vec{\numabsval}) \mathrel{\hat=} \{\vec{\numvar}\mid
   \mat{C}\vec{\numvar}\leq\vec{\numabsval}\} 
\]
\[
   \gamma_{SA}(\bar{\vec{\numabsval}}) \mathrel{\hat=} \{\vec{\numvar}\mid
   \mat{C}\vec{\numvar} > \vec{\numabsval}\} 
\]
\[
   \neg \gamma_{SA}(\bar{\vec{\numabsval}}) \mathrel{\hat=} \{\vec{\numvar}\mid
   \mat{C}\vec{\numvar} \leq \vec{\numabsval}\} 
\]
%
\begin{example}
Let an abstract SSA environment be $a = \{p:t,x\geq0,y\geq0\}$, 
then $\bar{a} = \{p:f,x<0,y<0\}$ and $\neg \bar{a} = \{p:t,x\geq0,y\geq0\}$.
\end{example}
%
\begin{example}
The precise complement of a meet irreducible $(x \leq 2)$ in
the interval domain over integers is $(x \geq 3)$, or the precise complement
of the meet irreducible $(x+y \leq 1)$ in the octagon domain over integers
is $(x+y \geq 2)$.  
\end{example}
%
\begin{table}
\scriptsize
\begin{center}
{
\begin{tabular}{l|l|l|l|l}
\hline
  Domain & Concrete & Abstract & Meet & Complemented Meet \\ 
  Name  & Domain & Elements & Irreducible & Irreducible \\ \hline
Interval & $\powerset(Var \rightarrow \mathbb{Z}) \cup \bot$  & $x = [l,u]$ & $x\leq N$ & $x > N$ \\ \hline
Octagon &  $\powerset(Var \times Var \rightarrow (\mathbb{Z} \cup
\{\infty\})) \cup \bot$ & $(\pm x_i \pm x_j \leq d)$ & $(x+y \leq N)$ & $(-x-y < N)$ \\ \hline
  Equality & $\powerset(Var \rightarrow \mathbb{Z}) \cup \bot$ & $(x=y)$ &
  $(x=y)$ & $(x \neq y)$ \\ \hline 
 Zones &  $\powerset(Var \times Var \rightarrow (\mathbb{Z} \cup
\{\infty\})) \cup \bot$ & $(x_i - x_j \leq d)$ & $(x+y < N)$ & $(-x-y \leq N)$ \\ \hline
\end{tabular}
}
\end{center}
\label{fig:complement}
\end{table}
%
Table~\ref{fig:complement} shows that most abstract domains admit precise
complements.  We now show that the meet irreducibles of 
$\widehat{\mathcal{SA}_{G}}$ have precise complements when instantiated over
arbitrary abstract domains. 
%
\begin{proposition} 
  $\gamma_{SA}(C_{i}\vec{x} \leq \vec{d_{i}}) = 
  (\neg \gamma_{SA}(C_{i}\vec{x} > \vec{d_{i}}))$
\end{proposition}
%
\begin{proof}
  Assuming that $\vec{d}$ is chosen from a non-relational domain such as 
  Interval domain $Itvs$, the meet 
  irreducibles are of the form $\{\vec{x} \diamond \vec{d}\}$, where
  $\diamond =
  \{<,>,\leq,\geq\}$. It is easy to see that the meet irreducibles of $Itvs$ are easily
  complementable.  Similarly, assuming that $\vec{d}$ is chosen from a
  relational domain such as Octagon 
  domain $Octs$, the meet irreducibles are of the form $\{\mat{C}\vec{x}
  \diamond \vec{d}\}$, which also admits precise complements.  However, for
  element $a \in \widehat{\mathcal{SA}_{G}}$, that does not admit precise 
  complementation, it can be decomposed into half spaces, 
  $decomp(a)=\{m_1, m_2, \ldots m_k\}$, such that each $m_i$ admits precise
  complements. 
\end{proof}
%
\Omit{
\para{Elements in $\mathcal{SA}_G$ gives a Program Trace}
%
We use the result of~\cite{Briggs:1998} in Lemma~\ref{ssa-model} that shows 
a model of $\varphi$ corresponds to a concrete counterexample trace in the 
original program. 
%
\begin{lemma}
  \[ \mathcal{T}_2(\gamma(a)) = \{\pi \mid \pi \in \Pi \wedge a \in
  \widehat{\mathcal{SA}_G} \} \]
\end{lemma}
%
\begin{proof}

\end{proof}
}
%

%%%%%%%%%%%%%%%%%%%%%%%%%%%%%%%%%%%%%%%%%%%%%%%%%%%%%%%%%%%%%%%%%%%

%%%%%%%%%%%%%%%%%%%%%%%%%%%%%%%%%%%%%%%%%%%%%%%%%%%%%%%%%%%%%%%%%%%
\section{An Example of Abstract Interpretation}
%
\begin{figure}[t]
\centering
\begin{tabular}{c|c}
\hline
Control-Flow Graph & Static Analysis Equation \\
\hline
\begin{minipage}{5.0cm}
\scalebox{.65}{\import{figures/}{ai-example.pspdftex}}
\end{minipage}
&
\begin{minipage}{7cm}
$\begin{array}{l}
     S_{n1} = \top, \\
     S_{n2} = post_{x:=2}(S_{n1}), \\
     S_{n3} = post_{x:=x*x}(S_{n2}) \cup post_{x:=x+2}(S_{n4}), \\ 
     S_{n4} = post_{x < 10}(S_{n3}), \\
     S_{n5} = post_{x\geq 10}(S_{n3}) \\
     S_{Error} = post_{x\neq10}(S_{n5}) \\
     S_{Safe} = post_{x=10}(S_{n5}) \\
\end{array}$
\end{minipage}
\\
\hline
\end{tabular}
\caption{\label{fig:se} A CFG and its static analysis equation}
\end{figure}
%
We present an example of a classical abstract interpretation of program.  
Figure~\ref{fig:se} shows a CFG and its corresponsing static analysis equations.  
A static analysis equation encodes the data-flow between individual control-flow 
nodes in the CFG and is given by a set valued variable $S_n$ for each location $n$ 
in the CFG.  Recall that each statement $s$ in the program is associated with a
postcondition transformer, $post_s(S_n)$, that computes the successor state of a 
statement $s$ starting from $S_n$ that can be reached in one step.  Static analysis 
using abstract interpretation usually computes a fixed point over the static analysis 
equation obtained from the Control Flow Graph (CFG) representation of a program~\cite{CC79}.  
%

Let us consider the equations in Figure~\ref{fig:se} that models the loop, 
$S_{n3} = post_{x:=x*x}(S_{n2}) \cup post_{x:=x+2}(S_{n4})$, 
$S_{n4} = post_{z \leq 10}(S_{n3})$.  These equations can be written as 
a function, $F(X)= \{4\} \cup \{x+2 \mid x \in X, x \leq 10\}$, where 
$post_{x:=x*x}(S_{n2}) = {4}$.  Assuming variable $x$ is an integer, the 
lattice of integers with a subset relation is $(\powerset(\mathbb{Z}),\subseteq)$. 
%

Standard means to infer loop invariant is to compute fixed points of function 
over this lattice structure~\cite{CC79,octagon}.  The fixed point of the above 
function gives the set $X$ that satisfies $F(X)=X$.  The loop invariant is given 
by $x\geq 4 \wedge x\leq 10 \wedge x\equiv 0\;(\bmod 2)$.  The set of values of 
$x$ satisfying the loop invariant is the least fixed point. 
%

Abstract interpretation of program computes a fixed point of an abstract
function (called abstract transformer), over an abstract lattice.  Assuming an
interval lattice $Intv$ which maps a set of integer values of a variable to 
its smallest interval that contains it, the function $F$ above is abstracted 
over an interval lattice as shown below.
%
\[ F^{\sharp}([a,b]) = [4,4] \sqcup ([a,b] \sqcap [-\infty,10]) +_{Intv} [2,2] \]
%
The initial values of $x$ is $[4,4]$ which is obtained by interval analyis of
the static analysis equation, $S_{n3} = post_{x:=x*x}(S_{n2})$. The 
function $F^{\sharp}$ computes an interval at each iteration where the interval 
below 10 is incremented by 2 and $(+_{Intv})$ denotes an addition operation in
the interval lattice. 
%
Figure~\ref{fig:fixpoint} shows the fixed point computation of the loop in 
Figure~\ref{fig:se}, over a lattice of intervals.  Each column denotes an
iteration of the fixed point computation which associates an interval with each
location in the program.  The initial value of $x$ is $\top$ at $n1$, while the
locations $n2$, $n3$, $n4$, $n5$, $Error$, $Safe$ are considered unreachable. 
Each iteration of the loop computes a bound on the variable $x$. 
The interval $x\colon[4,10]$ at the loop head $n3$ in the last iteration 
is the loop invariant. 
In practise, the total number of iterations may be too large to reach the fixed 
point. Hence, techniques like widening and narrowing are used to accelerate
convergence. 
%
\begin{figure}[t]
\centering
\begin{tabular}{ccccc}
\hline
  Control Location & Iteration 1 & Iteration 2 & Iteration 3 & Iteration 4\\
\hline
  $n1$ & $x\colon\top$ & $x\colon\top$ & $x\colon\top$  & $x\colon\top$ \\ 
  $n2$ & $x\colon[2,2]$ & $x\colon[2,2]$ & $x\colon[2,2]$ & $x\colon[2,2]$ \\
  $n3$ & $x\colon[4,4]$ & $x\colon[4,6]$ & $x\colon[4,8]$ & $x\colon[4,10]$ \\
  $n4$ & $x\colon[6,6]$ & $x\colon[6,8]$ & $x\colon[6,10]$ & $x\colon[6,10]$ \\
  $n5$ & $x\colon\bot$ & $x\colon\bot$ & $x\colon\bot$ & $x\colon[10,10]$ \\
  $Error$ &$x\colon\bot$ & $x\colon\bot$ & $x\colon\bot$ & $x\colon\bot$ \\
  $Safe$ &$x\colon\bot$ & $x\colon\bot$ & $x\colon\bot$ & $x\colon[10,10]$ \\
\hline
\end{tabular}
  \caption{\label{fig:fixpoint} Fixed point computation of program in
  Figure~\ref{fig:se}}
\end{figure}
%

%%%%%%%%%%%%%%%%%%%%%%%%%%%%%%%%%%%%%%%%%%%%%%%%%%%%%%%%%%%%%%%%%%%



%===============================================================================

%===============================================================================
\section{Abstract Conflict Driven Learning for Programs}~\label{acdlp}
%
\todo{Be CAREFUL about transformer defined in concrete SA lattice, it is 
better to declare that these transformers simulate transformers in trace lattice}
Figure~\ref{acdlp-top} present our framework called \emph{Abstract Conflict 
Driven Learning for Programs} that uses abstract model search and abstract 
conflict analysis procedures for safety verification of programs.  The model
search procedure operates on an over-approximate abstract domain using sound deduction 
transformers such as strongest post-condition or existential pre-condition 
transformers (see section~\ref{modelsearch}).  When the deduction transformers cannot 
infer any further information and is not $\gamma$-complete (see 
section~\ref{modelsearch}), then a decision (see section~\ref{decision}) is made  
that refines the current abstract element.  The decision and deduction step continues 
until either a satisfying assignment is obtained (corresponding deduction transformer 
is $\gamma$-complete) or a conflict is encountered. In the former case, ACDLP terminates 
with a counterexample trace and the program is \emph{unsafe}. Recall that a counterexample 
trace is trace that reaches the error location $\err$. 

However, if a conflict is encountered, then it implies that the corresponding 
program trace is either not valid or safe.  ACDLP then moves to the conflict 
analysis phase (see section~\ref{conflict-analysis}) where it learns the 
reason for the conflict.  Recall that a SAT solver uses conflict resolution to derive 
the reason for conflict (see section~\ref{sat-learning}).  For efficiency 
reasons, SAT solver picks only one conflict reason.  Conflict analysis 
in ACDLP operates on an under-approximate domain using sound abductive transformers 
such as universal post-condition or weakest pre-condition transformers 
(see section~\ref{conflict-analysis}).  There can be multiple incomparable 
reasons for a conflict, but ACDLP heuristically picks one reason and generalizes it. 
Intuitively, this means that a partial safety proof for $\mathcal{S}$ is obtained 
by generalizing $\mathcal{S}$ to a set of safe \rmcmt{or invalid} traces 
$\mathcal{S'}$ such that the generalized conflict reason still preserves the 
reachability of the error location $\err$.  A generalized conflict reason basically 
contains the common prefix of the set of safe traces $\mathcal{S'}$. A learned clause, 
which is the complement of conflict reason, contains an overapproximation of the set of 
unsafe \rmcmt{or valid traces}.  Learned clauses are implicitly represented as 
transformers (see section~\ref{learning}).  An invariant of the ACDLP algorithm is 
that the the set of transformers after learning preseves the error reachability. 
ACDLP backjumps to a consistent state and model search is repeated with the new 
learned transformer which drives the search away from the conflicting region.  
However, if no further backtracking is possible, then ACDLP terminates and 
returns \emph{safe}.  


In the subsequent sections, we present a theoretical framework and mathematical 
recipe to build a precise abstract interpretation framework for functional safety 
property verification using Abstract Conflict Driven Clause Learning procedure. 
%
\begin{figure}
\centering
\scalebox{.70}{\import{figures/}{acdlp-top.pspdftex}}
\caption{ACDLP: Abstract Conflict Driven Learning for Programs \label{acdlp-top}}
\end{figure}
%
%===============================================================================
\section{Abstract Model Search in Programs}~\label{modelsearch}
%===============================================================================
%
A model search in CDCL solver alternates between two phases -- \emph{decisions} 
and \emph{boolean constraint propagation}, until a satisfying assignment is
obtained or a conflict is encountered. ~\cite{sas12} shows that BCP computes a
greatest fixed point by applying a unit rule which is the best abstract
transformer over a partial assignments domain.  Here, we characterise model
search as a procedure to find a counterexample trace in programs.  To do so, 
we present abstract model search in program as an instance of the Global Bottom 
Problem, shown in algorithm~\ref{gbp}.  We now present various transformers that 
are required to compute a fixed point approximation of concrete unsafe trace 
transformer.  
%-------------------------------------------------------------------------------
\para{Abstract Deduction Transformer}
%
Recall that $f_{unsafe}^{G}$ is a lower closure operator, which may be approximated 
by computing a greatest fixed point in the abstract.  
%
We now formally define a transformer, $f_{aunsafe}^G$, over
$\widehat{\mathcal{SA}_{G}}$, that uses strongest postcondition and 
existential precondition to perform forward, backward and multi-way 
analysis.  This transformer soundly approximate $f_{unsafe}^{G}$.  
We will call this \emph{abstract deduction transformer}.
%
\begin{definition} (Unsafe Trace Transformer in $\widehat{\mathcal{SA}}$). 
  $f_{aunsafe}^{\rightarrow}, f_{aunsafe}^{\leftarrow} : \widehat{\mathcal{SA}_{G}}
  \rightarrow \widehat{\mathcal{SA}_{G}}$
  \[
    f_{aunsafe}^{\rightarrow}(A) \mathrel{\hat=} \mathit{gfp}\; Z.\;
    apost_{\constraints}(A \meet Z)
    \quad
    f_{aunsafe}^{\leftarrow}(A) \mathrel{\hat=} \mathit{gfp}\; Z.\;
    apre_{\constraints}(A \meet Z)
  \]
   \[
      f_{aunsafe}^{G} = f_{aunsafe}^{\leftarrow}(A) \meet f_{aunsafe}^{\rightarrow}(A) 
   \]  
\end{definition}
%
%\paragraph{Property of Unsafe Trace Transformer}
%
\Omit{
However, each of the transformers 
above is computed using a least fixed point.  Hence, the result is a nested fixed point.  
The nested computation of a greatest fixed point using forward and backward analysis
based on least fixed points is a well-known technique in program
analysis~\cite{Cousot99}, and is used to necessarily increase precision in the
abstract.}
%
\begin{theorem}
  The transformers $f_{aunsafe}^{\rightarrow}$, $f_{aunsafe}^{\leftarrow}$ and 
  $f_{aunsafe}^{G}$ soundly approximate $f_{unsafe}^{G}$.
\end{theorem}
\begin{proof}
 We prove that $f_{aunsafe}^{\rightarrow}$ soundly overapproximates 
 $f_{unsafe}^{G}$. 
  We prove that  
  $ f_{unsafe}^{G} \circ \gamma_F \circ \gamma_S \circ \gamma \subseteq 
     \gamma_F \circ \gamma_S \circ \gamma \circ f_{aunsafe}^{\rightarrow} 
  $.
 Recall that $\widehat{SA}_{G}$ overapproximates the concrete
 $\powerset(\Pi)$ via the function $(\alpha \circ \alpha_S \circ \alpha_F,
 \gamma \circ \gamma_S \circ \gamma_F)$. Also, $apost_s$ soundly approximate
  $tpost$. For any element $g \in \widehat{\mathcal{SA}_{G}}$, the abstract unsafe trace
  transformer $f_{aunsafe}^{\rightarrow}$, $Z \mapsto apost_s(Z \meet A)$ 
  soundly approximates the concrete unsafe trace transformer
  $f_{unsafe}^{G}$, $Z \mapsto \gamma(g) \cap (\mathcal{I} \cup tpost(Z))$.
  Since $\mathcal{F}_G$ overapproximates $\powerset(\Pi)$ following
  Proposition~\ref{ag} and $\mathcal{SA}_G$ is exact to $\mathcal{F}_G$, so
  $\widehat{\mathcal{SA}_G}$ overapproximates $\powerset(\Pi)$.  From fixed
  point transfer theorem~\cite{fpt}, a trace 
  $\pi \in f_{unsafe}^{G} \circ \gamma_F \circ \gamma_S \circ \gamma$ also
  implies $\pi \in \gamma_F \circ \gamma_S \circ \gamma \circ
  f_{aunsafe}^{\rightarrow}$. The proof for $f_{aunsafe}^{\leftarrow}$ is similar. 
  %a set of traces computed by $f_{unsafe}^G$ is
  %therefore also obtained from $f_{aunsafe}^{\rightarrow}$.
  \rmcmt{
  Assume a trace $\pi \in f_{unsafe}^{G} \circ \gamma_F \circ \gamma_S \circ
  \gamma(a)$.
  Then $\pi$ is a counterexample trace such that for all $((id,v),val) \in a$
  and $(v,\omega) \in \pi$, it holds that $\omega \in \gamma_F \circ \gamma_S \circ 
  \gamma \circ a(id,v)$.  }
\end{proof}
%
%===============================================================================
%\input{gamma-complete}
%===============================================================================
%
%
\para{Generalized Decision Operator}~\label{decision}
%
An abstract model search heuristically searches for counterexample trace or a conflict.
This process generates a downward iteration sequence in the lattice of
$\widehat{\mathcal{SA}_{G}}$.  A decision in CDCL solver heuristically picks 
an unassigned variable and assigns a value to it.  
Similarly, a decision in ACDLP refines a downwards iteration sequence when the 
transformers $apost_{s}$ and $apre_{s}$ fails to make a refinement.  Given 
an abstract element $a$ obtained from the fixed point iteration, a decision $f_{dec}$ in 
$\widehat{\mathcal{SA}_{G}}$ heuristically chooses a meet irreducible $(a')$, such that 
the resultant element is strictly smaller than the greatest lower bound of the 
pair $(a, a')$.  We formally define a generalized decision operator over 
$\widehat{\mathcal{SA}_{G}}$. 
%
\begin{definition} (Generalized Decision Operator) 
  $f_{dec} \colon \widehat{\mathcal{SA}_{G}} \times \widehat{\mathcal{SA}_{G}}
  \rightarrow \widehat{\mathcal{SA}_{G}}$  
   \[ f_{dec}(a, a') = 
        a \sqcap a', \text{where}\; a\neq a' \wedge (a \sqcap a' \sqsubseteq a)
        \wedge (a \sqcap a' \sqsubseteq a')
   \]     
\end{definition}
%

%===============================================================================
\para{Global Bottom Problem}~\label{gbpg}
%===============================================================================
%
\begin{definition}
Given an abstract static assignment lattice, 
$(\widehat{\mathcal{SA}_G}, \sqsubseteq_{SA})$, a 
transformer 
$f_{aunsafe}^{G} : \widehat{\mathcal{SA}_{G}} \rightarrow \widehat{\mathcal{SA}_{G}}$ 
is \emph{globally bottom} if $f_{aunsafe}^G(a) = \bot$ for all $a$.  The global bottom 
problem is to determine if a transformer 
$f_{aunsafe}^G$ on a lattice $\widehat{\mathcal{SA}_{G}}$ is globally bottom.
\end{definition}

%However, $f_{aunsafe}^G$ is not globally bottom if there exists 
An element $a \in \widehat{\mathcal{SA}_{G}}$ is a non-$\bot$ witness if 
$f_{aunsafe}^G(a) \neq \bot$.  In this thesis, we consider the global bottom 
problem for completely additive, reductive transformers on powerset lattices.  
The result below follows directly from the soundness of abstract interpretation. 

\begin{theorem}~\label{fpt}
Given a completely additive and reductive transformer $f_{unsafe}^G$ on powerset of traces 
$(\powerset(\Pi), \subseteq)$, and 
$f_{aunsafe}^G \colon \widehat{\mathcal{SA}_{G}} \rightarrow \widehat{\mathcal{SA}_{G}}$ is 
a sound overapproximation of $f_{unsafe}^G$, and $\gamma_{SA}(gfp(f_{aunsafe}^G)) = \bot$, 
then the transformer $f_{unsafe}^G$ is globally bottom.
\end{theorem}

We now provide a condition to check whether $f_{unsafe}^G$ is not globally bottom. 
For this, it is 
sufficient to check whether $f_{aunsafe}^G$ is $\gamma$-complete even though the 
underlying abstract domain and the transformer may still be imprecise. 

\begin{proposition}~\label{gcf}
If $f_{aunsafe}^G$ is $\gamma$-complete at an element 
$a \in \widehat{\mathcal{SA}_G}$ and $\gamma_{SA}(f_{aunsafe}^G(a)) \neq \bot$, 
then $f_{unsafe}^G$ is not globally bottom.
\end{proposition}
%
Algorithm~\ref{gbp} presents a procedure for computing a non-$\bot$ witness 
to the global bottom problem. The algorithm takes as input an over-approximate 
abstract deduction transformer, $f_{aunsafe}^G$, a generalized decision transformer, 
$f_{dec}$, and a stack $\trail$ that record the results of transformer application, 
that is, $\trail$ contains elements of $\widehat{\mathcal{SA}_G}$.  The data-structure 
$\trail$ is also called \emph{trail}. The trail $\trail$ 
is initialized to $\top$.  The expression $\bigsqcap \trail$ denotes conjunction of 
all elements in $\trail$.  The expression $\trail \leftarrow \trail.a$ denotes concatenating 
$\trail$ with the new element $a$ which is pushed into $\trail$. 
The algorithm  checks if $\underset{d}{\forall}(f_{aunsafe}^G(d) = \bot)$ where $d \sqsubset_{SA} a$, 
that is, $f_{aunsafe}^G$ is bottom for every elements below $a \in \widehat{\mathcal{SA}_G}$.  
The output of the algorithm is a tuple consisting of the result of abstract deduction 
transformer $(\bot$, non-$\bot$, UNKNOWN$)$ and the final content of $\trail$. 
Following theorem~\ref{fpt}, if 
$\gamma_{SA}(gfp(f_{aunsafe}^G(a))) = \bot$, then $\underset{c}{\forall}(f_{unsafe}^G(c)=\bot)$ 
where $c \subset_{SA} \gamma_{SA}(a)$, that is, $f_{unsafe}^G$ is bottom on every elements $c$ 
that are below $\gamma_{SA}(a)$. \rmcmt{Note that, if the initial value of $a = \top$, and 
\texttt{abstract-counterexample-search} returns $\bot$, that is,  $\gamma_{SA}(gfp(f_{aunsafe}^G(\top))) = \bot$, then 
$f_{unsafe}^G$ is \emph{globally bottom}.  Else, for intial value of $a \neq \top$, 
$\gamma_{SA}(gfp(f_{aunsafe}^G(a))) = \bot$ corresponds 
to partially safety proof.} However, if $f_{aunsafe}^G$ is $\gamma$-complete at a fixed-point 
$a^\dagger$, and $\gamma_{SA}(f_{aunsafe}^G(a^\dagger)) \neq \bot$, then following proposition~\ref{gcf}, 
$f_{unsafe}^G$ is \emph{not globally bottom.}  If none of these conditions holds true, then a 
generalized decision operator $f_{dec}$ is applied, which jumps under the greatest fixed-point and 
checks if $f_{aunsafe}^G$ is $\bot$ on elements below the fixed-point.  This step is used to 
improve the precision of the analysis.
%
\begin{algorithm2e}[t]
\DontPrintSemicolon
\SetKw{return}{return}
\SetKwRepeat{Do}{do}{while}
%\SetKwFunction{assume}{assume}
%\SetKwFunction{isf}{isFeasible}
\SetKwData{conflict}{conflict}
\SetKwData{safe}{safe}
\SetKwData{sat}{sat}
\SetKwData{unsafe}{unsafe}
\SetKwData{unknown}{unknown}
\SetKwData{true}{true}
\SetKwInOut{Input}{input}
\SetKwInOut{Output}{output}
\SetKwFor{Loop}{Loop}{}{}
\SetKw{KwNot}{not}
\begin{small}
\Input{$f_{aunsafe}^G \colon \widehat{\mathcal{SA}_G} \rightarrow 
	\widehat{\mathcal{SA}_G}$, 
	$f_{dec} \colon \widehat{\mathcal{SA}_G} \times \widehat{\mathcal{SA}_G} 
	\rightarrow \widehat{\mathcal{SA}_G}$, 
	$\trail \colon$ elements of $\widehat{\mathcal{SA}_G}$}
	\Output{A tuple $(result, \trail)$, where result can be $\bot$, non-$\bot$, 
	or UNKNOWN}
        $a \leftarrow \bigsqcap \trail$ \;   
	\Do{$a = a^\dagger\; \text{or}\; \gamma_{SA}(a)=\bot$} {
           $a^\dagger \leftarrow a$ \;
	   $a \leftarrow a \meet_{SA} f_{aunsafe}^G(a) $ \;
	}
        \lIf{$\gamma_{SA}(a)=\bot$} {\return $(\bot,\trail)$} 
        $\trail \leftarrow \trail.a$ \;
	\lIf{$f_{aunsafe}^G\; \text{is}\; \gamma-complete\; \text{at}\; a$} 
	{\return (non-$\bot, \trail)$}
	$d \leftarrow f_{dec}$\;
	\lIf {$d=a$} {\return (unknown, $a)$}
	$\trail \leftarrow \trail.d$ \;
	\return abstract-counterexample-search$(f_{aunsafe}^G,f_{dec},\trail)$\;

\end{small}
\caption{Abstract Search for a non-$\bot$ witness of Global Bottom Problem 
	abstract-counterexample-search$(f_{aunsafe}^G,f_{dec},\trail)$ \label{gbp}}
\end{algorithm2e}
%

\begin{example}
%
\begin{figure}
\centering
\scalebox{.90}{\import{figures/}{semantic-example.pspdftex}}
\caption{An example CFG \label{fig:ex-ac}}
\end{figure}
%
Figure~\ref{model-search} shows an example run for the abstract model search procedure 
for the CFG in~Fig.\ref{fig:ex-ac} over Interval domain.  The elements obtained 
using $f_{aunsafe}^{G}$ transformer with strongest postcondition is marked 
in blue in Figure~\ref{model-search}.  
%The example demonstrates that model search computes a downward iteration sequence.  
Starting from $\top$, forward analysis concludes that $x$ is between -2 and 2
from $apost_{x:=-2} \cup apost_{x:=0} \cup apost_{x:=2}$.  Note that the loop is completely 
unwound and all statements corresponding to the loop are collectively referred to as $loop$. 
A forward fixed-point analysis (marked by $apost_{loop}$) does not yield any new 
information. Clearly, the analysis
is not precise to infer anything about the reachability of the error location $Error$. 
Hence, we apply a decision by picking a meet irreducible $y\geq 2$ to increase the 
precision of analysis.  We then apply forward analysis from this decision which
yields a downward iteration sequence as shown in lower part of
Fig.~\ref{model-search}.  Forward analysis concludes that $\{y \geq 4, x \geq 2, x \leq 2\}$. 
Clearly, $apost_{(y < 0)}(y \geq 4 \wedge x \geq 2 \wedge x \leq 2)$ leads to \emph{conflict}, 
which is marked as $\bot$ in Figure~\ref{model-search}. Hence, the error location $Error$ is 
unreachable for this decision. 
\end{example}
%
\begin{figure}[t]
\centering
\vspace*{-0.2cm}
\scalebox{.75}{\import{figures/}{model_search.pspdftex}}
  \caption{Model Search as Downward Iteration Sequence with Decisions and
  Deductions}
\label{model-search}
\end{figure}
%
%===============================================================================
\section{Abstract Conflict Analysis in Programs}~\label{conflict-analysis}
%===============================================================================
%
A conflict analysis procedure in CDCL solver finds the reason for 
a conflict by analyzing the deductions made during the 
model search phase through \emph{conflict resolution}~\cite{cdcl}. 
Conflict analysis in CDCL solver is different from DPLL solver -- 
CDCL allows non-chronological backjumping which can discard 
multiple levels of decisions and deductions trail, CDCL learns a 
reason for a conflict called \emph{conflict clause} that prevents 
the model search from re-entering into the conflicting search space 
in the future.  A conflict clause in CDCL solver is a clause that 
expresses the fact that some combinations of variable assignment 
are not valid.  Haller et. al. in~\cite{sas12} shows that conflict 
analysis in CDCL solver operate over an underapproximate domain, 
which is a downward closed set of partial assignments. 
%


Conflict analysis in ACDLP solver can be seen as abductive
reasoning~\cite{abd1,dhk2013-popl}. 
\Omit{A logic-based abductive inference~\cite{abd1} tries to find an explanation
$\mathcal{A}$ from a statement $\formula$ such that the truth of $\mathcal{A}$ 
is sufficient to guarantee the truth of $\formula$ and $\mathcal{A}$ is 
consistent with the background theory.}
An abductive inference is a dual of deductive inference that takes an input 
set $\allval$ contradicting a $\formula$ and find a weakest reason or 
explanation for $\allval$, assuming that $\formula$ is unsatisfiable.  

\Omit{
Figure~\ref{conflict-cdcl} shows a conflict
analysis procedure in CDCL solver.  Starting from the initial conflict, $\{p:t,
q:t, r:f, s:f\}$, the solver derives a generalized conflict reason $\{p:t\}$ 
which underapproximate the initial conflict reason, through Unique Implication 
Point (UIP) based conflict resolution algorithm~\cite{uip,cdcl}.  It is important 
to note that the conflict analysis in CDCL operates on the downset abstraction 
of partial assignments domain.  
%
\begin{figure}[htbp]
\centering
\vspace*{-0.2cm}
\scalebox{.85}{\import{figures/}{conflict_cut.pspdftex}}
\caption{Generalizing Conflict Reason in CDCL Solvers \label{conflict-cdcl}}
\end{figure}
%
}

A safe trace transformer, $f_{safe}^{G}$, computes a set of safe or
invalid traces.  Thus, the transformer $f_{safe}^G$ is completely 
multiplicative and extensive.  An effective conflict analysis for program requires a set of 
transformers that underapproximate $f_{safe}^{G}$, over the 
downset abstraction of $\widehat{\mathcal{SA}_{G}}$.  The requirement 
for downset abstraction is motivated by the conflict 
minimisation~\cite{DBLP:conf/sat/SorenssonB09}
technique used by SAT solvers that tries to generalize the reason 
for a conflict. Given a partial assignment $\pi$ that leads to a conflict, 
conflict minimisation replace $\pi$ with $\pi'$ such that $\pi$ can be derived 
from $\pi'$ following unit rule.  Note that, there can be multiple incomparable 
partial assignments that leads to the conflict, so a downsets of partial 
assignments have to be considered.  In practise, SAT solvers maintains only 
one conflict reason since generating all minimisations for a conflict can be 
ineffecient.  A lattice theoretic view of the sets of downsets obtained from 
a partial assignment $\pi = \{P:t, Q:t, R:f, S:f\}$ that leads to a conflict, 
is presented in figure~\ref{downset-abs}.  Given a lattice with $\top$ and $\bot$, 
the innermost diamond represents the partial assignment $\pi$.  Let us assume 
that each partial assignment, $\pi_1 = \{P:t\}$, $\pi_2 = \{Q:t, R:f\}$ and  
$\pi_3 = \{S:f\}$, obtained from $\pi$, are sufficient to derive the conflict.  
Now, each $\pi_i. i \in \{1,2,3\}$, forms a downset ($\mathbb{D}(\pi_i)$), 
denoted by respective triangles in figure~\ref{downset-abs}.  Further, $\pi_1$, 
$\pi_2$ and $\pi_3$ are incomparable, as is evident from their position in the 
lattice (peak of respective triangles).  A point to note here is that if 
$\pi_i$ leads to conflict, then every element that refines $\pi_i$ must 
belong to the downset of $\pi_i$ and must also contribute to the conflict.  
Hence, the notion of downset provides a lattice theoretic formulation of 
conflict analysis procedure in ACDLP.
%
\begin{figure}
\centering
\scalebox{.65}{\import{figures/}{downset-abstraction.pspdftex}}
  \caption{\label{downset-abs} Conflict Reasons as Set of Downsets}
\end{figure}  
%
%
\begin{figure}[htbp]
\centering
\vspace*{-0.2cm}
\scalebox{.85}{\import{figures/}{downsets.pspdftex}}
\caption{Lattice for Conflict Analysis \label{downset}}
\end{figure}
%


Here, we characterise conflict analysis as a procedure to find a 
generalized reason for conflict from a partial safety proof of a safe 
trace. To do so, we present abstract conflict analysis in program as 
an instance of the Global Top Problem, shown in algorithm~\ref{gtp}.  


We now present various transformers that are required to compute a 
fixed point approximation of concrete safe trace transformer.  
Recall that a program transformer $s \in \Sigma$ transforms the memory state of a
program and is associated with a transition relation,
$\mathcal{ST}_{s}^{\Omega}$.  We can associate each $s \in \Sigma$ with a weakest
precondition $\widehat{pre_{s}}$ and a universal postcondition
$\widehat{post_{s}}$. 
\Omit{
We now define these statement transformers. 
%
\begin{definition} (Universal Postcondition and Weakest Precondition Transformers).
  \[
    \widehat{post_{s}}(A) \mathrel{\hat=} \{\omega' \mid \forall \omega \in
     \Omega. \omega \in A \vee (\omega,\omega') \not\in \mathcal{ST}_{s}^{\Omega}\} 
  \] 
  \[ 
     \widehat{pre_{s}}(A) \mathrel{\hat=} \{\omega \mid \forall \omega' \in
     \Omega. \omega' \in A \vee (\omega,\omega') \not\in \mathcal{ST}_{s}^{\Omega}\} 
  \]
\end{definition}
}
%
%
An abstract conflict analysis for program computes an underapproximation of 
the concrete safe trace transformer $f_{safe}^G$.
To compute an underapproximation of $f_{safe}^{G}$, we underapproximate the
state transformers, $\widehat{post_s}$ and $\widehat{pre_s}$.  Furthermore, 
the lattice  $\mathcal{SA}_{G}$ is underapproximated by the downset 
completion $\mathbb{D}(\widehat{\mathcal{SA}_{G}})$, which is the set of 
downsets of $\mathcal{SA}_{G}$, as shown in figure~\ref{downset}.  Note 
that a downset completion enriches a domain with disjunctions.  Elements of 
downset are abstracted using an abstraction function, $\alpha_D$, which 
overapproximates  
\rmcmt{In this thesis, we treat downsets as underapproximating abstractions.}

%The lattice used for conflict analysis is shown in Figure~\ref{downset}.
%
\begin{proposition}
\[
   (\mathcal{SA}_{G},\supseteq_{SA})
   \galois{\alpha_{D}}{\gamma_{D}}
   (\mathbb{D}(\widehat{\mathcal{SA}_{G}}),\sqsupseteq_{SA}^\dagger) 
\]
  \[
    \alpha_{D}(a) \mathrel{\hat=} \{a' \in \widehat{\mathcal{SA}_G} \mid \gamma_{SA}(a')
    \subseteq_{SA} a \}
    \qquad
    \gamma_{D}(b) \mathrel{\hat=} \{\bigcup_{b' \in decomp(b)}
    \gamma_{SA}(b')\}
  \]
\end{proposition}
%
%The pair $(\alpha_{D},\gamma_{D})$ forms a galois connection, 
%which follows directly from the proof of~\cite{Cousot92}.  
\rmcmt{We assume that $\gamma_D(\top) = \Pi$.}

\begin{proof} 
  We now prove that the pair $(\alpha_{\mathbb{D}},\gamma_{\mathbb{D}})$ forms a galois connection. 
  It is straightforward to see that $(\alpha_{\mathbb{D}}$ and $\gamma_{\mathbb{D}})$ are monotone. 
  We show that $(\alpha_{\mathbb{D}} \circ \gamma_{\mathbb{D}})$ is extensive and 
  $(\gamma_{\mathbb{D}} \circ \alpha_{\mathbb{D}})$ is reductive. 
 
  Let us assume that $a \in \mathbb{D}(\widehat{\mathcal{SA}_G})$.  Then, 
  $\gamma_D(a) \supseteq_{SA} \gamma_{SA}(a)$, and thus 
  $a \in \alpha_{\mathbb{D}} \circ \gamma_{\mathbb{D}}$.  Thus,  
  $\alpha_{\mathbb{D}} \circ \gamma_{\mathbb{D}}$ is extensive. 
  
  
  Let $c \in \gamma_D \circ \alpha_D(c')$. Then, there exists an element $a \in \alpha_D(c')$ 
  such that $c \in \gamma_{SA}(a)$ and $\gamma_{SA}(a) \subseteq_{SA} c'$. Therefore, 
  $c \in \gamma_D(a)$.  Thus, $\gamma_{\mathbb{D}} \circ \alpha_{\mathbb{D}}$ is reductive. 
\end{proof}
%
Let $\widehat{apre_s}$ and $\widehat{apost_s}$ be sound underapproximations of
weakest precondition transformer $\widehat{pre_s}$ and universal postcondition 
transformer $\widehat{post_s}$ over $\mathbb{D}(\widehat{\mathcal{SA}_{G}})$. 
%
The global abstract static assignment transformers for the 
lattice $\mathbb{D}(\widehat{\mathcal{SA}_G})$ are obtained 
from the underapproximate 
abstract state transformers, $\widehat{apost_s}$, $\widehat{apre_s}$. 
This is defined next.
%
\begin{definition} (Global Underapproximate Abstract Static Assignment
  Transformers for $\mathbb{D}(\widehat{\mathcal{SA}_G})$). 
  \[ 
     \widehat{apost_{\constraints}}, \widehat{apre_{\constraints}} : 
     \mathbb{D}(\widehat{\mathcal{SA}_G}) \rightarrow
     \mathbb{D}(\widehat{\mathcal{SA}_{G}}) 
   \]
   \[
     \widehat{apost_{\constraints}}(a) \mathrel{\hat=} 
     \underset{\sigma \in \constraints}{\bigsqcap} \widehat{apost_{\sigma}} \circ a 
   \]
  \[
    \widehat{apre_{\constraints}}(a) \mathrel{\hat=} 
    \underset{\sigma \in \constraints}{\bigsqcap} \widehat{apre_{\sigma}} \circ a 
   \]
\end{definition}
%
The transformers, $\widehat{apost_{\constraints}}$ and
$\widehat{apre_{\constraints}}$, soundly underapproximate 
their concrete counterparts, $\widehat{post_{\constraints}}$ 
and $\widehat{pre_{\constraints}}$ respectively. 
%


A conflict reason is derived by analyzing the deductions made 
from $f_{aunsafe}^{G}$ during the model search phase.  
%Recall that the deductions are obtained by approximating 
%a least fixed point using the strongest post-condition.  
Conflict analysis is performed by computing a
least fixed point of a weakest pre-condition transformer or an universal
postcondition transformer in the downset abstract domain
$\mathbb{D}(\widehat{\mathcal{SA}_{G}})$.  For this, we define
a transformer $f_{asafe}^G$ in $\mathbb{D}(\widehat{\mathcal{SA}_{G}})$, that 
is a sound under-approximation of completely multiplicative and 
extensive transformer, $f_{safe}^{G}$.  Below, $A$ is the downset 
closure of the original conflict reason. 
%
\begin{definition} (Safe Trace Transformers in
  $\mathbb{D}(\widehat{\mathcal{SA}})$). 
  \[
    f_{asafe}^{\rightarrow}, f_{asafe}^{\leftarrow} :
  \mathbb{D}(\widehat{\mathcal{SA}_{G}})
  \rightarrow \mathbb{D}(\widehat{\mathcal{SA}_{G}})
  \]
  \[
    f_{asafe}^{\rightarrow}(A) = \mathit{lfp}\; Z.\;
    \widehat{apost_{\constraints}}(A \join Z)
    \quad
    f_{asafe}^{\leftarrow}(A) = \mathit{lfp}\; Z.\; 
    \widehat{apre_{\constraints}}(A \join Z)
  \] 
\end{definition}
%
%
%===============================================================================
\para{Global Top Problem}
%===============================================================================
%
\begin{definition}
Given a downset lattice, 
$(\mathbb{D}(\widehat{\mathcal{SA}_G}), \sqsupseteq_{SA}^\dagger)$, 
a transformer 
$f_{asafe}^{G} : \mathbb{D}(\widehat{\mathcal{SA}_{G}}) \rightarrow \mathbb{D}(\widehat{\mathcal{SA}_{G}})$ 
is \emph{globally top} if $f_{asafe}^G(a) = \top$ for all 
$a \in \mathbb{D}(\widehat{\mathcal{SA}_G})$.  The global top
problem is to determine if a transformer $f_{asafe}^G$ on a lattice 
$\mathbb{D}(\widehat{\mathcal{SA}_{G}})$ is globally top.
\end{definition}
%
An element $a \in \mathbb{D}(\widehat{\mathcal{SA}_{G}})$ is a 
non-$\top$ witness if $f_{asafe}^G(a) \neq \top$.  Algorithm~\ref{gtp} 
present a procedure for checking if $f_{safe}^G$ is globally top.  
\rmcmt{If the result of $lfp(f_{asafe}^G)$ concretizes to $\top$, then we 
can infer the following.
\begin{enumerate}
	\item $f_{safe}$ is globally top.
	\item $f_{unsafe}$ is globally bottom since $gfp(f_{aunsafe}^G)$ concretizes to $\bot$
\end{enumerate}
}
%
Given a conflict, the procedure \texttt{analyze-partial-safety-proof} of Algorithm~\ref{gtp} is used to generalize 
an element that leads to the conflict.  To do so, it computes an 
under-approximation of the least fixed point using the upwards interpolation, 
$int \upharpoonright$.  Recall that upwards interpolation on a lattice $L$ 
is a function $int \upharpoonright \colon L \times L \rightarrow L$ such 
that $a \sqsubseteq b \implies a \sqsubseteq int \upharpoonright(a,b) \sqsubseteq b$ for all $a,b \in L$. 
To avoid considering sets of all generalizations of a conflict, $u \;int \upharpoonright \; a$ 
(in line 2 of Algorithm~\ref{gtp}) is used as a choice function, where $u$ and $a$ are downsets. 
Line 3 of Algorithm~\ref{gtp} suggest that if the weakest precondition of $\bot$ concretized to $\top$, 
then the $f_{safe}$ is globally top.


%
\begin{algorithm2e}[t]
\DontPrintSemicolon
\SetKw{return}{return}
\SetKwRepeat{Do}{do}{while}
%\SetKwFunction{assume}{assume}
%\SetKwFunction{isf}{isFeasible}
\SetKwData{conflict}{conflict}
\SetKwData{safe}{safe}
\SetKwData{sat}{sat}
\SetKwData{unsafe}{unsafe}
\SetKwData{unknown}{unknown}
\SetKwData{true}{true}
\SetKwInOut{Input}{input}
\SetKwInOut{Output}{output}
\SetKwFor{Loop}{Loop}{}{}
\SetKw{KwNot}{not}
\begin{small}
	\Input{$f_{asafe}^G \colon \mathbb{D}(\widehat{\mathcal{SA}_G}) \rightarrow 
	\mathbb{D}(\widehat{\mathcal{SA}_G})$, 
	$int \upharpoonright \colon \mathbb{D}(\widehat{\mathcal{SA}_G}) \times \mathbb{D}(\widehat{\mathcal{SA}_G}) 
	\rightarrow \mathbb{D}(\widehat{\mathcal{SA}_G})$, 
	$u \in \mathbb{D}(\widehat{\mathcal{SA}_G})$}
	\Output{A tuple $(result, t)$, where result can be $\top$, non-$\top$, 
	or UNKNOWN and $t \in \mathbb{D}(\widehat{\mathcal{SA}_G})$}

	$a \leftarrow u \sqcup_{SA}^\dagger f_{asafe}^G(u)$ \;
	$t \leftarrow u\; int \upharpoonright \; a$ \;
	
	\lIf{$\gamma_{D}(t)=\top$} {\return $(\top,t)$} 
        \lIf{$f_{asafe}^G$ is $\gamma$-complete at $t$} {\return (non-$\top$, $t$)} 
	\lIf{t=u} {\return (unknown, t)}
	{\return analyze-partial-safety-proof$(f_{asafe}^G, int \upharpoonright,t)$}
\end{small}
	\caption{Abstract Search for a non-$\top$ witness of Global Top Problem in $\mathbb{D}(\widehat{\mathcal{SA}_G})$,  
	analyze-partial-safety-proof$(f_{asafe}^G,int \upharpoonright,u)$ \label{gtp}}
\end{algorithm2e}
%

\begin{example}
  Let us revisit the example in Figure~\ref{fig:ex-ac} and the corresponding
  deductions in Figure~\ref{model-search}. The conflict analysis procedure is
  shown in Figure~\ref{conflict-example}.  We iteratively apply $\widehat{apre}$ 
  starting from the conflict element ($\bot$), the result of which is shown in
  bold text.  For example, $\widehat{apre_{y < 0}}(\bot)= \{y \geq 0\}$; whereas the
  result of $f_{aunsafe}^{G}$ transformer application via strongest postcondition 
  is $\{y\geq 4\}$.  So, we heuristically pick a generalized element $a$ such 
  that $\{y\geq 4\} \sqsubseteq a \sqsubseteq \{y \geq 0\}$; we pick $a=\{y \geq 0\}$ 
  through the application of upwards interpolation~\cite{leo-thesis},  
  \Omit{(corresponds to relaxation of narrowing operation in abstract interpretation)}
  $int\upharpoonright(y \geq 0, y \geq 4)$, marked in blue.  The heuristic 
  generalization of conflict reason for different abstract domains is explained 
  in section~\ref{heu-gen}.
  Note that the loop is completely unwound and all statements corresponding to the 
  loop are collectively referred to as $loop$.
  We then repeat the process marked by $\widehat{apre_{loop}}$. Subsequently, we derive 
  a generalized reason, $\{x>0, y \geq 0\}$, that strictly generalizes the decision 
  $(y \geq 2)$.  Hence, the partial safety proof corresponding to the set of traces 
  with prefix $\{y \geq 2\}$ gives us a generalized prefix that 
  includes the set of all safe traces with prefixes $\{x \geq 0, y \geq 0\}$. 
  Note that the concrete safe trace transformer $f_{safe}^G$ returns all safe traces which 
  includes traces with prefixes $\{ y < 0, y \geq 0 \}$.  However, $f_{aunsafe}^G$ returns 
  a conflict reason which underapproximates $f_{safe}^G$, that is, it does not include 
  the set of safe traces with prefix $\{y<0\}$, but generalizes the partial safety proof 
	for the set of safe traces with prefix $\{y \geq 2\}$. \todo{define prefix of trace}.
  %\rmcmt{define upwards interpolation}
\end{example}
%
\begin{figure}[t]
\centering
\vspace*{-0.2cm}
\scalebox{.70}{\import{figures/}{conflict_example.pspdftex}}
  \caption{Conflict Analysis with underapproximate weakest precondition and
  upwards interpolation}
\label{conflict-example}
\end{figure}
%
\subsection{Generalized Unit rule}~\label{learning}
%
Clause learning in ACDLP learns new abstract transformers $(AUnit)$ 
which are implicitly represented as clauses.  
%Since unit rule is the core component of the model search phase in SAT solvers, 
%so clause learning can be viewed as learning a unit-rule transformer, $(AUnit)$.  
Learning refines the transformer $f_{aunsafe}^G$ with the transformer $AUnit$. 
Intuitively, transformer learning through conflict analysis makes $f_{aunsafe}^G$ a 
closer approximation of the unsafe trace transformer, $f_{unsafe}^{G}$.  
%
The transformer $AUnit$ is a generalisation of the propositional unit rule~\cite{cdcl}.  
Recall that $\formula$ is a safety formula that is obtained through 
conjunction of all SSA elements in $\Sigma^{\dagger}$. 
For an abstract lattice $\widehat{\mathcal{SA}_{G}}$ with
complementable meet irreducibles and a set of meet irreducibles $\conflictset
\subseteq \widehat{\mathcal{SA}_{G}}$ such that $\bigsqcap
\conflictset$ does not satisfy $\formula$, $\aunit_\conflictset:
\widehat{\mathcal{SA}_G} \rightarrow \widehat{\mathcal{SA}_G}$ 
is formally defined as follows.
\[ \aunit_\conflictset(\absval) =
 \left\{\begin{array}{l@{\quad}l@{\qquad}l}
  \bot       & \text{if } \absval \sqsubseteq \bigsqcap \conflictset & (1)\\
  \bar{t}    & \text{if } t \in \conflictset \; \text{and} \; \forall t' \in \conflictset
  \setminus \{t\}. \absval  \sqsubseteq t' & (2) \\
  \top & \text{otherwise} & (3) \\
 \end{array}\right.
\]
%
Rule (1) shows $\aunit$ returns $\bot$ since 
$\absval \sqsubseteq \bigsqcap \conflictset$ is conflicting.  
Rule (2) of $\aunit$ infer a valid meet irreducible, 
which implies that $\conflictset$ is unit.  Rule (3) of  
$\aunit$ returns $\top$ which implies that the learned clause is not 
{\em asserting} after backtracking.  This would prevent any new 
deductions from the learned clause. Progress is then made by decisions. 
%
\begin{figure}[htbp]
\centering
\vspace*{-0.2cm}
\scalebox{.70}{\import{figures/}{learning.pspdftex}}
\caption{Downward Iteration Sequence with Learned transformer \label{learning}}
\end{figure}
%
\begin{example}
Fig.~\ref{learning} shows the sequence of fixed point iteration with the learned
transformer $y \leq 0$, obtained from $AUnit$.  Clearly, this also leads to 
\emph{conflict}. There are no further cases to explore. Thus, the procedure 
terminates and returns \emph{safe}.
\end{example}
%
\para{Algorithm for ACDLP}
%

{\begin{algorithm2e}[t]
\DontPrintSemicolon
\SetKw{return}{return}
\SetKwRepeat{Do}{do}{while}
%\SetKwFunction{assume}{assume}
%\SetKwFunction{isf}{isFeasible}
\SetKwData{conflict}{conflict}
\SetKwData{safe}{safe}
\SetKwData{sat}{sat}
\SetKwData{unsafe}{unsafe}
\SetKwData{unknown}{unknown}
\SetKwData{true}{true}
\SetKwInOut{Input}{input}
\SetKwInOut{Output}{output}
\SetKwFor{Loop}{Loop}{}{}
\SetKw{KwNot}{not}
\begin{small}
	\Input{
        $f_{aunsafe}^G \colon \widehat{\mathcal{SA}_G} \rightarrow 
	\widehat{\mathcal{SA}_G}$, 
	$f_{dec} \colon \widehat{\mathcal{SA}_G} \times \widehat{\mathcal{SA}_G} 
	\rightarrow \widehat{\mathcal{SA}_G}$, 
	$f_{asafe}^G \colon \mathbb{D}(\widehat{\mathcal{SA}_G}) \rightarrow 
	\mathbb{D}(\widehat{\mathcal{SA}_G})$, 
	$int \upharpoonright \colon \mathbb{D}(\widehat{\mathcal{SA}_G}) \times \mathbb{D}(\widehat{\mathcal{SA}_G}) 
	\rightarrow \mathbb{D}(\widehat{\mathcal{SA}_G})$ 
	}
	\Output{$\bot$ if $f_{aunsafe}^G$ is globally bottom, else 
	a counterexample $(\bigsqcap \trail)$}
        
	$\trail \leftarrow \top$ \;

	\Do{$\trail$ is empty} {
	   $(d,\trail) \leftarrow$ abstract-counterexample-search$(f_{aunsafe}^G,f_{dec},\trail)$ \;
	   \lIf{$d \neq \bot$} {\return $(d,\bigsqcap \trail)$} 
	   $a \leftarrow \bigsqcap \trail$ \; 
	   $(d,{t}) \leftarrow$ analyze-partial-safety-proof$(f_{asafe}^G,int \upharpoonright, a)$ \;
	   $learn \leftarrow AUnit_{t}$ \;
	   $\trail \leftarrow$ backjump$(learn,f_{aunsafe}^G,\trail)$ \;
	   $f_{aunsafe}^G \leftarrow f_{aunsafe}^G \sqcap learn$ \;
	}
	{\return \rmcmt{$\bot$}}
\end{small}
	\caption{Abstract Conflict Driven Clause Learning (ACDLP) \label{acdlp-algo}}
\end{algorithm2e}
%
Algorithm~\ref{acdlp-algo} describes the working of Abstract Conflict Driven Learning for Programs (ACDLP).  
The procedure \texttt{abstract-counterexample-search} procedure in ACDLP operates over an overapproximate domain $\mathcal{SA}_G$ 
with complementable meet irreducibles, whereas \texttt{analyze-partial-safety-proof} procedure operates over an 
underapproximate downset completion domain, $\mathbb{D}(\mathcal{SA}_G)$.  ACDLP starts by initializing $\trail$ to 
a singleton sequence $\top$. The outer loop of the algorithm terminates when $\trail$ is empty. In this case, 
\texttt{analyze-partial-safety-proof} returns $\top$ and there are no further cases to be explored, hence ACDLP 
terminates.  However, if \texttt{abstract-counterexample-search} returns non-$\bot$, then ACDLP terminates a 
counterexample $(\bigsqcap \trail)$.  The invariant of the ACDLP algorithm is that the transformer 
$f_{aunsafe}^G$ is a sound overapproximation of the concrete unsafe trace transformer $f_{unsafe}^G$.


Figure~\ref{acdlp-fig} describes the communication between the Abstract-Counterexample-Search (shown in left box) and Analyze-Partial-Safety-Proof (shown in right box) of ACDLP algorithm.  The conflicting elements obtained from a conflict are passed from Abstract-Counterexample-Search to Analyze-Partial-Safety-Proof.
Whereas, a learned transformer $learn$ is transferred from Analyze-Partial-Safety-Proof to Abstract-Counterexample-Search which 
refines the transformer $f_{aunsafe}^G$.  ACDLP backjumps after learning using the function \texttt{backjump}.  The backjump is 
asserting which implies that the learned transformer becomes \emph{unit} after backjumping, that is, the application of $learn$ 
transformer generates new deductions which guides the search away from the conflicting region.   The output of ACDLP is \emph{UNSAFE} 
in Figure~\ref{acdlp-fig} if $f_{aunsafe}^G$ is not globally bottom.  However, the output is \emph{SAFE} if $f_{aunsafe}^G$ is globally bottom.    
%
%
\begin{figure}[htbp]
\centering
\vspace*{-0.2cm}
\scalebox{.85}{\import{figures/}{acdcl.pspdftex}}
\caption{Abstract Interpretation Framework for Precise Safety Verification \label{acdlp-fig}}
\end{figure}
%
%

\para{Backjumping}
%
{\begin{algorithm2e}[t]
\DontPrintSemicolon
\SetKw{return}{return}
\SetKwRepeat{Do}{do}{while}
%\SetKwFunction{assume}{assume}
%\SetKwFunction{isf}{isFeasible}
\SetKwData{conflict}{conflict}
\SetKwData{safe}{safe}
\SetKwData{sat}{sat}
\SetKwData{unsafe}{unsafe}
\SetKwData{unknown}{unknown}
\SetKwData{true}{true}
\SetKwInOut{Input}{input}
\SetKwInOut{Output}{output}
\SetKwFor{Loop}{Loop}{}{}
\SetKw{KwNot}{not}
\begin{small}
	\Input{
	$learn \in \widehat{\mathcal{SA}_G}$, 
	$f_{aunsafe}^G \colon \widehat{\mathcal{SA}_G} \rightarrow 
	\widehat{\mathcal{SA}_G}$, 
	$\trail \colon$ elements of $\widehat{\mathcal{SA}_G}$}
 	
	\Output{$\trail$ after backtracking}
	
        $f_{aunsafe}^G \leftarrow f_{aunsafe}^G \meet learn$ \; 
	\Do{$\gamma_{SA}(f_{aunsafe}^G(\bigsqcap \trail)) = \bot$} {
		$\trail \leftarrow$ pop$(\trail)$ \;
	}
	{\return $\trail$}
\end{small}
	\caption{backjump $(learn, f_{aunsafe}^G, \trail)$ \label{backjump}}
\end{algorithm2e}
%
Algorithm~\ref{backjump} removes elements from $\trail$ as long as  
$\gamma_{SA}(f_{aunsafe}^G(\bigsqcap \trail)) = \bot$ holds true.  The 
function \emph{pop} is used to remove the top element of the trail $\trail$. 
The algorithm returns a modified $\trail$ that resets the state of $\trail$ 
before the element that is negated by the transformer $learn$, that is, the 
state of trail where $\gamma_{SA}(f_{aunsafe}^G(\bigsqcap \trail)) \neq \bot$ 
holds true.  If $\trail$ is empty after backjumping, then $f_{aunsafe}^G$ is 
globally bottom. 


\para{Asserting Backjump} The function \texttt{backjump}$(learn,f_{aunsafe}^G, \trail)$ 
is \emph{asserting} if the output trail $\trail'$ returned by \texttt{backjump} 
satisfies either of the following two conditions.
\begin{enumerate}
	\item $\trail' = \langle \rangle$, that is, $\trail'$ is empty
	\item \texttt{learn}$(\bigsqcap \trail')$ and $(\bigsqcap \trail)$ 
		are not comparable, where $\trail$ and $\trail'$
		are the trail before and after backjumping, respectively.
\end{enumerate}
%
Condition (2) above denotes that asserting backjumping drives the model search 
away from the conflicting region, that is, the trail $\trail'$ obtained after 
learning through \texttt{learn} transformer contains newly deduced elements 
that drives the model search into a new part of the search space. We now show 
that $learn(\bigsqcap \trail')$ and $(\bigsqcap \trail)$ are unordered 
in the lattice $\widehat{\mathcal{SA}_G}$.  We denote the transformer after learning 
as $f_{aunsafe}^{G \dagger}$, where $f_{aunsafe}^{G \dagger} = f_{aunsafe}^{G} \sqcap learn$.
%
\begin{lemma}~\label{bjump}
A run of ACDLP algorithm with asserting backjumping maintains the 
state of the trail in a way that the abstract element $\bigsqcap \trail$ 
from trail $\trail$ before backjumping is incomparable to the abstract 
element $f_{aunsafe}^{G \dagger}(\bigsqcap \trail')$ from trail $\trail'$ 
after backjumping. 
\end{lemma}
%
\begin{proof}
Given $\trail$ and $\trail'$ represent the trail before and after backjumping, 
respectively.  Let $a = \bigsqcap \trail$ and $a' = \bigsqcap \trail'$.  Recall 
that \texttt{analyze-partial-safety-proof} returns a tuple $(d, t)$, where 
$t \sqsupseteq_{SA} a$.  The generalized unit rule transformer $AUnit$ with 
respect to $t$ returns $AUnit_t(a') = a' \sqcap_{SA} \bar{l}$ after backjumping. 
Here, $\bar{l}$ is a meet irreducible of $\widehat{\mathcal{SA}_G}$.  It follows 
that $f_{aunsafe}^{G \dagger}(a') \sqsubseteq_{SA} \bar{l}$ and $l \sqsupseteq_{SA} t$, 
so $l \sqsupseteq_{SA} a$ also holds true.  We decompose the proof into two separate 
cases, each of which are proven by contradiction. \\


Case 1: Assume that $(\bigsqcap \trail) \sqsubseteq_{SA}$ \ltrans$(\bigsqcap \trail')$. 
That is, $a \sqsubseteq_{SA}$ \ltrans$(a')$. We can infer the following.
\begin{enumerate}
	\item $a \sqsubseteq_{SA}$ \ltrans$(a')$ implies $a \sqsubseteq_{SA} \bar{l}$. 
	\item $a \sqsubseteq_{SA} t$ and $t \sqsubseteq_{SA} l$ implies $a \sqsubseteq_{SA} l$.
\end{enumerate}
From the above, $\gamma_{SA}(a) = \bot$. This is a contradiction since 
\texttt{abstract-counterexample-search} never returns an empty trail, that is, a trail representing an empty set.  We 
not show that other side of the proof. \\

Case 2: Assume that $(\bigsqcap \trail) \sqsupseteq_{SA}$ \ltrans$(\bigsqcap \trail')$. 
That is, $a \sqsupseteq_{SA}$ \ltrans$(a')$. We can infer the following.
\begin{enumerate}
  \item From $a \sqsubseteq_{SA} l$, we can infer \ltrans$(a') \sqsubseteq_{SA} l$.
  \item We already know that \ltrans$(a') \sqsubseteq \bar{l}$.
\end{enumerate}
From the above, $\gamma_{SA}$(\ltrans$(a')) = \bot$. This is a contradiction since 
\texttt{backjump} never returns a conflicting trail. \\ 
	
Thus, we conclude that 
$(\bigsqcap \trail) \not\sqsubseteq_{SA}$ \ltrans$(\bigsqcap \trail')$ and 
\ltrans$(\bigsqcap \trail') \not\sqsubseteq_{SA} (\bigsqcap \trail)$. Hence, 
$(\bigsqcap \trail)$ and \ltrans$(\bigsqcap \trail')$ are unordered. 
\end{proof}
%
\begin{lemma}~\label{sound-transformer}
For a completely additive and reductive transformer $f_{unsafe}^G$ on powerset of traces 
$(\powerset(\Pi), \subseteq)$, let
$f_{aunsafe}^G \colon \widehat{\mathcal{SA}_{G}} \rightarrow \widehat{\mathcal{SA}_{G}}$ be 
a sound overapproximation of $f_{unsafe}^G$.  Let
$f_{asafe}^G \colon 
\mathbb{D}(\widehat{\mathbb{SA}_{G}}) \rightarrow \mathbb{D}(\widehat{\mathcal{SA}_{G}})$ be 
a sound underapproximation of a completely multiplicative and extensive transformer 
$f_{safe}^G$. Given an abstract element $a$, 
if $gfp(f_{aunsafe}^G)(a) = \emptyset$, then $f_{aunsafe}^G \sqcap learn$ 
is a sound overapproximation of $f_{unsafe}^G$. 
\end{lemma}

\begin{proof}[Proofsketch]
The procedure \texttt{analyze-partial-safety-proof} in ACDLP computes a least fixed-point
of the transformers $\widehat{apost}_{\Sigma^\dagger}$ and $\widehat{apre}_{\Sigma^\dagger}$.
The procedure returns an element $(d, t)$, such that $learn = AUnit_{t}$. The transformer 
$f_{aunsafe}^G \sqcap learn$ preserves the error reachability.  This can be explained as 
follows. For the given abstract element $a$, if $gfp(f_{aunsafe}^G)(a) = \emptyset$ and 
$\trail$ be the trail which leads to $\bot$, $\trail'$ be the trail after backjump, then
$\sqcap\trail$ is incomparable to $(f_{aunsafe}^G \sqcap learn)(\sqcap\trail')$ 
following lemma~\ref{bjump}. Thus, the transformer $AUnit_t$ drives the unsafe trace 
transformer $f_{aunsafe}^G$ away from the conflict, thereby preserving the error rechability. 
\end{proof}

\section{Putting it all together}
%
\begin{figure}[htbp]
\centering
\vspace*{-0.2cm}
\scalebox{.80}{\import{figures/}{schema.pspdftex}}
\caption{Building Blocks for Instantiating CDCL Architecture for Program Verification \label{schema}}
\end{figure}
%
Figure~\ref{schema} gives the basic building blocks for instantiating CDCL 
architecture for program verification using abstract interpretation framework.  
Elements of $\widehat{\mathcal{SA}_{G}}$ forms a lattice of approximation of program traces
ordered by $\sqsubseteq_{SA}$ where each element $a \in \widehat{\mathcal{SA}_{G}}$ 
represents a memory state of the program.  Each statement $s$ in the program
defines four transformers, strongest postcondition $post_s$, existential precondition 
$pre_s$, universal postcondition $\widehat{post_s}$, and weakest precondition 
$\widehat{pre_s}$.  Model search computes an overapproximation of unsafe trace 
transformer using fixed point over $f_{aunsafe}^{\leftarrow}$, $f_{aunsafe}^{\rightarrow}$ 
and decisions operator $f_{dec}$ to improve precision.  
Whereas, conflict analysis computes an underapproximation of safe trace transformer 
using fixed point over $f_{asafe}^{\leftarrow}$, $f_{asafe}^{\rightarrow}$ and
heuristic choice as upwards interpolation $int\upharpoonright$ to pick a conflict reason.
Learning overapproximates the unsafe trace transformer while still preserving the error reachability 
%is a reduction over $F^{\sharp}$ that 
and is parameterized by an element of downset abstract domain $\mathbb{D}(\widehat{\mathcal{SA}_{G}})$.
%
\begin{figure}[t]
\begin{minipage}{4cm}
\centering
\scalebox{.60}{\import{figures/}{acdl_fixpoint.pspdftex}}
  %\caption{Fixpoint computation during Abstract Model Search Phase (Red lines denote
  %decision, Blue lines denote deduction)}
  \label{model-fixpoint}
\end{minipage}%  
\hspace{10em}%
\begin{minipage}{4.0cm}
\centering
\scalebox{.60}{\import{figures/}{conflict-fixpoint.pspdftex}}
  %\caption{Conflict Analysis with Underapproximate Weakest
  %Precondition and Upwards Interpolation}
\label{conflict-fixpoint}
\end{minipage}
  \caption {
  A. Fixpoint computation during Abstract Model Search Phase (Red lines denote
  decision, Blue lines denote deduction), B. Conflict Analysis with 
  Underapproximate Weakest Precondition and Upwards Interpolation
  \label{fixpoint}}
\end{figure}
%

Fig.~\ref{fixpoint} shows a graphical execution of the model search and 
conflict analyis procedures over a lattice of fixpoints.  The region marked 
with green in figure~\ref{fixpoint}A and figure~\ref{fixpoint}B are the set of 
fixed points of $f_{aunsafe}^G$ and pre-fixed points $(f_{asafe}^G(C) \sqsupseteq C)$ 
respectively.   The least among the fixed points is marked as
$lfp(f_{aunsafe}^G)$, whereas the greatest among the fixed points is marked as
$gfp(f_{aunsafe}^G)$ in figure~\ref{fixpoint}A, and similarly for figure~\ref{fixpoint}B.  
Note that the concrete unsafe trace transformer $f_{unsafe}^G$ is a lower closure, 
so $lfp(f_{unsafe}^G) = \bot$.  Whereas the concrete safe transformer $f_{safe}$ is 
upper closure, so $gfp(f_{safe}^G) = \top$.  Since $f_{aunsafe}^G$ and $f_{asafe}^G$ 
are sound over and underapproximations of $f_{unsafe}^G$ and $f_{safe}^G$ respectively, 
so $lfp(f_{aunsafe}^G) = \bot$ in figure~\ref{fixpoint}A and $gfp(f_{asafe}^G) = \top$ in 
figure~\ref{fixpoint}B.

\Omit{
The region in the intersection of green and golden 
ellipse structure is the set of fixed-points, which is both a pre-fixed point and
post-fixed point.  
}
%

Fig.~\ref{fixpoint}(A) shows that model search procedure starts 
from $\top$, and make sequence of deductions (marked in blue) 
through the abstract transformer $f_{aunsafe}^G$, until it reaches 
the greatest fixed point.  Recall that $f_{aunsafe}^G$ is the
set of transformers in $\widehat{\mathcal{SA}_G}$ that are used for forward, 
backward and multi-way analysis. 
%until it reaches the greates fixed point. 
%Recall that BCP in CDCL solver computes a greatest fixed point through 
%repeated application of the unit rule transformer. 
When the deductions can not be refined further, a decision is made which 
jumps under the greatest fixed point, as shown by red arrows in 
Figure~\ref{fixpoint}A.   
\Omit{
A decision accelerates the fixed point 
computation and is necessary to quickly converge on a conflict or a 
counterexample trace. }
A sequence of decisions and deductions follows, which
terminates in the $\gamma$-complete region of the lattice (marked in pink). The
transformer $f_{aunsafe}^G$ is $\gamma$-complete on $d$ if $\gamma(f_{aunsafe}^G(d))
\subseteq f_{unsafe}^G(\gamma(d))$, where $f_{unsafe}^G$ is the concrete unsafe trace 
transformer.  This implies, that a counterexample is obtained and the procedure terminates.
%


Fig.~\ref{fixpoint}(B) shows that conflict analysis procedure starts from a
conflict $\{\mathcal{C}\}$, and iteratively generalizes an element by computing a
least fixed point using a weakest precondition, $\widehat{apre}_s$ (marked by
red arrows), followed by heuristically choosing a candidate reason for conflict 
through upwards interpolation (marked by blue arrows) that underapproximates 
$\widehat{apre}_s$.   

%Figure~\ref{fig:acdcl} presents an architectural view 
%of a learning based program analyzer using CDCL architecture.  
%
%


\Omit {
\begin{figure}
\centering
\vspace*{-0.2cm}
\scalebox{.70}{\import{figures/}{acdlp-top.pspdftex}}
\caption{Architectural View of ACDLP \label{conflict}}
\end{figure}
%
\rmcmt{define transformers}
%
\begin{algorithm2e}[t]
\DontPrintSemicolon
\SetKw{return}{return}
\SetKwRepeat{Do}{do}{while}
%\SetKwFunction{assume}{assume}
%\SetKwFunction{isf}{isFeasible}
\SetKwData{conflict}{conflict}
\SetKwData{safe}{safe}
\SetKwData{sat}{sat}
\SetKwData{unsafe}{unsafe}
\SetKwData{unknown}{unknown}
\SetKwData{true}{true}
\SetKwInOut{Input}{input}
\SetKwInOut{Output}{output}
\SetKwFor{Loop}{Loop}{}{}
\SetKw{KwNot}{not}
\begin{small}
\Input{A program in the form of a set of abstract transformers $\abstransset$.}
\Output{The status \safe or \unsafe. %and a counterexample if \unsafe.
}
$\trail \leftarrow \langle\rangle$, $\reasons \leftarrow []$ \;
$\mathit{result} \leftarrow \deduce_{\propheur}(\abstransset,\trail,\reasons)$ \;
\lIf{$\mathit{result}$ = \conflict} {
  \return \safe}
\While{$true$} 
{
\lIf{$\mathit{result}$ = \sat} {
  \return \unsafe}
  $\decisionvar \leftarrow \decide_{\decheur}(\abs(\trail))$ \;
  $\trail \leftarrow \trail \cdot \decisionvar$ \; 
  $\reasons[|\trail|] \leftarrow \top$ \;
  $\mathit{result} \leftarrow \deduce_{\propheur}(\abstransset,\trail,\reasons)$\;
  \Do{$\mathit{result} = \conflict$} {
    \lIf{$\neg \analyzeconflict_{\confheur}(\abstransset,\trail,\reasons)$} {
      \return \safe
    }
    $\mathit{result} \leftarrow \deduce_{\propheur}(\abstransset,\trail,\reasons)$ \;
  }
}
\end{small}
\caption{Abstract Conflict Driven Learning for Programs $ACDLP_{\propheur,\decheur,\confheur}(\abstransset)$ \label{Alg:acdcl}}
\end{algorithm2e}
%
In this section, we present our framework called \emph{Abstract Conflict 
Driven Learning} that uses abstract model search and abstract 
conflict analysis procedures for safety verification of programs. 
The input to ACDLP (Algorithm~\ref{Alg:acdcl}) is a
program in the form of a set of abstract transformers
$\abstransset=\{\abstrans{\domain}{\sigma}|\sigma\in\Sigma\}$
w.r.t.\ an abstract domain~$\domain$.  Recall that the safety 
formula $\bigwedge_{\constraint\in\constraints} \constraint$ 
is unsatisfiable if and only if the program is safe.  
The algorithm is parametrised by heuristics for propagation $(\propheur)$, 
decisions $(\decheur)$, and conflict analysis $(\confheur)$.
Approximation of the concrete transformers in 
$\abstransset$ are typically available in abstract domain in the 
form of strongest-post condition or weakest pre-condition. 
The algorithm maintains a propagation trail $\trail$ and 
a reason trail~$\reasons$.
The propagation trail stores all meet irreducibles inferred by 
the abstract model search phase (deductions and decisions).  
The reason trail maps the elements of the propagation trail to the
transformers $\abstransel{}\in\abstransset$ that were used to
derive them. 
%
\begin{definition} 
The \emph{abstract value} $\abs(\trail)$ corresponding to 
the propagation trail $\trail$ is the conjunction of the 
meet irreducibles on the trail:
$\abs(\trail)=\bigsqcap_{m \in \trail}m$ with
$\abs(\trail)=\top$ if $\trail$ is the empty sequence.
\end{definition}
%
The algorithm begins with an empty $\trail$, a empty $\reasons$, and the
abstract value $\top$.  The procedure $\deduce$ (details in
Section~\ref{sec:deduce}) computes a greatest fixed-point over the
transformers in $\abstransset$ that refines the abstract value,
similar to the Boolean Constraint Propagation
step in SAT solvers.  If the result of $\deduce$
is \textsf{conflict} ($\bot$), the algorithm terminates with
\textsf{safe}.  Otherwise, the analysis enters into the while loop at line 4
and makes a new decision by a call to $\decide$ (see
Section~\ref{sec:decide}), which returns a new meet irreducible
$\decisionvar$.
%
We concatenate $\decisionvar$ to the trail~$\trail$.  The decision
$\decisionvar$ refines the current abstract value $\abs(\trail)$ represented
by the trail, i.e., $\abs(\trail\cdot\decisionvar)\sqsubseteq \abs(\trail)$.
%
For example, a decision in the interval domain restricts the range of 
intervals for variables.
%
We set the corresponding entry in the reason trail~$\reasons$ to $\top$
to mark it as a decision.
%
The procedure $\deduce$ is called next to infer new meet irreducibles
based on the current decision.  The model search phase
alternates between the decision and deduction until $\deduce$ returns
either \textsf{sat} or \textsf{conflict}.  
%
If  $\deduce$ returns  \textsf{sat}, then 
we have found an abstract value that represents models of the safety formula, which
are counterexamples to the required safety property, and so ACDLP return
\textsf{unsafe}.
%
If  $\deduce$ returns  \textsf{conflict}, 
the algorithm enters in the $\analyzeconflict$ 
phase (see Section~\ref{sec:conflict}) to learn the reason for the conflict.   There can be multiple
incomparable reasons for conflict.  ACDLP heuristically 
chooses one reason~$\conflictset$ and learns it 
by adding it as an abstract transformer to $\abstransset$. The analysis 
backtracks by removing the content of $\trail$ up to a point where it does not 
conflict with $\conflictset$.  ACDLP then performs deductions with the learned 
transformer.  If $\analyzeconflict$ returns $\false$, then no further
backtracking is possible.  Thus, the safety formula is unsatisfiable
and ACDLP returns \textsf{safe}.
}

%===============================================================================

%===============================================================================
%\section{Completeness of ACDLP}
%
Leopold thesis, Page 113, 
"The completeness argument relies on the so-called "asserting" property of
learned clause ... "

\section{Termination of ACDLP}
%
Follow from paper mcSAT (Similar argument)
(A Model-Constructing Satisfiability Calculus 
http://csl.sri.com/users/dejan/papers/mcsat-vmcai2013.pdf)

Termination discussion with Leopold
===================================
Termination —
acyclicity ,
progress in finite case — finite lattice explore the whole lattice
gamma-complete counterexample —
Leopold POPL paper terminatation argument follow for finite lattice — progress finitely many steps — there is infinite sequence — guarantee of widening is finite — cannot go infinitely for in the lattice — width of the lattice — buckjump in the lattice can not reach same search same infinitely
When I step down in lattice, given a decision operator that does not infinitely make decisions — infinite intervals — smart decision operator makes finite range over intervals, either leads to conflict or gamma-complete in finite number of steps

    no infinite sequence of decisions
    no infinitely buckjump
    d when decide, b when backtrack — there is no sequence of algorithm that generate infinite sequence of b and d alternate — as long as we don’t reach some same space in lattice, hence not cyclic
    either an interval is unsat, or there is one value wherein is satisfiable
    — bounded from above intervals

case 2: For infinite lattice
— SSA variables are finite, SSA values are finite

%===============================================================================

%===============================================================================
\section{Soundness and Termination of ACDLP}
%
We now present the \emph{soundness} and \emph{termination} proof of the ACDLP algorithm.  

%\para{Soundness of ACDLP}
%\rmcmt{Leo ACDL paper Theorem 11}

\begin{theorem} (Soundness).  If ACDLP returns \emph{safe}, then $f_{unsafe}^G$ 
deduces $\emptyset$, that is, there exists no trace that terminate in error 
location $\err$.  If ACDLP returns \emph{unsafe}, then $f_{unsafe}^G$ is not 
empty, that is, there exists at least one trace that terminate in $\err$.
\end{theorem}

\begin{proof}

\end{proof}


\para{Correctness of ACDLP}
Recall that a program $P$ is translated into a set of constraints, $\constraints$ 
and represented as a safety formula, $\formula$.  The formula $\formula$ is a 
conjunction of SSA constraints $\constraint \in \constraints$ which are derived 
from $P$ through syntactic translation steps. 


Partial correctness of ACDLP is trivial since the algorithm either terminates with 
a model of $\formula$ ($P$ is \emph{unsafe}) or deduce $\bot$ ($P$ is safe).  It 
only remains to guarantee termination of ACDLP.  The termination is guaranteed 
by three key conditions -- \emph{Progress}, \emph{Finite Closure}, and \emph{Absence 
of deadlock}.  The following theorem shows that given these three conditions, ACDLP 
always terminates.
%
\begin{theorem} (Termination).
Given a set of constraints $\constraints$, and a finite set of meet irreducibles 
in abstract domain $\domain$ from which all the clauses during the run of the ACDLP 
procedure can be constructed, any derivation starting from an initial abstract 
element $\top$ will terminate either with \emph{safe}, when the set of constraints 
$\constraints$ is satisfiable, or with \emph{unsafe}, when the set of constraints 
$\constraints$ is unsatisfiable. 
\end{theorem}
%
\begin{proof}
The progress condition guarantees that model search phase eventually terminates, that 
is, there is \emph{no} infinite sequence of decisions ($f_{dec}$) and deductions, 
($apost_s$).  Starting from $\top$, the model search deduce sequence of abstract values 
which follow some partial order such that $a \sqsupseteq a'$, where $a'$ is closer to the 
solution in some heuristic sense.  The deduction sequence terminates either in $\bot$ 
(conflict) or some abstract value $a^\dagger$ such that the set of constraints 
$\constraints$ is $\gamma$-complete in $a^\dagger$.  


The progress of model search is explained by defining a partial order on trails, similar to 
~\cite{mcsat1}. The trail $\trail$ in ACDLP contains two different kinds of elements -- 
decision $(q)$ and propagations $(p)$.  We define a $cost$ function that maps elements of 
the trail into the set $\{1,2\}$, such that $cost(q)=1$ and $cost(p)=2$.  The intuition 
here is that the cost of propagations are higher than the cost of decisions.  We define 
$\trail\prec \trail'$ based on lexicographic ordering on the cost of the trail 
elements, which is shown below.
\[
   a.\trail \prec b.\trail' = cost(a) < cost(b) \vee (cost(a) = cost(b) \wedge \trail \prec \trail')
\]

Clearly, an empty trail is the minimal element, that is, $\langle \rangle \prec \trail$ if 
$\trail \neq \langle \rangle$.  Adding a new element $(a)$ to the trail $\trail$ 
increases its cost or makes it \emph{bigger} with respect to the partial order $\prec$, that is, 
$\trail \prec \trail.a$.  This condition holds true for decision and propagation steps.   
When the procedure backjumps, then the configuration of trail changes from $\trail.q.Z$ to $\trail.p$.  
In this case, $cost(q) \prec cost(p)$ and hence the trail is bigger with respect to $\prec$ 
after backjumping.  Thus, each step of ACDLP produce a bigger trail with respect to the partial 
order and ensures progress. However, the trail does not increase forever and thus guarantees 
termination.  


A deadlock freedom property guarantees that a new learnt clause can be added to the 
initial set of constraints $\constraints$ when a conflict takes place.  Let us 
assume that there is a derivation that does not terminate, that is, we must execute 
learning infinitely often.  However, this is a contradiction since each learning step 
must add a new clause to $\constraints$ which blocks the model search from re-entering 
the conflicting search space again and $\constraints$ is contained in the finite 
closure of the initial set of constraints $\constraints$. 


%The conflict analysis procedure always terminates in a finite number of steps through
%repeated application of the transformer $\widehat{a_{pre}}$. During the process, it 
%learns a new clause, backtracks and removes previously deduced elements. 

\end{proof}

%===============================================================================

%===============================================================================
%\input{ssa}
%===============================================================================

%===============================================================================
%\section{Abstract Conflict Driven Learning for Programs}~\label{acdlp}
%
\todo{Be CAREFUL about transformer defined in concrete SA lattice, it is 
better to declare that these transformers simulate transformers in trace lattice}
Figure~\ref{acdlp-top} present our framework called \emph{Abstract Conflict 
Driven Learning for Programs} that uses abstract model search and abstract 
conflict analysis procedures for safety verification of programs.  The model
search procedure operates on an over-approximate abstract domain using sound deduction 
transformers such as strongest post-condition or existential pre-condition 
transformers (see section~\ref{modelsearch}).  When the deduction transformers cannot 
infer any further information and is not $\gamma$-complete (see 
section~\ref{modelsearch}), then a decision (see section~\ref{decision}) is made  
that refines the current abstract element.  The decision and deduction step continues 
until either a satisfying assignment is obtained (corresponding deduction transformer 
is $\gamma$-complete) or a conflict is encountered. In the former case, ACDLP terminates 
with a counterexample trace and the program is \emph{unsafe}. Recall that a counterexample 
trace is trace that reaches the error location $\err$. 

However, if a conflict is encountered, then it implies that the corresponding 
program trace is either not valid or safe.  ACDLP then moves to the conflict 
analysis phase (see section~\ref{conflict-analysis}) where it learns the 
reason for the conflict.  Recall that a SAT solver uses conflict resolution to derive 
the reason for conflict (see section~\ref{sat-learning}).  For efficiency 
reasons, SAT solver picks only one conflict reason.  Conflict analysis 
in ACDLP operates on an under-approximate domain using sound abductive transformers 
such as universal post-condition or weakest pre-condition transformers 
(see section~\ref{conflict-analysis}).  There can be multiple incomparable 
reasons for a conflict, but ACDLP heuristically picks one reason and generalizes it. 
Intuitively, this means that a partial safety proof for $\mathcal{S}$ is obtained 
by generalizing $\mathcal{S}$ to a set of safe \rmcmt{or invalid} traces 
$\mathcal{S'}$ such that the generalized conflict reason still preserves the 
reachability of the error location $\err$.  A generalized conflict reason basically 
contains the common prefix of the set of safe traces $\mathcal{S'}$. A learned clause, 
which is the complement of conflict reason, contains an overapproximation of the set of 
unsafe \rmcmt{or valid traces}.  Learned clauses are implicitly represented as 
transformers (see section~\ref{learning}).  An invariant of the ACDLP algorithm is 
that the the set of transformers after learning preseves the error reachability. 
ACDLP backjumps to a consistent state and model search is repeated with the new 
learned transformer which drives the search away from the conflicting region.  
However, if no further backtracking is possible, then ACDLP terminates and 
returns \emph{safe}.  


In the subsequent sections, we present a theoretical framework and mathematical 
recipe to build a precise abstract interpretation framework for functional safety 
property verification using Abstract Conflict Driven Clause Learning procedure. 
%
\begin{figure}
\centering
\scalebox{.70}{\import{figures/}{acdlp-top.pspdftex}}
\caption{ACDLP: Abstract Conflict Driven Learning for Programs \label{acdlp-top}}
\end{figure}
%
%===============================================================================
\section{Abstract Model Search in Programs}~\label{modelsearch}
%===============================================================================
%
A model search in CDCL solver alternates between two phases -- \emph{decisions} 
and \emph{boolean constraint propagation}, until a satisfying assignment is
obtained or a conflict is encountered. ~\cite{sas12} shows that BCP computes a
greatest fixed point by applying a unit rule which is the best abstract
transformer over a partial assignments domain.  Here, we characterise model
search as a procedure to find a counterexample trace in programs.  To do so, 
we present abstract model search in program as an instance of the Global Bottom 
Problem, shown in algorithm~\ref{gbp}.  We now present various transformers that 
are required to compute a fixed point approximation of concrete unsafe trace 
transformer.  
%-------------------------------------------------------------------------------
\para{Abstract Deduction Transformer}
%
Recall that $f_{unsafe}^{G}$ is a lower closure operator, which may be approximated 
by computing a greatest fixed point in the abstract.  
%
We now formally define a transformer, $f_{aunsafe}^G$, over
$\widehat{\mathcal{SA}_{G}}$, that uses strongest postcondition and 
existential precondition to perform forward, backward and multi-way 
analysis.  This transformer soundly approximate $f_{unsafe}^{G}$.  
We will call this \emph{abstract deduction transformer}.
%
\begin{definition} (Unsafe Trace Transformer in $\widehat{\mathcal{SA}}$). 
  $f_{aunsafe}^{\rightarrow}, f_{aunsafe}^{\leftarrow} : \widehat{\mathcal{SA}_{G}}
  \rightarrow \widehat{\mathcal{SA}_{G}}$
  \[
    f_{aunsafe}^{\rightarrow}(A) \mathrel{\hat=} \mathit{gfp}\; Z.\;
    apost_{\constraints}(A \meet Z)
    \quad
    f_{aunsafe}^{\leftarrow}(A) \mathrel{\hat=} \mathit{gfp}\; Z.\;
    apre_{\constraints}(A \meet Z)
  \]
   \[
      f_{aunsafe}^{G} = f_{aunsafe}^{\leftarrow}(A) \meet f_{aunsafe}^{\rightarrow}(A) 
   \]  
\end{definition}
%
%\paragraph{Property of Unsafe Trace Transformer}
%
\Omit{
However, each of the transformers 
above is computed using a least fixed point.  Hence, the result is a nested fixed point.  
The nested computation of a greatest fixed point using forward and backward analysis
based on least fixed points is a well-known technique in program
analysis~\cite{Cousot99}, and is used to necessarily increase precision in the
abstract.}
%
\begin{theorem}
  The transformers $f_{aunsafe}^{\rightarrow}$, $f_{aunsafe}^{\leftarrow}$ and 
  $f_{aunsafe}^{G}$ soundly approximate $f_{unsafe}^{G}$.
\end{theorem}
\begin{proof}
 We prove that $f_{aunsafe}^{\rightarrow}$ soundly overapproximates 
 $f_{unsafe}^{G}$. 
  We prove that  
  $ f_{unsafe}^{G} \circ \gamma_F \circ \gamma_S \circ \gamma \subseteq 
     \gamma_F \circ \gamma_S \circ \gamma \circ f_{aunsafe}^{\rightarrow} 
  $.
 Recall that $\widehat{SA}_{G}$ overapproximates the concrete
 $\powerset(\Pi)$ via the function $(\alpha \circ \alpha_S \circ \alpha_F,
 \gamma \circ \gamma_S \circ \gamma_F)$. Also, $apost_s$ soundly approximate
  $tpost$. For any element $g \in \widehat{\mathcal{SA}_{G}}$, the abstract unsafe trace
  transformer $f_{aunsafe}^{\rightarrow}$, $Z \mapsto apost_s(Z \meet A)$ 
  soundly approximates the concrete unsafe trace transformer
  $f_{unsafe}^{G}$, $Z \mapsto \gamma(g) \cap (\mathcal{I} \cup tpost(Z))$.
  Since $\mathcal{F}_G$ overapproximates $\powerset(\Pi)$ following
  Proposition~\ref{ag} and $\mathcal{SA}_G$ is exact to $\mathcal{F}_G$, so
  $\widehat{\mathcal{SA}_G}$ overapproximates $\powerset(\Pi)$.  From fixed
  point transfer theorem~\cite{fpt}, a trace 
  $\pi \in f_{unsafe}^{G} \circ \gamma_F \circ \gamma_S \circ \gamma$ also
  implies $\pi \in \gamma_F \circ \gamma_S \circ \gamma \circ
  f_{aunsafe}^{\rightarrow}$. The proof for $f_{aunsafe}^{\leftarrow}$ is similar. 
  %a set of traces computed by $f_{unsafe}^G$ is
  %therefore also obtained from $f_{aunsafe}^{\rightarrow}$.
  \rmcmt{
  Assume a trace $\pi \in f_{unsafe}^{G} \circ \gamma_F \circ \gamma_S \circ
  \gamma(a)$.
  Then $\pi$ is a counterexample trace such that for all $((id,v),val) \in a$
  and $(v,\omega) \in \pi$, it holds that $\omega \in \gamma_F \circ \gamma_S \circ 
  \gamma \circ a(id,v)$.  }
\end{proof}
%
%===============================================================================
%\input{gamma-complete}
%===============================================================================
%
%
\para{Generalized Decision Operator}~\label{decision}
%
An abstract model search heuristically searches for counterexample trace or a conflict.
This process generates a downward iteration sequence in the lattice of
$\widehat{\mathcal{SA}_{G}}$.  A decision in CDCL solver heuristically picks 
an unassigned variable and assigns a value to it.  
Similarly, a decision in ACDLP refines a downwards iteration sequence when the 
transformers $apost_{s}$ and $apre_{s}$ fails to make a refinement.  Given 
an abstract element $a$ obtained from the fixed point iteration, a decision $f_{dec}$ in 
$\widehat{\mathcal{SA}_{G}}$ heuristically chooses a meet irreducible $(a')$, such that 
the resultant element is strictly smaller than the greatest lower bound of the 
pair $(a, a')$.  We formally define a generalized decision operator over 
$\widehat{\mathcal{SA}_{G}}$. 
%
\begin{definition} (Generalized Decision Operator) 
  $f_{dec} \colon \widehat{\mathcal{SA}_{G}} \times \widehat{\mathcal{SA}_{G}}
  \rightarrow \widehat{\mathcal{SA}_{G}}$  
   \[ f_{dec}(a, a') = 
        a \sqcap a', \text{where}\; a\neq a' \wedge (a \sqcap a' \sqsubseteq a)
        \wedge (a \sqcap a' \sqsubseteq a')
   \]     
\end{definition}
%

%===============================================================================
\para{Global Bottom Problem}~\label{gbpg}
%===============================================================================
%
\begin{definition}
Given an abstract static assignment lattice, 
$(\widehat{\mathcal{SA}_G}, \sqsubseteq_{SA})$, a 
transformer 
$f_{aunsafe}^{G} : \widehat{\mathcal{SA}_{G}} \rightarrow \widehat{\mathcal{SA}_{G}}$ 
is \emph{globally bottom} if $f_{aunsafe}^G(a) = \bot$ for all $a$.  The global bottom 
problem is to determine if a transformer 
$f_{aunsafe}^G$ on a lattice $\widehat{\mathcal{SA}_{G}}$ is globally bottom.
\end{definition}

%However, $f_{aunsafe}^G$ is not globally bottom if there exists 
An element $a \in \widehat{\mathcal{SA}_{G}}$ is a non-$\bot$ witness if 
$f_{aunsafe}^G(a) \neq \bot$.  In this thesis, we consider the global bottom 
problem for completely additive, reductive transformers on powerset lattices.  
The result below follows directly from the soundness of abstract interpretation. 

\begin{theorem}~\label{fpt}
Given a completely additive and reductive transformer $f_{unsafe}^G$ on powerset of traces 
$(\powerset(\Pi), \subseteq)$, and 
$f_{aunsafe}^G \colon \widehat{\mathcal{SA}_{G}} \rightarrow \widehat{\mathcal{SA}_{G}}$ is 
a sound overapproximation of $f_{unsafe}^G$, and $\gamma_{SA}(gfp(f_{aunsafe}^G)) = \bot$, 
then the transformer $f_{unsafe}^G$ is globally bottom.
\end{theorem}

We now provide a condition to check whether $f_{unsafe}^G$ is not globally bottom. 
For this, it is 
sufficient to check whether $f_{aunsafe}^G$ is $\gamma$-complete even though the 
underlying abstract domain and the transformer may still be imprecise. 

\begin{proposition}~\label{gcf}
If $f_{aunsafe}^G$ is $\gamma$-complete at an element 
$a \in \widehat{\mathcal{SA}_G}$ and $\gamma_{SA}(f_{aunsafe}^G(a)) \neq \bot$, 
then $f_{unsafe}^G$ is not globally bottom.
\end{proposition}
%
Algorithm~\ref{gbp} presents a procedure for computing a non-$\bot$ witness 
to the global bottom problem. The algorithm takes as input an over-approximate 
abstract deduction transformer, $f_{aunsafe}^G$, a generalized decision transformer, 
$f_{dec}$, and a stack $\trail$ that record the results of transformer application, 
that is, $\trail$ contains elements of $\widehat{\mathcal{SA}_G}$.  The data-structure 
$\trail$ is also called \emph{trail}. The trail $\trail$ 
is initialized to $\top$.  The expression $\bigsqcap \trail$ denotes conjunction of 
all elements in $\trail$.  The expression $\trail \leftarrow \trail.a$ denotes concatenating 
$\trail$ with the new element $a$ which is pushed into $\trail$. 
The algorithm  checks if $\underset{d}{\forall}(f_{aunsafe}^G(d) = \bot)$ where $d \sqsubset_{SA} a$, 
that is, $f_{aunsafe}^G$ is bottom for every elements below $a \in \widehat{\mathcal{SA}_G}$.  
The output of the algorithm is a tuple consisting of the result of abstract deduction 
transformer $(\bot$, non-$\bot$, UNKNOWN$)$ and the final content of $\trail$. 
Following theorem~\ref{fpt}, if 
$\gamma_{SA}(gfp(f_{aunsafe}^G(a))) = \bot$, then $\underset{c}{\forall}(f_{unsafe}^G(c)=\bot)$ 
where $c \subset_{SA} \gamma_{SA}(a)$, that is, $f_{unsafe}^G$ is bottom on every elements $c$ 
that are below $\gamma_{SA}(a)$. \rmcmt{Note that, if the initial value of $a = \top$, and 
\texttt{abstract-counterexample-search} returns $\bot$, that is,  $\gamma_{SA}(gfp(f_{aunsafe}^G(\top))) = \bot$, then 
$f_{unsafe}^G$ is \emph{globally bottom}.  Else, for intial value of $a \neq \top$, 
$\gamma_{SA}(gfp(f_{aunsafe}^G(a))) = \bot$ corresponds 
to partially safety proof.} However, if $f_{aunsafe}^G$ is $\gamma$-complete at a fixed-point 
$a^\dagger$, and $\gamma_{SA}(f_{aunsafe}^G(a^\dagger)) \neq \bot$, then following proposition~\ref{gcf}, 
$f_{unsafe}^G$ is \emph{not globally bottom.}  If none of these conditions holds true, then a 
generalized decision operator $f_{dec}$ is applied, which jumps under the greatest fixed-point and 
checks if $f_{aunsafe}^G$ is $\bot$ on elements below the fixed-point.  This step is used to 
improve the precision of the analysis.
%
\begin{algorithm2e}[t]
\DontPrintSemicolon
\SetKw{return}{return}
\SetKwRepeat{Do}{do}{while}
%\SetKwFunction{assume}{assume}
%\SetKwFunction{isf}{isFeasible}
\SetKwData{conflict}{conflict}
\SetKwData{safe}{safe}
\SetKwData{sat}{sat}
\SetKwData{unsafe}{unsafe}
\SetKwData{unknown}{unknown}
\SetKwData{true}{true}
\SetKwInOut{Input}{input}
\SetKwInOut{Output}{output}
\SetKwFor{Loop}{Loop}{}{}
\SetKw{KwNot}{not}
\begin{small}
\Input{$f_{aunsafe}^G \colon \widehat{\mathcal{SA}_G} \rightarrow 
	\widehat{\mathcal{SA}_G}$, 
	$f_{dec} \colon \widehat{\mathcal{SA}_G} \times \widehat{\mathcal{SA}_G} 
	\rightarrow \widehat{\mathcal{SA}_G}$, 
	$\trail \colon$ elements of $\widehat{\mathcal{SA}_G}$}
	\Output{A tuple $(result, \trail)$, where result can be $\bot$, non-$\bot$, 
	or UNKNOWN}
        $a \leftarrow \bigsqcap \trail$ \;   
	\Do{$a = a^\dagger\; \text{or}\; \gamma_{SA}(a)=\bot$} {
           $a^\dagger \leftarrow a$ \;
	   $a \leftarrow a \meet_{SA} f_{aunsafe}^G(a) $ \;
	}
        \lIf{$\gamma_{SA}(a)=\bot$} {\return $(\bot,\trail)$} 
        $\trail \leftarrow \trail.a$ \;
	\lIf{$f_{aunsafe}^G\; \text{is}\; \gamma-complete\; \text{at}\; a$} 
	{\return (non-$\bot, \trail)$}
	$d \leftarrow f_{dec}$\;
	\lIf {$d=a$} {\return (unknown, $a)$}
	$\trail \leftarrow \trail.d$ \;
	\return abstract-counterexample-search$(f_{aunsafe}^G,f_{dec},\trail)$\;

\end{small}
\caption{Abstract Search for a non-$\bot$ witness of Global Bottom Problem 
	abstract-counterexample-search$(f_{aunsafe}^G,f_{dec},\trail)$ \label{gbp}}
\end{algorithm2e}
%

\begin{example}
%
\begin{figure}
\centering
\scalebox{.90}{\import{figures/}{semantic-example.pspdftex}}
\caption{An example CFG \label{fig:ex-ac}}
\end{figure}
%
Figure~\ref{model-search} shows an example run for the abstract model search procedure 
for the CFG in~Fig.\ref{fig:ex-ac} over Interval domain.  The elements obtained 
using $f_{aunsafe}^{G}$ transformer with strongest postcondition is marked 
in blue in Figure~\ref{model-search}.  
%The example demonstrates that model search computes a downward iteration sequence.  
Starting from $\top$, forward analysis concludes that $x$ is between -2 and 2
from $apost_{x:=-2} \cup apost_{x:=0} \cup apost_{x:=2}$.  Note that the loop is completely 
unwound and all statements corresponding to the loop are collectively referred to as $loop$. 
A forward fixed-point analysis (marked by $apost_{loop}$) does not yield any new 
information. Clearly, the analysis
is not precise to infer anything about the reachability of the error location $Error$. 
Hence, we apply a decision by picking a meet irreducible $y\geq 2$ to increase the 
precision of analysis.  We then apply forward analysis from this decision which
yields a downward iteration sequence as shown in lower part of
Fig.~\ref{model-search}.  Forward analysis concludes that $\{y \geq 4, x \geq 2, x \leq 2\}$. 
Clearly, $apost_{(y < 0)}(y \geq 4 \wedge x \geq 2 \wedge x \leq 2)$ leads to \emph{conflict}, 
which is marked as $\bot$ in Figure~\ref{model-search}. Hence, the error location $Error$ is 
unreachable for this decision. 
\end{example}
%
\begin{figure}[t]
\centering
\vspace*{-0.2cm}
\scalebox{.75}{\import{figures/}{model_search.pspdftex}}
  \caption{Model Search as Downward Iteration Sequence with Decisions and
  Deductions}
\label{model-search}
\end{figure}
%
%===============================================================================
\section{Abstract Conflict Analysis in Programs}~\label{conflict-analysis}
%===============================================================================
%
A conflict analysis procedure in CDCL solver finds the reason for 
a conflict by analyzing the deductions made during the 
model search phase through \emph{conflict resolution}~\cite{cdcl}. 
Conflict analysis in CDCL solver is different from DPLL solver -- 
CDCL allows non-chronological backjumping which can discard 
multiple levels of decisions and deductions trail, CDCL learns a 
reason for a conflict called \emph{conflict clause} that prevents 
the model search from re-entering into the conflicting search space 
in the future.  A conflict clause in CDCL solver is a clause that 
expresses the fact that some combinations of variable assignment 
are not valid.  Haller et. al. in~\cite{sas12} shows that conflict 
analysis in CDCL solver operate over an underapproximate domain, 
which is a downward closed set of partial assignments. 
%


Conflict analysis in ACDLP solver can be seen as abductive
reasoning~\cite{abd1,dhk2013-popl}. 
\Omit{A logic-based abductive inference~\cite{abd1} tries to find an explanation
$\mathcal{A}$ from a statement $\formula$ such that the truth of $\mathcal{A}$ 
is sufficient to guarantee the truth of $\formula$ and $\mathcal{A}$ is 
consistent with the background theory.}
An abductive inference is a dual of deductive inference that takes an input 
set $\allval$ contradicting a $\formula$ and find a weakest reason or 
explanation for $\allval$, assuming that $\formula$ is unsatisfiable.  

\Omit{
Figure~\ref{conflict-cdcl} shows a conflict
analysis procedure in CDCL solver.  Starting from the initial conflict, $\{p:t,
q:t, r:f, s:f\}$, the solver derives a generalized conflict reason $\{p:t\}$ 
which underapproximate the initial conflict reason, through Unique Implication 
Point (UIP) based conflict resolution algorithm~\cite{uip,cdcl}.  It is important 
to note that the conflict analysis in CDCL operates on the downset abstraction 
of partial assignments domain.  
%
\begin{figure}[htbp]
\centering
\vspace*{-0.2cm}
\scalebox{.85}{\import{figures/}{conflict_cut.pspdftex}}
\caption{Generalizing Conflict Reason in CDCL Solvers \label{conflict-cdcl}}
\end{figure}
%
}

A safe trace transformer, $f_{safe}^{G}$, computes a set of safe or
invalid traces.  Thus, the transformer $f_{safe}^G$ is completely 
multiplicative and extensive.  An effective conflict analysis for program requires a set of 
transformers that underapproximate $f_{safe}^{G}$, over the 
downset abstraction of $\widehat{\mathcal{SA}_{G}}$.  The requirement 
for downset abstraction is motivated by the conflict 
minimisation~\cite{DBLP:conf/sat/SorenssonB09}
technique used by SAT solvers that tries to generalize the reason 
for a conflict. Given a partial assignment $\pi$ that leads to a conflict, 
conflict minimisation replace $\pi$ with $\pi'$ such that $\pi$ can be derived 
from $\pi'$ following unit rule.  Note that, there can be multiple incomparable 
partial assignments that leads to the conflict, so a downsets of partial 
assignments have to be considered.  In practise, SAT solvers maintains only 
one conflict reason since generating all minimisations for a conflict can be 
ineffecient.  A lattice theoretic view of the sets of downsets obtained from 
a partial assignment $\pi = \{P:t, Q:t, R:f, S:f\}$ that leads to a conflict, 
is presented in figure~\ref{downset-abs}.  Given a lattice with $\top$ and $\bot$, 
the innermost diamond represents the partial assignment $\pi$.  Let us assume 
that each partial assignment, $\pi_1 = \{P:t\}$, $\pi_2 = \{Q:t, R:f\}$ and  
$\pi_3 = \{S:f\}$, obtained from $\pi$, are sufficient to derive the conflict.  
Now, each $\pi_i. i \in \{1,2,3\}$, forms a downset ($\mathbb{D}(\pi_i)$), 
denoted by respective triangles in figure~\ref{downset-abs}.  Further, $\pi_1$, 
$\pi_2$ and $\pi_3$ are incomparable, as is evident from their position in the 
lattice (peak of respective triangles).  A point to note here is that if 
$\pi_i$ leads to conflict, then every element that refines $\pi_i$ must 
belong to the downset of $\pi_i$ and must also contribute to the conflict.  
Hence, the notion of downset provides a lattice theoretic formulation of 
conflict analysis procedure in ACDLP.
%
\begin{figure}
\centering
\scalebox{.65}{\import{figures/}{downset-abstraction.pspdftex}}
  \caption{\label{downset-abs} Conflict Reasons as Set of Downsets}
\end{figure}  
%
%
\begin{figure}[htbp]
\centering
\vspace*{-0.2cm}
\scalebox{.85}{\import{figures/}{downsets.pspdftex}}
\caption{Lattice for Conflict Analysis \label{downset}}
\end{figure}
%


Here, we characterise conflict analysis as a procedure to find a 
generalized reason for conflict from a partial safety proof of a safe 
trace. To do so, we present abstract conflict analysis in program as 
an instance of the Global Top Problem, shown in algorithm~\ref{gtp}.  


We now present various transformers that are required to compute a 
fixed point approximation of concrete safe trace transformer.  
Recall that a program transformer $s \in \Sigma$ transforms the memory state of a
program and is associated with a transition relation,
$\mathcal{ST}_{s}^{\Omega}$.  We can associate each $s \in \Sigma$ with a weakest
precondition $\widehat{pre_{s}}$ and a universal postcondition
$\widehat{post_{s}}$. 
\Omit{
We now define these statement transformers. 
%
\begin{definition} (Universal Postcondition and Weakest Precondition Transformers).
  \[
    \widehat{post_{s}}(A) \mathrel{\hat=} \{\omega' \mid \forall \omega \in
     \Omega. \omega \in A \vee (\omega,\omega') \not\in \mathcal{ST}_{s}^{\Omega}\} 
  \] 
  \[ 
     \widehat{pre_{s}}(A) \mathrel{\hat=} \{\omega \mid \forall \omega' \in
     \Omega. \omega' \in A \vee (\omega,\omega') \not\in \mathcal{ST}_{s}^{\Omega}\} 
  \]
\end{definition}
}
%
%
An abstract conflict analysis for program computes an underapproximation of 
the concrete safe trace transformer $f_{safe}^G$.
To compute an underapproximation of $f_{safe}^{G}$, we underapproximate the
state transformers, $\widehat{post_s}$ and $\widehat{pre_s}$.  Furthermore, 
the lattice  $\mathcal{SA}_{G}$ is underapproximated by the downset 
completion $\mathbb{D}(\widehat{\mathcal{SA}_{G}})$, which is the set of 
downsets of $\mathcal{SA}_{G}$, as shown in figure~\ref{downset}.  Note 
that a downset completion enriches a domain with disjunctions.  Elements of 
downset are abstracted using an abstraction function, $\alpha_D$, which 
overapproximates  
\rmcmt{In this thesis, we treat downsets as underapproximating abstractions.}

%The lattice used for conflict analysis is shown in Figure~\ref{downset}.
%
\begin{proposition}
\[
   (\mathcal{SA}_{G},\supseteq_{SA})
   \galois{\alpha_{D}}{\gamma_{D}}
   (\mathbb{D}(\widehat{\mathcal{SA}_{G}}),\sqsupseteq_{SA}^\dagger) 
\]
  \[
    \alpha_{D}(a) \mathrel{\hat=} \{a' \in \widehat{\mathcal{SA}_G} \mid \gamma_{SA}(a')
    \subseteq_{SA} a \}
    \qquad
    \gamma_{D}(b) \mathrel{\hat=} \{\bigcup_{b' \in decomp(b)}
    \gamma_{SA}(b')\}
  \]
\end{proposition}
%
%The pair $(\alpha_{D},\gamma_{D})$ forms a galois connection, 
%which follows directly from the proof of~\cite{Cousot92}.  
\rmcmt{We assume that $\gamma_D(\top) = \Pi$.}

\begin{proof} 
  We now prove that the pair $(\alpha_{\mathbb{D}},\gamma_{\mathbb{D}})$ forms a galois connection. 
  It is straightforward to see that $(\alpha_{\mathbb{D}}$ and $\gamma_{\mathbb{D}})$ are monotone. 
  We show that $(\alpha_{\mathbb{D}} \circ \gamma_{\mathbb{D}})$ is extensive and 
  $(\gamma_{\mathbb{D}} \circ \alpha_{\mathbb{D}})$ is reductive. 
 
  Let us assume that $a \in \mathbb{D}(\widehat{\mathcal{SA}_G})$.  Then, 
  $\gamma_D(a) \supseteq_{SA} \gamma_{SA}(a)$, and thus 
  $a \in \alpha_{\mathbb{D}} \circ \gamma_{\mathbb{D}}$.  Thus,  
  $\alpha_{\mathbb{D}} \circ \gamma_{\mathbb{D}}$ is extensive. 
  
  
  Let $c \in \gamma_D \circ \alpha_D(c')$. Then, there exists an element $a \in \alpha_D(c')$ 
  such that $c \in \gamma_{SA}(a)$ and $\gamma_{SA}(a) \subseteq_{SA} c'$. Therefore, 
  $c \in \gamma_D(a)$.  Thus, $\gamma_{\mathbb{D}} \circ \alpha_{\mathbb{D}}$ is reductive. 
\end{proof}
%
Let $\widehat{apre_s}$ and $\widehat{apost_s}$ be sound underapproximations of
weakest precondition transformer $\widehat{pre_s}$ and universal postcondition 
transformer $\widehat{post_s}$ over $\mathbb{D}(\widehat{\mathcal{SA}_{G}})$. 
%
The global abstract static assignment transformers for the 
lattice $\mathbb{D}(\widehat{\mathcal{SA}_G})$ are obtained 
from the underapproximate 
abstract state transformers, $\widehat{apost_s}$, $\widehat{apre_s}$. 
This is defined next.
%
\begin{definition} (Global Underapproximate Abstract Static Assignment
  Transformers for $\mathbb{D}(\widehat{\mathcal{SA}_G})$). 
  \[ 
     \widehat{apost_{\constraints}}, \widehat{apre_{\constraints}} : 
     \mathbb{D}(\widehat{\mathcal{SA}_G}) \rightarrow
     \mathbb{D}(\widehat{\mathcal{SA}_{G}}) 
   \]
   \[
     \widehat{apost_{\constraints}}(a) \mathrel{\hat=} 
     \underset{\sigma \in \constraints}{\bigsqcap} \widehat{apost_{\sigma}} \circ a 
   \]
  \[
    \widehat{apre_{\constraints}}(a) \mathrel{\hat=} 
    \underset{\sigma \in \constraints}{\bigsqcap} \widehat{apre_{\sigma}} \circ a 
   \]
\end{definition}
%
The transformers, $\widehat{apost_{\constraints}}$ and
$\widehat{apre_{\constraints}}$, soundly underapproximate 
their concrete counterparts, $\widehat{post_{\constraints}}$ 
and $\widehat{pre_{\constraints}}$ respectively. 
%


A conflict reason is derived by analyzing the deductions made 
from $f_{aunsafe}^{G}$ during the model search phase.  
%Recall that the deductions are obtained by approximating 
%a least fixed point using the strongest post-condition.  
Conflict analysis is performed by computing a
least fixed point of a weakest pre-condition transformer or an universal
postcondition transformer in the downset abstract domain
$\mathbb{D}(\widehat{\mathcal{SA}_{G}})$.  For this, we define
a transformer $f_{asafe}^G$ in $\mathbb{D}(\widehat{\mathcal{SA}_{G}})$, that 
is a sound under-approximation of completely multiplicative and 
extensive transformer, $f_{safe}^{G}$.  Below, $A$ is the downset 
closure of the original conflict reason. 
%
\begin{definition} (Safe Trace Transformers in
  $\mathbb{D}(\widehat{\mathcal{SA}})$). 
  \[
    f_{asafe}^{\rightarrow}, f_{asafe}^{\leftarrow} :
  \mathbb{D}(\widehat{\mathcal{SA}_{G}})
  \rightarrow \mathbb{D}(\widehat{\mathcal{SA}_{G}})
  \]
  \[
    f_{asafe}^{\rightarrow}(A) = \mathit{lfp}\; Z.\;
    \widehat{apost_{\constraints}}(A \join Z)
    \quad
    f_{asafe}^{\leftarrow}(A) = \mathit{lfp}\; Z.\; 
    \widehat{apre_{\constraints}}(A \join Z)
  \] 
\end{definition}
%
%
%===============================================================================
\para{Global Top Problem}
%===============================================================================
%
\begin{definition}
Given a downset lattice, 
$(\mathbb{D}(\widehat{\mathcal{SA}_G}), \sqsupseteq_{SA}^\dagger)$, 
a transformer 
$f_{asafe}^{G} : \mathbb{D}(\widehat{\mathcal{SA}_{G}}) \rightarrow \mathbb{D}(\widehat{\mathcal{SA}_{G}})$ 
is \emph{globally top} if $f_{asafe}^G(a) = \top$ for all 
$a \in \mathbb{D}(\widehat{\mathcal{SA}_G})$.  The global top
problem is to determine if a transformer $f_{asafe}^G$ on a lattice 
$\mathbb{D}(\widehat{\mathcal{SA}_{G}})$ is globally top.
\end{definition}
%
An element $a \in \mathbb{D}(\widehat{\mathcal{SA}_{G}})$ is a 
non-$\top$ witness if $f_{asafe}^G(a) \neq \top$.  Algorithm~\ref{gtp} 
present a procedure for checking if $f_{safe}^G$ is globally top.  
\rmcmt{If the result of $lfp(f_{asafe}^G)$ concretizes to $\top$, then we 
can infer the following.
\begin{enumerate}
	\item $f_{safe}$ is globally top.
	\item $f_{unsafe}$ is globally bottom since $gfp(f_{aunsafe}^G)$ concretizes to $\bot$
\end{enumerate}
}
%
Given a conflict, the procedure \texttt{analyze-partial-safety-proof} of Algorithm~\ref{gtp} is used to generalize 
an element that leads to the conflict.  To do so, it computes an 
under-approximation of the least fixed point using the upwards interpolation, 
$int \upharpoonright$.  Recall that upwards interpolation on a lattice $L$ 
is a function $int \upharpoonright \colon L \times L \rightarrow L$ such 
that $a \sqsubseteq b \implies a \sqsubseteq int \upharpoonright(a,b) \sqsubseteq b$ for all $a,b \in L$. 
To avoid considering sets of all generalizations of a conflict, $u \;int \upharpoonright \; a$ 
(in line 2 of Algorithm~\ref{gtp}) is used as a choice function, where $u$ and $a$ are downsets. 
Line 3 of Algorithm~\ref{gtp} suggest that if the weakest precondition of $\bot$ concretized to $\top$, 
then the $f_{safe}$ is globally top.


%
\begin{algorithm2e}[t]
\DontPrintSemicolon
\SetKw{return}{return}
\SetKwRepeat{Do}{do}{while}
%\SetKwFunction{assume}{assume}
%\SetKwFunction{isf}{isFeasible}
\SetKwData{conflict}{conflict}
\SetKwData{safe}{safe}
\SetKwData{sat}{sat}
\SetKwData{unsafe}{unsafe}
\SetKwData{unknown}{unknown}
\SetKwData{true}{true}
\SetKwInOut{Input}{input}
\SetKwInOut{Output}{output}
\SetKwFor{Loop}{Loop}{}{}
\SetKw{KwNot}{not}
\begin{small}
	\Input{$f_{asafe}^G \colon \mathbb{D}(\widehat{\mathcal{SA}_G}) \rightarrow 
	\mathbb{D}(\widehat{\mathcal{SA}_G})$, 
	$int \upharpoonright \colon \mathbb{D}(\widehat{\mathcal{SA}_G}) \times \mathbb{D}(\widehat{\mathcal{SA}_G}) 
	\rightarrow \mathbb{D}(\widehat{\mathcal{SA}_G})$, 
	$u \in \mathbb{D}(\widehat{\mathcal{SA}_G})$}
	\Output{A tuple $(result, t)$, where result can be $\top$, non-$\top$, 
	or UNKNOWN and $t \in \mathbb{D}(\widehat{\mathcal{SA}_G})$}

	$a \leftarrow u \sqcup_{SA}^\dagger f_{asafe}^G(u)$ \;
	$t \leftarrow u\; int \upharpoonright \; a$ \;
	
	\lIf{$\gamma_{D}(t)=\top$} {\return $(\top,t)$} 
        \lIf{$f_{asafe}^G$ is $\gamma$-complete at $t$} {\return (non-$\top$, $t$)} 
	\lIf{t=u} {\return (unknown, t)}
	{\return analyze-partial-safety-proof$(f_{asafe}^G, int \upharpoonright,t)$}
\end{small}
	\caption{Abstract Search for a non-$\top$ witness of Global Top Problem in $\mathbb{D}(\widehat{\mathcal{SA}_G})$,  
	analyze-partial-safety-proof$(f_{asafe}^G,int \upharpoonright,u)$ \label{gtp}}
\end{algorithm2e}
%

\begin{example}
  Let us revisit the example in Figure~\ref{fig:ex-ac} and the corresponding
  deductions in Figure~\ref{model-search}. The conflict analysis procedure is
  shown in Figure~\ref{conflict-example}.  We iteratively apply $\widehat{apre}$ 
  starting from the conflict element ($\bot$), the result of which is shown in
  bold text.  For example, $\widehat{apre_{y < 0}}(\bot)= \{y \geq 0\}$; whereas the
  result of $f_{aunsafe}^{G}$ transformer application via strongest postcondition 
  is $\{y\geq 4\}$.  So, we heuristically pick a generalized element $a$ such 
  that $\{y\geq 4\} \sqsubseteq a \sqsubseteq \{y \geq 0\}$; we pick $a=\{y \geq 0\}$ 
  through the application of upwards interpolation~\cite{leo-thesis},  
  \Omit{(corresponds to relaxation of narrowing operation in abstract interpretation)}
  $int\upharpoonright(y \geq 0, y \geq 4)$, marked in blue.  The heuristic 
  generalization of conflict reason for different abstract domains is explained 
  in section~\ref{heu-gen}.
  Note that the loop is completely unwound and all statements corresponding to the 
  loop are collectively referred to as $loop$.
  We then repeat the process marked by $\widehat{apre_{loop}}$. Subsequently, we derive 
  a generalized reason, $\{x>0, y \geq 0\}$, that strictly generalizes the decision 
  $(y \geq 2)$.  Hence, the partial safety proof corresponding to the set of traces 
  with prefix $\{y \geq 2\}$ gives us a generalized prefix that 
  includes the set of all safe traces with prefixes $\{x \geq 0, y \geq 0\}$. 
  Note that the concrete safe trace transformer $f_{safe}^G$ returns all safe traces which 
  includes traces with prefixes $\{ y < 0, y \geq 0 \}$.  However, $f_{aunsafe}^G$ returns 
  a conflict reason which underapproximates $f_{safe}^G$, that is, it does not include 
  the set of safe traces with prefix $\{y<0\}$, but generalizes the partial safety proof 
	for the set of safe traces with prefix $\{y \geq 2\}$. \todo{define prefix of trace}.
  %\rmcmt{define upwards interpolation}
\end{example}
%
\begin{figure}[t]
\centering
\vspace*{-0.2cm}
\scalebox{.70}{\import{figures/}{conflict_example.pspdftex}}
  \caption{Conflict Analysis with underapproximate weakest precondition and
  upwards interpolation}
\label{conflict-example}
\end{figure}
%
\subsection{Generalized Unit rule}~\label{learning}
%
Clause learning in ACDLP learns new abstract transformers $(AUnit)$ 
which are implicitly represented as clauses.  
%Since unit rule is the core component of the model search phase in SAT solvers, 
%so clause learning can be viewed as learning a unit-rule transformer, $(AUnit)$.  
Learning refines the transformer $f_{aunsafe}^G$ with the transformer $AUnit$. 
Intuitively, transformer learning through conflict analysis makes $f_{aunsafe}^G$ a 
closer approximation of the unsafe trace transformer, $f_{unsafe}^{G}$.  
%
The transformer $AUnit$ is a generalisation of the propositional unit rule~\cite{cdcl}.  
Recall that $\formula$ is a safety formula that is obtained through 
conjunction of all SSA elements in $\Sigma^{\dagger}$. 
For an abstract lattice $\widehat{\mathcal{SA}_{G}}$ with
complementable meet irreducibles and a set of meet irreducibles $\conflictset
\subseteq \widehat{\mathcal{SA}_{G}}$ such that $\bigsqcap
\conflictset$ does not satisfy $\formula$, $\aunit_\conflictset:
\widehat{\mathcal{SA}_G} \rightarrow \widehat{\mathcal{SA}_G}$ 
is formally defined as follows.
\[ \aunit_\conflictset(\absval) =
 \left\{\begin{array}{l@{\quad}l@{\qquad}l}
  \bot       & \text{if } \absval \sqsubseteq \bigsqcap \conflictset & (1)\\
  \bar{t}    & \text{if } t \in \conflictset \; \text{and} \; \forall t' \in \conflictset
  \setminus \{t\}. \absval  \sqsubseteq t' & (2) \\
  \top & \text{otherwise} & (3) \\
 \end{array}\right.
\]
%
Rule (1) shows $\aunit$ returns $\bot$ since 
$\absval \sqsubseteq \bigsqcap \conflictset$ is conflicting.  
Rule (2) of $\aunit$ infer a valid meet irreducible, 
which implies that $\conflictset$ is unit.  Rule (3) of  
$\aunit$ returns $\top$ which implies that the learned clause is not 
{\em asserting} after backtracking.  This would prevent any new 
deductions from the learned clause. Progress is then made by decisions. 
%
\begin{figure}[htbp]
\centering
\vspace*{-0.2cm}
\scalebox{.70}{\import{figures/}{learning.pspdftex}}
\caption{Downward Iteration Sequence with Learned transformer \label{learning}}
\end{figure}
%
\begin{example}
Fig.~\ref{learning} shows the sequence of fixed point iteration with the learned
transformer $y \leq 0$, obtained from $AUnit$.  Clearly, this also leads to 
\emph{conflict}. There are no further cases to explore. Thus, the procedure 
terminates and returns \emph{safe}.
\end{example}
%
\para{Algorithm for ACDLP}
%

{\begin{algorithm2e}[t]
\DontPrintSemicolon
\SetKw{return}{return}
\SetKwRepeat{Do}{do}{while}
%\SetKwFunction{assume}{assume}
%\SetKwFunction{isf}{isFeasible}
\SetKwData{conflict}{conflict}
\SetKwData{safe}{safe}
\SetKwData{sat}{sat}
\SetKwData{unsafe}{unsafe}
\SetKwData{unknown}{unknown}
\SetKwData{true}{true}
\SetKwInOut{Input}{input}
\SetKwInOut{Output}{output}
\SetKwFor{Loop}{Loop}{}{}
\SetKw{KwNot}{not}
\begin{small}
	\Input{
        $f_{aunsafe}^G \colon \widehat{\mathcal{SA}_G} \rightarrow 
	\widehat{\mathcal{SA}_G}$, 
	$f_{dec} \colon \widehat{\mathcal{SA}_G} \times \widehat{\mathcal{SA}_G} 
	\rightarrow \widehat{\mathcal{SA}_G}$, 
	$f_{asafe}^G \colon \mathbb{D}(\widehat{\mathcal{SA}_G}) \rightarrow 
	\mathbb{D}(\widehat{\mathcal{SA}_G})$, 
	$int \upharpoonright \colon \mathbb{D}(\widehat{\mathcal{SA}_G}) \times \mathbb{D}(\widehat{\mathcal{SA}_G}) 
	\rightarrow \mathbb{D}(\widehat{\mathcal{SA}_G})$ 
	}
	\Output{$\bot$ if $f_{aunsafe}^G$ is globally bottom, else 
	a counterexample $(\bigsqcap \trail)$}
        
	$\trail \leftarrow \top$ \;

	\Do{$\trail$ is empty} {
	   $(d,\trail) \leftarrow$ abstract-counterexample-search$(f_{aunsafe}^G,f_{dec},\trail)$ \;
	   \lIf{$d \neq \bot$} {\return $(d,\bigsqcap \trail)$} 
	   $a \leftarrow \bigsqcap \trail$ \; 
	   $(d,{t}) \leftarrow$ analyze-partial-safety-proof$(f_{asafe}^G,int \upharpoonright, a)$ \;
	   $learn \leftarrow AUnit_{t}$ \;
	   $\trail \leftarrow$ backjump$(learn,f_{aunsafe}^G,\trail)$ \;
	   $f_{aunsafe}^G \leftarrow f_{aunsafe}^G \sqcap learn$ \;
	}
	{\return \rmcmt{$\bot$}}
\end{small}
	\caption{Abstract Conflict Driven Clause Learning (ACDLP) \label{acdlp-algo}}
\end{algorithm2e}
%
Algorithm~\ref{acdlp-algo} describes the working of Abstract Conflict Driven Learning for Programs (ACDLP).  
The procedure \texttt{abstract-counterexample-search} procedure in ACDLP operates over an overapproximate domain $\mathcal{SA}_G$ 
with complementable meet irreducibles, whereas \texttt{analyze-partial-safety-proof} procedure operates over an 
underapproximate downset completion domain, $\mathbb{D}(\mathcal{SA}_G)$.  ACDLP starts by initializing $\trail$ to 
a singleton sequence $\top$. The outer loop of the algorithm terminates when $\trail$ is empty. In this case, 
\texttt{analyze-partial-safety-proof} returns $\top$ and there are no further cases to be explored, hence ACDLP 
terminates.  However, if \texttt{abstract-counterexample-search} returns non-$\bot$, then ACDLP terminates a 
counterexample $(\bigsqcap \trail)$.  The invariant of the ACDLP algorithm is that the transformer 
$f_{aunsafe}^G$ is a sound overapproximation of the concrete unsafe trace transformer $f_{unsafe}^G$.


Figure~\ref{acdlp-fig} describes the communication between the Abstract-Counterexample-Search (shown in left box) and Analyze-Partial-Safety-Proof (shown in right box) of ACDLP algorithm.  The conflicting elements obtained from a conflict are passed from Abstract-Counterexample-Search to Analyze-Partial-Safety-Proof.
Whereas, a learned transformer $learn$ is transferred from Analyze-Partial-Safety-Proof to Abstract-Counterexample-Search which 
refines the transformer $f_{aunsafe}^G$.  ACDLP backjumps after learning using the function \texttt{backjump}.  The backjump is 
asserting which implies that the learned transformer becomes \emph{unit} after backjumping, that is, the application of $learn$ 
transformer generates new deductions which guides the search away from the conflicting region.   The output of ACDLP is \emph{UNSAFE} 
in Figure~\ref{acdlp-fig} if $f_{aunsafe}^G$ is not globally bottom.  However, the output is \emph{SAFE} if $f_{aunsafe}^G$ is globally bottom.    
%
%
\begin{figure}[htbp]
\centering
\vspace*{-0.2cm}
\scalebox{.85}{\import{figures/}{acdcl.pspdftex}}
\caption{Abstract Interpretation Framework for Precise Safety Verification \label{acdlp-fig}}
\end{figure}
%
%

\para{Backjumping}
%
{\begin{algorithm2e}[t]
\DontPrintSemicolon
\SetKw{return}{return}
\SetKwRepeat{Do}{do}{while}
%\SetKwFunction{assume}{assume}
%\SetKwFunction{isf}{isFeasible}
\SetKwData{conflict}{conflict}
\SetKwData{safe}{safe}
\SetKwData{sat}{sat}
\SetKwData{unsafe}{unsafe}
\SetKwData{unknown}{unknown}
\SetKwData{true}{true}
\SetKwInOut{Input}{input}
\SetKwInOut{Output}{output}
\SetKwFor{Loop}{Loop}{}{}
\SetKw{KwNot}{not}
\begin{small}
	\Input{
	$learn \in \widehat{\mathcal{SA}_G}$, 
	$f_{aunsafe}^G \colon \widehat{\mathcal{SA}_G} \rightarrow 
	\widehat{\mathcal{SA}_G}$, 
	$\trail \colon$ elements of $\widehat{\mathcal{SA}_G}$}
 	
	\Output{$\trail$ after backtracking}
	
        $f_{aunsafe}^G \leftarrow f_{aunsafe}^G \meet learn$ \; 
	\Do{$\gamma_{SA}(f_{aunsafe}^G(\bigsqcap \trail)) = \bot$} {
		$\trail \leftarrow$ pop$(\trail)$ \;
	}
	{\return $\trail$}
\end{small}
	\caption{backjump $(learn, f_{aunsafe}^G, \trail)$ \label{backjump}}
\end{algorithm2e}
%
Algorithm~\ref{backjump} removes elements from $\trail$ as long as  
$\gamma_{SA}(f_{aunsafe}^G(\bigsqcap \trail)) = \bot$ holds true.  The 
function \emph{pop} is used to remove the top element of the trail $\trail$. 
The algorithm returns a modified $\trail$ that resets the state of $\trail$ 
before the element that is negated by the transformer $learn$, that is, the 
state of trail where $\gamma_{SA}(f_{aunsafe}^G(\bigsqcap \trail)) \neq \bot$ 
holds true.  If $\trail$ is empty after backjumping, then $f_{aunsafe}^G$ is 
globally bottom. 


\para{Asserting Backjump} The function \texttt{backjump}$(learn,f_{aunsafe}^G, \trail)$ 
is \emph{asserting} if the output trail $\trail'$ returned by \texttt{backjump} 
satisfies either of the following two conditions.
\begin{enumerate}
	\item $\trail' = \langle \rangle$, that is, $\trail'$ is empty
	\item \texttt{learn}$(\bigsqcap \trail')$ and $(\bigsqcap \trail)$ 
		are not comparable, where $\trail$ and $\trail'$
		are the trail before and after backjumping, respectively.
\end{enumerate}
%
Condition (2) above denotes that asserting backjumping drives the model search 
away from the conflicting region, that is, the trail $\trail'$ obtained after 
learning through \texttt{learn} transformer contains newly deduced elements 
that drives the model search into a new part of the search space. We now show 
that $learn(\bigsqcap \trail')$ and $(\bigsqcap \trail)$ are unordered 
in the lattice $\widehat{\mathcal{SA}_G}$.  We denote the transformer after learning 
as $f_{aunsafe}^{G \dagger}$, where $f_{aunsafe}^{G \dagger} = f_{aunsafe}^{G} \sqcap learn$.
%
\begin{lemma}~\label{bjump}
A run of ACDLP algorithm with asserting backjumping maintains the 
state of the trail in a way that the abstract element $\bigsqcap \trail$ 
from trail $\trail$ before backjumping is incomparable to the abstract 
element $f_{aunsafe}^{G \dagger}(\bigsqcap \trail')$ from trail $\trail'$ 
after backjumping. 
\end{lemma}
%
\begin{proof}
Given $\trail$ and $\trail'$ represent the trail before and after backjumping, 
respectively.  Let $a = \bigsqcap \trail$ and $a' = \bigsqcap \trail'$.  Recall 
that \texttt{analyze-partial-safety-proof} returns a tuple $(d, t)$, where 
$t \sqsupseteq_{SA} a$.  The generalized unit rule transformer $AUnit$ with 
respect to $t$ returns $AUnit_t(a') = a' \sqcap_{SA} \bar{l}$ after backjumping. 
Here, $\bar{l}$ is a meet irreducible of $\widehat{\mathcal{SA}_G}$.  It follows 
that $f_{aunsafe}^{G \dagger}(a') \sqsubseteq_{SA} \bar{l}$ and $l \sqsupseteq_{SA} t$, 
so $l \sqsupseteq_{SA} a$ also holds true.  We decompose the proof into two separate 
cases, each of which are proven by contradiction. \\


Case 1: Assume that $(\bigsqcap \trail) \sqsubseteq_{SA}$ \ltrans$(\bigsqcap \trail')$. 
That is, $a \sqsubseteq_{SA}$ \ltrans$(a')$. We can infer the following.
\begin{enumerate}
	\item $a \sqsubseteq_{SA}$ \ltrans$(a')$ implies $a \sqsubseteq_{SA} \bar{l}$. 
	\item $a \sqsubseteq_{SA} t$ and $t \sqsubseteq_{SA} l$ implies $a \sqsubseteq_{SA} l$.
\end{enumerate}
From the above, $\gamma_{SA}(a) = \bot$. This is a contradiction since 
\texttt{abstract-counterexample-search} never returns an empty trail, that is, a trail representing an empty set.  We 
not show that other side of the proof. \\

Case 2: Assume that $(\bigsqcap \trail) \sqsupseteq_{SA}$ \ltrans$(\bigsqcap \trail')$. 
That is, $a \sqsupseteq_{SA}$ \ltrans$(a')$. We can infer the following.
\begin{enumerate}
  \item From $a \sqsubseteq_{SA} l$, we can infer \ltrans$(a') \sqsubseteq_{SA} l$.
  \item We already know that \ltrans$(a') \sqsubseteq \bar{l}$.
\end{enumerate}
From the above, $\gamma_{SA}$(\ltrans$(a')) = \bot$. This is a contradiction since 
\texttt{backjump} never returns a conflicting trail. \\ 
	
Thus, we conclude that 
$(\bigsqcap \trail) \not\sqsubseteq_{SA}$ \ltrans$(\bigsqcap \trail')$ and 
\ltrans$(\bigsqcap \trail') \not\sqsubseteq_{SA} (\bigsqcap \trail)$. Hence, 
$(\bigsqcap \trail)$ and \ltrans$(\bigsqcap \trail')$ are unordered. 
\end{proof}
%
\begin{lemma}~\label{sound-transformer}
For a completely additive and reductive transformer $f_{unsafe}^G$ on powerset of traces 
$(\powerset(\Pi), \subseteq)$, let
$f_{aunsafe}^G \colon \widehat{\mathcal{SA}_{G}} \rightarrow \widehat{\mathcal{SA}_{G}}$ be 
a sound overapproximation of $f_{unsafe}^G$.  Let
$f_{asafe}^G \colon 
\mathbb{D}(\widehat{\mathbb{SA}_{G}}) \rightarrow \mathbb{D}(\widehat{\mathcal{SA}_{G}})$ be 
a sound underapproximation of a completely multiplicative and extensive transformer 
$f_{safe}^G$. Given an abstract element $a$, 
if $gfp(f_{aunsafe}^G)(a) = \emptyset$, then $f_{aunsafe}^G \sqcap learn$ 
is a sound overapproximation of $f_{unsafe}^G$. 
\end{lemma}

\begin{proof}[Proofsketch]
The procedure \texttt{analyze-partial-safety-proof} in ACDLP computes a least fixed-point
of the transformers $\widehat{apost}_{\Sigma^\dagger}$ and $\widehat{apre}_{\Sigma^\dagger}$.
The procedure returns an element $(d, t)$, such that $learn = AUnit_{t}$. The transformer 
$f_{aunsafe}^G \sqcap learn$ preserves the error reachability.  This can be explained as 
follows. For the given abstract element $a$, if $gfp(f_{aunsafe}^G)(a) = \emptyset$ and 
$\trail$ be the trail which leads to $\bot$, $\trail'$ be the trail after backjump, then
$\sqcap\trail$ is incomparable to $(f_{aunsafe}^G \sqcap learn)(\sqcap\trail')$ 
following lemma~\ref{bjump}. Thus, the transformer $AUnit_t$ drives the unsafe trace 
transformer $f_{aunsafe}^G$ away from the conflict, thereby preserving the error rechability. 
\end{proof}

\section{Putting it all together}
%
\begin{figure}[htbp]
\centering
\vspace*{-0.2cm}
\scalebox{.80}{\import{figures/}{schema.pspdftex}}
\caption{Building Blocks for Instantiating CDCL Architecture for Program Verification \label{schema}}
\end{figure}
%
Figure~\ref{schema} gives the basic building blocks for instantiating CDCL 
architecture for program verification using abstract interpretation framework.  
Elements of $\widehat{\mathcal{SA}_{G}}$ forms a lattice of approximation of program traces
ordered by $\sqsubseteq_{SA}$ where each element $a \in \widehat{\mathcal{SA}_{G}}$ 
represents a memory state of the program.  Each statement $s$ in the program
defines four transformers, strongest postcondition $post_s$, existential precondition 
$pre_s$, universal postcondition $\widehat{post_s}$, and weakest precondition 
$\widehat{pre_s}$.  Model search computes an overapproximation of unsafe trace 
transformer using fixed point over $f_{aunsafe}^{\leftarrow}$, $f_{aunsafe}^{\rightarrow}$ 
and decisions operator $f_{dec}$ to improve precision.  
Whereas, conflict analysis computes an underapproximation of safe trace transformer 
using fixed point over $f_{asafe}^{\leftarrow}$, $f_{asafe}^{\rightarrow}$ and
heuristic choice as upwards interpolation $int\upharpoonright$ to pick a conflict reason.
Learning overapproximates the unsafe trace transformer while still preserving the error reachability 
%is a reduction over $F^{\sharp}$ that 
and is parameterized by an element of downset abstract domain $\mathbb{D}(\widehat{\mathcal{SA}_{G}})$.
%
\begin{figure}[t]
\begin{minipage}{4cm}
\centering
\scalebox{.60}{\import{figures/}{acdl_fixpoint.pspdftex}}
  %\caption{Fixpoint computation during Abstract Model Search Phase (Red lines denote
  %decision, Blue lines denote deduction)}
  \label{model-fixpoint}
\end{minipage}%  
\hspace{10em}%
\begin{minipage}{4.0cm}
\centering
\scalebox{.60}{\import{figures/}{conflict-fixpoint.pspdftex}}
  %\caption{Conflict Analysis with Underapproximate Weakest
  %Precondition and Upwards Interpolation}
\label{conflict-fixpoint}
\end{minipage}
  \caption {
  A. Fixpoint computation during Abstract Model Search Phase (Red lines denote
  decision, Blue lines denote deduction), B. Conflict Analysis with 
  Underapproximate Weakest Precondition and Upwards Interpolation
  \label{fixpoint}}
\end{figure}
%

Fig.~\ref{fixpoint} shows a graphical execution of the model search and 
conflict analyis procedures over a lattice of fixpoints.  The region marked 
with green in figure~\ref{fixpoint}A and figure~\ref{fixpoint}B are the set of 
fixed points of $f_{aunsafe}^G$ and pre-fixed points $(f_{asafe}^G(C) \sqsupseteq C)$ 
respectively.   The least among the fixed points is marked as
$lfp(f_{aunsafe}^G)$, whereas the greatest among the fixed points is marked as
$gfp(f_{aunsafe}^G)$ in figure~\ref{fixpoint}A, and similarly for figure~\ref{fixpoint}B.  
Note that the concrete unsafe trace transformer $f_{unsafe}^G$ is a lower closure, 
so $lfp(f_{unsafe}^G) = \bot$.  Whereas the concrete safe transformer $f_{safe}$ is 
upper closure, so $gfp(f_{safe}^G) = \top$.  Since $f_{aunsafe}^G$ and $f_{asafe}^G$ 
are sound over and underapproximations of $f_{unsafe}^G$ and $f_{safe}^G$ respectively, 
so $lfp(f_{aunsafe}^G) = \bot$ in figure~\ref{fixpoint}A and $gfp(f_{asafe}^G) = \top$ in 
figure~\ref{fixpoint}B.

\Omit{
The region in the intersection of green and golden 
ellipse structure is the set of fixed-points, which is both a pre-fixed point and
post-fixed point.  
}
%

Fig.~\ref{fixpoint}(A) shows that model search procedure starts 
from $\top$, and make sequence of deductions (marked in blue) 
through the abstract transformer $f_{aunsafe}^G$, until it reaches 
the greatest fixed point.  Recall that $f_{aunsafe}^G$ is the
set of transformers in $\widehat{\mathcal{SA}_G}$ that are used for forward, 
backward and multi-way analysis. 
%until it reaches the greates fixed point. 
%Recall that BCP in CDCL solver computes a greatest fixed point through 
%repeated application of the unit rule transformer. 
When the deductions can not be refined further, a decision is made which 
jumps under the greatest fixed point, as shown by red arrows in 
Figure~\ref{fixpoint}A.   
\Omit{
A decision accelerates the fixed point 
computation and is necessary to quickly converge on a conflict or a 
counterexample trace. }
A sequence of decisions and deductions follows, which
terminates in the $\gamma$-complete region of the lattice (marked in pink). The
transformer $f_{aunsafe}^G$ is $\gamma$-complete on $d$ if $\gamma(f_{aunsafe}^G(d))
\subseteq f_{unsafe}^G(\gamma(d))$, where $f_{unsafe}^G$ is the concrete unsafe trace 
transformer.  This implies, that a counterexample is obtained and the procedure terminates.
%


Fig.~\ref{fixpoint}(B) shows that conflict analysis procedure starts from a
conflict $\{\mathcal{C}\}$, and iteratively generalizes an element by computing a
least fixed point using a weakest precondition, $\widehat{apre}_s$ (marked by
red arrows), followed by heuristically choosing a candidate reason for conflict 
through upwards interpolation (marked by blue arrows) that underapproximates 
$\widehat{apre}_s$.   

%Figure~\ref{fig:acdcl} presents an architectural view 
%of a learning based program analyzer using CDCL architecture.  
%
%


\Omit {
\begin{figure}
\centering
\vspace*{-0.2cm}
\scalebox{.70}{\import{figures/}{acdlp-top.pspdftex}}
\caption{Architectural View of ACDLP \label{conflict}}
\end{figure}
%
\rmcmt{define transformers}
%
\begin{algorithm2e}[t]
\DontPrintSemicolon
\SetKw{return}{return}
\SetKwRepeat{Do}{do}{while}
%\SetKwFunction{assume}{assume}
%\SetKwFunction{isf}{isFeasible}
\SetKwData{conflict}{conflict}
\SetKwData{safe}{safe}
\SetKwData{sat}{sat}
\SetKwData{unsafe}{unsafe}
\SetKwData{unknown}{unknown}
\SetKwData{true}{true}
\SetKwInOut{Input}{input}
\SetKwInOut{Output}{output}
\SetKwFor{Loop}{Loop}{}{}
\SetKw{KwNot}{not}
\begin{small}
\Input{A program in the form of a set of abstract transformers $\abstransset$.}
\Output{The status \safe or \unsafe. %and a counterexample if \unsafe.
}
$\trail \leftarrow \langle\rangle$, $\reasons \leftarrow []$ \;
$\mathit{result} \leftarrow \deduce_{\propheur}(\abstransset,\trail,\reasons)$ \;
\lIf{$\mathit{result}$ = \conflict} {
  \return \safe}
\While{$true$} 
{
\lIf{$\mathit{result}$ = \sat} {
  \return \unsafe}
  $\decisionvar \leftarrow \decide_{\decheur}(\abs(\trail))$ \;
  $\trail \leftarrow \trail \cdot \decisionvar$ \; 
  $\reasons[|\trail|] \leftarrow \top$ \;
  $\mathit{result} \leftarrow \deduce_{\propheur}(\abstransset,\trail,\reasons)$\;
  \Do{$\mathit{result} = \conflict$} {
    \lIf{$\neg \analyzeconflict_{\confheur}(\abstransset,\trail,\reasons)$} {
      \return \safe
    }
    $\mathit{result} \leftarrow \deduce_{\propheur}(\abstransset,\trail,\reasons)$ \;
  }
}
\end{small}
\caption{Abstract Conflict Driven Learning for Programs $ACDLP_{\propheur,\decheur,\confheur}(\abstransset)$ \label{Alg:acdcl}}
\end{algorithm2e}
%
In this section, we present our framework called \emph{Abstract Conflict 
Driven Learning} that uses abstract model search and abstract 
conflict analysis procedures for safety verification of programs. 
The input to ACDLP (Algorithm~\ref{Alg:acdcl}) is a
program in the form of a set of abstract transformers
$\abstransset=\{\abstrans{\domain}{\sigma}|\sigma\in\Sigma\}$
w.r.t.\ an abstract domain~$\domain$.  Recall that the safety 
formula $\bigwedge_{\constraint\in\constraints} \constraint$ 
is unsatisfiable if and only if the program is safe.  
The algorithm is parametrised by heuristics for propagation $(\propheur)$, 
decisions $(\decheur)$, and conflict analysis $(\confheur)$.
Approximation of the concrete transformers in 
$\abstransset$ are typically available in abstract domain in the 
form of strongest-post condition or weakest pre-condition. 
The algorithm maintains a propagation trail $\trail$ and 
a reason trail~$\reasons$.
The propagation trail stores all meet irreducibles inferred by 
the abstract model search phase (deductions and decisions).  
The reason trail maps the elements of the propagation trail to the
transformers $\abstransel{}\in\abstransset$ that were used to
derive them. 
%
\begin{definition} 
The \emph{abstract value} $\abs(\trail)$ corresponding to 
the propagation trail $\trail$ is the conjunction of the 
meet irreducibles on the trail:
$\abs(\trail)=\bigsqcap_{m \in \trail}m$ with
$\abs(\trail)=\top$ if $\trail$ is the empty sequence.
\end{definition}
%
The algorithm begins with an empty $\trail$, a empty $\reasons$, and the
abstract value $\top$.  The procedure $\deduce$ (details in
Section~\ref{sec:deduce}) computes a greatest fixed-point over the
transformers in $\abstransset$ that refines the abstract value,
similar to the Boolean Constraint Propagation
step in SAT solvers.  If the result of $\deduce$
is \textsf{conflict} ($\bot$), the algorithm terminates with
\textsf{safe}.  Otherwise, the analysis enters into the while loop at line 4
and makes a new decision by a call to $\decide$ (see
Section~\ref{sec:decide}), which returns a new meet irreducible
$\decisionvar$.
%
We concatenate $\decisionvar$ to the trail~$\trail$.  The decision
$\decisionvar$ refines the current abstract value $\abs(\trail)$ represented
by the trail, i.e., $\abs(\trail\cdot\decisionvar)\sqsubseteq \abs(\trail)$.
%
For example, a decision in the interval domain restricts the range of 
intervals for variables.
%
We set the corresponding entry in the reason trail~$\reasons$ to $\top$
to mark it as a decision.
%
The procedure $\deduce$ is called next to infer new meet irreducibles
based on the current decision.  The model search phase
alternates between the decision and deduction until $\deduce$ returns
either \textsf{sat} or \textsf{conflict}.  
%
If  $\deduce$ returns  \textsf{sat}, then 
we have found an abstract value that represents models of the safety formula, which
are counterexamples to the required safety property, and so ACDLP return
\textsf{unsafe}.
%
If  $\deduce$ returns  \textsf{conflict}, 
the algorithm enters in the $\analyzeconflict$ 
phase (see Section~\ref{sec:conflict}) to learn the reason for the conflict.   There can be multiple
incomparable reasons for conflict.  ACDLP heuristically 
chooses one reason~$\conflictset$ and learns it 
by adding it as an abstract transformer to $\abstransset$. The analysis 
backtracks by removing the content of $\trail$ up to a point where it does not 
conflict with $\conflictset$.  ACDLP then performs deductions with the learned 
transformer.  If $\analyzeconflict$ returns $\false$, then no further
backtracking is possible.  Thus, the safety formula is unsatisfiable
and ACDLP returns \textsf{safe}.
}

%\input{deductions}
%===============================================================================

%===============================================================================
%\section{Program Structure Aids SAT Solver}
%
Figure~\ref{fig:sat-program} shows that program structure assist satisfiability 
solvers to efficiently reason about program properties.  This is manifested by
the use of assumption (marked in red) as decision that restrict the intervals 
of variable $y$.  The right hand column in Figure~\ref{fig:sat-program} clearly 
demonstrate the effect of this assumption on the proof generated by 
MiniSAT~2.2.0~\cite{minisat} solver.  The figures without bracket and within 
bracket in right column of Figure~\ref{fig:sat-program} shows the proof statistics 
without the use of assumption and with the use of assumption respectively. The
assumption clearly assist the solver in restricting the decision to variable $y$
which immediately leads to the proof.  The total number of decision without assumption 
is 15 and with assumption, the solver just makes a single decision.  Our \rmcmt{ACDCL} 
based program analyzer support a decision heuristic that exploit the high-level structure 
of the program to derive decision variables (here varaible $y$) that participate in 
branching conditions. 
%
\begin{figure}[t]
\centering
\small
\begin{tabular}{|c|c|}
\hline
  Program & Proof Statistics \\
\hline
\scriptsize
\begin{lstlisting}[mathescape=true,language=C,style=base]
int main() {
 int y, x=0;
 @**********************
 assume(y>=0 && y<=10);
 ***********************@
 if(y==0)
   x=y+1;
 else if(y==1)
   x=y+1;
 else if(y==2)
   x=x*2+y;
 else if(y==3)
   x=x-y;
 else 
   x=1;
 assert(x>0);
}
\end{lstlisting}
&
\begin{lstlisting}[mathescape=true,language=C,style=base]
  restarts           : 1@(1)@
  conflicts          : 4@(0)@           
  decisions          : 15@(1)@
  propagations       : 165@(134)@         
  conflict literals  : 6@(0)@ 
\end{lstlisting}
\\
\hline
\end{tabular}
\caption{\label{fig:sat-program} Program and its Proof Statistics using
  MiniSAT~2.2.0}
\end{figure}
%
\begin{figure}[t]
\centering
\small
\begin{tabular}{|c|c|}
\hline
  Program & Proof Statistics \\
\hline
\scriptsize
\begin{lstlisting}[mathescape=true,language=C,style=base]
  unsigned x, y;
  bool op;
  assume(x==y);
  if(op)
   x++;
  else
   x--;

  if(op)
   y=y+1;
  else
   y=y-1;
  assert(x==y);
\end{lstlisting}
&
\begin{lstlisting}[mathescape=true,language=C,style=base]
  restarts           : 4@(0)@
  conflicts          : 423@(0)@           
  decisions          : 2516@(0)@
  propagations       : 38617@(173)@         
  conflict literals  : 2905@(0)@ 
\end{lstlisting}
\\
\hline
\end{tabular}
\caption{\label{fig:sat-program} Program and its Proof Statistics using
  MiniSAT~2.2.0}
\end{figure}
%

%===============================================================================

%===============================================================================
\section{Experimental Results}

We have implemented ACDLP for bounded safety verification of C programs.  
ACDLP is implemented in C++ on top of the
\textsc{CPROVER}~\cite{cprover} framework as an extension of 2LS~\cite{2ls}
and consists of around 9~KLOC. 
The template polyhedra domain is implemented in C++ in 10~KLOC.  Templates
can be intervals, octagons, zones, equalities, or restricted polyhedra.  Our
domain handles all C operators, including bit-wise ones, and supports
precise complementation of meet irreducibles, which is necessary for
conflict-driven learning.  Our tool and benchmarks are available 
at~\url{http://www.cprover.org/acdcl/}.

We verified a total of~85 ANSI-C benchmarks.  These are derived from:
(1)~the bit-vector regression category in SV-COMP'16; (2)~ANSI-C models of
hardware circuits auto-generated by v2c~\cite{mtk2016} from VIS Verilog
models and opencores.org; (3)~controller code with varying loop bounds 
auto-generated from Simulink model and control 
intensive programs with nested loops containing relational properties. 
%The software models drawn from hardware benchmarks contains complex bit-wise
%operations, which are handled out-of-the-box by our domain implementation. 
All the programs with bounded loops are completely unrolled before
analysis.  

We~compare ACDLP with the state-of-the-art SAT-based bounded model checker
CBMC (\cite{cbmc}, version 5.5) and a commercial static analysis tool,
Astr{\'e}e (\cite{astree}, version 14.10).  CBMC uses MiniSAT~2.2.1 in the
backend.  Astr{\'e}e uses a range of abstract domains, which includes
interval, bit-field, congruence, trace partitioning, and relational domains
(octagons, polyhedra, zones, equalities, filter).  To enable fair comparison
using Astr{\'e}e, all bounded loops in the program are completely unwound up
to a given bound before passing to Astr{\'e}e.  This prevents Astr{\'e}e
from widening loops.
%
ACDLP is instantiated to a product of the Booleans and the Interval or
Octagon domain.  ACDLP is also configured with a decision heuristic 
(ordered, random, activity-based), propagation (forward, backward and multi-way), 
and conflict-analysis (learning UIP, DPLL-style).  The timeout for our
experiments is set to~200 seconds.
%
\Omit {
To enable precise analysis using Astr{\'e}e, all our benchmarks are 
manually instrumented with partition directives which provides external 
hint to the tool to guide the trace partitioning heuristics.  Usually, 
such high-precision is not needed for static analysis, since it makes 
the analysis very expensive.  Without trace partitioning, the 
analysis using Astr{\'e}e shows high degree of imprecision. 
}

%%%%%%%%%%%%%%%%%%%%%%%%%% scatter plots %%%%%%%%%%%%%%%%%%%%%%%%%%%
\begin{figure}[t]
  \centering
\begin{tabular}{@{\hspace{-0.5em}}c@{\hspace{1.5em}}c}
\begin{tikzpicture}[scale=1]
\begin{loglogaxis} [xmin=0.1,xmax=4000, ymin=0.1, ymax=4000, xlabel= SAT (Decisions),
			ylabel=ACDLP (Decisions),
                        width=0.45\linewidth,
			legend pos = north west,
			%legend style={at={(0.8,0.15)},
			%anchor=north,legend columns=-1 },
			]
\addplot [only marks,scatter,point meta=explicit symbolic,
	scatter/classes={s={mark=square,mark size=1.5},u={mark=triangle*,blue,mark size=1.5}},] 
	table [meta=label] {plotdata/scatter-decision.dat};
	\legend{Safe,Unsafe}
\addplot [domain=.1:4000] {x};
%\addplot [red,sharp plot, domain=.1:1500] {900}
%          node [below] at (axis cs:10,850) {timeout};
%\addplot [red,sharp plot, domain=.1:1500] coordinates{(900,.1) (900,1500)}
%          node [left,rotate=90] at (axis cs:700,10) {timeout}
 %node [right,black] at (axis cs:10,3) {portfolio faster}
 %node [right,black] at (axis cs:1,55) {kIkI faster};
%\addplot [red,sharp plot, update limits=false] coordinates{(900,.1) (900,1500)}
%	node [left] at {axis cs:700,200} {timeout};
\end{loglogaxis}
\end{tikzpicture}
 &
\begin{tikzpicture}[scale=1]
  \centering
	\begin{loglogaxis} [xmin=.1,xmax=83000, ymin=.1, ymax=83000, xlabel=SAT (Propagations),
			ylabel=ACDLP (Propagations),
                        width=0.45\linewidth,
			legend pos = north west,
      %legend style={at={(0.8,0.15)},
			%anchor=north,legend columns=-1 },
			]
\addplot [only marks,scatter,point meta=explicit symbolic,
	scatter/classes={s={mark=square,mark size=1.5},u={mark=triangle*,blue,mark size=1.5}},] 
	table [meta=label] {plotdata/scatter-propagation.dat};
	\legend{Safe,Unsafe}
\addplot [domain=.1:83000] {x};
%\addplot [red,sharp plot, domain=.1:1500] {900}
%          node [below] at (axis cs:10,850) {timeout};
%\addplot [red,sharp plot, domain=.1:1500] coordinates{(900,.1) (900,1500)}
%          node [left,rotate=90] at (axis cs:700,150) {timeout}
 %node [right,black] at (axis cs:10,3) {CPAchecker faster}
 %node [right,black] at (axis cs:1,55) {kIkI faster};
\end{loglogaxis}
\end{tikzpicture} \\
(a) & (b)
\end{tabular}
\caption{\label{fig:results}
Comparing SAT-based BMC and ACDLP: number of decisions and propagations}
\end{figure}
%%%%%%%%%%%%%%%%%%%%%%%%%%%%%%%%%%%%%%%%%%%%%%%%%%%%%%%%%%%%%%%%%%%%%%%%%%%%%%%%

\begin{figure}[t]
  \centering
  \begin{tikzpicture}[scale=1]

\pgfplotscreateplotcyclelist{markstyles}{%
solid, every mark/.append style={solid, fill=white}, mark=square*, mark size=1.5\\%
solid, every mark/.append style={solid, blue}, mark=triangle*,mark size=1.5\\%
solid, every mark/.append style={solid, fill=black}, mark=otimes*, mark size=1.5\\%
}
    
 	%axis
  \begin{axis}[
    width=\linewidth, height=4.5cm,
    xlabel={Benchmark Number},
    ylabel={Time (seconds)},
    domain = 1:85,
    xmin=1, xmax=85,
    ymin=0, ymax=200,
    %ytick={0,20,...,200},
    xtick={1,5,10,...,85},
    ymode = log,
    %log basis x={2},
    %xticklabel=\pgfmathparse{2^\tick}\pgfmathprintnumber{\pgfmathresult},
    legend pos = south east,
    grid = major,
    major grid style={line width=.2pt,draw=gray!50},
    cycle list name=markstyles
  ]
	
  %plots
    \addplot table [only marks, y=Time, x=Benchmarks]{plotdata/cbmc.dat};
	\addlegendentry{CBMC}
  
  \addplot table [only marks, y=Time, x=Benchmarks]{plotdata/acdlp.dat};
  \addlegendentry{ACDLP}
  
  \addplot table [only marks, y=Time, x=Benchmarks]{plotdata/astree.dat};
    \addlegendentry{Astr{\'e}e}
	
  \end{axis}  
\end{tikzpicture}
\caption{\label{fig:runtimes}
  Runtime comparison between CBMC, Astr{\'e}e and ACDLP}
\end{figure}

%

\medskip

\noindent \textbf{ACDLP versus CBMC}
Fig.~\ref{fig:results} presents a comparison between CBMC
and ACDLP.  Fig.~\ref{fig:results}(a) clearly shows that the SAT-based analysis 
makes significantly more decisions than ACDLP for all the benchmarks. 
The points on the extreme right below the diagonal in
Fig.~\ref{fig:results}(b) show that the number of propagations in the SAT-based 
analysis is maximal for benchmarks that exhibit relational behaviour.  These
benchmarks are solved by the octagon domain in ACDLP.  We see a reduction of at 
least two orders of magnitude in the total number of decisions, propagations 
and conflicts compared to analysis using CBMC.  

Out of 85 benchmarks, SAT-based analysis could prove only 26
benchmarks without any restarts.  The solver was restarted in the other 59 
cases to avoid spending too much time in ``hopeless'' branches.  By contrast, 
ACDLP solved all 85 benchmarks without restarts.  
The runtime comparison between ACDLP and CBMC is shown in 
Figure~\ref{fig:runtimes}.  ACDLP is~1.5X faster than CBMC. 
The superior performance of ACDLP is attributed to the decision heuristics, 
which exploit the high-level structure of the program, combined with the 
precise deduction by multi-way transformer and stronger learnt clauses aided 
by the abstract domains. 
%

\medskip

\noindent \textbf{ACDLP versus Astr{\'e}e}
%
To enable precise analysis with Astr{\'e}e, we manually instrument the
benchmarks with partition directives \texttt{\_\_ASTREE\_partition\_control}
at various control-flow joins.  These directives provide external hints to
Astr{\'e}e to guide its internal trace partitioning domain. 
Figure~\ref{fig:runtimes} demonstrates that Astr{\'e}e is~2X faster than
ACDLP for {37}\% cases (32 out of 85); but the analysis using Astr{\'e}e
shows a high degree of imprecision (marked as timeout in
Figure~\ref{fig:runtimes}).  Astr{\'e}e reported~53 false alarms among~85
benchmarks.  By contrast, the analysis using ACDLP produces correct results
for~81 benchmarks.  ACDLP times out for~4 benchmarks.  Clearly, ACDLP has
higher precision than Astr{\'e}e.  A detailed comparison between ACDLP, 
CBMC and Astr{\'e}e is available at~\cite{extended}.
%
%url{http://www.cprover.org/acdcl/}.
%presented in \rmcmt{Appendix~\ref{appendix:extended_result}.}  


\rmcmt{Our experimental evaluation suggests that ACDLP can be seen as a
technique to improve the efficiency of SAT-based BMC.  Additionally, ACDLP can
also be perceived as an automatic way to improve the precision of conventional
abstract interpretation over non-distributive lattices through automatic
partitioning techniques such as decisions and transformer learning.}


%\section{Detailed Experimental Results}\label{appendix:extended_result}
Table~\ref{detailed_result} gives a detailed comparison between CBMC version
5.5 and ACDLP.  Columns~1--4 in Table~\ref{detailed_result} contain the
name of the tool, the benchmark category, the number of lines of code (LOC),
and the total number of safe and unsafe benchmarks in the respective
categories (labelled as Safe/Unsafe).  The solver statistics ({\em
Decisions, Propagations, Conflicts, Conflict Literals, Restarts}) 
for CBMC and ACDLP are in columns~5--9.
%
%Column~10 reports the total time for verification per category.  

We classify our benchmarks into separate categories. 
We label the benchmarks in bit-vector regression category from SV-COMP'16 
as {\em Bit-vector}, ANSI-C models of hardware circuits auto-generated by v2c 
tool as {\em Verilog-C} and auto-generated Controller code and control-intensive 
benchmarks as {\em Control-Flow} category.  The total number of benchmarks in 
{\em bit-vector} category are 13, {\em Control-Flow} category contains 
55 benchmarks and {\em Verilog-C} category has 17 benchmarks. The timeout for 
our experiments is set to~200 seconds.  All times in Table~\ref{detailed_result} 
and Table~\ref{ai-result} are in seconds. 

The Bit-vector category contains a total of~13 benchmarks, out of which~6
are safe and the remaining~7 are unsafe benchmarks.  The benchmarks in the
control-flow category contains simple bounded loop analysis with relational
properties to more complex controller code containing nested loops with
varying loop bounds.  Out of~55 benchmarks in this category, 35 are safe and
20 are unsafe.  We verified a total of~17 hardware benchmarks, which are
given in Verilog RTL language.  Out of these~17 benchmarks,~10 are safe and
the remaining~7 are unsafe.  The software models (in ANSI-C) for the Verilog
circuits are obtained via a Verilog to C translator tool, {\em v2c}.  These
software models are then fed to CBMC and ACDLP.  The hardware benchmarks
include an implementation of a Instruction buffer logic, FIFO arbiter,
traffic light controller, cache coherence protocol, Dekker's mutual
exclusion algorithm among others.  The largest benchmark is the cache
coherence protocol which consists of~890 LOC and the smallest benchmark is
TicTacToe with~67 LOC.  The software models of these Verilog circuits uses
several complex bit-wise logic to map hardware operations into an equivalent
C syntax.  We emphasize that our implementation can handle bit-wise
operations out-of-the-box.

\begin{table}[!b]
\begin{center}
{
\begin{tabular}{l|l|r|r|r|r|r|r|r}
\hline
           &          &     & Safe/  &           & Propa-  &           & Conflict &          \\
  Verifier & Category & LOC & Unsafe & Decisions & gations & Conflicts & literals & Restarts \\ \hline
  CBMC & \multirow{2}{*}{Bit-vector} & \multirow{2}{*}{501} &
  \multirow{2}{*}{6/7} & 1011 & 1190 & 0 & 0 & 7 \\
  ACDLP & & & & 0 & 44 & 0 & 0 & 0 \\ \hline
  CBMC & \multirow{2}{*}{Control-Flow} & \multirow{2}{*}{1387} & 
  \multirow{2}{*}{35/20} & 29382 & 379727 & 4520 & 37160 & 62 \\ 
  ACDLP & & & & 414 & 6487 & 195 & 180 & 0  \\ \hline
  CBMC & \multirow{2}{*}{Verilog-C} & \multirow{2}{*}{4210} & 
  \multirow{2}{*}{10/7} & 131932 & 322707 & 69 & 349 & 6 \\ 
  ACDLP & & & & 625 & 8196 & 22 & 22 & 0 \\ \hline
\end{tabular}
}
\end{center}
\caption{CBMC versus ACDLP}
\label{detailed_result}
\end{table}

The statistics for ACDLP in Table~\ref{detailed_result} is obtained using an
ordered decision heuristic, multi-way propagation heuristic and a first-UIP
learning heuristic.  Note that the deductions using a multi-way heuristic is
more precise than forward or backward heuristics, but multi-way heuristic
takes longer time to reach the fixed-point.  Furthermore, multi-way
heuristic significantly reduces the total number of decisions, propagations
and learning iterations due to higher precision of the deductions made in
the abstract domain.  Overall, ACDLP reduces the total number of decisions,
propagations, conflicts and restarts by a factor of~20X compared to CBMC.

\begin{table}[t]
\begin{center}
{
\begin{tabular}{l|l|r|r|r}
\hline
  Verifier & Category & \#Proved (safe/unsafe) & \#Inconclusive & \#False Positives \\ \hline
  Astr{\'e}e & \multirow{2}{*}{Bit-vector} & 5/7 & 0 & 1 \\
  ACDLP & & 6/7 & 0 & 0 \\ \hline
  Aste{\'e}e & \multirow{2}{*}{Control-Flow} & 24/9 & 0 & 22 \\
  ACDLP & & 35/17 & 3 & 0 \\ \hline
  Astr{\'e}e & \multirow{2}{*}{Verilog-C} & 2/4 & 0 & 11 \\
  ACDLP & & 9/7 & 1 & 0 \\ \hline
\end{tabular}
}
\end{center}
  \caption{Astr{\'e}e versus ACDLP}
\label{ai-result}
\end{table}

Table~\ref{ai-result} gives a detailed comparison between Astr{\'e}e and
ACDLP.  Columns~1--5 in Table~\ref{ai-result} gives the name of the
tool, the benchmark category, the total number of instances proved safe or
unsafe (labelled as safe/unsafe), the total number of inconclusive
benchmarks and total number of false positives per category.

Table~\ref{ai-result} shows that ACDLP solved twice more benchmarks than
Astr{\'e}e.  The total number of inconclusive results in ACDLP is~4.  The
inconclusive results is because of timeout.  By~contrast, Astr{\'e}e reports
a total of~53 false positives among~85 benchmarks.  Clearly, ACDLP is more
precise than Astr{\'e}e.

\section{Decision Heuristics in ACDLP}~\label{decision}
%
We have implemented several decision heuristics in ACDLP: {\em ordered}, 
{\em longest-range}, {\em random}, and the {\em activity based} 
decision heuristic.  The {\em ordered} decision heuristic 
%creates an ordering among meet irreducibles, 
makes decisions on meet irreducibles that involve conditional 
variables (variables that appear in conditional branches) first 
before choosing meet irreducibles with numerical variables.  
%The ordered heuristic gives an effect of trace partitioning~\cite{toplas07}.
%
The {\em longest-range} heuristic simply keeps track of the bounds
$\numabsval_l,\numabsval_u$ of matching template rows, which are 
%\footnote{These are template rows with row vectors $\vec{c}$, $\vec{c}'$ such that $\vec{c}=-\vec{c}'$.}
row vectors $\vec{c}$, $\vec{c}'$ such that $\vec{c}=-\vec{c}'$.
%\pscmt{[that has become a bit hard to understand since some of the definitions have been removed]} 
$\numabsval_l\leq \vec{c}\vec{x}\leq \numabsval_u$, picks the one with the longest range
$\numabsval_u-\numabsval_l$, and randomly returns the meet irreducible
$\vec{c}\vec{x}\leq
\lfloor\frac{\numabsval_l+\numabsval_u}{2}\rfloor$ or its
complement. This ensures a fairness policy in selecting a variable
since it guarantees that the intervals of meet irreducibles are
uniformly restricted.
%
The {\em random} decision heuristic arbitrarily picks a meet irreducible  
for making decision. 
%
%The {\em relational} decision heuristics is only relevant for relational 
%abstract domains.  
%
The {\em activity based} decision heuristic is inspired by the 
decision heuristic used in the Berkmin SAT solver.  
The activity based heuristic %is currently implemented for interval constraints only.  
%The heuristic 
keeps track of the activity of meet irreducibles that 
participate in conflict clauses.  Based on the most 
active meet irreducible, ranges are split similar to 
the {\em longest-range} heuristic.
%
\section{Decisions, Propagations and Learning in ACDLP}
%

%%%%%%%%%%%%%%%%%%%%%%%%%%%%%%%%%%%%%%%%%%%%%%%%%%%%%%%%%%%%%%%%%%%%%%%%%%%%%%%%
\begin{figure}[t]
\begin{tabular}{@{\hspace{-1.5em}}c@{\hspace{1em}}c}
\begin{tikzpicture}[scale=0.75]
	\begin{loglogaxis} [
      xmin=0.1,xmax=200, ymin=0.1, ymax=200, 
      xlabel=Forward Propagation, ylabel=Multi-way Propagation, 
      title={Performance of Propagation Heuristics},
			legend pos = north west,
			%legend style={at={(0.8,0.15)},
			%anchor=north,legend columns=-1 },
			]
\addplot [mark size=1pt,only marks,scatter,point meta=explicit symbolic,
	scatter/classes={s={mark=square,mark size=2.5},u={mark=triangle*,blue,mark size=2.5}}] 
	table [meta=label] {plotdata/scatter-chaotic-forward.dat};
	\legend{Safe,Unsafe}
  \addplot [domain=.1:200] {x};
\end{loglogaxis}
\end{tikzpicture}
 &
\begin{tikzpicture}[scale=0.75]
\pgfplotscreateplotcyclelist{markstyles}{%
mark=diamond\\%
mark=square*,red\\%
mark=triangle*,blue\\%
}
	\begin{axis} [
  title={Performance of Decision Heuristics}, ymode = log,
  xmin=0, xmax=85, only marks,
  ymin=0, ymax=200, 
  xlabel={Benchmarks},
  ylabel={time (in seconds)},
  xtick={},
  ytick={}, %{0, 0.5, 1.0, 10, 50, 100, 200, 500, 700, 1000, 1500},
	legend pos=north west, mark size=1pt, cycle list name=markstyles,
 ]
 \addplot table[x=benchmark,y=time] {plotdata/berkmin.dat};
 \addplot table[x=benchmark,y=time] {plotdata/random.dat};
 \addplot table[x=benchmark,y=time] {plotdata/ordered.dat};
 \addlegendentry{Activity}
 \addlegendentry{Random}
 \addlegendentry{Ordered}
 %\addplot [domain=.1:2500] {};
 \end{axis}
\end{tikzpicture} \\
(a) & (b)
\end{tabular}
\caption{\label{prop-dec}
Effect of Propagation Heuristics and Decision Heuristics in ACDLP
}
\end{figure}
%%%%%%%%%%%%%%%%%%%%%%%%%%%%%%%%%%%%%%%%%%%%%%%%%%%%%%%%%%%%%%%%

%
\paragraph{Propagation Strategy.}
%
Fig.~\ref{prop-dec}(a) presents a comparison between the {\em forward} 
and {\em multi-way} propagation strategy in ACDLP.  The
choice of strategy influences the total number of decisions and clause 
learning iterations.  Hence, the propagation strategy has a
significant influence on the runtime, which can be seen in
Fig.~\ref{prop-dec}(a).  We did not report the performance 
of {\em backward} propagation strategy due to large 
number of timeouts.  Compared to forward propagation, the multi-way
strategy may take more iterations to reach the fixed-point, but it
subsequently reduces the total number of decisions and conflicts to prove the
program.  This is attributed to the higher precision of the meet irreducibles 
inferred by the multi-way strategy, which subsequently aids the decision 
heuristics to make better decisions.  

\paragraph{Decision Heuristics.}
%
Fig.~\ref{prop-dec}(b) shows the performance of different decision
heuristics in ACDLP.  Note that the runtimes for all decision heuristics are
obtained using the multi-way propagation strategy.  The runtimes are very
close, but we can still discern some key characteristics of these
heuristics.  The activity based heuristic performs consistently well for most safe
benchmarks and all bit-vector category benchmarks.  By contrast, the ordered
heuristic performs better for programs with conditional branches since it
prioritises decisions on meet irreducibles that appear in conditionals.  The
runtimes for the random heuristic are marginally higher than the other
two.  This suggests that domain-specific decision heuristics are important
for ACDLP.
%
\Omit{
Whereas activity-based heuristics such as Berkmin heuristic which 
works well in propositional cases performs best for benchmarks 
that encountered the maximum number of conflicts to prove safety, 
thus allowing the heuristics to choose the decision variable among the set of learnt clauses.   
}

\paragraph{Learning.}
%
Learning has a significant influence on the runtime of ACDLP.  We~compare
the UIP-based learning technique with an analysis that performs classical 
DPLL-style analysis.
%chronoligical backtracking without learning.  
The effect of UIP computation allows ACDLP to backtrack non-chronologically 
and guide the model search with a learnt transformer.  But classical 
DPLL-style analysis exhibits case-enumeration behaviour and could not finish 
within the time bound for 20\% of our benchmarks.
%


%===============================================================================

%===============================================================================
%===============================================================================
\section{Related Work}
%===============================================================================
%
\rmcmt{Refer ATVA paper}
The abstract satisfaction framework in~\cite{leo-thesis} can be perceived from
three different perspectives -- 1) characterizing satisfiability procedure as 
abstract interpretation, 2) using decision procedures to refine static analysis, 
and 3) lifting solver and deduction algorithms to new logics.   

Imprecision in static analysis may be due to joins, or over-approximate abstract
transformers or imprecise fixed point iteration. Precision loss due to joins was 
addressed through application of CDCL-style reasoning~\cite{tacas12}, 
DPLL(T)~\cite{SMPP}, or through unification~\cite{cade07}.  The works of   
of~\cite{vmcai04} and~\cite{cav12} synthesize best abstract transformer using
satisfiability solvers. Whereas, the precision loss due to fixed point iteration
is addressed by Monniaux et. al. in~\cite{sas11}.

The DPLL(T) framework provides an unified approach to developing decision 
procedures~\cite{dpllt}.  However, the seperation between the Boolean and 
theory solver in DPLL(T) can lead to performance issues.  Hence, several 
framework evolved based on DPLL(T) in the past.  Some of the attempts in 
this direction are Abstract DPLL framework~\cite{adpll}, generalized 
DPLL~\cite{dpll}, natural domain SMT~\cite{ndsmt}.

The work of~\cite{DBLP:journals/fmsd/BrainDGHK14} lifted CDCL to floating-point 
arithmetic, whereas Nelson-Oppen combination procedure was lifted to abstract
domains~\cite{cav12} and ~\cite{pldi06} lifted St$\mathring{\text{a}}$lmarck's 
method to arithmetic logic. 

%===============================================================================

%===============================================================================
\section{Conclusion}
%===============================================================================
%
The lattice theoretic account of CDCL algorithm provides a natural extension 
to other domains.  In this paper, we present a full abstract interpretation
account of bounded model checking using CDCL architecture.  To this end, we
first show that finding non-trivial trace abstraction in presence of loops, 
that satisfy properties of CDCL can be achieved through logical encoding of
program semantics using static single assignment form.  We then present a
characterization of CDCL as a tool to compute fixed point approximation of unsafe 
trace transformer over a lattice of program traces.  We showed that decision 
and learning in CDCL automatically allows the analysis to recover from
imprecision.  Our framework provides a mathematical basis for instantiating 
CDCL architecture for program analysis using the framework of abstract interpretation. 
A practical benefit of our work is a new, learning based architecture for implementing 
precise \rmcmt{program or safety?} analysis tools, through combination of abstraction
interpretation and CDCL-based SAT Solvers.  

\bibliographystyle{splncs03}
\bibliography{biblio.bib}


\end{document}
