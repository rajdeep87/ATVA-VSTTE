\section{Fixed-point Characterisations of Propositional Satisfiability}
%
Abstract Conflict Driven Clause Learning (ACDCL)~\cite{dhk2013-popl} is a 
lattice-based generalization of CDCL.  Silva et. al.~\cite{sas12}
shows the correspondences between propositional CDCL and lattice-based
abstractions, which are summarized in Table~\ref{connection}.  
%
\begin{table}[]
\centering
\caption{Components in propositional solver and their counterparts in lattice-theoretic generalization}
\label{connection}
\begin{tabular}{ll}
\toprule
  Propositional Solver & Abstract Interpretation \\
  \midrule
Partial assignment & Abstract Domain with complementable meet irreducibles \\
Singleton assignments & Meet Irreducibles   \\
CNF formula & Abstract Deduction Transformer    \\
Unit rule & Best Abstract Transformer \\
BCP & Greatest Fixed-Point Computation \\
Decision & Dual Widening \\ 
Conflict Analysis & Abductive Reasoning \\
Clause Learning & Synthesizing Abstract Transformer for negation \\ 
\bottomrule
\end{tabular}
\end{table}
%


%The lattice theoretic generalization of the propositional satisfiability 
%and the connection between the propositions and partial assignments and that of lattice and
%abstraction is summarized in Table~\ref{connection}.  
Silva et. al.~\cite{popl2014} proposed an abstract satisfaction framework that 
presents an application of abstract interpretation to design satisfiability algorithms 
following the correspondances in Table~\ref{connection}.  \rmcmt{We reinterpret the 
lattice based generalizations of the propositional satisfiablity algorithm 
in subsequent section for lifting the abstraction satisfaction framework to 
relational lattice structures for safety verification.}
%
We define fixed-point characterisations of the models and countermodels of a
formula and express satisfiability and validity in terms of these fixed-points.
\rmcmt{Change "we" to passive since this is background}


Let $Prop$ be a set of propositional variables.  A literal is a variable or its
negation.  Let $\mathbb{B} \mathrel{\hat=} \{t,f\}$ be the set of truth values. 
An assignment $\tau: Prop \rightarrow \mathbb{B}$ maps variables to truth
values. 
%
\begin{definition}
  An assignment $\tau$ is a \emph{model} of a formula $\varphi$, denoted $\tau
  \models \varphi$, if $\tau$ satisfies $\varphi$.
\end{definition}
%
\begin{definition}
  An assignment $\tau$ is a \emph{countermodel} of a formula $\varphi$, denoted $\tau
  \not\models \varphi$, if $\tau$ does not satisfy $\varphi$.
\end{definition}

\rmcmt{Recall} that a formula $\varphi$ is satisfiable if has a model and
unsatisfiable if there is no model of $\varphi$. 

Let $Asgn \mathrel{\hat=} Prop \rightarrow \mathbb{B}$ be the set of
concrete assignments.  A concrete domain over $Asgn$ is given by
$(\powerset{(Asgn)},\subseteq,\cup,\cap)$. Properties of $\varphi$ can be
expressed with the following transformers. A \emph{concrete model transformer} 
($f_{mod}^\varphi$) over a set of assignments $Z$ removes all countermodels of 
$\varphi$ from the set $Z$ such that it contains only those assignments that
satisfy $\varphi$.  Whereas, a \emph{concrete countermodel transformer} ($f_{cmod}^\varphi$) 
over a set of assignments $Z$ removes all models of $\varphi$ from $Z$ such that
it contains only those assignments that does not satisfy $\varphi$.
\[ f_{mod}^\varphi(Z) \mathrel{\hat=} \{\tau \in Z | \tau \models \varphi \}
\quad  
f_{cmod}^\varphi(Z) \mathrel{\hat=} \{\tau \in Z | \tau \not\models \varphi \}
\]
We define an \emph{universal countermodel transformer}, $f_{ucmod}^\varphi$, 
over a set of assignments $Z$, that adds all countermodels of $\varphi$ to 
$Z$. 
\[
f_{ucmod}^\varphi(Z) \mathrel{\hat=} \{\tau \in Asgn | \tau \not\models 
\varphi\; \text{or}\; \tau \in Z\}
\]
%
A fixed-point characterization of satisfiability can be expressed by suitably
exploiting the algebraic properties of these transformers, $f_{mod}$ and
$f_{cmod}$.  The concrete model and countermodel transformers has the following 
properties. 
\begin{enumerate}
  \item $f_{mod}^\varphi$ and $f_{cmod}^\varphi$ are idempotent.   
  \item $f_{mod}^\varphi(Asgn)$ is the set of models of $\varphi$ 
    and $f_{cmod}^\varphi(Asgn)$ is the set of countermodels of $\varphi$.
  \item There exists galois connection as shown below.
    \[ \powerset{(Asgn)} \galois{f_{mod}^\varphi}{f_{cmod}^\varphi}
    \powerset{(Asgn)} \]
\end{enumerate}

\begin{theorem}
  A formula $\varphi$ is unsatisfiable iff $f_{mod}^\varphi(Asgn)$ is empty 
  and $f_{cmod}^\varphi(Asgn)$ contains set of all assignments. 
\end{theorem}

\begin{theorem}
  A formula $\varphi$ is unsatisfiable iff greatest fixed-point of
  $f_{mod}^\varphi$, that is $gfp(f_{mod}^\varphi)$, is empty.
\end{theorem}

\subsection{Abstract Interpretation of Satisfiability}
%
Given a concrete domain of assignments, $(\powerset{(Asgn)},\subseteq,\cup,\cap)$, let 
$(O,\sqsubseteq,\sqcup,\sqcap)$ and $(U,\preceq,\sqcup,\sqcap)$ be an
overapproximation and underapproximation of the domain of assignments
respectively. The subset ordering $\subseteq$ of assignments are refined 
such that $a \sqsubseteq b$ implies $\gamma(a) \subseteq \gamma(b)$ and 
$p \preceq q$ implies $\gamma(p) \subseteq \gamma(q)$.  The galois connection
between the concrete domain of assignments and the approximating domains, $O$
and $U$ are given by 
\[  
    (\powerset{(Asgn)},\subseteq,\cup,\cap)
    \galois{\alpha_O}{\gamma_O}
    (O,\sqsubseteq,\sqcup,\sqcap)
\]
\rmcmt{
\[  
    (\powerset{(Asgn)},\supseteq,\cup,\cap) \galois{\alpha_U}{\gamma_U}
    (U,\succeq,\sqcup,\sqcap)
\]
}
%
We now define the abstract transformers over the domain $(O, \sqsubseteq)$ and
$(U, \preceq)$ that soundly approximates the concrete transformers, 
$f_{mod}^\varphi$ and $f_{cmod}^\varphi$.  Let $f_{aomod}^\varphi : O \rightarrow O$ 
be an abstract over-approximation of $f_{mod}^\varphi$ and $f_{aucmod}^\varphi :
U\rightarrow U$ be an abstarct under-approximation of concrete countermodel transformer 
$f_{cmod}^\varphi$. The abstract transformers satisfy the following constraints. 
\[
f_{mod}^\varphi \circ \gamma_O  \subseteq \gamma_O \circ f_{aomod}^\varphi
\qquad 
f_{cmod}^\varphi \circ \gamma_U  \supseteq \gamma_U \circ f_{aucmod}^\varphi
\]
%
\begin{theorem}
  A formula $\varphi$ is unsatisfiable iff $\gamma_O(gfp(f_{aomod}^\varphi))$ is empty 
\end{theorem}
\begin{proof}
  Theorem \rmcmt{number theorem} follows from the soundness proof of Abstract
  Interpretation. 
\end{proof}
%
\subsection{SAT solvers works on Partial Assignments Domain}
Given a CNF formula $\varphi$, a \emph{partial assignment} $\pi$ is 
a truth assignment to a subset of variables of $\varphi$, that is a 
partial function in $Prop \rightarrow \mathbb{B}$. We use 
$\modformula{\varphi}{\pi}$ to denote a \emph{simplified formula} 
that is obtained by replacing variables in $\pi$ with their respective
values in $\varphi$, removing all satisfiable clauses (clauses with at least 
one true literal) and deleting all false literals from remaining clauses.
%
\begin{example}
  Consider a CNF formula $\varphi = (\neg p \vee q) \wedge (p \vee \neg q)  
  \wedge (p \vee r)$ and a partial assignment $\pi=\{p:f,q:t\}$.  Clearly,
  $\varphi$ is not satisfied since $(p \vee \neg q)$ evaluates to \emph{false} 
  even without assignment to variable $r$. The partial assignment cannot be
  extended further since every extension of $\pi$ would evaluate to
  \emph{false} due to unsatisfiability of the clause $(p \vee \neg q)$. This
  leads to a \emph{conflict} state, denoted by $\bot$.  
\end{example}
%
We model partial assignments as a total function, 
$\pi: Prop \rightarrow \{t,f,\top\}$ over partial assignments domain $PAsgn$, 
where for each variable $p$, $\pi(p) = \top$ if $\pi$ is undefined on $p$. 
The \emph{partial assignments domain} is given by, 
$PAsgn \mathrel{\hat=} (Prop \rightarrow \{t,f,\top\}) \cup \{\bot\}$. 
The set $PAsgn$ contains a special element $\bot$ that denotes \emph{conflict}. 
%The \emph{partial assignments domain} is given by $(PAsgn,\sqsubseteq)$ where 
The ordering of elements in $(PAsgn, \sqsubseteq)$
is $t \sqsubseteq \top$ and $f \sqsubseteq \top$ with the empty set denoted by
$\bot$. Figure~\ref{fig:pasgn} presents a partial assignments domain over two 
variables, $p$ and $q$.  The element $(p:t)$ in $PAsgn$ denotes a partial 
assignment in which $p$ is \emph{true} and the remaining variables are mapped 
to $\top$. 
%
\begin{figure}[htbp]
\centering
\vspace*{-0.2cm}
\scalebox{.90}{\import{chapter5/figures/}{bool_lattice.pspdftex}}
\caption{Partial Assignments Domain over two variables \label{fig:pasgn}}
\end{figure}
%
\rmcmt{concrete boolean domain}
The galois connection between the concrete domain of assignments $Asgn$ and 
the over-approximating partial assignments domain, $PAsgn$ is given by 
\[  
    (\powerset{(Asgn)},\subseteq,\cup,\cap)
    \galois{\alpha_{PAsgn}}{\gamma_{PAsgn}}
    (PAsgn,\sqsubseteq,\sqcup,\sqcap)
\]
%
The abstraction function, $\alpha_{PAsgn}$, and concretisation function,
$\gamma_{PAsgn}$ are defined by,
\begin{equation*}
  \alpha_{PAsgn}(\emptyset) \mathrel{\hat=} \bot
  \quad 
  \alpha_{PAsgn}(C) \mathrel{\hat=} \{ p \mapsto \bigsqcup \{\pi(p) \mid \pi \in
  C\; \text{and}\; p \in Prop\} \} 
\end{equation*}

\begin{equation*}
  \gamma_{PAsgn}(\bot) \mathrel{\hat=} \emptyset
  \quad 
  \gamma_{PAsgn}(\pi) \mathrel{\hat=} \{ \tau \in Asgn \mid \tau(p) \sqsubseteq
  \pi(p), \forall p \in Prop \} 
\end{equation*}
%
\begin{figure}[htbp]
\centering
\vspace*{-0.2cm}
\scalebox{.90}{\import{chapter5/figures/}{cdcl.pspdftex}}
\caption{Architecture of CDCL solver \label{fig:cdcl}}
\end{figure}
\section{Model Search in Propositional Solver}
Model Search in CDCL solver alternates between the Boolean Constraint
Propagation (BCP) phase and the decision phase until a satisfying assignment is
found or a conflict is derived.  In the former case, the solver terminates with
the satisfying assignment, while in the later case the CDCL solver enters into
conflict analysis phase.  Figure~\ref{fig:cdcl} shows the high-level
architectural view of CDCL solver.  Below we describe the abstract interpretation 
account of BCP and decision.
%
\subsection{Boolean Constraint Propagation}
%
A \emph{unit clause rule} is a key component in a SAT solver. The unit rule
states that if all but one literals in a clause are false, then the remaining 
unassigned literal must be assigned \emph{true} value.  Boolean constraint
propagation is the repeated application of the unit rule.  Thus, unit rule helps
to derive logical consequences of a formula and is applied after every branching
step (that is decision).  The unit rule $(Unit)$ for a clause $c$ under partial
assignment $\pi$ is expressed by the function, $Unit \colon c \times PAsgn \rightarrow PAsgn$, 
and is given by Equation~\ref{eq:unit}.
%
\begin{equation}\label{eq:unit}
  Unit(c, \pi) = \left\{\begin{array}{lr}
    \bot, & \text{if } \pi(c)=f \\
    \pi \cup \{l \mapsto t\}, & \text{if } c=\theta \vee l\; \text{and }
    \pi(\theta)=f \\ 
    \pi \cup \{l \mapsto f\}, & \text{if } c=\theta \vee \neg l\; \text{and }
    \pi(\theta)=f \\ 
    \pi, & \text{otherwise}
\end{array}\right\}
\end{equation}
%
Algorithm~\ref{Alg:bcp} presents the boolean constraint propagation. Given a 
formula $\varphi$ and partial assignment $\pi$, the BCP procedure is a function
given by $bcp \colon \varphi \times PAsgn \rightarrow PAsgn$. 
%
\begin{algorithm2e}[t]
\DontPrintSemicolon
\SetKw{return}{return}
\SetKw{continue}{continue}
\SetKwRepeat{Do}{do}{while}
\SetKwData{conflict}{conflict}
\SetKwData{safe}{safe}
\SetKwData{sat}{sat}
\SetKwData{unsafe}{unsafe}
\SetKwData{unknown}{unknown}
\SetKwData{true}{true}
\SetKwInOut{Input}{input}
\SetKwInOut{Output}{output}
\SetKwFor{Loop}{Loop}{}{}
\SetKw{KwNot}{not}
\begin{small}
  \Input{A propositional formula $\varphi$ in CNF and a partial assignment $\pi$} 
  \Output{Partial assignment $\pi'$}
  \Do {$\pi' \neq \pi$} {
    $\pi' \leftarrow \pi$ \;
    \ForEach{$Clause\; c\; \in \varphi$} {
      $\pi' \leftarrow Unit(c,\pi)$ \; 
    }
  }
  \return $\pi'$ \;
\end{small}
\caption{Boolean Constraint Propagation $bcp(\varphi, \pi)$
  \label{Alg:bcp}}
\end{algorithm2e}
%
\begin{example}
  \textbf{Application of Unit Rule} 
  Consider a clause $c=l_1 \vee \neg l_2 \vee \neg l_3$ and a partial assignment
  $\pi=\{l_1:f,l_3:t\}$. Clearly, the clause $c$ is \em{unit}.  The application
  of Unit rule $Unit(c, \pi)$ yields a partial assignment, $\pi'=\{l_1:f,l_2:f,l_3:t\}$. 
\end{example}
%
\begin{example}
  \textbf{GFP computation with BCP} 
  Consider a CNF formula $\varphi = p \wedge (\neg p \vee q) \wedge (q \vee \neg
  r) \wedge (r \vee \neg s)$ and a partial assignment $\pi_0=\{\top\}$. Staring
  from an empty partial assignment $\top$, we repeatedly apply unit rule to
  improve the precision of the deduction. 
\end{example}
  $\begin{array}{l@{}c@{}l}
    Unit(\varphi, \top) &=& Unit(p, \top) \sqcap Unit((\neg p \vee q), \top)
  \sqcap Unit((q \vee \neg r), \top) \sqcap Unit((r \vee \neg s), \top) \\
   &=& \{p:t\} = \pi_1 \\ \\
    
    Unit(\varphi, \pi_1) &=& Unit(p, \pi_1) \sqcap Unit((\neg p \vee q),
    \pi_1) \sqcap Unit((q \vee \neg r), \pi_1) \sqcap Unit((r \vee \neg s),
    \pi_1) \\
    &=& \{p:t, q:t \}  = \pi_2 \\ \\ 
    
    Unit(\varphi, \pi_2) &=& Unit(p, \pi_2) \sqcap Unit((\neg p \vee q),
    \pi_2) \sqcap Unit((q \vee \neg r), \pi_2) \sqcap Unit((r \vee \neg s),
    \pi_2) \\
    &=& \{p:t, q:t, r:f \}  = \pi_3 \\ \\ 

    Unit(\varphi, \pi_3) &=& Unit(p, \pi_3) \sqcap Unit((\neg p \vee q),
    \pi_3) \sqcap Unit((q \vee \neg r), \pi_3) \sqcap Unit((r \vee \neg s),
    \pi_3) \\
    &=& \{p:t, q:t, r:f, s:f \}  = \pi_4 \\ \\ 
  \end{array}$

  The application of boolean constraint propagation to $\varphi$, $bcp(\varphi,\pi)$
  yields the following sequence of deductions, which computes greatest
  fixed-point in the partial assignments domain. 
  \[\pi_1=\{p:t\} \qquad \pi_2=\{p:t,q:t\} \] 
  \[
  \pi_3=\{p:t,q:t,r:f\} \qquad \pi_4=\{p:t,q:t,r:f,s:f\}
  \]
  All clauses in $\varphi$ is satisfiable at this point. Hence, BCP terminates
  with the satisfying assignment $\pi_4=\{p:t,q:t,r:f,s:f\}$. Note that the
  initial partial assignment $\pi_0$ in BCP need not begin with a $\top$ but 
  may contain any elements of partial assignments domain, for example,
  $pi_0=\{p:t\}$.

\begin{example}
  \textbf{BCP is Incomplete} 
  Consider a variant of above CNF formula $\varphi = p \wedge (\neg p \vee q) \wedge (q \vee \neg
  r) \wedge (r \vee \neg s \vee t)$ and a partial assignment $\pi=\{\top\}$. The
  application of boolean constraint propagation to $\varphi$, $bcp(\varphi,\pi)$
  yields the following sequence of deductions.
\end{example}
  \[\pi_0=\{\top\} \quad \pi_1=\{p:t\} \quad \pi_2=\{p:t,q:t\} \quad
  \pi_3=\{p:t,q:t,r:f\} 
  \]
  The remaining clause $(r \vee \neg s \vee t)$ have literals $s$ and $t$
  unassigned. So, BCP stops with the partial assignment $\pi_3=\{p:t,q:t,r:f\}$.
  Hence, BCP is sound but incomplete deduction procedure. 

\begin{example}
  \textbf{BCP leads to Conflict} 
  Consider a partial assignment, $\pi=\{p:t,q:t,s:t\}$, and the CNF formula 
  $\varphi = p \wedge (\neg p \vee q) \wedge (\neg q \vee \neg r) \wedge (r \vee \neg s)$, 
  $bcp(\varphi,\pi)$ infers $r:f$ from the clause $(\neg q \vee \neg r)$ and
  $r:t$ from the clause $(r \vee \neg s)$.  This leads to a conflict, denoted by
  $\bot$. Hence, $\pi$ can not be extended further to satisfy $\varphi$. 
\end{example}
%
Recall that the concrete model transformer is given by the function, 
$f_{mod}^\varphi \colon \powerset{(Asgn)} \rightarrow \powerset{(Asgn)}$.  
We denote the concrete model transformer under partial assignment $\pi$ by
$f_{mod,\pi}^\varphi : PAsgn \times \powerset{(Asgn)} \rightarrow
\powerset{(Asgn)}$ such that $f_{mod,\pi}^\varphi(p) = f_{mod}^\varphi(p \cap \gamma(\pi))$.

The \emph{abstract deduction transformer}, $ded_{\varphi,\pi} \colon PAsgn
\times PAsgn \rightarrow PAsgn $ soundly over-approximates the concrete model transformer 
$f_{mod, \pi}^\varphi$, such that $f_{mod, \pi}^\varphi \circ \gamma_{PAsgn}
\subseteq \gamma_{PAsgn} \circ ded_{\varphi,\pi}$. That is, all solutions of
$ded_{\varphi,\pi}$ is satisfied by the models of $\varphi$. Assuming that
$f_{aomod}^c$ is the best abstract transformer of $f_{mod}^c$ for a clause $c
\in \varphi$, the abstract deduction transformer, $ded_{\varphi,\pi}$ is defined below.
\[
  ded_{\varphi,\pi}(p) \mathrel{\hat=} \bigsqcap \{f_{aomod}^c(p \sqcap \pi)
  \mid c \in \varphi\}
\]

The following abstract interpretation characterisations of unit rule and BCP 
procedure is shown in~\cite{leo-thesis}.
%
\begin{enumerate}
  \item Unit rule is the best abstract transformer for a clause $c$ and
    a partial assignment $\pi \in PAsgn$, $Unit(c, \pi) = \alpha_{PAsgn} \circ
    f_{mod}^c \circ \gamma_{PAsgn}(\pi)$
  \item The result of BCP is equivalent to the greatest fixed point of the 
    abstract deduction transformer, $gfp(ded_{\varphi,\pi})$.
\end{enumerate}
%
\subsection{Decision}
%
Decisions underapproximate the result of $gfp(ded_{\varphi,\top})$. In lattice
theoretic terms, a decision is a downward extrapolation function on a lattice 
of partial assignments, that is, $ext \downharpoonright \colon PASgn \rightarrow 
PAsgn$.  Decision can be seen as a mechanism to increase the precision of the analysis by
iterating below $gfp(ded_{\varphi,\top})$. A widening operator is used to move
upwards in the lattice in a least fixed point computation to achieve
convergence. Hence, decision in CDCL solver is dual widening without any
requirement on convergence.  The decision step is formalized below. 
\[
  ext \downharpoonright (\pi) \mathrel{\hat=} \pi \sqcap \{x:v\}, \text{where x
  is a decision variable and}\; v \in \mathbb{B}
\]
%
\section{Conflict Analysis}
CDCL solver records the deductions and decisions during model search in an
implication graph. Conflict analysis in CDCL solver aims to find a cut of the
implication graph that represents a sufficient \emph{reason} for a conflict. A cut
in the implication graph is a conjunction of literals.  Hence, a cut can be
represented by a partial assignment, $\pi$.  There can be multiple incomparable
reasons for a conflict. Hence, there can be set of cuts which represent set of
partial assignments. The disjunction of set of cuts is a sufficient reason
for conflict. For effeciency reason, a CDCL solver heuristically choose a single 
cut as the reason for conflict~\cite{cdcl}.  Thus, conflict analysis involves two steps -- 
1) heuristically choose a cut, and 2) generalize the cut.

A reason for a partial assignment $\pi$ is an element $\pi'$ such that
$f_{mod}^\varphi \circ \gamma_{PAsgn}(\pi') \subseteq \gamma_{PAsgn}(\pi)$.
Conflict analysis finds a reason for a conflict element $\bot$. 
In lattice theoretic terms, a conflict analysis is performed when 
$gfp(ded_{\varphi,\top}) = \bot$. Recall that the universal countermodel
transformer, $f_{ucmod}^\varphi(X)$ adds all countermodels of $\varphi$ to 
$X$. Conflict analysis can be seen as underapproximating the set of
countermodels computed by $f_{ucmod}^\varphi$ through underapproximating 
domain and underapproximating transformer.

Starting from an initial conflict reason represented by the partial assignment
$\pi_{cut0}$, let $\{cut0\}$ denote the set of vertices representing
$\pi_{cut0}$ in the implication graph.  Let $\{cut1\}$ denote a set of vertices 
in the implication graph and if $\{cut0\}$ is contained in $\{cut1\}$, then the
ordering of corresponding partial assignments satisfy $\pi_{cut1} \sqsubseteq
\pi_{cut0}$.  Conflict analysis operates on the downward closed sets 
(downsets) of partial assignments, $(\mathcal{D}(PAsgn), \subseteq)$. 
Note that the downset lattice is also called the \emph{disjunctive completion} of 
an abstract domain. 
The lattice $(\mathcal{D}(PAsgn))$ is an under-approximating abstract domain
which underapproximates the concrete domain of assignments.  The abstraction and
concretisation functions of $(\mathcal{D}(PAsgn))$ are defined below. 
\begin{equation*}
  \alpha_{\mathcal{D}(PAsgn)}(Z) \mathrel{\hat=} \bigcup \{\pi \mid 
  \gamma_{PAsgn}(\pi) \subseteq Z \}
\end{equation*}
%
\begin{equation*}
  \gamma_{\mathcal{D}(PAsgn)}(Z) \mathrel{\hat=} \{ 
  \gamma_{PAsgn}(\pi) \mid \pi \in Z \}
\end{equation*}
%
Recall that the universal countermodel transformer is given by the function, 
$f_{ucmod}^\varphi \colon \powerset{(Asgn)} \rightarrow \powerset{(Asgn)}$.  
We denote the universal countermodel transformer under partial assignment $\pi$ by
$f_{ucmod,\pi}^\varphi : PAsgn \times \powerset{(Asgn)} \rightarrow
\powerset{(Asgn)}$ such that $f_{ucmod,\pi}^\varphi(p) = f_{ucmod}^\varphi(p
\cup \gamma_{PAsgn}(\pi))$.

The \emph{abstract abduction transformer}, $abd_{\varphi,\pi} \colon PAsgn
\times \mathcal{D}(PAsgn) \rightarrow \mathcal{D}(PAsgn) $ soundly 
under-approximates the universal countermodel transformer 
$f_{ucmod, \pi}^\varphi$, such that 
$\gamma_{\mathcal{D}(PAsgn)} \circ abd_{\varphi,\pi} \subseteq 
f_{ucmod, \pi}^\varphi \circ \gamma_{\mathcal{D}(PAsgn)}$.
%
\begin{figure}[htbp]
\centering
\vspace*{-0.2cm}
\scalebox{.90}{\import{chapter5/figures/}{conflict_cut.pspdftex}}
\caption{Generalizing Conflict Reason \label{conflict}}
\end{figure}
%

Conflict analysis in CDCL solver heuristically choose a single partial
assignment among the set of conflict reasons due to effeciency purpose.
The heuristic choice can be seen as upwards interpolation functon, 
$int \upharpoonright \colon \mathcal{D}(PAsgn) \times \mathcal{D}(PAsgn)
\rightarrow \mathcal{D}(PAsgn)$. An upward interpolation for partial 
assignments $\pi$ and $\pi'$ where $\pi \sqsubseteq \pi'$ is described below. 
Note that the term \emph{interpolation} have different meaning than \emph{craig
interpolants.}
\[
  int \upharpoonright (\pi, \pi') \mathrel{\hat=} \{\pi^\dagger \mid \pi
 \sqsubseteq \pi^\dagger \sqsubseteq \pi'\}
\]
\begin{example}
Figure~\ref{conflict} shows the conflict analysis in CDCL solver. The initial
  conflict is $\pi = \{p:t,q:t,r:f,s:f\}$. A CDCL solver heuristically picks a single 
  conflict reason as shown in the right of Figure~\ref{conflict}. The graph cuts 
  over the intial conflict reason produce the candidate partial assignment set, 
  $\pi' = \{(p:t,q:t)(r:f,s:f)\}$. The heuristic choice is an upwards interpolation, 
  $int \upharpoonright (\pi, \pi') = \{(p:t,q:t)\}$.
\end{example}
\rmcmt{write about mimisation transformer in defferent sense}
%
Figure~\ref{fig:acdcl} shows the abstract interpretation account of CDCL
algorithm.  The work of~\cite{leo-thesis} shows that the building blocks 
of a CDCL based SAT solver can be understood in terms of lattice structures, 
transformers and fixed point iteration. An objective behind this unification is 
to achieve an uniform theoretical and practical treatment of the SAT/SMT solving 
and static analysis using lattice theoretic techniques.  The expectation is that
this will contribute to the development of new class of solvers and program
analyser technologies that are easy to develop and have improved performance. 
%
\begin{figure}[htbp]
\centering
\vspace*{-0.2cm}
\scalebox{.90}{\import{chapter5/figures/}{acdcl.pspdftex}}
\caption{Abstract Interpretation Characterisation of CDCL \label{fig:acdcl}}
\end{figure}

