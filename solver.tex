\section{Solver Architectures}
%
We first discuss various architectures of DPLL based satisfiability solvers and then 
concentrate on extending CDCL solver to richer logics.  From architectural standpoint, 
the following classification of DPLL family of solvers gives a clear overview 
of the various similarities and differences between them.  

\para{CDCL}
Figure~\ref{cdcl} gives the architecture of CDCL solver~\cite{cdcl}, which 
has two key components -- \emph{model search} and \emph{conflict analysis}.  
%
\begin{figure}
\centering
\scalebox{.65}{\import{figures/}{cdcl.pspdftex}}
  \caption{Architecture of CDCL \label{cdcl}}
\end{figure} 
%
%
\begin{figure}
\centering
\scalebox{.65}{\import{figures/}{dpll.pspdftex}}
  \caption{Architecture of Classical DPLL(T) \label{dpll}}
\end{figure} 
%
\begin{figure}
\centering
\scalebox{.65}{\import{figures/}{natural_domain_smt.pspdftex}}
  \caption{Architecture of Natural Domain SMT Solver \label{natural-domain-smt}}
\end{figure} 
%
\begin{figure}
\centering
\scalebox{.65}{\import{figures/}{natural_domain_smt.pspdftex}}
  \caption{\rmcmt{Architecture of Model Constructing Satisfiability Calculus} \label{mcsat}}
\end{figure} 
%
%%%%%%%%%%%%%%%%%%%% DPLL(T) %%%%%%%%%%%%%%%%%%%%%%%%%%
\para{DPLL(T)}
Figure~\ref{dpll} gives the architecture of DPLL(T) solver~\cite{cdcl,smt1,smt2}, also 
called \emph{Lazy SMT}.  
DPLL(T) uses CDCL solver as a black box which is used to enumerate the 
assignments of the boolean abstraction of the formula. The candidate boolean 
assignment is then checked with a solver for conjunctive fragment of theory to 
check satisfiability of quantifier-free formula in a theory.  Thus, decisions and 
learning in DPLL(T) is delegated to the propositional CDCL solver which is used to 
enumerate theory facts.  Whereas, the theory solver adds blocking clauses in terms of 
the existing atoms that represent partial assingment infeasible in the theory.   The 
work of~\cite{} introduced \emph{Splitting on demand} 
which encodes new theory facts as propositional variable and adds to a formula.  

There are various first-order 
theories such as arithmetic, bit-vector, arrays, uninterpreted functions, and datatypes.  
One of the benefits of the DPLL(T) architecture is that it is directly impacted by any 
advances and performance improvements of the CDCL SAT solver.    


%
%%%%%%%%%%%%%%%%%%%% Natural Domain SMT %%%%%%%%%%%%%%%%%%%%%%%%%%
\para{Natural Domain SMT}
Figure~\ref{natural-domain-smt} gives the architecture of natural domain 
SMT solver, proposed by Cotton et al., that integrates the theory solvers 
inside the CDCL procedure with techniques such as theory propagation and theory learning.  
The unification of the theory solver and CDCL search into a common architecture 
enables direct search of the model of formula over the natural domain of variables. 
This lead to the development of new solver procedures that performs direct model 
construction complemented with conflict resolution in some restricted first-order 
domains such as floating-point~\cite{DBLP:journals/fmsd/BrainDGHK14}, linear real
arithmetic~\cite{ndsmt,linear-real}, linear integer arithmetic~\cite{linear-int}, 
nonlinear arithmetic~\cite{nonlinear}.

%%%%%%%%%%%%%%%%%%%% Generalized DPLL %%%%%%%%%%%%%%%%%%%%%%%%%%
\para{Generalised DPLL}
McMillan et al.~\cite{DBLP:conf/cav/McMillanKS09} presented a Generalised DPLL 
procedure called \emph{GDPLL} which performs decisions and learning directly 
in theory.  A key difference between lazy SMT and GDPLL is that the theory 
deductions in GDPLL only occurs in response to a conflict in model search in contrast 
to lazy SMT which restricts theory deductions only in response to satisfying 
partial assignments.  Additionally, GDPLL allows theory learning in contrast to lazy SMT 
approach which cannot learn theory facts.  Similar theory-specific approaches are presented for 
equality~\cite{DBLP:journals/iandc/BadbanPTZ07} and integer linear 
arithmetic~\cite{DBLP:conf/cade/JovanovicM11}.


%%%%%%%%%%%%%%%%%%%% MCSAT %%%%%%%%%%%%%%%%%%%%%%%%%%
\para{Model Constructing Satisfiability Calculus}
While natural domain SMT procedures are quite effective in their respective
first-order theories, they have limitations in pure boolean reasoning and
are not compatible with DPLL(T) procedures.  A \emph{model-constructing 
satisfiability calculus} (mcSAT)~\cite{mcsat1,mcsat2} extends DPLL(T) 
procedure through \emph{model assignment} which allows decisions on theory variables.  
Furthermore, it allows propagations and learning in terms of theory variables 
that are not present in the given formula.

\begin{example}~\label{mcsatex}
%
Let us illustrate mcSAT procedure through a simple example. Consider a formula 
$\phi = (x \geq 5) \wedge (\neg (x \geq 2) \vee y \geq 2) \wedge (x^2 + y^2 \leq 2 \vee xy > 2)$.  
Let clause $C1 = (x \geq 5)$, $C2 = (\neg (x \geq 2) \vee y \geq 2)$ and 
$C3 = (x^2 + y^2 \leq 2 \vee xy > 2)$.  The procedure performs obvious deductions initially 
that sets $C1$ to \emph{true}, that is, $(x \geq 5)$ is deduced.  This infers $x \geq 2$ 
in $C2$ which implies $y \geq 2$ in $C2$.  The current partial assignment 
is extended by a decision which sets $x^2 + y^2 \leq 2$ to \emph{true} and performs a 
\emph{model assignment}, $x \rightarrow 5$.  Decisions in mcSAT constitute a truth 
assignment to decided literal as well as concrete assignments to first-order uninterpreted 
symbol.  The later is a model assignment.  The trail after decision is shown in figure~\ref{dec}.
Clearly, the decision leads to a conflict since the trail element $x^2 + y^2  \leq 2$ is 
unsatisfiable under the current model assignment.  The procedure investigates the 
reason for the conflict and learns a generalized clause, 
$C4 = \neg (x^2 + y^2 \leq 2) \vee (x \leq 1)$.  An alternative but not so productive 
learned clause could be $\neg(x^2 + y^2 \leq 2) \vee \neg(x \rightarrow 5)$.  
The contents of the trail at this point are $\{x \geq 5, x \geq 2, y \geq 2, x^2+y^2 \leq 2 \}$.  Clause 
$C4$ infers a new element, $x \leq 1$ which contradicts with the trail element $x \geq 5$.  
Conflict resolution between clauses $C4 =  \neg (x^2 + y^2 \leq 2) \vee (x \leq 1)$ and 
$\neg (x \geq 5) \vee \neg (x \leq 1)$ yields a learned clause $C5 = \neg(x \geq 5) \vee \neg(x^2 + y^2 \leq 2)$.  
The contents of the trail after conflict analysis are $\{x \geq 5, x \geq 2, y \geq 2 \}$. Clause $C5$ infers a 
new element, $\neg (x^2 + y^2 \leq 2)$ which is added to the trail.  The trail obtained 
through propagation after learning is shown in figure~\ref{mcfinal}.  The final content of 
the trail is $\{x \geq 5, x \geq 2, y \geq 2, \neg(x^2 + y^2 \leq 2) \}$ which satisfies 
all clauses.  Hence, the satisfying solution is $\{x=5, y=2\}$.
\end{example}
%
\begin{figure}
\centering
\scalebox{.70}{\import{figures/}{trail_mc.pspdftex}}
  \caption{Content of trail after decision for example~\ref{mcsatex} \label{dec}}
\end{figure} 
%
%
\begin{figure}
\centering
\scalebox{.70}{\import{figures/}{trail_mc_final.pspdftex}}
  \caption{Content of trail after learning for example~\ref{mcsatex} \label{mcfinal}}
\end{figure} 
%

\para{ACDLP versus DPLL/CDCL-based solvers}
%
There are few fundamental differences between ACDLP and solver procedures discussed above. 
ACDLP is different from DPLL(T) procedure in that it does not require a propositional solver and 
the reasoning happens directly in an abstract domain which corresponds to a fragment of theory.  
ACDLP is also different from GDPLL.  This can be explained with an example borrowed from
~\cite{DBLP:conf/cav/McMillanKS09}.  Consider the formula, 
$\phi = (a < b) \wedge (a < c) \wedge (b < d \vee c < d) \wedge (d < a)$, where $a,b,c,d$ are integers.  
GDPLL determines that $\phi$ is unsatisfiable using decisions and shadow rule for conflict analysis.  
Whereas, ACDLP do not need to make any decision and can prove unsatisfiability through deductions 
in the abstract domain of Inequality.  An Inequality abstract domain contains only conjunctions of 
inequalities. The abstract value computed through evaluating each clause in the Inequality abstract 
domain is given by $\pi = \{a < b, a < c, d < a\}$.  The greatest fixed point iteration for $\phi$ under
$\pi$ is computed by first applying best abstract transformer for each singleton clause, which gives 
$\pi'= \pi \cup \{ d < b, d < c \}$.  Now, the clause $(b < d \vee c < d)$ is unsatisfiable under 
$\pi'$.  Hence $\phi$ is unsatisfiable. 


Whereas, one of the fundamental differences between ACDLP and mcSAT is that decisions in mcSAT 
are restricted to concrete model assignments to theory variables and boolean assignments to 
theory atoms in the formula.  Whereas, decisions in ACDLP can be made over set of all meet 
irreducibles representable by the underlying abstract domains.



Another key difference between ACDLP and the solver procedures discussed above is that 
ACDLP directly operates over SSA representation of a program whereas a DPLL(T) based solver 
operates on a formula which is a conjunction of sentences in a given logic where a sentence 
(or clause) is a disjunction of theory atoms. This formula is usually specified in 
SMT-LIB2~\cite{smt-lib2} language.  For example, a sentence in the logic formula of 
quatifier-free linear rational arithmetic (QFLRA) is of the form 
$c_1x_1 + c_2x_2 + \ldots + c_nx_n \diamond c_0$, where $c_i$ is a rational constant 
and $x_i$ is a rational variable, and $\diamond$ is either $\geq$ or $>$.  A model of 
QFLRA is an assignment of rational values to variables $x_i$ such that it satisfies 
every clause of the formula.  \rmcmt{In contrast, an SSA may not be exactly represented 
in a clausal form.  Recall from section~\cite{ssa-cite}, elements of SSA are obtained directly 
from program statements through syntactic translation steps.  Hence, a clause in SSA 
representation constitute an equality or inequality (obtained from assignment statements 
or an assertion) or disjunctions (obtained from control-flow branches).} 
