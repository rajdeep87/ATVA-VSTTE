%===============================================================================
\section{Related Work}
%===============================================================================
%
\rmcmt{Refer ATVA paper}
The abstract satisfaction framework in~\cite{leo-thesis} can be perceived from
three different perspectives -- 1) characterizing satisfiability procedure as 
abstract interpretation, 2) using decision procedures to refine static analysis, 
and 3) lifting solver and deduction algorithms to new logics.   

Imprecision in static analysis may be due to joins, or over-approximate abstract
transformers or imprecise fixed point iteration. Precision loss due to joins was 
addressed through application of CDCL-style reasoning~\cite{tacas12}, 
DPLL(T)~\cite{SMPP}, or through unification~\cite{cade07}.  The works of   
of~\cite{vmcai04} and~\cite{cav12} synthesize best abstract transformer using
satisfiability solvers. Whereas, the precision loss due to fixed point iteration
is addressed by Monniaux et. al. in~\cite{sas11}.

The DPLL(T) framework provides an unified approach to developing decision 
procedures~\cite{dpllt}.  However, the seperation between the Boolean and 
theory solver in DPLL(T) can lead to performance issues.  Hence, several 
framework evolved based on DPLL(T) in the past.  Some of the attempts in 
this direction are Abstract DPLL framework~\cite{adpll}, generalized 
DPLL~\cite{dpll}, natural domain SMT~\cite{ndsmt}.

The work of~\cite{DBLP:journals/fmsd/BrainDGHK14} lifted CDCL to floating-point 
arithmetic, whereas Nelson-Oppen combination procedure was lifted to abstract
domains~\cite{cav12} and ~\cite{pldi06} lifted St$\mathring{\text{a}}$lmarck's 
method to arithmetic logic. 
