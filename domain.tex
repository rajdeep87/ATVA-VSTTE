%%%%%%%%%%%%%%%%%%%%%%% Trace-based Abstraction  %%%%%%%%%%%%%%%%%%%%%
%
\section{Lattice for Approximation of Trace Semantics}~\label{semantic-trace}
%
In this section, we describe the various lattices over the static analysis
equations and static single assignment form for a given control-flow graph.  
Figure~\ref{fig:lattice} shows the various lattices that approximate the
concrete trace semantics.
%


An important property of a clause learning SAT solver is that each meet
irreducibles of the partial assignments domain admits precise
complements~\cite{sas12}.  This property enables learning of domain 
elements that navigates the search away from the conflicting region.   
In order to lift CDCL to program analysis, it is imperative to find a suitable
trace based abstraction that admits precise complementation property. 
To this end, we first show that conventional means to represent a program through 
abstraction of a static analysis equation does not admit precise 
complements with respect to unsafe traces.   In this paper, we present an 
abstraction of program traces using logical encoding of CFG that bounded 
model checkers use.  We will show in Section~\ref{complement} that this 
representation allows us to lift CDCL to program analysis.  
%
%Hence, our formalizations can be understood as a full abstract interpretation 
%account of bounded model checking.
%\rmcmt{with the hope that this leads to ideas of how
%loops can be approximated in bounded model checking.} 
%
\begin{figure}[htbp]
\centering
\vspace*{-0.2cm}
  \scalebox{.70}{\import{figures/}{lattice.pspdftex}}
  \caption{Approximation of Trace Semantics used by ACDLP (right) and Abstract
  Interpretation (left)
  \label{fig:lattice}}
\end{figure}
%
\subsection{Concrete Flow Lattice}
%
Abstract interpretation technique for program analysis computes 
a fixed-point over static analysis equations~\cite{CC79,octagon}.  
The advantage of using approximations of static analysis equation 
over trace based abstraction is that the former only requires 
abstractions of memory states.  \\
%
We now define the \emph{concrete flow lattice} that describes these static 
analysis equations with respect to the set of control locations. 
%
\begin{definition} (Concrete Flow Lattice). A \emph{concrete flow lattice}
  corresponding to a CFG $G= (V, E, \mathcal{S}, \mathcal{T}, I)$  
  is a complete lattice $(\mathcal{F}_{G}, \sqsubseteq, \sqcap, \sqcup)$, 
  defined as follows. 
\end{definition}
  \[
    \mathcal{F}_{G} \mathrel{\hat=} V \rightarrow \powerset({\Omega}) \qquad \forall a,b \in
     \mathcal{F}_{G}.\; a \sqsubseteq b\; \text{if}\; \forall v \in V.\; a(v)
     \subseteq b(v) 
  \]

  \[
     a \sqcap b \mathrel{\hat=} v \mapsto a(v) \cap b(v) \qquad 
     a \sqcup b \mathrel{\hat=} v \mapsto a(v) \cup b(v)
  \]
%
%The static analysis equations corresponding to a control flow graph is shown in
%Figure~\ref{fig:se}.
%
\Omit{
Recall that a CFG statement $s$ corresponds to a state transition, that is, 
$\mathcal{ST}_s \subseteq \Omega \times \Omega$.  Hence, we define a
\emph{postcondition transformer} and \emph{precondition transformer} 
for $\mathcal{T}(u,v)=s$.
%
\[ 
  post_{u,v}(S) \mathrel{\hat=} \{\omega_k \mid \exists\; \omega_j \in S.\;
  (\omega_j,\omega_k) \in \mathcal{ST}_s \}
\]
%
\[ 
  pre_{u,v}(S) \mathrel{\hat=} \{\omega_j \mid \exists\; \omega_k \in S.\;
  (\omega_j,\omega_k) \in \mathcal{ST}_s \} 
\]
%
\begin{definition} (Concrete Flow Lattice Transformers). We define a strongest
  postcondition transformer and an existential precondition transformer of a 
  concrete flow lattice, $\mathcal{F}_{G}$ as follows.  
\end{definition}
  \[
    fpost \colon \mathcal{F}_{G} \rightarrow \mathcal{F}_{G} 
    \qquad 
    fpre \colon \mathcal{F}_{G} \rightarrow \mathcal{F}_{G} 
  \]
  \[
    fpost(v \mapsto a(v)) \mathrel{\hat=} \underset{(u,v) \in E}{\bigcup}
    post_{u,v} \circ a(u) 
    \qquad 
    fpre(v \mapsto a(v)) \mathrel{\hat=} \underset{(v,w) \in E}{\bigcup}
    pre_{v,w} \circ a(w) 
  \]
%
}
The concrete flow lattice is an over-approximation of the concrete powerset of
traces and there exists a galois connection between the two as shown below. 
\[
  (\powerset({\Pi}),\subseteq,\cup,\cap)
    \galois{\alpha_{F}}{\gamma_{F}}
    (\mathcal{F}_{G},\sqsubseteq,\sqcup,\sqcap)
\]
We define the abstraction and concretization functions between the concrete
trace domain $\powerset({\Pi})$ and concrete flow lattice $\mathcal{F}_{G}$.
\[
  \alpha_{F}(T) \mathrel{\hat=} \lambda v. \{\omega \mid \exists
  \pi \in T.\; (v,\omega) \in \pi \} 
\]
\[
  \gamma_{F}(a) \mathrel{\hat=} \{\pi \in \Pi \mid \forall (v,\omega) \in \pi.\;
  \omega \in a(v) \wedge \pi \in \Pi_{wf} \}
\]
%
It is easy to see that $\alpha_F \circ \gamma_{F}$ is deflationary while 
$\gamma_F \circ \alpha_F$ is inflationary, and the pair $(\alpha_F, \gamma_F)$
forms a galois connection. We prove this in proposition~\ref{ag}. 
%


For the purpose of static analysis, the concrete semantics of program's
control-flow graph is over-approximated using abstract interpretation
methodology~\cite{CC79}. The concrete domain of memory states 
$\powerset({\Omega})$ is abstracted to construct an abstract program
model which is then used for abstract program analysis.  Most static program
analyzers build an abstraction $A$ of the memory states $\Omega$, shown below. 
\[
  (\powerset({\Omega}),\subseteq)
    \galois{\alpha_{AF}}{\gamma_{AF}}
    (\mathcal{A},\sqsubseteq_{AF})
\]
%
\begin{definition} (Abstract Flow Lattice). An \emph{Abstract Flow Lattice}
  corresponding to a CFG $G= (V, E, \mathcal{S}, \mathcal{T}, I)$ and 
  an abstraction $\mathcal{A}$ over concrete memory states $\Omega$
  is a complete lattice $(\widehat{\mathcal{F}_{G}}, \sqsubseteq_{AF}, \sqcap_{AF},
  \sqcup_{AF})$, 
  which is defined as follows. 
\end{definition}
  \[
    \widehat{\mathcal{F}_{G}} \mathrel{\hat=} V \rightarrow \mathcal{A} \qquad \forall a,b \in
     \widehat{\mathcal{F}_{G}}.\; a \sqsubseteq b\; \text{if}\; \forall v \in V.\; a(v)
     \sqsubseteq_{AF} b(v) 
  \]
%
  \[
    a \sqcap_{AF} b \mathrel{\hat=} v \mapsto a(v) \sqcap_{AF} b(v) \qquad 
     a \sqcup_{AF} b \mathrel{\hat=} v \mapsto a(v) \sqcup_{AF} b(v)
  \]
%
\subsection{Complementation in Abstract Flow Lattice}~\label{complement-fg}
%
A meet irreducible of abstract flow lattice $\widehat{\mathcal{F}_{G}}$ is defined as
follows.
\[
  \sqcap_{irrd}(\widehat{\mathcal{F}_{G}}) \mathrel{\hat=} 
  \left\{\begin{array}{l@{\quad}l}
    n \mapsto a & \text{if}\; n \in V \wedge a \in \sqcap_{irrd}(\mathcal{A}) \\
    n \mapsto \top & \text{otherwise} \\
  \end{array}\right.
\]  
%
Here, a node $n \in V$ is mapped to an abstract memory state $a \in \mathcal{A}$, 
where $a$ is a meet irreducible of $A$.  Let $n \mapsto \bar{a}$ be a precise 
complement of $n \mapsto a$, where $\bar{a}$ is a precise complement of $a$.    
However, the two sets are not equal, that is, 
$\gamma_{AF}(n \mapsto a) \neq \neg \gamma_{AF}(n \mapsto \bar{a})$.
The element $\neg \gamma_{AF}(n \mapsto \bar{a})$ gives the set of traces 
where some occurrences of $n$ are contained in the concretization of
$\gamma_{AF}(a)$.  Whereas, the element $\gamma_{AF}(n \mapsto a)$ gives 
the set of traces where all occurrences of $n$ are contained in the 
concretization of $\gamma_{AF}(a)$.  This is explained below.
%
\[
  \gamma_{AF}(n \mapsto a) \mathrel{\hat=} \{\pi \in \Pi \mid \forall (n,\omega) \in
  \pi. \omega \in \gamma_{AF}(a) \}
\]
%
\[
  \neg \gamma_{AF}(n \mapsto \bar{a}) \mathrel{\hat=} \neg \{\pi \in \Pi \mid \forall (n,\omega) \in
  \pi. \omega \not\in \gamma_{AF}(a) \}
\]
%
\[
  \neg \gamma_{AF}(n \mapsto \bar{a}) \mathrel{\hat=} \{\pi \in \Pi \mid \exists (n,\omega) \in \pi. \omega \in \gamma_{AF}(a) \}
\]
%
%An important requirement of ACDLP is that the meet irreducibles in the domain 
%must be precisely complementable with respect to the unsafe trace transformer.  
The abstract control flow lattice $\widehat{\mathcal{F}_{G}}$ does not always have 
complementable meet irreducibles.  So, we define a domain over logical encodings 
of the CFG, the elements of which admits precise complementation property.
%
\subsection{Lattice Structure over SSA}~\label{state-transformer}
%
We now define a lattice over SSA structures that admits precise complementation 
property which is necessary for lifting CDCL to program analysis. Let
$Var_{ssa}$ denote the set of SSA variables in the SSA form and $Value$ denote the
set of values that these SSA variables can take.  Note that the set $Var_{ssa}$ is
obtained from the CFG $G= (V, E, \mathcal{S}, \mathcal{T}, I)$ through program
transformation $\mathcal{T}_2$. 
%
\begin{definition} (Static Assignment Lattice). A \emph{Static Assignment Lattice}
  corresponding to a CFG $G= (V, E, \mathcal{S}, \mathcal{T}, I)$ is a complete 
  lattice $(\mathcal{SA}_{G}, \subseteq_{SA}, \cup_{SA}, \cap_{SA})$, defined 
  as follows.
\end{definition}
%
  \[
    \mathcal{SA}_{G} \mathrel{\hat=} \powerset(Var_{ssa} \rightarrow Value)
    \quad \forall A,B \in
    \mathcal{SA}_{G}.\; A \subseteq_{SA} B\; \text{if}\; \forall a \in A.\; 
    a \in B 
  \]
\rmcmt{  
  \[
    A \cap_{SA} B \mathrel{\hat=} \lambda\; x. ( \{f(x) \cap g(x) \mid 
    f \in A, g \in B \}) 
    \qquad 
    A \cup_{SA} B \mathrel{\hat=} \lambda\; x. ( \{f(x) \cup g(x) \mid 
    f \in A, g \in B \}) 
  \]
}
%
%%%%%%%%%%%%%%%% Static Single Assignment Lattice %%%%%%%%%%%%%%%%%%
%
Figure~\ref{fig:concrete} gives an example of concrete static assignment lattice
over SSA variables $p$ of type boolean and $x$, $y$ of numerical types. Note
that the lattice $\mathcal{SA}_G$ is a set of concrete environments over SSA
variables.  We will call these \emph{concrete SSA environments}.  Following the
result of Briggs et. al.~\cite{Briggs:1998}, the elements of $\mathcal{SA}_G$
can be mapped back to a trace of the original program.  The elements marked in bold 
in Figure~\ref{fig:concrete} corresponds to concrete assignments to SSA variables 
that maps to a concrete program trace. 
%
\begin{figure}[htbp]
\centering
\vspace*{-0.2cm}
  \scalebox{.90}{\import{figures/}{concrete_env.pspdftex}}
\caption{Concrete Static Assignment Lattice over a boolean SSA variable $p$ and two
  numerical SSA variables $x$ and $y$ that takes values in Integer domain \label{fig:concrete}}
\end{figure}
%


An SSA statement (or program transformer) $s \in \constraints$ transforms the 
memory state of a program and is 
therefore associated with a strongest postcondition or an existential 
precondition transformer. Let us assume that each SSA statement is also 
associated with a transition relation, $\mathcal{ST}_{s}^{\Omega}$. 
%
\begin{definition} (Strongest Postcondition and Existential Precondition).
  \[
     post_{s}(A) \mathrel{\hat=} \{\omega' \mid \exists \omega \in
     \Omega. \omega \in A \wedge (\omega,\omega') \in \mathcal{ST}_{s}^{\Omega} \} 
  \] 
  \[ 
     pre_{s}(A) \mathrel{\hat=} \{\omega \mid \exists \omega' \in
     \Omega. \omega' \in A \wedge (\omega,\omega') \in \mathcal{ST}_{s}^{\Omega}\} 
  \]
\end{definition}
%
\begin{definition} (Weakest Precondition and Universal Postcondition). 
  \[
    \widehat{pre_s}(A) \mathrel{\hat=} \{\omega \mid \forall \omega' \in
    \Omega.\; \omega' \in A \vee (\omega,\omega') \not\in \mathcal{ST}_{s}^{\Omega} \}.
  \]
  \[ 
     \widehat{post_s}(A) \mathrel{\hat=} \{\omega' \mid \forall \omega \in
     \Omega.\; \omega \in A \vee (\omega,\omega') \not\in
     \mathcal{ST}_{s}^{\Omega} \}.
  \]
\end{definition}
%
\Omit{
\begin{definition} (Static Assignment Transformers). 
   $
     post_{\mathcal{SA}}, pre_{\mathcal{SA}} : \mathcal{SA}_G
      \rightarrow \mathcal{SA}_{G} 
   $
   \[
     post_{\mathcal{SA}}(a) \mathrel{\hat=} \bigcup_{s \in \rmcmt{\constraints}}
     post_{s}(a) 
     \quad
     pre_{\mathcal{SA}}(a) \mathrel{\hat=} \bigcup_{s \in \rmcmt{\constraints}}
     pre_{s}(a) 
   \]
\end{definition}
}
%
\begin{figure}[t]
\centering
\vspace*{-0.2cm}
\scalebox{.70}{\import{figures/}{state_transformer.pspdftex}}
  \caption{Strongest postcondition $post_s$ (Blue), Existential precondition
  $pre_s$ (Red), Universal postcondition $\widehat{post_s}$ (Pink), 
  Weakest precondition $\widehat{pre_s}$ (Green)}
\label{state-transformer}
\end{figure}
%
Figure~\ref{state-transformer} graphically shows the result of various
state transformers for a given memory state $A$. A strongest postcondition 
maps a set of states $A$ to the set of all successor states that can be reached
in one step, whereas an existential precondition maps a set of states $A$ to
states the program \emph{may} have before executing $s$. A weakest precondition maps a set of states $A$ to the set
of states which can \emph{only} reach elements of $A$. Whereas, an universal
postcondition maps $A$ to set of states the program \emph{must} reach after
executing $s$.
%

Recall that the trace transformers $tpost$ and $tpre$ are defined over
$\powerset(\Pi)$. The transformers, $post_{s}$ and $pre_{s}$, soundly 
approximate $tpost$ and $tpre$ respectively.  \rmcmt{how ? --relate to syntactic
transformation T2}
%
The global static assignment transformers for the lattice $\mathcal{SA}_G$ over 
a set of SSA constraints $\constraints$ can be easily derived from state
transformers, $post_s, pre_s$.  This is defined next.
%
\begin{definition} (Global Static Assignment Transformers). 
  \[ 
  post_{\constraints}, pre_{\constraints}, 
  \widehat{post_{\constraints}}, \widehat{pre_{\constraints}}, 
  \colon \mathcal{SA}_G \rightarrow \mathcal{SA}_G
  \]
    
  \[
  post_{\constraints}(a) \mathrel{\hat=} \underset{\sigma \in
  \constraints}{\bigcap} post_\sigma \circ a
  \qquad  
  pre_{\constraints}(a) \mathrel{\hat=} \underset{\sigma \in
  \constraints}{\bigcap} pre_\sigma \circ a
  \]
  
  \[
  \widehat{post_{\constraints}}(a) \mathrel{\hat=} \underset{\sigma \in
  \constraints}{\bigcap} \widehat{post_\sigma} \circ a
  \qquad   
  \widehat{pre_{\constraints}}(a) \mathrel{\hat=} \underset{\sigma \in
  \constraints}{\bigcap} \widehat{pre_\sigma} \circ a
  \]
\end{definition}
%
\rmcmt{The transformers, $post_{\constraints}$ and $pre_{\constraints}$, soundly 
approximate $tpost$ and $tpre$ respectively. (say how?) }
%
We now define an abstraction of static assignment lattice which we 
call \emph{abstract static assignment} domain. 
The concrete static assignment domain is a complete lattice of concrete 
SSA environments.  The abstractions of concrete static assignment domain fall
into two categories; an abstraction that preserves the relationship between 
SSA variables and the other which does not preserve any relation. 
%

First, we give an abstraction of static assignment lattice that does not
preserve relationship between SSA variables.  Then, we present a generic 
abstraction using a \emph{template-based abstract domain} that can express 
relational as well as non-relational abstractions of SSA environments.  
%An advantage of a template-based domain is that it can be instantiated with 
%arbitrary templates over relational or non-relational domains.  
This provides the flexibility to instantiate CDCL over arbitrary abstract domains for 
program analysis.  
%
\para{Abstract Static Assignment Lattice Over Non-relational Domains}
%
An abstraction of $\mathcal{SA}_G$ over non-relational domain requires each SSA
variables in $SSA_{var}$ to be abstracted independently of each other.  
The concrete domain of $\powerset(Var_{ssa} \rightarrow Value)$ is abstracted 
through a function which maps each SSA variable to an abstract value.  We 
assume that there is a galois connection $(\alpha_v,\gamma_v)$ such that
  $(Value, \subseteq)
   \galois{\alpha_{v}}{\gamma_{v}}
   (\widehat{Value}, \sqsubseteq)$.   
%
Then, an abstraction of $\mathcal{SA}_G$ over non-relational domain, denoted by
$\mathcal{SA}_G^\dagger = \powerset(Var_{ssa} \rightarrow \widehat{Value})$, 
can be constructed using the following galois connection 
$(\alpha^\dagger,\gamma^\dagger)$.
\[
   (\mathcal{SA}_{G},\subseteq_{SA})
   \galois{\alpha^\dagger}{\gamma^\dagger}
   ({\mathcal{SA}_{G}}^\dagger,\sqsubseteq^\dagger)
\]
\[
  \alpha^\dagger(c) \mathrel{\hat=} \lambda x.\alpha_v(\bigcup\{\tau(x) \mid \tau
  \in c\})
\]
\[
  \gamma^\dagger(a) \mathrel{\hat=} \{\lambda x.v' \mid v' \in (\gamma_v \circ
  a)(x)\}
\]
%
\begin{example}
%
An example of $({\mathcal{SA}_{G}}^\dagger,\sqsubseteq^\dagger)$ over an
Interval domain $Itv$ maps each SSA variable on an Interval, where 
$(\alpha_v,\gamma_v) = (\alpha_{Itv}, \gamma_{Itv})$.  Consider an SSA 
$c=\{x_1:=2, x_2 \geq 5\}$, then $\alpha^\dagger(c) = \{x_1=[2,2],x_2=[5,\infty]\}$
\end{example}
%
\para{Abstract Static Assignment Lattice Over Template-based Domains}
%
A folk wisdom in abstract interpretation community is that an analysis using 
non-relational domains are effective but imprecise. However, for practical
purposes, it is imperative to keep track of relations between program variables.  
A relational domain can express relationship between variables, though with varying
expressivity, depending on a weak or strong relational domain.  We now construct
an abstraction of $\mathcal{SA}_G$ using a template-based domain, denoted by 
$\widehat{\mathcal{SA}_{G}}$, which can be instantiated with arbitrary templates 
over relational or non-relational domains. Note that from this point onwards, we 
will operate on $\widehat{\mathcal{SA}_{G}}$ instead of ${\mathcal{SA}_{G}}^\dagger$. 
%
\begin{definition}~\label{assl} (Abstract Static Assignment Lattice). An \emph{Abstract Static 
Assignment Lattice} corresponding to a CFG $G= (V, E, \mathcal{S}, \mathcal{T}, I)$ 
and a set $X$ of values of vector $\vec{x}$ of variables in the 
corresponding SSA, is a complete lattice 
  $(\widehat{\mathcal{SA}_{G}}, \sqsubseteq_{SA}, \sqcap_{SA}, \sqcup_{SA})$, which 
is defined as follows. 
\[
  \widehat{\mathcal{SA}_{G}} \mathrel{\hat=} C\vec{x} \leq \vec{d},\; \text{where}\; 
  \vec{d}\;\text{is a constant vector and C is a coefficient matrix} 
\]
\end{definition}
%
The abstraction and concretization functions $(\alpha_{SA}, \gamma_{SA})$ form 
a galois connection.
%
\[
  (\mathcal{SA}_{G},\subseteq_{SA})
   \galois{\alpha_{SA}}{\gamma_{SA}}
   (\widehat{\mathcal{SA}_{G}},\sqsubseteq_{SA})
\]
%
\[
  \alpha_{SA}(\numconcval) = \min \{\vec{\numabsval}\mid
  \mat{C}\vec{\numvar}\leq\vec{\numabsval}, \vec{\numvar}\in
  \numconcval\},\;\text{where}\; \min\; \text{is applied component-wise}.  
\]  
%  
\[  
   \gamma_{SA}(\vec{\numabsval}) \mathrel{\hat=} \{\vec{\numvar}\mid
   \mat{C}\vec{\numvar}\leq\vec{\numabsval}, \vec{\numvar} \in X\} 
   \qquad \gamma_{SA}(\bot)=\emptyset
\]
%



%
Figure~\ref{fig:abstract} shows the abstract static assignment lattice over an
Interval abstract domain.  The elements marked in bold corresponds to concrete 
assignments to SSA variables that maps to a concrete program trace. 
%
\begin{figure}[htbp]
\centering
\vspace*{-0.2cm}
  \scalebox{.90}{\import{figures/}{abstract_env.pspdftex}}
\caption{Abstract Static Assignment Lattice over a boolean SSA variable $p$ and two
  numerical SSA variables $x$ and $y$ that takes values in Interval domain
  \label{fig:abstract}}
\end{figure}
%
\begin{example}
For example, consider the SSA elements $\{x_1\geq 0, x_1-z\leq 30\}$, then
abstract domain value is $\vec{\numabsval}=\vecv{0}{30}$,
with $\mat{C}=\qmat{-1}{0}{1}{-1}$ and $\vec{\numvar}=\vecv{x_1}{z}$. \\ 
\end{example}
%
\begin{definition} (Meet Irreducibles of $\widehat{\mathcal{SA}_G})$
%

A meet irreducible of \emph{abstract static assignment} domain
$\widehat{\mathcal{SA}_{G}}$ where $\mat{C_i}$ is $i$-th row vector  
of a matrix of size $N \times M$ and $\vec{\numabsval}$ is a vector 
  of size $N$, is defined below.  \rmcmt{Here, $MAX$ is the largest 
  value of $\vec{d}$ determined by the matrix $\mat{C}$}.
%   
\rmcmt{
  \[
     \meet_{irrd}(\widehat{\mathcal{SA}_{G}}) \mathrel{\hat=} 
     \{\vec{d} \mid \exists_{=1}\;d_i \in \vec{\numabsval}.\;(d_i \neq MAX) \wedge
     C_{i}\vec{x} \leq d_i\} \;\text{where}\;(i \leq N)
  \]
}  
Informally, a meet irreducible of $\widehat{\mathcal{SA}_{G}}$ is the abstract
value $\vec{d}$ such that there exists exactly one element of $\vec{d}$ that is
not $MAX$.
%
\end{definition}
%
%\rmcmt{Show how MAX is computed}.
We now discuss how $MAX$ is computed.  For as SSA element 
$\hmat{1}{-1} \vecv{x}{y} \leq d$ where $x$ and $y$ are 
32-bit signed integers (marked as s32 in figure~\ref{fig:max}), 
the total number of bits to represent $d$ is 34.  Thus, the 
value of $MAX$ is the largest value representable in 34 bits.
%
\begin{figure}[htbp]
\centering
\vspace*{-0.2cm}
  \scalebox{.90}{\import{figures/}{max.pspdftex}}
\caption{Computing $MAX$ from bit-width of $d$
  \label{fig:max}}
\end{figure}
%


Meet irreducibles of $\widehat{\mathcal{SA}_G}$ over an \emph{Interval} domain 
with template $[l,u]$ such that $\vecv{-1}{1}(\vec{x})\leq \vecv{l}{u}$ is given
by $\{\vecv{-l}{MAX}, \vecv{MAX}{u}\}$.
%
\begin{example}
  An example of meet irreducible in $\widehat{\mathcal{SA}_{G}}$ over 
  $\vec{\numvar} = \vecv{a}{b}$ that takes values in \emph{Interval} domain 
  is given by, $\meet_{irrd}(\{a:[5,7]\}) = \{\vecv{-5}{MAX}, \vecv{7}{MAX}\}$ 
  with $\mat{C}=\qmat{1}{0}{0}{0}$. Here, the abstract values of variable 
  $b$ is equal to $MAX$. \\
\end{example}  
%
\begin{example} \rmcmt{Octagon}
  An example of meet irreducible in $\widehat{\mathcal{SA}_{G}}$ over 
  $\vec{\numvar} = \vecv{a}{b}$ that takes values in \emph{Octagon} domain 
  is given by, 
  \[\meet_{irrd}(\{a+b \leq 5\}) =
  \vecvvv{MAX}{MAX}{MAX}{MAX}{5}{MAX}{MAX}{MAX}\]
    
  \[\text{with}\;\; \mat{C}=
  \begin{bmatrix}
    0 & 1 \\
    1 & 0 \\
    0 & -1 \\
    -1 & 0 \\
    1 & 1 \\
    1 & -1 \\
    -1 & 1 \\
    -1 & -1
  \end{bmatrix}
  \]
  %\mat{c}=\qqqmat{0}{1}{1}{0}{0}{-1}{-1}{0}{1}{1}{1}{-1}{-1}{1}{-1}{-1}\] 
   Here, the abstract values of $a$, $b$, $-a$, $-b$, $(a-b)$, $(-a+b)$, $(-a-b)$ are
  equal to $MAX$. \\
\end{example}  
%


An advantage of abstract static assignment domain $\widehat{\mathcal{SA}_{G}}$ 
is that it can be instantiated over arbitrary relational or non-relational 
numerical abstract domains.  This gives the flexibility to instantiate CDCL for 
safety verification over arbitrary abstract domains. 
%
We will use the template-based abstraction for the rest of the paper. 
%%%%%%%%%%%%%%%%%%%%%%%%% PROOF %%%%%%%%%%%%%%%%%%%%%%%%
\begin{proposition}~\label{ag}
  $
  (\powerset({\Pi}),\subseteq)
    \galois{\alpha_{T}}{\gamma_{T}}
  (\mathcal{SA}_{G},\subseteq_{SA})
   \galois{\alpha_{SA}}{\gamma_{SA}}
   (\widehat{\mathcal{SA}_{G}},\sqsubseteq_{SA})
  $ 
\end{proposition}
%
\begin{proof}
  The lattice $\mathcal{SA}_G$ is obtained from the concrete lattice of 
  traces $\powerset(\Pi)$ via syntactic translation steps $\mathcal{T}_1$ and
  $\mathcal{T}_2$, as shown in figure~\ref{fig:semantic}.  Both translation
  steps are exact, that is, $\powerset(\Pi)$ and $\mathcal{SA}_G$ form an 
  exact abstraction via the galois 
  connection $(\alpha_T, \gamma_T)$. This implies, $\alpha_T(\pi) = \tau$ where
  $\tau \in \mathcal{SA}_G$.  Also, $\gamma_T(\tau) = \pi \in \powerset(\Pi)$,
  since an SSA can be exactly mapped back to the program trace following the
  syntactic translation steps proposed in~\cite{Briggs:1998}.   
  Thus, there is no loss or gain of information between elements of 
  $\mathcal{SA}_{G}$ and $\powerset(\Pi)$.


  Furthermore, the lattice $\widehat{\mathcal{SA}_{G}}$ is obtained via 
  two-step galois connection $(\alpha_{T} \circ \alpha_{SA})$ from
  the concrete lattice of traces $\powerset(\Pi)$. The abstraction $\alpha_{T}$
  is exact, whereas, $\alpha_{SA}$ is an over-approxation by
  definition~\ref{assl}.
  By \rmcmt{transitivity, (find reference)}, the lattice 
  $\widehat{\mathcal{SA}_{G}}$ over-approximates $\powerset(\Pi)$.
\end{proof}
%
\Omit{
\begin{definition}
A trace $\pi$ in abstract static assignment domain $\widehat{\mathcal{SA}_{G}}$ 
is a sequence of abstract states of the form $\{C_{ij}\vec{x} \leq \vec{d_{i}}$,
starting with an abstract element $\top$. We use $\Pi$ to denote set of traces. 
\end{definition}
}
%
The abstract transformers for the lattice $\widehat{\mathcal{SA}_{G}}$ 
transforms a memory state of a program and therefore associated with state 
transformers.  For every SSA statement $s \in \constraints$, let
$apost_{s}$ and $apre{s}$ be sound over-approximations of $post_{s}$ and
$pre_{s}$ in the lattice $\widehat{\mathcal{SA}_G} \rightarrow 
\widehat{\mathcal{SA}_G}$, respectively. 
%
Th global abstract static assignment transformers for the 
lattice $\widehat{\mathcal{SA}_G}$ over a set of SSA constraints 
$\constraints$ are obtained from abstract state transformers, 
$apost_s, apre_s$. This is defined next.
%
\begin{definition} (Global Overapproximate Abstract Static Assignment Transformers). 
  \[ 
     apost_{\constraints}, apre_{\constraints} : \widehat{\mathcal{SA}_G}
     \rightarrow \widehat{\mathcal{SA}_{G}} 
   \]
   \[
     apost_{\constraints}(a) \mathrel{\hat=} 
     \underset{\sigma \in \constraints}{\bigsqcap} apost_{\sigma} \circ a 
   \]
  \[
     apre_{\constraints}(a) \mathrel{\hat=} 
     \underset{\sigma \in \constraints}{\bigsqcap} apre_{\sigma} \circ a 
   \]
\end{definition}
%
The transformers $apost_{\constraints}$ and $apre_{\constraints}$ soundly overapproximates 
$post_{\constraints}$ and $pre_{\constraints}$, respectively. 
%
\subsection{Complementation in Abstract Static Assignment Lattice}~\label{complement}
%
Recall that every meet irreducibles of a partial assignments domain admits a
precise complement.  This property of the partial assignments domain enables a
CDCL solver to learn a conflict clause that is obtained by complementing a 
conflict reason.  In section~\ref{complement-fg}, we showed that the elements 
of abstract flow lattice $\widehat{\mathcal{F}_G}$ does not always have precise 
complements.  In this section, we show that the meet irreducibles of 
$\widehat{\mathcal{SA}_G}$ have precise complements. 
%
\begin{definition}~\label{meet-decomp}
A \emph{meet decomposition} $\decomp(\absval)$ of an abstract 
element $\absval \in \widehat{\mathcal{SA}_{G}}$ is a set of meet 
irreducibles $M \subseteq \widehat{\mathcal{SA}_{G}}$ such that 
$\absval=\bigsqcap_{m\in M} m$.
\end{definition}
%
\begin{example} The meet decomposition of the interval domain element
$decomp(2\leq x\leq 4 \wedge 3\leq y\leq 5)$ is
the set $\{x\geq 2, x\leq 4, y\geq 3, y\leq 5\}$.
\end{example}
%
Recall that a precise complement of an element $a \in \widehat{\mathcal{SA}_G}$ 
is an element $\bar{a} \in \widehat{\mathcal{SA}_G}$ such that $\neg \gamma_{SA}(\bar{a}) =
\gamma_{SA}(a)$.  The meet irreducibles of Abstract Static Assignment lattice 
$\widehat{\mathcal{SA}_G}$ admits precise complements. This is explained below. 
%
\[
   \gamma_{SA}(\vec{\numabsval}) \mathrel{\hat=} \{\vec{\numvar}\mid
   \mat{C}\vec{\numvar}\leq\vec{\numabsval}\} 
\]
\[
   \gamma_{SA}(\bar{\vec{\numabsval}}) \mathrel{\hat=} \{\vec{\numvar}\mid
   \mat{C}\vec{\numvar} > \vec{\numabsval}\} 
\]
\[
   \neg \gamma_{SA}(\bar{\vec{\numabsval}}) \mathrel{\hat=} \{\vec{\numvar}\mid
   \mat{C}\vec{\numvar} \leq \vec{\numabsval}\} 
\]
%
\begin{example}
Let an abstract SSA environment be $a = \{p:t,x\geq0,y\geq0\}$, 
then $\bar{a} = \{p:f,x<0,y<0\}$ and $\neg \bar{a} = \{p:t,x\geq0,y\geq0\}$.
\end{example}
%
\begin{example}
The precise complement of a meet irreducible $(x \leq 2)$ in
the interval domain over integers is $(x \geq 3)$, or the precise complement
of the meet irreducible $(x+y \leq 1)$ in the octagon domain over integers
is $(x+y \geq 2)$.  
\end{example}
%
\begin{table}
\scriptsize
\begin{center}
{
\begin{tabular}{l|l|l|l|l}
\hline
  Domain & Concrete & Abstract & Meet & Complemented Meet \\ 
  Name  & Domain & Elements & Irreducible & Irreducible \\ \hline
Interval & $\powerset(Var \rightarrow \mathbb{Z}) \cup \bot$  & $x = [l,u]$ & $x\leq N$ & $x > N$ \\ \hline
Octagon &  $\powerset(Var \times Var \rightarrow (\mathbb{Z} \cup
\{\infty\})) \cup \bot$ & $(\pm x_i \pm x_j \leq d)$ & $(x+y \leq N)$ & $(-x-y < N)$ \\ \hline
  Equality & $\powerset(Var \rightarrow \mathbb{Z}) \cup \bot$ & $(x=y)$ &
  $(x=y)$ & $(x \neq y)$ \\ \hline 
 Zones &  $\powerset(Var \times Var \rightarrow (\mathbb{Z} \cup
\{\infty\})) \cup \bot$ & $(x_i - x_j \leq d)$ & $(x+y < N)$ & $(-x-y \leq N)$ \\ \hline
\end{tabular}
}
\end{center}
\label{fig:complement}
\end{table}
%
Table~\ref{fig:complement} shows that most abstract domains admit precise
complements.  We now show that the meet irreducibles of 
$\widehat{\mathcal{SA}_{G}}$ have precise complements when instantiated over
arbitrary abstract domains. 
%
\begin{proposition} 
  $\gamma_{SA}(C_{i}\vec{x} \leq \vec{d_{i}}) = 
  (\neg \gamma_{SA}(C_{i}\vec{x} > \vec{d_{i}}))$
\end{proposition}
%
\begin{proof}
  Assuming that $\vec{d}$ is chosen from a non-relational domain such as 
  Interval domain $Itvs$, the meet 
  irreducibles are of the form $\{\vec{x} \diamond \vec{d}\}$, where
  $\diamond =
  \{<,>,\leq,\geq\}$. It is easy to see that the meet irreducibles of $Itvs$ are easily
  complementable.  Similarly, assuming that $\vec{d}$ is chosen from a
  relational domain such as Octagon 
  domain $Octs$, the meet irreducibles are of the form $\{\mat{C}\vec{x}
  \diamond \vec{d}\}$, which also admits precise complements.  However, for
  element $a \in \widehat{\mathcal{SA}_{G}}$, that does not admit precise 
  complementation, it can be decomposed into half spaces, 
  $decomp(a)=\{m_1, m_2, \ldots m_k\}$, such that each $m_i$ admits precise
  complements. 
\end{proof}
%
\Omit{
\para{Elements in $\mathcal{SA}_G$ gives a Program Trace}
%
We use the result of~\cite{Briggs:1998} in Lemma~\ref{ssa-model} that shows 
a model of $\varphi$ corresponds to a concrete counterexample trace in the 
original program. 
%
\begin{lemma}
  \[ \mathcal{T}_2(\gamma(a)) = \{\pi \mid \pi \in \Pi \wedge a \in
  \widehat{\mathcal{SA}_G} \} \]
\end{lemma}
%
\begin{proof}

\end{proof}
}
%
