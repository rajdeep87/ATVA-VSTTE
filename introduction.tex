\section{Introduction}

\para{Conflict Driven Clause Learning Solvers} 
%
A Conflict Driven Clause Learning algorithm~\cite{cdcl} is a boolean satisfiability
procedure that alternates between a model search phase and conflict analysis
phase to solve a propositional formula given in conjunctive normal form (CNF).  
The model search phase uses \emph{decisions} and \emph{Boolean Constraint
Propagation} (BCP) to search for a satisfying assignment of the formula.  A 
decision uses branching to assign the chosen value to a chosen branching variable.  
Decision is followed by BCP step. BCP uses repeated application of unit rule where 
an unit rule identifies variables which must be assigned specific Boolean value. 
If BCP identifies an unsatisfied clause, the CDCL algorithm is said to be in 
\emph{conflict} and the algorithm \emph{backtracks} non-chronologically. A
backjumping procedure undoes all branching steps until the decision level 
where the solver state is consistent (non-conflicting).  This is a key 
difference with a DPLL solver which just backtracks to the previous 
decison level and flips it. The backjumping level in CDCL is determined 
by analyzing the most recent conflict and learning a new clause from the 
conflict.  If the backtracking level is the root of the search tree, then 
the solver terminates with no models, that is, the formula is \emph{unsatisfiable}.  
Else, the model search phase is repeated with the learnt clause and the procedure 
continues until all variables has been assigned in which case, the 
algorithm returns a satisfiable assignment, that is, the formula is \emph{satisfiable}. 
%

\para{Connection between CDCL and Program Analysis}
%
Thus, CDCL solvers can be understood as a procedure to compute approximations of fixed 
points over a lattice of partial assignments~\cite{sas12}.  A key insight 
that connects CDCL to program analysis is that they use an imprecise 
over-approximate domain of partial assignments to gain efficiency and 
techniques like decision and clause learning to improve precision.  
%
Silva et. al. in~\cite{popl2014} propose an understanding of CDCL in the language
of lattices and transformers suggesting a ``\emph{Grand Unification}" of SAT
and static analysis.
%
In this paper, we take one step towards this unification goal by characterizing CDCL
as a procedure for computing fixed point approximations over a lattice
of program traces to determine program safety.  
%learning based program analyzers.  
A practical benefit of this characterization is that the CDCL architecture can
be used to build a precise learning based static analyzers that operates over 
arbitrary non-distributive abstract domains~\cite{atva2017}. \rmcmt{update
reference} \\
%


CDCL can be understood as a general algorithmic framework, parameterized 
by a concrete domain $C$ and an abstract domain $A$, where $C$ is the 
set of propositional truth assignment and $A$ the domain of partial assignments.  
A characterization 
of CDCL as a program analyzer requires the concrete domain to 
be instantiated over a lattice of program traces that may lead to an error.  
Given a \emph{safe trace transformer} which returns a set of safe or 
invalid traces and an \emph{unsafe trace transformer} which returns a set 
of unsafe traces, satisfiability can then be seen as a property of fixed points of
unsafe trace transformer over this lattice.  A model search
overapproximates the unsafe trace transformer and conflict analysis
underapproximates the safe trace transformer.  Decisions refines a downward
iteration sequence and learning overapproximates set 
of unsafe traces.  This paper presents a theoretical recipe and a mathematical
basis for instantiating CDCL architecture for program analysis using the
framework of abstract interpretation.  
%to formalize an abstract interpretation account of Bounded Model Checking using CDCL
%architecture.  
In this paper, we restrict our formalizations to programs 
with bounded loops and finite recursion depth.
We expect that the characterizations presented in this paper would contribute 
to the development of precise and efficient static analyzers that can 
automatically recover from imprecision without employing expensive 
abstract domains such as trace partitioning~\cite{toplas07}.  
One such practical instantiation of CDCL as learning based program
analyzer~\cite{atva2017} \rmcmt{update reference} over a template based abstract domains is 
demonstrated in this paper. 
%
\para{Structure of the paper}
%
In this paper, we first briefly explain the work of Silva et. al.~\cite{sas12,
popl2014} that gives an abstract interpretation account of CDCL algorithm. 
We then present a novel abstract interpretation account of bounded model
checking using CDCL algorithm.  To this end, we first show that conventional
trace based abstractions are not ideal for lifting CDCL to program analysis due
to lack of precise complementation property. So,
we introduce an abstraction of program traces over a logical encoding of program 
using Static Single Assignment (SSA) form.  We then present a novel safety
verification algorithm called \emph{Abstract Conflict Driven Learning for Programs (ACDLP)} 
that uses decision and learning techniques to precisely reason over a lattice of
SSA elements.  We conclude that a practical benefit of abstract interpretation 
account of BMC using CDCL algorithm gives a new, learning based architecture for 
implementing precise program analysis tools.   
%
\para{Contributions}
In this paper, we make the following contributions.
%
\begin{enumerate}
  \item We characterize satisfiability as a property of fixed points of
  unsafe trace transformer over the lattice of program traces.
  \item We present an abstraction of trace semantics that is suitable for
  lifting CDCL architecture to program analysis. 
  \item We characterize model search as a technique to compute a greatest
    fixed point of an unsafe trace transformer using deductions and decisions.
  \item A conflict analysis is characterized as a technique to compute
    a least fixed point of safe trace transformer over a downset lattice 
    with heuristic choice of conflict reasons.  We show that learning is 
    a reduction over unsafe trace transformers, parameterized by an element of downset lattice.  
\end{enumerate}
%

