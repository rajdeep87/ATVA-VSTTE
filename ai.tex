\section{An Example of Abstract Interpretation}
%
\begin{figure}[t]
\centering
\begin{tabular}{c|c}
\hline
Control-Flow Graph & Static Analysis Equation \\
\hline
\begin{minipage}{5.0cm}
\scalebox{.65}{\import{figures/}{ai-example.pspdftex}}
\end{minipage}
&
\begin{minipage}{7cm}
$\begin{array}{l}
     S_{n1} = \top, \\
     S_{n2} = post_{x:=2}(S_{n1}), \\
     S_{n3} = post_{x:=x*x}(S_{n2}) \cup post_{x:=x+2}(S_{n4}), \\ 
     S_{n4} = post_{x < 10}(S_{n3}), \\
     S_{n5} = post_{x\geq 10}(S_{n3}) \\
     S_{Error} = post_{x\neq10}(S_{n5}) \\
     S_{Safe} = post_{x=10}(S_{n5}) \\
\end{array}$
\end{minipage}
\\
\hline
\end{tabular}
\caption{\label{fig:se} A CFG and its static analysis equation}
\end{figure}
%
We present an example of a classical abstract interpretation of program.  
Figure~\ref{fig:se} shows a CFG and its corresponsing static analysis equations.  
A static analysis equation encodes the data-flow between individual control-flow 
nodes in the CFG and is given by a set valued variable $S_n$ for each location $n$ 
in the CFG.  Recall that each statement $s$ in the program is associated with a
postcondition transformer, $post_s(S_n)$, that computes the successor state of a 
statement $s$ starting from $S_n$ that can be reached in one step.  Static analysis 
using abstract interpretation usually computes a fixed point over the static analysis 
equation obtained from the Control Flow Graph (CFG) representation of a program~\cite{CC79}.  
%

Let us consider the equations in Figure~\ref{fig:se} that models the loop, 
$S_{n3} = post_{x:=x*x}(S_{n2}) \cup post_{x:=x+2}(S_{n4})$, 
$S_{n4} = post_{z \leq 10}(S_{n3})$.  These equations can be written as 
a function, $F(X)= \{4\} \cup \{x+2 \mid x \in X, x \leq 10\}$, where 
$post_{x:=x*x}(S_{n2}) = {4}$.  Assuming variable $x$ is an integer, the 
lattice of integers with a subset relation is $(\powerset(\mathbb{Z}),\subseteq)$. 
%

Standard means to infer loop invariant is to compute fixed points of function 
over this lattice structure~\cite{CC79,octagon}.  The fixed point of the above 
function gives the set $X$ that satisfies $F(X)=X$.  The loop invariant is given 
by $x\geq 4 \wedge x\leq 10 \wedge x\equiv 0\;(\bmod 2)$.  The set of values of 
$x$ satisfying the loop invariant is the least fixed point. 
%

Abstract interpretation of program computes a fixed point of an abstract
function (called abstract transformer), over an abstract lattice.  Assuming an
interval lattice $Intv$ which maps a set of integer values of a variable to 
its smallest interval that contains it, the function $F$ above is abstracted 
over an interval lattice as shown below.
%
\[ F^{\sharp}([a,b]) = [4,4] \sqcup ([a,b] \sqcap [-\infty,10]) +_{Intv} [2,2] \]
%
The initial values of $x$ is $[4,4]$ which is obtained by interval analyis of
the static analysis equation, $S_{n3} = post_{x:=x*x}(S_{n2})$. The 
function $F^{\sharp}$ computes an interval at each iteration where the interval 
below 10 is incremented by 2 and $(+_{Intv})$ denotes an addition operation in
the interval lattice. 
%
Figure~\ref{fig:fixpoint} shows the fixed point computation of the loop in 
Figure~\ref{fig:se}, over a lattice of intervals.  Each column denotes an
iteration of the fixed point computation which associates an interval with each
location in the program.  The initial value of $x$ is $\top$ at $n1$, while the
locations $n2$, $n3$, $n4$, $n5$, $Error$, $Safe$ are considered unreachable. 
Each iteration of the loop computes a bound on the variable $x$. 
The interval $x\colon[4,10]$ at the loop head $n3$ in the last iteration 
is the loop invariant. 
In practise, the total number of iterations may be too large to reach the fixed 
point. Hence, techniques like widening and narrowing are used to accelerate
convergence. 
%
\begin{figure}[t]
\centering
\begin{tabular}{ccccc}
\hline
  Control Location & Iteration 1 & Iteration 2 & Iteration 3 & Iteration 4\\
\hline
  $n1$ & $x\colon\top$ & $x\colon\top$ & $x\colon\top$  & $x\colon\top$ \\ 
  $n2$ & $x\colon[2,2]$ & $x\colon[2,2]$ & $x\colon[2,2]$ & $x\colon[2,2]$ \\
  $n3$ & $x\colon[4,4]$ & $x\colon[4,6]$ & $x\colon[4,8]$ & $x\colon[4,10]$ \\
  $n4$ & $x\colon[6,6]$ & $x\colon[6,8]$ & $x\colon[6,10]$ & $x\colon[6,10]$ \\
  $n5$ & $x\colon\bot$ & $x\colon\bot$ & $x\colon\bot$ & $x\colon[10,10]$ \\
  $Error$ &$x\colon\bot$ & $x\colon\bot$ & $x\colon\bot$ & $x\colon\bot$ \\
  $Safe$ &$x\colon\bot$ & $x\colon\bot$ & $x\colon\bot$ & $x\colon[10,10]$ \\
\hline
\end{tabular}
  \caption{\label{fig:fixpoint} Fixed point computation of program in
  Figure~\ref{fig:se}}
\end{figure}
%
